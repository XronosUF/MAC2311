%%%%%%%%%%%%%%%%%%%%%%%
%%\tagged{Cat@One, Cat@Two, Cat@Three, Cat@Four, Cat@Five, Ans@ShortAns, Type@Compute, Topic@Limit, Sub@Poly, Sub@Trig, Sub@Continuity}{

\latexProblemContent{
\begin{problem}
Determine if the limit approaches a finite number, infinity, or does not exist. (If the limit does not exist, write DNE)
\input{./Limits/2311-Compute-Limit-0009.HELP.tex}
\[\lim_{x\to{-5}}{-4 \, {\left(x^{3} + 8 \, x^{2} + 11 \, x - 20\right)} \tan\left(\frac{1}{6} \, \pi x\right)}=\answer{0}\]

\end{problem}}%}

%%%%%%%%%%%%%%%%%%%%%%




\latexProblemContent{
\begin{problem}
Determine if the limit approaches a finite number, infinity, or does not exist. (If the limit does not exist, write DNE)
\input{./Limits/2311-Compute-Limit-0009.HELP.tex}
\[\lim_{x\to{-3}}{5 \, {\left(x^{3} - 12 \, x^{2} + 45 \, x - 50\right)} \cos\left(\frac{1}{3} \, \pi x\right)}=\answer{1600}\]

\end{problem}}%}

%%%%%%%%%%%%%%%%%%%%%%




\latexProblemContent{
\begin{problem}
Determine if the limit approaches a finite number, infinity, or does not exist. (If the limit does not exist, write DNE)
\input{./Limits/2311-Compute-Limit-0009.HELP.tex}
\[\lim_{x\to{-3}}{3 \, {\left(x - 1\right)} {\left(x - 3\right)} {\left(x - 4\right)} \cos\left(\frac{4}{3} \, \pi x\right)}=\answer{-504}\]

\end{problem}}%}

%%%%%%%%%%%%%%%%%%%%%%




\latexProblemContent{
\begin{problem}
Determine if the limit approaches a finite number, infinity, or does not exist. (If the limit does not exist, write DNE)
\input{./Limits/2311-Compute-Limit-0009.HELP.tex}
\[\lim_{x\to{3}}{-2 \, {\left(x^{3} + 11 \, x^{2} + 38 \, x + 40\right)} \sin\left(-\frac{5}{2} \, \pi x\right)}=\answer{-560}\]

\end{problem}}%}

%%%%%%%%%%%%%%%%%%%%%%




\latexProblemContent{
\begin{problem}
Determine if the limit approaches a finite number, infinity, or does not exist. (If the limit does not exist, write DNE)
\input{./Limits/2311-Compute-Limit-0009.HELP.tex}
\[\lim_{x\to{-2}}{{\left(x^{3} - 5 \, x^{2} - x + 5\right)} \cos\left(\frac{5}{2} \, \pi x\right)}=\answer{21}\]

\end{problem}}%}

%%%%%%%%%%%%%%%%%%%%%%




\latexProblemContent{
\begin{problem}
Determine if the limit approaches a finite number, infinity, or does not exist. (If the limit does not exist, write DNE)
\input{./Limits/2311-Compute-Limit-0009.HELP.tex}
\[\lim_{x\to{2}}{4 \, {\left(x^{2} - 16\right)} \tan\left(-2 \, \pi x\right)}=\answer{0}\]

\end{problem}}%}

%%%%%%%%%%%%%%%%%%%%%%




\latexProblemContent{
\begin{problem}
Determine if the limit approaches a finite number, infinity, or does not exist. (If the limit does not exist, write DNE)
\input{./Limits/2311-Compute-Limit-0009.HELP.tex}
\[\lim_{x\to{-3}}{3 \, {\left(x^{3} - 9 \, x^{2} + 26 \, x - 24\right)} \tan\left(\frac{2}{3} \, \pi x\right)}=\answer{0}\]

\end{problem}}%}

%%%%%%%%%%%%%%%%%%%%%%




\latexProblemContent{
\begin{problem}
Determine if the limit approaches a finite number, infinity, or does not exist. (If the limit does not exist, write DNE)
\input{./Limits/2311-Compute-Limit-0009.HELP.tex}
\[\lim_{x\to{-4}}{-3 \, {\left(x^{2} + 8 \, x + 15\right)} \sin\left(-\frac{5}{3} \, \pi x\right)}=\answer{\frac{1}{2} \, \left(3 \, \sqrt{3}\right)}\]

\end{problem}}%}

%%%%%%%%%%%%%%%%%%%%%%




\latexProblemContent{
\begin{problem}
Determine if the limit approaches a finite number, infinity, or does not exist. (If the limit does not exist, write DNE)
\input{./Limits/2311-Compute-Limit-0009.HELP.tex}
\[\lim_{x\to{-1}}{{\left(x + 5\right)} {\left(x - 1\right)} {\left(x - 5\right)} \sin\left(5 \, \pi x\right)}=\answer{0}\]

\end{problem}}%}

%%%%%%%%%%%%%%%%%%%%%%




\latexProblemContent{
\begin{problem}
Determine if the limit approaches a finite number, infinity, or does not exist. (If the limit does not exist, write DNE)
\input{./Limits/2311-Compute-Limit-0009.HELP.tex}
\[\lim_{x\to{-1}}{-2 \, {\left(x + 3\right)}^{2} {\left(x + 2\right)} \sin\left(-\pi x\right)}=\answer{0}\]

\end{problem}}%}

%%%%%%%%%%%%%%%%%%%%%%




\latexProblemContent{
\begin{problem}
Determine if the limit approaches a finite number, infinity, or does not exist. (If the limit does not exist, write DNE)
\input{./Limits/2311-Compute-Limit-0009.HELP.tex}
\[\lim_{x\to{2}}{4 \, {\left(x^{3} - 5 \, x^{2} + 2 \, x + 8\right)} \sin\left(\frac{1}{3} \, \pi x\right)}=\answer{0}\]

\end{problem}}%}

%%%%%%%%%%%%%%%%%%%%%%




\latexProblemContent{
\begin{problem}
Determine if the limit approaches a finite number, infinity, or does not exist. (If the limit does not exist, write DNE)
\input{./Limits/2311-Compute-Limit-0009.HELP.tex}
\[\lim_{x\to{1}}{-2 \, {\left(x + 2\right)} \tan\left(-4 \, \pi x\right)}=\answer{0}\]

\end{problem}}%}

%%%%%%%%%%%%%%%%%%%%%%




\latexProblemContent{
\begin{problem}
Determine if the limit approaches a finite number, infinity, or does not exist. (If the limit does not exist, write DNE)
\input{./Limits/2311-Compute-Limit-0009.HELP.tex}
\[\lim_{x\to{-2}}{-4 \, {\left(x + 4\right)} {\left(x - 5\right)} \cos\left(\frac{5}{3} \, \pi x\right)}=\answer{-28}\]

\end{problem}}%}

%%%%%%%%%%%%%%%%%%%%%%




\latexProblemContent{
\begin{problem}
Determine if the limit approaches a finite number, infinity, or does not exist. (If the limit does not exist, write DNE)
\input{./Limits/2311-Compute-Limit-0009.HELP.tex}
\[\lim_{x\to{-3}}{-5 \, {\left(x^{3} + 4 \, x^{2} - 11 \, x - 30\right)} \tan\left(-\frac{1}{3} \, \pi x\right)}=\answer{0}\]

\end{problem}}%}

%%%%%%%%%%%%%%%%%%%%%%




\latexProblemContent{
\begin{problem}
Determine if the limit approaches a finite number, infinity, or does not exist. (If the limit does not exist, write DNE)
\input{./Limits/2311-Compute-Limit-0009.HELP.tex}
\[\lim_{x\to{-2}}{3 \, {\left(x - 3\right)} \sin\left(\frac{1}{3} \, \pi x\right)}=\answer{\frac{5}{2} \, \left(3 \, \sqrt{3}\right)}\]

\end{problem}}%}

%%%%%%%%%%%%%%%%%%%%%%




\latexProblemContent{
\begin{problem}
Determine if the limit approaches a finite number, infinity, or does not exist. (If the limit does not exist, write DNE)
\input{./Limits/2311-Compute-Limit-0009.HELP.tex}
\[\lim_{x\to{1}}{3 \, {\left(x^{3} - 5 \, x^{2} - 2 \, x + 24\right)} \cos\left(\frac{4}{3} \, \pi x\right)}=\answer{-27}\]

\end{problem}}%}

%%%%%%%%%%%%%%%%%%%%%%




\latexProblemContent{
\begin{problem}
Determine if the limit approaches a finite number, infinity, or does not exist. (If the limit does not exist, write DNE)
\input{./Limits/2311-Compute-Limit-0009.HELP.tex}
\[\lim_{x\to{1}}{2 \, {\left(x - 2\right)} {\left(x - 3\right)} \sin\left(\frac{1}{2} \, \pi x\right)}=\answer{4}\]

\end{problem}}%}

%%%%%%%%%%%%%%%%%%%%%%




\latexProblemContent{
\begin{problem}
Determine if the limit approaches a finite number, infinity, or does not exist. (If the limit does not exist, write DNE)
\input{./Limits/2311-Compute-Limit-0009.HELP.tex}
\[\lim_{x\to{-2}}{-3 \, {\left(x + 3\right)} \tan\left(\frac{2}{3} \, \pi x\right)}=\answer{3^{\frac{3}{2}}}\]

\end{problem}}%}

%%%%%%%%%%%%%%%%%%%%%%




\latexProblemContent{
\begin{problem}
Determine if the limit approaches a finite number, infinity, or does not exist. (If the limit does not exist, write DNE)
\input{./Limits/2311-Compute-Limit-0009.HELP.tex}
\[\lim_{x\to{-4}}{-2 \, {\left(x^{2} - 3 \, x - 10\right)} \sin\left(\frac{5}{2} \, \pi x\right)}=\answer{0}\]

\end{problem}}%}

%%%%%%%%%%%%%%%%%%%%%%




\latexProblemContent{
\begin{problem}
Determine if the limit approaches a finite number, infinity, or does not exist. (If the limit does not exist, write DNE)
\input{./Limits/2311-Compute-Limit-0009.HELP.tex}
\[\lim_{x\to{-5}}{-{\left(x + 1\right)} {\left(x - 1\right)} {\left(x - 4\right)} \sin\left(2 \, \pi x\right)}=\answer{0}\]

\end{problem}}%}

%%%%%%%%%%%%%%%%%%%%%%




\latexProblemContent{
\begin{problem}
Determine if the limit approaches a finite number, infinity, or does not exist. (If the limit does not exist, write DNE)
\input{./Limits/2311-Compute-Limit-0009.HELP.tex}
\[\lim_{x\to{1}}{-5 \, {\left(x^{3} + 6 \, x^{2} - x - 30\right)} \sin\left(-\pi x\right)}=\answer{0}\]

\end{problem}}%}

%%%%%%%%%%%%%%%%%%%%%%




\latexProblemContent{
\begin{problem}
Determine if the limit approaches a finite number, infinity, or does not exist. (If the limit does not exist, write DNE)
\input{./Limits/2311-Compute-Limit-0009.HELP.tex}
\[\lim_{x\to{-2}}{-4 \, {\left(x^{3} - 12 \, x + 16\right)} \cos\left(\frac{2}{3} \, \pi x\right)}=\answer{64}\]

\end{problem}}%}

%%%%%%%%%%%%%%%%%%%%%%




\latexProblemContent{
\begin{problem}
Determine if the limit approaches a finite number, infinity, or does not exist. (If the limit does not exist, write DNE)
\input{./Limits/2311-Compute-Limit-0009.HELP.tex}
\[\lim_{x\to{-5}}{{\left(x + 5\right)} {\left(x - 1\right)} {\left(x - 3\right)} \sin\left(3 \, \pi x\right)}=\answer{0}\]

\end{problem}}%}

%%%%%%%%%%%%%%%%%%%%%%




\latexProblemContent{
\begin{problem}
Determine if the limit approaches a finite number, infinity, or does not exist. (If the limit does not exist, write DNE)
\input{./Limits/2311-Compute-Limit-0009.HELP.tex}
\[\lim_{x\to{-3}}{3 \, {\left(x^{3} - 7 \, x - 6\right)} \tan\left(-\frac{1}{3} \, \pi x\right)}=\answer{0}\]

\end{problem}}%}

%%%%%%%%%%%%%%%%%%%%%%




\latexProblemContent{
\begin{problem}
Determine if the limit approaches a finite number, infinity, or does not exist. (If the limit does not exist, write DNE)
\input{./Limits/2311-Compute-Limit-0009.HELP.tex}
\[\lim_{x\to{5}}{4 \, {\left(x - 4\right)} \tan\left(\frac{1}{2} \, \pi x\right)}=\answer{\infty}\]

\end{problem}}%}

%%%%%%%%%%%%%%%%%%%%%%




\latexProblemContent{
\begin{problem}
Determine if the limit approaches a finite number, infinity, or does not exist. (If the limit does not exist, write DNE)
\input{./Limits/2311-Compute-Limit-0009.HELP.tex}
\[\lim_{x\to{3}}{-5 \, {\left(x + 5\right)} {\left(x + 3\right)} {\left(x + 2\right)} \sin\left(-\frac{1}{3} \, \pi x\right)}=\answer{0}\]

\end{problem}}%}

%%%%%%%%%%%%%%%%%%%%%%




\latexProblemContent{
\begin{problem}
Determine if the limit approaches a finite number, infinity, or does not exist. (If the limit does not exist, write DNE)
\input{./Limits/2311-Compute-Limit-0009.HELP.tex}
\[\lim_{x\to{-2}}{4 \, {\left(x + 4\right)} {\left(x - 2\right)} {\left(x - 4\right)} \tan\left(2 \, \pi x\right)}=\answer{0}\]

\end{problem}}%}

%%%%%%%%%%%%%%%%%%%%%%




\latexProblemContent{
\begin{problem}
Determine if the limit approaches a finite number, infinity, or does not exist. (If the limit does not exist, write DNE)
\input{./Limits/2311-Compute-Limit-0009.HELP.tex}
\[\lim_{x\to{3}}{{\left(x - 1\right)} \sin\left(-\frac{1}{2} \, \pi x\right)}=\answer{2}\]

\end{problem}}%}

%%%%%%%%%%%%%%%%%%%%%%




\latexProblemContent{
\begin{problem}
Determine if the limit approaches a finite number, infinity, or does not exist. (If the limit does not exist, write DNE)
\input{./Limits/2311-Compute-Limit-0009.HELP.tex}
\[\lim_{x\to{1}}{-4 \, {\left(x^{3} + 8 \, x^{2} + 19 \, x + 12\right)} \tan\left(-\frac{1}{3} \, \pi x\right)}=\answer{160 \, \sqrt{3}}\]

\end{problem}}%}

%%%%%%%%%%%%%%%%%%%%%%




\latexProblemContent{
\begin{problem}
Determine if the limit approaches a finite number, infinity, or does not exist. (If the limit does not exist, write DNE)
\input{./Limits/2311-Compute-Limit-0009.HELP.tex}
\[\lim_{x\to{4}}{5 \, {\left(x + 1\right)}^{2} {\left(x - 5\right)} \sin\left(-\pi x\right)}=\answer{0}\]

\end{problem}}%}

%%%%%%%%%%%%%%%%%%%%%%




\latexProblemContent{
\begin{problem}
Determine if the limit approaches a finite number, infinity, or does not exist. (If the limit does not exist, write DNE)
\input{./Limits/2311-Compute-Limit-0009.HELP.tex}
\[\lim_{x\to{2}}{3 \, {\left(x - 2\right)} {\left(x - 3\right)} \cos\left(\pi x\right)}=\answer{0}\]

\end{problem}}%}

%%%%%%%%%%%%%%%%%%%%%%




\latexProblemContent{
\begin{problem}
Determine if the limit approaches a finite number, infinity, or does not exist. (If the limit does not exist, write DNE)
\input{./Limits/2311-Compute-Limit-0009.HELP.tex}
\[\lim_{x\to{5}}{-2 \, {\left(x^{3} + 6 \, x^{2} + 11 \, x + 6\right)} \tan\left(-\frac{1}{3} \, \pi x\right)}=\answer{-224 \, \left(3 \, \sqrt{3}\right)}\]

\end{problem}}%}

%%%%%%%%%%%%%%%%%%%%%%




\latexProblemContent{
\begin{problem}
Determine if the limit approaches a finite number, infinity, or does not exist. (If the limit does not exist, write DNE)
\input{./Limits/2311-Compute-Limit-0009.HELP.tex}
\[\lim_{x\to{-2}}{3 \, {\left(x^{3} - 5 \, x^{2} - 2 \, x + 24\right)} \cos\left(-\frac{2}{3} \, \pi x\right)}=\answer{0}\]

\end{problem}}%}

%%%%%%%%%%%%%%%%%%%%%%




\latexProblemContent{
\begin{problem}
Determine if the limit approaches a finite number, infinity, or does not exist. (If the limit does not exist, write DNE)
\input{./Limits/2311-Compute-Limit-0009.HELP.tex}
\[\lim_{x\to{-3}}{3 \, {\left(x^{2} - 4 \, x + 3\right)} \cos\left(\frac{1}{6} \, \pi x\right)}=\answer{0}\]

\end{problem}}%}

%%%%%%%%%%%%%%%%%%%%%%




\latexProblemContent{
\begin{problem}
Determine if the limit approaches a finite number, infinity, or does not exist. (If the limit does not exist, write DNE)
\input{./Limits/2311-Compute-Limit-0009.HELP.tex}
\[\lim_{x\to{2}}{4 \, {\left(x - 4\right)} \tan\left(\frac{1}{2} \, \pi x\right)}=\answer{0}\]

\end{problem}}%}

%%%%%%%%%%%%%%%%%%%%%%




\latexProblemContent{
\begin{problem}
Determine if the limit approaches a finite number, infinity, or does not exist. (If the limit does not exist, write DNE)
\input{./Limits/2311-Compute-Limit-0009.HELP.tex}
\[\lim_{x\to{-1}}{{\left(x - 1\right)} \cos\left(\frac{5}{6} \, \pi x\right)}=\answer{\sqrt{3}}\]

\end{problem}}%}

%%%%%%%%%%%%%%%%%%%%%%




\latexProblemContent{
\begin{problem}
Determine if the limit approaches a finite number, infinity, or does not exist. (If the limit does not exist, write DNE)
\input{./Limits/2311-Compute-Limit-0009.HELP.tex}
\[\lim_{x\to{-3}}{-2 \, {\left(x^{3} + 8 \, x^{2} + 21 \, x + 18\right)} \tan\left(-\frac{3}{2} \, \pi x\right)}=\answer{0}\]

\end{problem}}%}

%%%%%%%%%%%%%%%%%%%%%%




\latexProblemContent{
\begin{problem}
Determine if the limit approaches a finite number, infinity, or does not exist. (If the limit does not exist, write DNE)
\input{./Limits/2311-Compute-Limit-0009.HELP.tex}
\[\lim_{x\to{-1}}{-5 \, {\left(x + 5\right)} \cos\left(-\pi x\right)}=\answer{20}\]

\end{problem}}%}

%%%%%%%%%%%%%%%%%%%%%%




\latexProblemContent{
\begin{problem}
Determine if the limit approaches a finite number, infinity, or does not exist. (If the limit does not exist, write DNE)
\input{./Limits/2311-Compute-Limit-0009.HELP.tex}
\[\lim_{x\to{4}}{3 \, {\left(x - 2\right)} {\left(x - 3\right)} \cos\left(\pi x\right)}=\answer{6}\]

\end{problem}}%}

%%%%%%%%%%%%%%%%%%%%%%




\latexProblemContent{
\begin{problem}
Determine if the limit approaches a finite number, infinity, or does not exist. (If the limit does not exist, write DNE)
\input{./Limits/2311-Compute-Limit-0009.HELP.tex}
\[\lim_{x\to{-4}}{3 \, {\left(x - 3\right)} \sin\left(-\frac{1}{2} \, \pi x\right)}=\answer{0}\]

\end{problem}}%}

%%%%%%%%%%%%%%%%%%%%%%




\latexProblemContent{
\begin{problem}
Determine if the limit approaches a finite number, infinity, or does not exist. (If the limit does not exist, write DNE)
\input{./Limits/2311-Compute-Limit-0009.HELP.tex}
\[\lim_{x\to{-5}}{2 \, {\left(x + 3\right)} {\left(x - 2\right)} \sin\left(-3 \, \pi x\right)}=\answer{0}\]

\end{problem}}%}

%%%%%%%%%%%%%%%%%%%%%%




\latexProblemContent{
\begin{problem}
Determine if the limit approaches a finite number, infinity, or does not exist. (If the limit does not exist, write DNE)
\input{./Limits/2311-Compute-Limit-0009.HELP.tex}
\[\lim_{x\to{-2}}{-{\left(x + 2\right)} {\left(x + 1\right)} {\left(x - 2\right)} \tan\left(\frac{2}{3} \, \pi x\right)}=\answer{0}\]

\end{problem}}%}

%%%%%%%%%%%%%%%%%%%%%%




\latexProblemContent{
\begin{problem}
Determine if the limit approaches a finite number, infinity, or does not exist. (If the limit does not exist, write DNE)
\input{./Limits/2311-Compute-Limit-0009.HELP.tex}
\[\lim_{x\to{2}}{5 \, {\left(x + 3\right)} {\left(x - 1\right)} {\left(x - 5\right)} \tan\left(\frac{1}{2} \, \pi x\right)}=\answer{0}\]

\end{problem}}%}

%%%%%%%%%%%%%%%%%%%%%%




\latexProblemContent{
\begin{problem}
Determine if the limit approaches a finite number, infinity, or does not exist. (If the limit does not exist, write DNE)
\input{./Limits/2311-Compute-Limit-0009.HELP.tex}
\[\lim_{x\to{-1}}{{\left(x^{3} - 8 \, x^{2} + 19 \, x - 12\right)} \cos\left(\pi x\right)}=\answer{40}\]

\end{problem}}%}

%%%%%%%%%%%%%%%%%%%%%%




\latexProblemContent{
\begin{problem}
Determine if the limit approaches a finite number, infinity, or does not exist. (If the limit does not exist, write DNE)
\input{./Limits/2311-Compute-Limit-0009.HELP.tex}
\[\lim_{x\to{-2}}{{\left(x + 2\right)} {\left(x - 1\right)} \cos\left(-2 \, \pi x\right)}=\answer{0}\]

\end{problem}}%}

%%%%%%%%%%%%%%%%%%%%%%




\latexProblemContent{
\begin{problem}
Determine if the limit approaches a finite number, infinity, or does not exist. (If the limit does not exist, write DNE)
\input{./Limits/2311-Compute-Limit-0009.HELP.tex}
\[\lim_{x\to{-2}}{5 \, {\left(x - 4\right)} {\left(x - 5\right)} \sin\left(\frac{4}{3} \, \pi x\right)}=\answer{-35 \, \left(3 \, \sqrt{3}\right)}\]

\end{problem}}%}

%%%%%%%%%%%%%%%%%%%%%%




\latexProblemContent{
\begin{problem}
Determine if the limit approaches a finite number, infinity, or does not exist. (If the limit does not exist, write DNE)
\input{./Limits/2311-Compute-Limit-0009.HELP.tex}
\[\lim_{x\to{-2}}{2 \, {\left(x^{3} + 3 \, x^{2} - 4 \, x - 12\right)} \cos\left(-\frac{1}{2} \, \pi x\right)}=\answer{0}\]

\end{problem}}%}

%%%%%%%%%%%%%%%%%%%%%%




\latexProblemContent{
\begin{problem}
Determine if the limit approaches a finite number, infinity, or does not exist. (If the limit does not exist, write DNE)
\input{./Limits/2311-Compute-Limit-0009.HELP.tex}
\[\lim_{x\to{-3}}{-{\left(x^{2} - 4 \, x - 5\right)} \sin\left(\frac{5}{6} \, \pi x\right)}=\answer{16}\]

\end{problem}}%}

%%%%%%%%%%%%%%%%%%%%%%




\latexProblemContent{
\begin{problem}
Determine if the limit approaches a finite number, infinity, or does not exist. (If the limit does not exist, write DNE)
\input{./Limits/2311-Compute-Limit-0009.HELP.tex}
\[\lim_{x\to{3}}{2 \, {\left(x + 2\right)} {\left(x - 2\right)}^{2} \sin\left(-\pi x\right)}=\answer{0}\]

\end{problem}}%}

%%%%%%%%%%%%%%%%%%%%%%




\latexProblemContent{
\begin{problem}
Determine if the limit approaches a finite number, infinity, or does not exist. (If the limit does not exist, write DNE)
\input{./Limits/2311-Compute-Limit-0009.HELP.tex}
\[\lim_{x\to{-1}}{2 \, {\left(x - 2\right)} \cos\left(-\frac{1}{2} \, \pi x\right)}=\answer{0}\]

\end{problem}}%}

%%%%%%%%%%%%%%%%%%%%%%




\latexProblemContent{
\begin{problem}
Determine if the limit approaches a finite number, infinity, or does not exist. (If the limit does not exist, write DNE)
\input{./Limits/2311-Compute-Limit-0009.HELP.tex}
\[\lim_{x\to{-1}}{4 \, {\left(x - 4\right)} \cos\left(\frac{2}{3} \, \pi x\right)}=\answer{10}\]

\end{problem}}%}

%%%%%%%%%%%%%%%%%%%%%%




\latexProblemContent{
\begin{problem}
Determine if the limit approaches a finite number, infinity, or does not exist. (If the limit does not exist, write DNE)
\input{./Limits/2311-Compute-Limit-0009.HELP.tex}
\[\lim_{x\to{-1}}{-3 \, {\left(x^{3} + 7 \, x^{2} + 7 \, x - 15\right)} \tan\left(\frac{1}{2} \, \pi x\right)}=\answer{\infty}\]

\end{problem}}%}

%%%%%%%%%%%%%%%%%%%%%%




\latexProblemContent{
\begin{problem}
Determine if the limit approaches a finite number, infinity, or does not exist. (If the limit does not exist, write DNE)
\input{./Limits/2311-Compute-Limit-0009.HELP.tex}
\[\lim_{x\to{-2}}{-4 \, {\left(x + 4\right)} \tan\left(-\pi x\right)}=\answer{0}\]

\end{problem}}%}

%%%%%%%%%%%%%%%%%%%%%%




\latexProblemContent{
\begin{problem}
Determine if the limit approaches a finite number, infinity, or does not exist. (If the limit does not exist, write DNE)
\input{./Limits/2311-Compute-Limit-0009.HELP.tex}
\[\lim_{x\to{-5}}{-3 \, {\left(x + 3\right)} \cos\left(\frac{4}{3} \, \pi x\right)}=\answer{-3}\]

\end{problem}}%}

%%%%%%%%%%%%%%%%%%%%%%




\latexProblemContent{
\begin{problem}
Determine if the limit approaches a finite number, infinity, or does not exist. (If the limit does not exist, write DNE)
\input{./Limits/2311-Compute-Limit-0009.HELP.tex}
\[\lim_{x\to{3}}{-2 \, {\left(x^{3} + x^{2} - 8 \, x - 12\right)} \tan\left(\pi x\right)}=\answer{0}\]

\end{problem}}%}

%%%%%%%%%%%%%%%%%%%%%%




\latexProblemContent{
\begin{problem}
Determine if the limit approaches a finite number, infinity, or does not exist. (If the limit does not exist, write DNE)
\input{./Limits/2311-Compute-Limit-0009.HELP.tex}
\[\lim_{x\to{3}}{4 \, {\left(x + 3\right)} {\left(x - 3\right)} {\left(x - 4\right)} \cos\left(\frac{1}{2} \, \pi x\right)}=\answer{0}\]

\end{problem}}%}

%%%%%%%%%%%%%%%%%%%%%%




\latexProblemContent{
\begin{problem}
Determine if the limit approaches a finite number, infinity, or does not exist. (If the limit does not exist, write DNE)
\input{./Limits/2311-Compute-Limit-0009.HELP.tex}
\[\lim_{x\to{1}}{4 \, {\left(x + 2\right)} {\left(x - 2\right)} {\left(x - 4\right)} \sin\left(-\frac{2}{3} \, \pi x\right)}=\answer{-2 \, \left(9 \, \sqrt{3}\right)}\]

\end{problem}}%}

%%%%%%%%%%%%%%%%%%%%%%




\latexProblemContent{
\begin{problem}
Determine if the limit approaches a finite number, infinity, or does not exist. (If the limit does not exist, write DNE)
\input{./Limits/2311-Compute-Limit-0009.HELP.tex}
\[\lim_{x\to{4}}{-3 \, {\left(x + 3\right)} {\left(x - 1\right)} {\left(x - 5\right)} \sin\left(\frac{1}{3} \, \pi x\right)}=\answer{-\frac{7}{2} \, \left(9 \, \sqrt{3}\right)}\]

\end{problem}}%}

%%%%%%%%%%%%%%%%%%%%%%




\latexProblemContent{
\begin{problem}
Determine if the limit approaches a finite number, infinity, or does not exist. (If the limit does not exist, write DNE)
\input{./Limits/2311-Compute-Limit-0009.HELP.tex}
\[\lim_{x\to{-2}}{4 \, {\left(x^{3} - 2 \, x^{2} - 23 \, x + 60\right)} \tan\left(-5 \, \pi x\right)}=\answer{0}\]

\end{problem}}%}

%%%%%%%%%%%%%%%%%%%%%%




\latexProblemContent{
\begin{problem}
Determine if the limit approaches a finite number, infinity, or does not exist. (If the limit does not exist, write DNE)
\input{./Limits/2311-Compute-Limit-0009.HELP.tex}
\[\lim_{x\to{5}}{2 \, {\left(x + 2\right)} {\left(x + 1\right)} {\left(x - 2\right)} \sin\left(-2 \, \pi x\right)}=\answer{0}\]

\end{problem}}%}

%%%%%%%%%%%%%%%%%%%%%%




\latexProblemContent{
\begin{problem}
Determine if the limit approaches a finite number, infinity, or does not exist. (If the limit does not exist, write DNE)
\input{./Limits/2311-Compute-Limit-0009.HELP.tex}
\[\lim_{x\to{-4}}{{\left(x - 1\right)} \cos\left(\frac{4}{3} \, \pi x\right)}=\answer{\frac{5}{2}}\]

\end{problem}}%}

%%%%%%%%%%%%%%%%%%%%%%




\latexProblemContent{
\begin{problem}
Determine if the limit approaches a finite number, infinity, or does not exist. (If the limit does not exist, write DNE)
\input{./Limits/2311-Compute-Limit-0009.HELP.tex}
\[\lim_{x\to{-2}}{-5 \, {\left(x + 5\right)} {\left(x + 3\right)} {\left(x + 1\right)} \cos\left(-3 \, \pi x\right)}=\answer{15}\]

\end{problem}}%}

%%%%%%%%%%%%%%%%%%%%%%




\latexProblemContent{
\begin{problem}
Determine if the limit approaches a finite number, infinity, or does not exist. (If the limit does not exist, write DNE)
\input{./Limits/2311-Compute-Limit-0009.HELP.tex}
\[\lim_{x\to{-3}}{-2 \, {\left(x + 2\right)} {\left(x - 5\right)} \cos\left(\frac{5}{2} \, \pi x\right)}=\answer{0}\]

\end{problem}}%}

%%%%%%%%%%%%%%%%%%%%%%




\latexProblemContent{
\begin{problem}
Determine if the limit approaches a finite number, infinity, or does not exist. (If the limit does not exist, write DNE)
\input{./Limits/2311-Compute-Limit-0009.HELP.tex}
\[\lim_{x\to{5}}{4 \, {\left(x + 4\right)} {\left(x - 3\right)} {\left(x - 4\right)} \cos\left(3 \, \pi x\right)}=\answer{-72}\]

\end{problem}}%}

%%%%%%%%%%%%%%%%%%%%%%




\latexProblemContent{
\begin{problem}
Determine if the limit approaches a finite number, infinity, or does not exist. (If the limit does not exist, write DNE)
\input{./Limits/2311-Compute-Limit-0009.HELP.tex}
\[\lim_{x\to{-4}}{4 \, {\left(x^{3} - 2 \, x^{2} - 7 \, x - 4\right)} \cos\left(-\frac{1}{2} \, \pi x\right)}=\answer{-288}\]

\end{problem}}%}

%%%%%%%%%%%%%%%%%%%%%%




\latexProblemContent{
\begin{problem}
Determine if the limit approaches a finite number, infinity, or does not exist. (If the limit does not exist, write DNE)
\input{./Limits/2311-Compute-Limit-0009.HELP.tex}
\[\lim_{x\to{-5}}{-{\left(x^{3} - 7 \, x - 6\right)} \cos\left(\frac{3}{2} \, \pi x\right)}=\answer{0}\]

\end{problem}}%}

%%%%%%%%%%%%%%%%%%%%%%




\latexProblemContent{
\begin{problem}
Determine if the limit approaches a finite number, infinity, or does not exist. (If the limit does not exist, write DNE)
\input{./Limits/2311-Compute-Limit-0009.HELP.tex}
\[\lim_{x\to{2}}{{\left(x + 3\right)} {\left(x - 1\right)}^{2} \cos\left(-\frac{1}{2} \, \pi x\right)}=\answer{-5}\]

\end{problem}}%}

%%%%%%%%%%%%%%%%%%%%%%




\latexProblemContent{
\begin{problem}
Determine if the limit approaches a finite number, infinity, or does not exist. (If the limit does not exist, write DNE)
\input{./Limits/2311-Compute-Limit-0009.HELP.tex}
\[\lim_{x\to{4}}{-4 \, {\left(x + 4\right)} \sin\left(-\pi x\right)}=\answer{0}\]

\end{problem}}%}

%%%%%%%%%%%%%%%%%%%%%%




\latexProblemContent{
\begin{problem}
Determine if the limit approaches a finite number, infinity, or does not exist. (If the limit does not exist, write DNE)
\input{./Limits/2311-Compute-Limit-0009.HELP.tex}
\[\lim_{x\to{-2}}{-5 \, {\left(x^{3} + 7 \, x^{2} + 11 \, x + 5\right)} \cos\left(-\frac{1}{6} \, \pi x\right)}=\answer{-\frac{15}{2}}\]

\end{problem}}%}

%%%%%%%%%%%%%%%%%%%%%%




\latexProblemContent{
\begin{problem}
Determine if the limit approaches a finite number, infinity, or does not exist. (If the limit does not exist, write DNE)
\input{./Limits/2311-Compute-Limit-0009.HELP.tex}
\[\lim_{x\to{1}}{-{\left(x + 1\right)} \cos\left(2 \, \pi x\right)}=\answer{-2}\]

\end{problem}}%}

%%%%%%%%%%%%%%%%%%%%%%




\latexProblemContent{
\begin{problem}
Determine if the limit approaches a finite number, infinity, or does not exist. (If the limit does not exist, write DNE)
\input{./Limits/2311-Compute-Limit-0009.HELP.tex}
\[\lim_{x\to{-2}}{-{\left(x + 2\right)} {\left(x + 1\right)} \cos\left(-2 \, \pi x\right)}=\answer{0}\]

\end{problem}}%}

%%%%%%%%%%%%%%%%%%%%%%




\latexProblemContent{
\begin{problem}
Determine if the limit approaches a finite number, infinity, or does not exist. (If the limit does not exist, write DNE)
\input{./Limits/2311-Compute-Limit-0009.HELP.tex}
\[\lim_{x\to{-4}}{4 \, {\left(x - 2\right)} {\left(x - 4\right)} \cos\left(\pi x\right)}=\answer{192}\]

\end{problem}}%}

%%%%%%%%%%%%%%%%%%%%%%




\latexProblemContent{
\begin{problem}
Determine if the limit approaches a finite number, infinity, or does not exist. (If the limit does not exist, write DNE)
\input{./Limits/2311-Compute-Limit-0009.HELP.tex}
\[\lim_{x\to{-5}}{5 \, {\left(x - 5\right)} \tan\left(-\frac{1}{3} \, \pi x\right)}=\answer{50 \, \sqrt{3}}\]

\end{problem}}%}

%%%%%%%%%%%%%%%%%%%%%%




\latexProblemContent{
\begin{problem}
Determine if the limit approaches a finite number, infinity, or does not exist. (If the limit does not exist, write DNE)
\input{./Limits/2311-Compute-Limit-0009.HELP.tex}
\[\lim_{x\to{-5}}{-4 \, {\left(x + 4\right)} {\left(x + 2\right)} {\left(x - 5\right)} \tan\left(-2 \, \pi x\right)}=\answer{0}\]

\end{problem}}%}

%%%%%%%%%%%%%%%%%%%%%%




\latexProblemContent{
\begin{problem}
Determine if the limit approaches a finite number, infinity, or does not exist. (If the limit does not exist, write DNE)
\input{./Limits/2311-Compute-Limit-0009.HELP.tex}
\[\lim_{x\to{-5}}{3 \, {\left(x^{2} - 2 \, x - 3\right)} \sin\left(-\frac{1}{6} \, \pi x\right)}=\answer{48}\]

\end{problem}}%}

%%%%%%%%%%%%%%%%%%%%%%




\latexProblemContent{
\begin{problem}
Determine if the limit approaches a finite number, infinity, or does not exist. (If the limit does not exist, write DNE)
\input{./Limits/2311-Compute-Limit-0009.HELP.tex}
\[\lim_{x\to{-4}}{-2 \, {\left(x + 3\right)} {\left(x + 2\right)} \tan\left(-\frac{1}{2} \, \pi x\right)}=\answer{0}\]

\end{problem}}%}

%%%%%%%%%%%%%%%%%%%%%%




\latexProblemContent{
\begin{problem}
Determine if the limit approaches a finite number, infinity, or does not exist. (If the limit does not exist, write DNE)
\input{./Limits/2311-Compute-Limit-0009.HELP.tex}
\[\lim_{x\to{-3}}{4 \, {\left(x^{3} - 2 \, x^{2} - 11 \, x + 12\right)} \sin\left(\frac{1}{3} \, \pi x\right)}=\answer{0}\]

\end{problem}}%}

%%%%%%%%%%%%%%%%%%%%%%




\latexProblemContent{
\begin{problem}
Determine if the limit approaches a finite number, infinity, or does not exist. (If the limit does not exist, write DNE)
\input{./Limits/2311-Compute-Limit-0009.HELP.tex}
\[\lim_{x\to{2}}{-5 \, {\left(x + 5\right)} {\left(x + 3\right)} {\left(x - 1\right)} \tan\left(\pi x\right)}=\answer{0}\]

\end{problem}}%}

%%%%%%%%%%%%%%%%%%%%%%




\latexProblemContent{
\begin{problem}
Determine if the limit approaches a finite number, infinity, or does not exist. (If the limit does not exist, write DNE)
\input{./Limits/2311-Compute-Limit-0009.HELP.tex}
\[\lim_{x\to{2}}{-5 \, {\left(x^{2} - 25\right)} \sin\left(\frac{5}{2} \, \pi x\right)}=\answer{0}\]

\end{problem}}%}

%%%%%%%%%%%%%%%%%%%%%%




\latexProblemContent{
\begin{problem}
Determine if the limit approaches a finite number, infinity, or does not exist. (If the limit does not exist, write DNE)
\input{./Limits/2311-Compute-Limit-0009.HELP.tex}
\[\lim_{x\to{-2}}{2 \, {\left(x^{2} - 3 \, x + 2\right)} \cos\left(\frac{1}{2} \, \pi x\right)}=\answer{-24}\]

\end{problem}}%}

%%%%%%%%%%%%%%%%%%%%%%




\latexProblemContent{
\begin{problem}
Determine if the limit approaches a finite number, infinity, or does not exist. (If the limit does not exist, write DNE)
\input{./Limits/2311-Compute-Limit-0009.HELP.tex}
\[\lim_{x\to{4}}{2 \, {\left(x^{2} - 3 \, x + 2\right)} \tan\left(\frac{1}{2} \, \pi x\right)}=\answer{0}\]

\end{problem}}%}

%%%%%%%%%%%%%%%%%%%%%%




\latexProblemContent{
\begin{problem}
Determine if the limit approaches a finite number, infinity, or does not exist. (If the limit does not exist, write DNE)
\input{./Limits/2311-Compute-Limit-0009.HELP.tex}
\[\lim_{x\to{-4}}{-{\left(x^{2} + 3 \, x + 2\right)} \sin\left(-\frac{1}{3} \, \pi x\right)}=\answer{3^{\frac{3}{2}}}\]

\end{problem}}%}

%%%%%%%%%%%%%%%%%%%%%%




\latexProblemContent{
\begin{problem}
Determine if the limit approaches a finite number, infinity, or does not exist. (If the limit does not exist, write DNE)
\input{./Limits/2311-Compute-Limit-0009.HELP.tex}
\[\lim_{x\to{-1}}{4 \, {\left(x^{3} - 5 \, x^{2} + 2 \, x + 8\right)} \sin\left(\frac{1}{3} \, \pi x\right)}=\answer{0}\]

\end{problem}}%}

%%%%%%%%%%%%%%%%%%%%%%




\latexProblemContent{
\begin{problem}
Determine if the limit approaches a finite number, infinity, or does not exist. (If the limit does not exist, write DNE)
\input{./Limits/2311-Compute-Limit-0009.HELP.tex}
\[\lim_{x\to{-4}}{3 \, {\left(x^{3} - 7 \, x^{2} + 15 \, x - 9\right)} \tan\left(\frac{1}{3} \, \pi x\right)}=\answer{245 \, \left(3 \, \sqrt{3}\right)}\]

\end{problem}}%}

%%%%%%%%%%%%%%%%%%%%%%




\latexProblemContent{
\begin{problem}
Determine if the limit approaches a finite number, infinity, or does not exist. (If the limit does not exist, write DNE)
\input{./Limits/2311-Compute-Limit-0009.HELP.tex}
\[\lim_{x\to{3}}{2 \, {\left(x + 5\right)} {\left(x + 1\right)} {\left(x - 2\right)} \sin\left(-\frac{5}{6} \, \pi x\right)}=\answer{-64}\]

\end{problem}}%}

%%%%%%%%%%%%%%%%%%%%%%




\latexProblemContent{
\begin{problem}
Determine if the limit approaches a finite number, infinity, or does not exist. (If the limit does not exist, write DNE)
\input{./Limits/2311-Compute-Limit-0009.HELP.tex}
\[\lim_{x\to{-1}}{5 \, {\left(x^{2} - 4 \, x - 5\right)} \cos\left(-\pi x\right)}=\answer{0}\]

\end{problem}}%}

%%%%%%%%%%%%%%%%%%%%%%




\latexProblemContent{
\begin{problem}
Determine if the limit approaches a finite number, infinity, or does not exist. (If the limit does not exist, write DNE)
\input{./Limits/2311-Compute-Limit-0009.HELP.tex}
\[\lim_{x\to{4}}{-3 \, {\left(x + 4\right)} {\left(x + 3\right)} \sin\left(-4 \, \pi x\right)}=\answer{0}\]

\end{problem}}%}

%%%%%%%%%%%%%%%%%%%%%%




\latexProblemContent{
\begin{problem}
Determine if the limit approaches a finite number, infinity, or does not exist. (If the limit does not exist, write DNE)
\input{./Limits/2311-Compute-Limit-0009.HELP.tex}
\[\lim_{x\to{2}}{-3 \, {\left(x + 3\right)} \tan\left(\frac{1}{2} \, \pi x\right)}=\answer{0}\]

\end{problem}}%}

%%%%%%%%%%%%%%%%%%%%%%




\latexProblemContent{
\begin{problem}
Determine if the limit approaches a finite number, infinity, or does not exist. (If the limit does not exist, write DNE)
\input{./Limits/2311-Compute-Limit-0009.HELP.tex}
\[\lim_{x\to{-1}}{-4 \, {\left(x^{2} + 9 \, x + 20\right)} \sin\left(-\frac{5}{3} \, \pi x\right)}=\answer{8 \, \left(3 \, \sqrt{3}\right)}\]

\end{problem}}%}

%%%%%%%%%%%%%%%%%%%%%%




\latexProblemContent{
\begin{problem}
Determine if the limit approaches a finite number, infinity, or does not exist. (If the limit does not exist, write DNE)
\input{./Limits/2311-Compute-Limit-0009.HELP.tex}
\[\lim_{x\to{1}}{-2 \, {\left(x^{2} - 4\right)} \sin\left(2 \, \pi x\right)}=\answer{0}\]

\end{problem}}%}

%%%%%%%%%%%%%%%%%%%%%%




\latexProblemContent{
\begin{problem}
Determine if the limit approaches a finite number, infinity, or does not exist. (If the limit does not exist, write DNE)
\input{./Limits/2311-Compute-Limit-0009.HELP.tex}
\[\lim_{x\to{2}}{4 \, {\left(x + 4\right)} {\left(x - 4\right)} \sin\left(-2 \, \pi x\right)}=\answer{0}\]

\end{problem}}%}

%%%%%%%%%%%%%%%%%%%%%%




\latexProblemContent{
\begin{problem}
Determine if the limit approaches a finite number, infinity, or does not exist. (If the limit does not exist, write DNE)
\input{./Limits/2311-Compute-Limit-0009.HELP.tex}
\[\lim_{x\to{-5}}{{\left(x^{2} - 1\right)} \sin\left(-\frac{1}{3} \, \pi x\right)}=\answer{-4 \, \left(3 \, \sqrt{3}\right)}\]

\end{problem}}%}

%%%%%%%%%%%%%%%%%%%%%%




\latexProblemContent{
\begin{problem}
Determine if the limit approaches a finite number, infinity, or does not exist. (If the limit does not exist, write DNE)
\input{./Limits/2311-Compute-Limit-0009.HELP.tex}
\[\lim_{x\to{5}}{3 \, {\left(x^{3} + 7 \, x^{2} - 5 \, x - 75\right)} \sin\left(-\frac{5}{2} \, \pi x\right)}=\answer{-600}\]

\end{problem}}%}

%%%%%%%%%%%%%%%%%%%%%%




\latexProblemContent{
\begin{problem}
Determine if the limit approaches a finite number, infinity, or does not exist. (If the limit does not exist, write DNE)
\input{./Limits/2311-Compute-Limit-0009.HELP.tex}
\[\lim_{x\to{4}}{2 \, {\left(x - 2\right)} \sin\left(\frac{1}{3} \, \pi x\right)}=\answer{-2 \, \sqrt{3}}\]

\end{problem}}%}

%%%%%%%%%%%%%%%%%%%%%%




\latexProblemContent{
\begin{problem}
Determine if the limit approaches a finite number, infinity, or does not exist. (If the limit does not exist, write DNE)
\input{./Limits/2311-Compute-Limit-0009.HELP.tex}
\[\lim_{x\to{5}}{2 \, {\left(x + 5\right)} {\left(x + 4\right)} {\left(x - 2\right)} \tan\left(-\frac{5}{2} \, \pi x\right)}=\answer{\infty}\]

\end{problem}}%}

%%%%%%%%%%%%%%%%%%%%%%




\latexProblemContent{
\begin{problem}
Determine if the limit approaches a finite number, infinity, or does not exist. (If the limit does not exist, write DNE)
\input{./Limits/2311-Compute-Limit-0009.HELP.tex}
\[\lim_{x\to{-4}}{{\left(x^{3} + x^{2} - 17 \, x + 15\right)} \sin\left(-\frac{5}{6} \, \pi x\right)}=\answer{-\frac{35}{2} \, \sqrt{3}}\]

\end{problem}}%}

%%%%%%%%%%%%%%%%%%%%%%




\latexProblemContent{
\begin{problem}
Determine if the limit approaches a finite number, infinity, or does not exist. (If the limit does not exist, write DNE)
\input{./Limits/2311-Compute-Limit-0009.HELP.tex}
\[\lim_{x\to{-4}}{5 \, {\left(x^{2} - 8 \, x + 15\right)} \sin\left(\pi x\right)}=\answer{0}\]

\end{problem}}%}

%%%%%%%%%%%%%%%%%%%%%%




\latexProblemContent{
\begin{problem}
Determine if the limit approaches a finite number, infinity, or does not exist. (If the limit does not exist, write DNE)
\input{./Limits/2311-Compute-Limit-0009.HELP.tex}
\[\lim_{x\to{1}}{-2 \, {\left(x + 2\right)} {\left(x - 5\right)} \cos\left(\frac{5}{3} \, \pi x\right)}=\answer{12}\]

\end{problem}}%}

%%%%%%%%%%%%%%%%%%%%%%




\latexProblemContent{
\begin{problem}
Determine if the limit approaches a finite number, infinity, or does not exist. (If the limit does not exist, write DNE)
\input{./Limits/2311-Compute-Limit-0009.HELP.tex}
\[\lim_{x\to{3}}{4 \, {\left(x - 2\right)} {\left(x - 4\right)} \tan\left(\frac{2}{3} \, \pi x\right)}=\answer{0}\]

\end{problem}}%}

%%%%%%%%%%%%%%%%%%%%%%




\latexProblemContent{
\begin{problem}
Determine if the limit approaches a finite number, infinity, or does not exist. (If the limit does not exist, write DNE)
\input{./Limits/2311-Compute-Limit-0009.HELP.tex}
\[\lim_{x\to{-2}}{-3 \, {\left(x + 5\right)} {\left(x + 3\right)} {\left(x - 3\right)} \sin\left(-\frac{5}{6} \, \pi x\right)}=\answer{-\frac{5}{2} \, \left(9 \, \sqrt{3}\right)}\]

\end{problem}}%}

%%%%%%%%%%%%%%%%%%%%%%




\latexProblemContent{
\begin{problem}
Determine if the limit approaches a finite number, infinity, or does not exist. (If the limit does not exist, write DNE)
\input{./Limits/2311-Compute-Limit-0009.HELP.tex}
\[\lim_{x\to{3}}{4 \, {\left(x + 5\right)} {\left(x - 4\right)} \cos\left(-5 \, \pi x\right)}=\answer{32}\]

\end{problem}}%}

%%%%%%%%%%%%%%%%%%%%%%




\latexProblemContent{
\begin{problem}
Determine if the limit approaches a finite number, infinity, or does not exist. (If the limit does not exist, write DNE)
\input{./Limits/2311-Compute-Limit-0009.HELP.tex}
\[\lim_{x\to{-3}}{-{\left(x + 3\right)} {\left(x + 1\right)} \cos\left(-\pi x\right)}=\answer{0}\]

\end{problem}}%}

%%%%%%%%%%%%%%%%%%%%%%




\latexProblemContent{
\begin{problem}
Determine if the limit approaches a finite number, infinity, or does not exist. (If the limit does not exist, write DNE)
\input{./Limits/2311-Compute-Limit-0009.HELP.tex}
\[\lim_{x\to{5}}{-5 \, {\left(x^{2} + 7 \, x + 10\right)} \cos\left(-\pi x\right)}=\answer{350}\]

\end{problem}}%}

%%%%%%%%%%%%%%%%%%%%%%




\latexProblemContent{
\begin{problem}
Determine if the limit approaches a finite number, infinity, or does not exist. (If the limit does not exist, write DNE)
\input{./Limits/2311-Compute-Limit-0009.HELP.tex}
\[\lim_{x\to{1}}{-2 \, {\left(x + 3\right)} {\left(x + 2\right)} \sin\left(-\frac{3}{2} \, \pi x\right)}=\answer{-24}\]

\end{problem}}%}

%%%%%%%%%%%%%%%%%%%%%%




\latexProblemContent{
\begin{problem}
Determine if the limit approaches a finite number, infinity, or does not exist. (If the limit does not exist, write DNE)
\input{./Limits/2311-Compute-Limit-0009.HELP.tex}
\[\lim_{x\to{-2}}{-{\left(x^{2} + 5 \, x + 4\right)} \cos\left(-\frac{4}{3} \, \pi x\right)}=\answer{-1}\]

\end{problem}}%}

%%%%%%%%%%%%%%%%%%%%%%




\latexProblemContent{
\begin{problem}
Determine if the limit approaches a finite number, infinity, or does not exist. (If the limit does not exist, write DNE)
\input{./Limits/2311-Compute-Limit-0009.HELP.tex}
\[\lim_{x\to{4}}{-5 \, {\left(x + 5\right)} {\left(x + 2\right)} \sin\left(-2 \, \pi x\right)}=\answer{0}\]

\end{problem}}%}

%%%%%%%%%%%%%%%%%%%%%%




\latexProblemContent{
\begin{problem}
Determine if the limit approaches a finite number, infinity, or does not exist. (If the limit does not exist, write DNE)
\input{./Limits/2311-Compute-Limit-0009.HELP.tex}
\[\lim_{x\to{-2}}{-4 \, {\left(x + 4\right)} \cos\left(-\frac{1}{6} \, \pi x\right)}=\answer{-4}\]

\end{problem}}%}

%%%%%%%%%%%%%%%%%%%%%%




\latexProblemContent{
\begin{problem}
Determine if the limit approaches a finite number, infinity, or does not exist. (If the limit does not exist, write DNE)
\input{./Limits/2311-Compute-Limit-0009.HELP.tex}
\[\lim_{x\to{4}}{-3 \, {\left(x^{2} + 8 \, x + 15\right)} \sin\left(-\frac{5}{2} \, \pi x\right)}=\answer{0}\]

\end{problem}}%}

%%%%%%%%%%%%%%%%%%%%%%




\latexProblemContent{
\begin{problem}
Determine if the limit approaches a finite number, infinity, or does not exist. (If the limit does not exist, write DNE)
\input{./Limits/2311-Compute-Limit-0009.HELP.tex}
\[\lim_{x\to{4}}{5 \, {\left(x^{2} - 25\right)} \cos\left(-5 \, \pi x\right)}=\answer{-45}\]

\end{problem}}%}

%%%%%%%%%%%%%%%%%%%%%%




\latexProblemContent{
\begin{problem}
Determine if the limit approaches a finite number, infinity, or does not exist. (If the limit does not exist, write DNE)
\input{./Limits/2311-Compute-Limit-0009.HELP.tex}
\[\lim_{x\to{-5}}{-2 \, {\left(x + 5\right)} {\left(x + 2\right)} \tan\left(-\frac{5}{6} \, \pi x\right)}=\answer{0}\]

\end{problem}}%}

%%%%%%%%%%%%%%%%%%%%%%




\latexProblemContent{
\begin{problem}
Determine if the limit approaches a finite number, infinity, or does not exist. (If the limit does not exist, write DNE)
\input{./Limits/2311-Compute-Limit-0009.HELP.tex}
\[\lim_{x\to{2}}{{\left(x^{2} + 2 \, x - 3\right)} \tan\left(-\frac{1}{2} \, \pi x\right)}=\answer{0}\]

\end{problem}}%}

%%%%%%%%%%%%%%%%%%%%%%




\latexProblemContent{
\begin{problem}
Determine if the limit approaches a finite number, infinity, or does not exist. (If the limit does not exist, write DNE)
\input{./Limits/2311-Compute-Limit-0009.HELP.tex}
\[\lim_{x\to{-1}}{{\left(x - 1\right)} \sin\left(-\pi x\right)}=\answer{0}\]

\end{problem}}%}

%%%%%%%%%%%%%%%%%%%%%%




\latexProblemContent{
\begin{problem}
Determine if the limit approaches a finite number, infinity, or does not exist. (If the limit does not exist, write DNE)
\input{./Limits/2311-Compute-Limit-0009.HELP.tex}
\[\lim_{x\to{-2}}{4 \, {\left(x^{2} - 6 \, x + 8\right)} \cos\left(\frac{1}{3} \, \pi x\right)}=\answer{-48}\]

\end{problem}}%}

%%%%%%%%%%%%%%%%%%%%%%




\latexProblemContent{
\begin{problem}
Determine if the limit approaches a finite number, infinity, or does not exist. (If the limit does not exist, write DNE)
\input{./Limits/2311-Compute-Limit-0009.HELP.tex}
\[\lim_{x\to{-1}}{-5 \, {\left(x + 5\right)} \cos\left(-2 \, \pi x\right)}=\answer{-20}\]

\end{problem}}%}

%%%%%%%%%%%%%%%%%%%%%%




\latexProblemContent{
\begin{problem}
Determine if the limit approaches a finite number, infinity, or does not exist. (If the limit does not exist, write DNE)
\input{./Limits/2311-Compute-Limit-0009.HELP.tex}
\[\lim_{x\to{2}}{{\left(x + 5\right)} {\left(x - 1\right)} \tan\left(-\frac{5}{2} \, \pi x\right)}=\answer{0}\]

\end{problem}}%}

%%%%%%%%%%%%%%%%%%%%%%




\latexProblemContent{
\begin{problem}
Determine if the limit approaches a finite number, infinity, or does not exist. (If the limit does not exist, write DNE)
\input{./Limits/2311-Compute-Limit-0009.HELP.tex}
\[\lim_{x\to{-3}}{2 \, {\left(x^{3} - 3 \, x^{2} - 4 \, x + 12\right)} \tan\left(3 \, \pi x\right)}=\answer{0}\]

\end{problem}}%}

%%%%%%%%%%%%%%%%%%%%%%




\latexProblemContent{
\begin{problem}
Determine if the limit approaches a finite number, infinity, or does not exist. (If the limit does not exist, write DNE)
\input{./Limits/2311-Compute-Limit-0009.HELP.tex}
\[\lim_{x\to{-5}}{5 \, {\left(x - 5\right)} \tan\left(-\frac{1}{2} \, \pi x\right)}=\answer{\infty}\]

\end{problem}}%}

%%%%%%%%%%%%%%%%%%%%%%




\latexProblemContent{
\begin{problem}
Determine if the limit approaches a finite number, infinity, or does not exist. (If the limit does not exist, write DNE)
\input{./Limits/2311-Compute-Limit-0009.HELP.tex}
\[\lim_{x\to{-4}}{-2 \, {\left(x + 4\right)} {\left(x + 2\right)} \sin\left(-\frac{4}{3} \, \pi x\right)}=\answer{0}\]

\end{problem}}%}

%%%%%%%%%%%%%%%%%%%%%%




\latexProblemContent{
\begin{problem}
Determine if the limit approaches a finite number, infinity, or does not exist. (If the limit does not exist, write DNE)
\input{./Limits/2311-Compute-Limit-0009.HELP.tex}
\[\lim_{x\to{-4}}{3 \, {\left(x^{3} - 6 \, x^{2} + 11 \, x - 6\right)} \cos\left(\pi x\right)}=\answer{-630}\]

\end{problem}}%}

%%%%%%%%%%%%%%%%%%%%%%




\latexProblemContent{
\begin{problem}
Determine if the limit approaches a finite number, infinity, or does not exist. (If the limit does not exist, write DNE)
\input{./Limits/2311-Compute-Limit-0009.HELP.tex}
\[\lim_{x\to{-4}}{{\left(x + 3\right)} {\left(x + 1\right)} {\left(x - 1\right)} \cos\left(-\frac{3}{2} \, \pi x\right)}=\answer{-15}\]

\end{problem}}%}

%%%%%%%%%%%%%%%%%%%%%%




\latexProblemContent{
\begin{problem}
Determine if the limit approaches a finite number, infinity, or does not exist. (If the limit does not exist, write DNE)
\input{./Limits/2311-Compute-Limit-0009.HELP.tex}
\[\lim_{x\to{4}}{2 \, {\left(x + 4\right)} {\left(x - 2\right)} {\left(x - 3\right)} \cos\left(3 \, \pi x\right)}=\answer{32}\]

\end{problem}}%}

%%%%%%%%%%%%%%%%%%%%%%




\latexProblemContent{
\begin{problem}
Determine if the limit approaches a finite number, infinity, or does not exist. (If the limit does not exist, write DNE)
\input{./Limits/2311-Compute-Limit-0009.HELP.tex}
\[\lim_{x\to{-5}}{-2 \, {\left(x + 2\right)} \cos\left(4 \, \pi x\right)}=\answer{6}\]

\end{problem}}%}

%%%%%%%%%%%%%%%%%%%%%%




\latexProblemContent{
\begin{problem}
Determine if the limit approaches a finite number, infinity, or does not exist. (If the limit does not exist, write DNE)
\input{./Limits/2311-Compute-Limit-0009.HELP.tex}
\[\lim_{x\to{3}}{5 \, {\left(x^{3} - 10 \, x^{2} + 31 \, x - 30\right)} \tan\left(\pi x\right)}=\answer{0}\]

\end{problem}}%}

%%%%%%%%%%%%%%%%%%%%%%




\latexProblemContent{
\begin{problem}
Determine if the limit approaches a finite number, infinity, or does not exist. (If the limit does not exist, write DNE)
\input{./Limits/2311-Compute-Limit-0009.HELP.tex}
\[\lim_{x\to{-1}}{-4 \, {\left(x + 4\right)} {\left(x - 1\right)} \sin\left(\pi x\right)}=\answer{0}\]

\end{problem}}%}

%%%%%%%%%%%%%%%%%%%%%%




\latexProblemContent{
\begin{problem}
Determine if the limit approaches a finite number, infinity, or does not exist. (If the limit does not exist, write DNE)
\input{./Limits/2311-Compute-Limit-0009.HELP.tex}
\[\lim_{x\to{-2}}{{\left(x^{2} - 6 \, x + 5\right)} \cos\left(\frac{5}{2} \, \pi x\right)}=\answer{-21}\]

\end{problem}}%}

%%%%%%%%%%%%%%%%%%%%%%




\latexProblemContent{
\begin{problem}
Determine if the limit approaches a finite number, infinity, or does not exist. (If the limit does not exist, write DNE)
\input{./Limits/2311-Compute-Limit-0009.HELP.tex}
\[\lim_{x\to{-3}}{5 \, {\left(x - 1\right)} {\left(x - 2\right)} {\left(x - 5\right)} \tan\left(\frac{1}{6} \, \pi x\right)}=\answer{\infty}\]

\end{problem}}%}

%%%%%%%%%%%%%%%%%%%%%%




\latexProblemContent{
\begin{problem}
Determine if the limit approaches a finite number, infinity, or does not exist. (If the limit does not exist, write DNE)
\input{./Limits/2311-Compute-Limit-0009.HELP.tex}
\[\lim_{x\to{3}}{5 \, {\left(x - 5\right)} \cos\left(4 \, \pi x\right)}=\answer{-10}\]

\end{problem}}%}

%%%%%%%%%%%%%%%%%%%%%%




\latexProblemContent{
\begin{problem}
Determine if the limit approaches a finite number, infinity, or does not exist. (If the limit does not exist, write DNE)
\input{./Limits/2311-Compute-Limit-0009.HELP.tex}
\[\lim_{x\to{3}}{{\left(x^{2} + 4 \, x - 5\right)} \sin\left(-\frac{5}{3} \, \pi x\right)}=\answer{0}\]

\end{problem}}%}

%%%%%%%%%%%%%%%%%%%%%%




\latexProblemContent{
\begin{problem}
Determine if the limit approaches a finite number, infinity, or does not exist. (If the limit does not exist, write DNE)
\input{./Limits/2311-Compute-Limit-0009.HELP.tex}
\[\lim_{x\to{-5}}{4 \, {\left(x - 4\right)} \cos\left(\frac{5}{3} \, \pi x\right)}=\answer{-18}\]

\end{problem}}%}

%%%%%%%%%%%%%%%%%%%%%%




\latexProblemContent{
\begin{problem}
Determine if the limit approaches a finite number, infinity, or does not exist. (If the limit does not exist, write DNE)
\input{./Limits/2311-Compute-Limit-0009.HELP.tex}
\[\lim_{x\to{4}}{-3 \, {\left(x + 3\right)} {\left(x - 5\right)} \tan\left(\frac{5}{2} \, \pi x\right)}=\answer{0}\]

\end{problem}}%}

%%%%%%%%%%%%%%%%%%%%%%




\latexProblemContent{
\begin{problem}
Determine if the limit approaches a finite number, infinity, or does not exist. (If the limit does not exist, write DNE)
\input{./Limits/2311-Compute-Limit-0009.HELP.tex}
\[\lim_{x\to{5}}{-2 \, {\left(x^{2} - 2 \, x - 8\right)} \sin\left(2 \, \pi x\right)}=\answer{0}\]

\end{problem}}%}

%%%%%%%%%%%%%%%%%%%%%%




\latexProblemContent{
\begin{problem}
Determine if the limit approaches a finite number, infinity, or does not exist. (If the limit does not exist, write DNE)
\input{./Limits/2311-Compute-Limit-0009.HELP.tex}
\[\lim_{x\to{-5}}{-{\left(x + 1\right)} {\left(x - 4\right)} \sin\left(\frac{2}{3} \, \pi x\right)}=\answer{-2 \, \left(9 \, \sqrt{3}\right)}\]

\end{problem}}%}

%%%%%%%%%%%%%%%%%%%%%%




\latexProblemContent{
\begin{problem}
Determine if the limit approaches a finite number, infinity, or does not exist. (If the limit does not exist, write DNE)
\input{./Limits/2311-Compute-Limit-0009.HELP.tex}
\[\lim_{x\to{-2}}{{\left(x^{3} - 6 \, x^{2} + 9 \, x - 4\right)} \sin\left(\frac{2}{3} \, \pi x\right)}=\answer{-\left(27 \, \sqrt{3}\right)}\]

\end{problem}}%}

%%%%%%%%%%%%%%%%%%%%%%




\latexProblemContent{
\begin{problem}
Determine if the limit approaches a finite number, infinity, or does not exist. (If the limit does not exist, write DNE)
\input{./Limits/2311-Compute-Limit-0009.HELP.tex}
\[\lim_{x\to{-5}}{-4 \, {\left(x + 4\right)} {\left(x - 2\right)} \tan\left(2 \, \pi x\right)}=\answer{0}\]

\end{problem}}%}

%%%%%%%%%%%%%%%%%%%%%%




\latexProblemContent{
\begin{problem}
Determine if the limit approaches a finite number, infinity, or does not exist. (If the limit does not exist, write DNE)
\input{./Limits/2311-Compute-Limit-0009.HELP.tex}
\[\lim_{x\to{3}}{-2 \, {\left(x + 2\right)} \sin\left(\frac{1}{2} \, \pi x\right)}=\answer{10}\]

\end{problem}}%}

%%%%%%%%%%%%%%%%%%%%%%




\latexProblemContent{
\begin{problem}
Determine if the limit approaches a finite number, infinity, or does not exist. (If the limit does not exist, write DNE)
\input{./Limits/2311-Compute-Limit-0009.HELP.tex}
\[\lim_{x\to{-5}}{3 \, {\left(x^{3} - 4 \, x^{2} + x + 6\right)} \tan\left(-\frac{1}{2} \, \pi x\right)}=\answer{\infty}\]

\end{problem}}%}

%%%%%%%%%%%%%%%%%%%%%%




\latexProblemContent{
\begin{problem}
Determine if the limit approaches a finite number, infinity, or does not exist. (If the limit does not exist, write DNE)
\input{./Limits/2311-Compute-Limit-0009.HELP.tex}
\[\lim_{x\to{-2}}{3 \, {\left(x - 2\right)} {\left(x - 3\right)} {\left(x - 4\right)} \cos\left(\frac{2}{3} \, \pi x\right)}=\answer{180}\]

\end{problem}}%}

%%%%%%%%%%%%%%%%%%%%%%




\latexProblemContent{
\begin{problem}
Determine if the limit approaches a finite number, infinity, or does not exist. (If the limit does not exist, write DNE)
\input{./Limits/2311-Compute-Limit-0009.HELP.tex}
\[\lim_{x\to{-1}}{{\left(x^{3} - 8 \, x^{2} + 19 \, x - 12\right)} \tan\left(\pi x\right)}=\answer{0}\]

\end{problem}}%}

%%%%%%%%%%%%%%%%%%%%%%




\latexProblemContent{
\begin{problem}
Determine if the limit approaches a finite number, infinity, or does not exist. (If the limit does not exist, write DNE)
\input{./Limits/2311-Compute-Limit-0009.HELP.tex}
\[\lim_{x\to{2}}{{\left(x + 1\right)} {\left(x - 1\right)} \cos\left(-\frac{1}{2} \, \pi x\right)}=\answer{-3}\]

\end{problem}}%}

%%%%%%%%%%%%%%%%%%%%%%




\latexProblemContent{
\begin{problem}
Determine if the limit approaches a finite number, infinity, or does not exist. (If the limit does not exist, write DNE)
\input{./Limits/2311-Compute-Limit-0009.HELP.tex}
\[\lim_{x\to{-4}}{-2 \, {\left(x^{3} - x^{2} - 4 \, x + 4\right)} \cos\left(\frac{2}{3} \, \pi x\right)}=\answer{-60}\]

\end{problem}}%}

%%%%%%%%%%%%%%%%%%%%%%




\latexProblemContent{
\begin{problem}
Determine if the limit approaches a finite number, infinity, or does not exist. (If the limit does not exist, write DNE)
\input{./Limits/2311-Compute-Limit-0009.HELP.tex}
\[\lim_{x\to{-1}}{5 \, {\left(x + 2\right)} {\left(x - 5\right)} \cos\left(-\frac{2}{3} \, \pi x\right)}=\answer{15}\]

\end{problem}}%}

%%%%%%%%%%%%%%%%%%%%%%




\latexProblemContent{
\begin{problem}
Determine if the limit approaches a finite number, infinity, or does not exist. (If the limit does not exist, write DNE)
\input{./Limits/2311-Compute-Limit-0009.HELP.tex}
\[\lim_{x\to{2}}{-3 \, {\left(x^{3} + 3 \, x^{2} - 16 \, x - 48\right)} \sin\left(\frac{4}{3} \, \pi x\right)}=\answer{10 \, \left(9 \, \sqrt{3}\right)}\]

\end{problem}}%}

%%%%%%%%%%%%%%%%%%%%%%




\latexProblemContent{
\begin{problem}
Determine if the limit approaches a finite number, infinity, or does not exist. (If the limit does not exist, write DNE)
\input{./Limits/2311-Compute-Limit-0009.HELP.tex}
\[\lim_{x\to{-2}}{-4 \, {\left(x^{3} + 5 \, x^{2} + 2 \, x - 8\right)} \tan\left(\frac{1}{6} \, \pi x\right)}=\answer{0}\]

\end{problem}}%}

%%%%%%%%%%%%%%%%%%%%%%




\latexProblemContent{
\begin{problem}
Determine if the limit approaches a finite number, infinity, or does not exist. (If the limit does not exist, write DNE)
\input{./Limits/2311-Compute-Limit-0009.HELP.tex}
\[\lim_{x\to{-1}}{{\left(x + 4\right)} {\left(x - 1\right)} \cos\left(-2 \, \pi x\right)}=\answer{-6}\]

\end{problem}}%}

%%%%%%%%%%%%%%%%%%%%%%




\latexProblemContent{
\begin{problem}
Determine if the limit approaches a finite number, infinity, or does not exist. (If the limit does not exist, write DNE)
\input{./Limits/2311-Compute-Limit-0009.HELP.tex}
\[\lim_{x\to{-5}}{2 \, {\left(x^{3} - 3 \, x^{2} + 4\right)} \tan\left(-\frac{1}{3} \, \pi x\right)}=\answer{392 \, \sqrt{3}}\]

\end{problem}}%}

%%%%%%%%%%%%%%%%%%%%%%




\latexProblemContent{
\begin{problem}
Determine if the limit approaches a finite number, infinity, or does not exist. (If the limit does not exist, write DNE)
\input{./Limits/2311-Compute-Limit-0009.HELP.tex}
\[\lim_{x\to{3}}{5 \, {\left(x^{2} - 8 \, x + 15\right)} \tan\left(\pi x\right)}=\answer{0}\]

\end{problem}}%}

%%%%%%%%%%%%%%%%%%%%%%




%%%%%%%%%%%%%%%%%%%%%%




\latexProblemContent{
\begin{problem}
Determine if the limit approaches a finite number, infinity, or does not exist. (If the limit does not exist, write DNE)
\input{./Limits/2311-Compute-Limit-0009.HELP.tex}
\[\lim_{x\to{4}}{2 \, {\left(x + 2\right)} {\left(x - 2\right)} \sin\left(-\frac{2}{3} \, \pi x\right)}=\answer{-4 \, \left(3 \, \sqrt{3}\right)}\]

\end{problem}}%}

%%%%%%%%%%%%%%%%%%%%%%




\latexProblemContent{
\begin{problem}
Determine if the limit approaches a finite number, infinity, or does not exist. (If the limit does not exist, write DNE)
\input{./Limits/2311-Compute-Limit-0009.HELP.tex}
\[\lim_{x\to{-5}}{-2 \, {\left(x + 2\right)} \tan\left(-\frac{5}{3} \, \pi x\right)}=\answer{2 \, \left(3 \, \sqrt{3}\right)}\]

\end{problem}}%}

%%%%%%%%%%%%%%%%%%%%%%




\latexProblemContent{
\begin{problem}
Determine if the limit approaches a finite number, infinity, or does not exist. (If the limit does not exist, write DNE)
\input{./Limits/2311-Compute-Limit-0009.HELP.tex}
\[\lim_{x\to{-1}}{{\left(x - 1\right)} {\left(x - 2\right)} \sin\left(2 \, \pi x\right)}=\answer{0}\]

\end{problem}}%}

%%%%%%%%%%%%%%%%%%%%%%




\latexProblemContent{
\begin{problem}
Determine if the limit approaches a finite number, infinity, or does not exist. (If the limit does not exist, write DNE)
\input{./Limits/2311-Compute-Limit-0009.HELP.tex}
\[\lim_{x\to{-3}}{2 \, {\left(x^{2} + 2 \, x - 8\right)} \cos\left(-4 \, \pi x\right)}=\answer{-10}\]

\end{problem}}%}

%%%%%%%%%%%%%%%%%%%%%%




\latexProblemContent{
\begin{problem}
Determine if the limit approaches a finite number, infinity, or does not exist. (If the limit does not exist, write DNE)
\input{./Limits/2311-Compute-Limit-0009.HELP.tex}
\[\lim_{x\to{-4}}{3 \, {\left(x + 3\right)} {\left(x - 3\right)} \sin\left(-3 \, \pi x\right)}=\answer{0}\]

\end{problem}}%}

%%%%%%%%%%%%%%%%%%%%%%




\latexProblemContent{
\begin{problem}
Determine if the limit approaches a finite number, infinity, or does not exist. (If the limit does not exist, write DNE)
\input{./Limits/2311-Compute-Limit-0009.HELP.tex}
\[\lim_{x\to{3}}{-5 \, {\left(x + 5\right)} {\left(x + 4\right)} {\left(x + 2\right)} \tan\left(-4 \, \pi x\right)}=\answer{0}\]

\end{problem}}%}

%%%%%%%%%%%%%%%%%%%%%%




\latexProblemContent{
\begin{problem}
Determine if the limit approaches a finite number, infinity, or does not exist. (If the limit does not exist, write DNE)
\input{./Limits/2311-Compute-Limit-0009.HELP.tex}
\[\lim_{x\to{-2}}{-{\left(x + 1\right)} {\left(x - 3\right)} \cos\left(\frac{1}{2} \, \pi x\right)}=\answer{5}\]

\end{problem}}%}

%%%%%%%%%%%%%%%%%%%%%%




\latexProblemContent{
\begin{problem}
Determine if the limit approaches a finite number, infinity, or does not exist. (If the limit does not exist, write DNE)
\input{./Limits/2311-Compute-Limit-0009.HELP.tex}
\[\lim_{x\to{4}}{{\left(x - 1\right)} \sin\left(\frac{4}{3} \, \pi x\right)}=\answer{-\frac{1}{2} \, \left(3 \, \sqrt{3}\right)}\]

\end{problem}}%}

%%%%%%%%%%%%%%%%%%%%%%




\latexProblemContent{
\begin{problem}
Determine if the limit approaches a finite number, infinity, or does not exist. (If the limit does not exist, write DNE)
\input{./Limits/2311-Compute-Limit-0009.HELP.tex}
\[\lim_{x\to{2}}{5 \, {\left(x^{3} + x^{2} - 22 \, x - 40\right)} \tan\left(-\frac{4}{3} \, \pi x\right)}=\answer{-40 \, \left(9 \, \sqrt{3}\right)}\]

\end{problem}}%}

%%%%%%%%%%%%%%%%%%%%%%




\latexProblemContent{
\begin{problem}
Determine if the limit approaches a finite number, infinity, or does not exist. (If the limit does not exist, write DNE)
\input{./Limits/2311-Compute-Limit-0009.HELP.tex}
\[\lim_{x\to{-3}}{-2 \, {\left(x + 2\right)} \tan\left(\frac{1}{2} \, \pi x\right)}=\answer{\infty}\]

\end{problem}}%}

%%%%%%%%%%%%%%%%%%%%%%




\latexProblemContent{
\begin{problem}
Determine if the limit approaches a finite number, infinity, or does not exist. (If the limit does not exist, write DNE)
\input{./Limits/2311-Compute-Limit-0009.HELP.tex}
\[\lim_{x\to{4}}{{\left(x^{2} - 3 \, x + 2\right)} \cos\left(\pi x\right)}=\answer{6}\]

\end{problem}}%}

%%%%%%%%%%%%%%%%%%%%%%




\latexProblemContent{
\begin{problem}
Determine if the limit approaches a finite number, infinity, or does not exist. (If the limit does not exist, write DNE)
\input{./Limits/2311-Compute-Limit-0009.HELP.tex}
\[\lim_{x\to{2}}{-2 \, {\left(x^{2} - 2 \, x - 8\right)} \cos\left(4 \, \pi x\right)}=\answer{16}\]

\end{problem}}%}

%%%%%%%%%%%%%%%%%%%%%%




\latexProblemContent{
\begin{problem}
Determine if the limit approaches a finite number, infinity, or does not exist. (If the limit does not exist, write DNE)
\input{./Limits/2311-Compute-Limit-0009.HELP.tex}
\[\lim_{x\to{4}}{-3 \, {\left(x + 3\right)} \sin\left(\pi x\right)}=\answer{0}\]

\end{problem}}%}

%%%%%%%%%%%%%%%%%%%%%%




\latexProblemContent{
\begin{problem}
Determine if the limit approaches a finite number, infinity, or does not exist. (If the limit does not exist, write DNE)
\input{./Limits/2311-Compute-Limit-0009.HELP.tex}
\[\lim_{x\to{2}}{-5 \, {\left(x + 5\right)} {\left(x + 3\right)} {\left(x - 2\right)} \tan\left(\pi x\right)}=\answer{0}\]

\end{problem}}%}

%%%%%%%%%%%%%%%%%%%%%%




\latexProblemContent{
\begin{problem}
Determine if the limit approaches a finite number, infinity, or does not exist. (If the limit does not exist, write DNE)
\input{./Limits/2311-Compute-Limit-0009.HELP.tex}
\[\lim_{x\to{1}}{-3 \, {\left(x + 4\right)} {\left(x + 3\right)} {\left(x - 5\right)} \sin\left(-\frac{4}{3} \, \pi x\right)}=\answer{40 \, \left(3 \, \sqrt{3}\right)}\]

\end{problem}}%}

%%%%%%%%%%%%%%%%%%%%%%




\latexProblemContent{
\begin{problem}
Determine if the limit approaches a finite number, infinity, or does not exist. (If the limit does not exist, write DNE)
\input{./Limits/2311-Compute-Limit-0009.HELP.tex}
\[\lim_{x\to{-2}}{-{\left(x + 1\right)} {\left(x - 4\right)} \sin\left(\frac{2}{3} \, \pi x\right)}=\answer{-\left(3 \, \sqrt{3}\right)}\]

\end{problem}}%}

%%%%%%%%%%%%%%%%%%%%%%




\latexProblemContent{
\begin{problem}
Determine if the limit approaches a finite number, infinity, or does not exist. (If the limit does not exist, write DNE)
\input{./Limits/2311-Compute-Limit-0009.HELP.tex}
\[\lim_{x\to{-3}}{-2 \, {\left(x + 3\right)} {\left(x + 2\right)} \cos\left(-\frac{1}{2} \, \pi x\right)}=\answer{0}\]

\end{problem}}%}

%%%%%%%%%%%%%%%%%%%%%%




\latexProblemContent{
\begin{problem}
Determine if the limit approaches a finite number, infinity, or does not exist. (If the limit does not exist, write DNE)
\input{./Limits/2311-Compute-Limit-0009.HELP.tex}
\[\lim_{x\to{-4}}{-{\left(x + 1\right)} \tan\left(-\frac{5}{3} \, \pi x\right)}=\answer{-\left(3 \, \sqrt{3}\right)}\]

\end{problem}}%}

%%%%%%%%%%%%%%%%%%%%%%




\latexProblemContent{
\begin{problem}
Determine if the limit approaches a finite number, infinity, or does not exist. (If the limit does not exist, write DNE)
\input{./Limits/2311-Compute-Limit-0009.HELP.tex}
\[\lim_{x\to{-1}}{-5 \, {\left(x + 5\right)} \sin\left(\frac{5}{6} \, \pi x\right)}=\answer{10}\]

\end{problem}}%}

%%%%%%%%%%%%%%%%%%%%%%




\latexProblemContent{
\begin{problem}
Determine if the limit approaches a finite number, infinity, or does not exist. (If the limit does not exist, write DNE)
\input{./Limits/2311-Compute-Limit-0009.HELP.tex}
\[\lim_{x\to{2}}{3 \, {\left(x + 5\right)} {\left(x + 4\right)} {\left(x - 3\right)} \sin\left(-2 \, \pi x\right)}=\answer{0}\]

\end{problem}}%}

%%%%%%%%%%%%%%%%%%%%%%




\latexProblemContent{
\begin{problem}
Determine if the limit approaches a finite number, infinity, or does not exist. (If the limit does not exist, write DNE)
\input{./Limits/2311-Compute-Limit-0009.HELP.tex}
\[\lim_{x\to{3}}{-{\left(x^{3} + 9 \, x^{2} + 23 \, x + 15\right)} \tan\left(-\frac{5}{3} \, \pi x\right)}=\answer{0}\]

\end{problem}}%}

%%%%%%%%%%%%%%%%%%%%%%




\latexProblemContent{
\begin{problem}
Determine if the limit approaches a finite number, infinity, or does not exist. (If the limit does not exist, write DNE)
\input{./Limits/2311-Compute-Limit-0009.HELP.tex}
\[\lim_{x\to{-3}}{2 \, {\left(x + 4\right)} {\left(x + 3\right)} {\left(x - 2\right)} \tan\left(-\frac{3}{2} \, \pi x\right)}=\answer{-\frac{20}{3 \, \pi}}\]

\end{problem}}%}

%%%%%%%%%%%%%%%%%%%%%%




\latexProblemContent{
\begin{problem}
Determine if the limit approaches a finite number, infinity, or does not exist. (If the limit does not exist, write DNE)
\input{./Limits/2311-Compute-Limit-0009.HELP.tex}
\[\lim_{x\to{4}}{-{\left(x + 1\right)} \cos\left(-\frac{1}{3} \, \pi x\right)}=\answer{\frac{5}{2}}\]

\end{problem}}%}

%%%%%%%%%%%%%%%%%%%%%%




\latexProblemContent{
\begin{problem}
Determine if the limit approaches a finite number, infinity, or does not exist. (If the limit does not exist, write DNE)
\input{./Limits/2311-Compute-Limit-0009.HELP.tex}
\[\lim_{x\to{5}}{-3 \, {\left(x + 3\right)} {\left(x - 3\right)} {\left(x - 4\right)} \sin\left(\frac{4}{3} \, \pi x\right)}=\answer{-8 \, \left(3 \, \sqrt{3}\right)}\]

\end{problem}}%}

%%%%%%%%%%%%%%%%%%%%%%




\latexProblemContent{
\begin{problem}
Determine if the limit approaches a finite number, infinity, or does not exist. (If the limit does not exist, write DNE)
\input{./Limits/2311-Compute-Limit-0009.HELP.tex}
\[\lim_{x\to{1}}{2 \, {\left(x + 4\right)} {\left(x - 2\right)} \cos\left(-\frac{4}{3} \, \pi x\right)}=\answer{5}\]

\end{problem}}%}

%%%%%%%%%%%%%%%%%%%%%%




\latexProblemContent{
\begin{problem}
Determine if the limit approaches a finite number, infinity, or does not exist. (If the limit does not exist, write DNE)
\input{./Limits/2311-Compute-Limit-0009.HELP.tex}
\[\lim_{x\to{3}}{-4 \, {\left(x^{3} + 5 \, x^{2} - 2 \, x - 24\right)} \tan\left(-\pi x\right)}=\answer{0}\]

\end{problem}}%}

%%%%%%%%%%%%%%%%%%%%%%




\latexProblemContent{
\begin{problem}
Determine if the limit approaches a finite number, infinity, or does not exist. (If the limit does not exist, write DNE)
\input{./Limits/2311-Compute-Limit-0009.HELP.tex}
\[\lim_{x\to{1}}{-3 \, {\left(x + 3\right)} {\left(x - 1\right)} {\left(x - 2\right)} \cos\left(\pi x\right)}=\answer{0}\]

\end{problem}}%}

%%%%%%%%%%%%%%%%%%%%%%




\latexProblemContent{
\begin{problem}
Determine if the limit approaches a finite number, infinity, or does not exist. (If the limit does not exist, write DNE)
\input{./Limits/2311-Compute-Limit-0009.HELP.tex}
\[\lim_{x\to{4}}{-{\left(x^{2} - x - 2\right)} \cos\left(\frac{1}{3} \, \pi x\right)}=\answer{5}\]

\end{problem}}%}

%%%%%%%%%%%%%%%%%%%%%%




\latexProblemContent{
\begin{problem}
Determine if the limit approaches a finite number, infinity, or does not exist. (If the limit does not exist, write DNE)
\input{./Limits/2311-Compute-Limit-0009.HELP.tex}
\[\lim_{x\to{-4}}{-{\left(x^{2} - 2 \, x - 3\right)} \cos\left(\frac{3}{2} \, \pi x\right)}=\answer{-21}\]

\end{problem}}%}

%%%%%%%%%%%%%%%%%%%%%%




\latexProblemContent{
\begin{problem}
Determine if the limit approaches a finite number, infinity, or does not exist. (If the limit does not exist, write DNE)
\input{./Limits/2311-Compute-Limit-0009.HELP.tex}
\[\lim_{x\to{4}}{-3 \, {\left(x + 3\right)}^{2} {\left(x - 4\right)} \tan\left(4 \, \pi x\right)}=\answer{0}\]

\end{problem}}%}

%%%%%%%%%%%%%%%%%%%%%%




\latexProblemContent{
\begin{problem}
Determine if the limit approaches a finite number, infinity, or does not exist. (If the limit does not exist, write DNE)
\input{./Limits/2311-Compute-Limit-0009.HELP.tex}
\[\lim_{x\to{2}}{3 \, {\left(x + 1\right)} {\left(x - 1\right)} {\left(x - 3\right)} \sin\left(\frac{1}{3} \, \pi x\right)}=\answer{-\frac{1}{2} \, \left(9 \, \sqrt{3}\right)}\]

\end{problem}}%}

%%%%%%%%%%%%%%%%%%%%%%




\latexProblemContent{
\begin{problem}
Determine if the limit approaches a finite number, infinity, or does not exist. (If the limit does not exist, write DNE)
\input{./Limits/2311-Compute-Limit-0009.HELP.tex}
\[\lim_{x\to{-3}}{4 \, {\left(x - 4\right)} \sin\left(-\frac{5}{3} \, \pi x\right)}=\answer{0}\]

\end{problem}}%}

%%%%%%%%%%%%%%%%%%%%%%




\latexProblemContent{
\begin{problem}
Determine if the limit approaches a finite number, infinity, or does not exist. (If the limit does not exist, write DNE)
\input{./Limits/2311-Compute-Limit-0009.HELP.tex}
\[\lim_{x\to{-3}}{-2 \, {\left(x + 5\right)} {\left(x + 2\right)} {\left(x - 3\right)} \cos\left(\frac{3}{2} \, \pi x\right)}=\answer{0}\]

\end{problem}}%}

%%%%%%%%%%%%%%%%%%%%%%




\latexProblemContent{
\begin{problem}
Determine if the limit approaches a finite number, infinity, or does not exist. (If the limit does not exist, write DNE)
\input{./Limits/2311-Compute-Limit-0009.HELP.tex}
\[\lim_{x\to{5}}{-2 \, {\left(x^{2} + 6 \, x + 8\right)} \cos\left(-\frac{4}{3} \, \pi x\right)}=\answer{63}\]

\end{problem}}%}

%%%%%%%%%%%%%%%%%%%%%%




\latexProblemContent{
\begin{problem}
Determine if the limit approaches a finite number, infinity, or does not exist. (If the limit does not exist, write DNE)
\input{./Limits/2311-Compute-Limit-0009.HELP.tex}
\[\lim_{x\to{-3}}{-5 \, {\left(x + 5\right)} {\left(x + 2\right)} {\left(x - 5\right)} \sin\left(-\pi x\right)}=\answer{0}\]

\end{problem}}%}

%%%%%%%%%%%%%%%%%%%%%%




\latexProblemContent{
\begin{problem}
Determine if the limit approaches a finite number, infinity, or does not exist. (If the limit does not exist, write DNE)
\input{./Limits/2311-Compute-Limit-0009.HELP.tex}
\[\lim_{x\to{-4}}{-3 \, {\left(x + 3\right)}^{2} {\left(x - 3\right)} \cos\left(\frac{3}{2} \, \pi x\right)}=\answer{21}\]

\end{problem}}%}

%%%%%%%%%%%%%%%%%%%%%%




\latexProblemContent{
\begin{problem}
Determine if the limit approaches a finite number, infinity, or does not exist. (If the limit does not exist, write DNE)
\input{./Limits/2311-Compute-Limit-0009.HELP.tex}
\[\lim_{x\to{-2}}{3 \, {\left(x^{3} + 4 \, x^{2} - 11 \, x - 30\right)} \tan\left(-\frac{1}{3} \, \pi x\right)}=\answer{0}\]

\end{problem}}%}

%%%%%%%%%%%%%%%%%%%%%%




\latexProblemContent{
\begin{problem}
Determine if the limit approaches a finite number, infinity, or does not exist. (If the limit does not exist, write DNE)
\input{./Limits/2311-Compute-Limit-0009.HELP.tex}
\[\lim_{x\to{1}}{-{\left(x + 2\right)} {\left(x + 1\right)} {\left(x - 1\right)} \tan\left(\frac{1}{3} \, \pi x\right)}=\answer{0}\]

\end{problem}}%}

%%%%%%%%%%%%%%%%%%%%%%




\latexProblemContent{
\begin{problem}
Determine if the limit approaches a finite number, infinity, or does not exist. (If the limit does not exist, write DNE)
\input{./Limits/2311-Compute-Limit-0009.HELP.tex}
\[\lim_{x\to{-2}}{-4 \, {\left(x + 4\right)} {\left(x + 3\right)} {\left(x + 2\right)} \sin\left(-\pi x\right)}=\answer{0}\]

\end{problem}}%}

%%%%%%%%%%%%%%%%%%%%%%




\latexProblemContent{
\begin{problem}
Determine if the limit approaches a finite number, infinity, or does not exist. (If the limit does not exist, write DNE)
\input{./Limits/2311-Compute-Limit-0009.HELP.tex}
\[\lim_{x\to{1}}{3 \, {\left(x + 4\right)} {\left(x - 3\right)} \sin\left(-\frac{2}{3} \, \pi x\right)}=\answer{5 \, \left(3 \, \sqrt{3}\right)}\]

\end{problem}}%}

%%%%%%%%%%%%%%%%%%%%%%




\latexProblemContent{
\begin{problem}
Determine if the limit approaches a finite number, infinity, or does not exist. (If the limit does not exist, write DNE)
\input{./Limits/2311-Compute-Limit-0009.HELP.tex}
\[\lim_{x\to{3}}{4 \, {\left(x - 4\right)} \tan\left(\frac{1}{3} \, \pi x\right)}=\answer{0}\]

\end{problem}}%}

%%%%%%%%%%%%%%%%%%%%%%




\latexProblemContent{
\begin{problem}
Determine if the limit approaches a finite number, infinity, or does not exist. (If the limit does not exist, write DNE)
\input{./Limits/2311-Compute-Limit-0009.HELP.tex}
\[\lim_{x\to{-4}}{-3 \, {\left(x + 5\right)}^{2} {\left(x + 3\right)} \tan\left(-\frac{5}{2} \, \pi x\right)}=\answer{0}\]

\end{problem}}%}

%%%%%%%%%%%%%%%%%%%%%%




\latexProblemContent{
\begin{problem}
Determine if the limit approaches a finite number, infinity, or does not exist. (If the limit does not exist, write DNE)
\input{./Limits/2311-Compute-Limit-0009.HELP.tex}
\[\lim_{x\to{5}}{2 \, {\left(x + 4\right)} {\left(x + 3\right)} {\left(x - 2\right)} \tan\left(-\frac{3}{2} \, \pi x\right)}=\answer{\infty}\]

\end{problem}}%}

%%%%%%%%%%%%%%%%%%%%%%




\latexProblemContent{
\begin{problem}
Determine if the limit approaches a finite number, infinity, or does not exist. (If the limit does not exist, write DNE)
\input{./Limits/2311-Compute-Limit-0009.HELP.tex}
\[\lim_{x\to{2}}{-2 \, {\left(x + 2\right)} \cos\left(-2 \, \pi x\right)}=\answer{-8}\]

\end{problem}}%}

%%%%%%%%%%%%%%%%%%%%%%




\latexProblemContent{
\begin{problem}
Determine if the limit approaches a finite number, infinity, or does not exist. (If the limit does not exist, write DNE)
\input{./Limits/2311-Compute-Limit-0009.HELP.tex}
\[\lim_{x\to{-2}}{-3 \, {\left(x^{3} + 4 \, x^{2} + x - 6\right)} \tan\left(\frac{1}{6} \, \pi x\right)}=\answer{0}\]

\end{problem}}%}

%%%%%%%%%%%%%%%%%%%%%%




\latexProblemContent{
\begin{problem}
Determine if the limit approaches a finite number, infinity, or does not exist. (If the limit does not exist, write DNE)
\input{./Limits/2311-Compute-Limit-0009.HELP.tex}
\[\lim_{x\to{-3}}{-2 \, {\left(x^{2} + 5 \, x + 6\right)} \cos\left(-3 \, \pi x\right)}=\answer{0}\]

\end{problem}}%}

%%%%%%%%%%%%%%%%%%%%%%




\latexProblemContent{
\begin{problem}
Determine if the limit approaches a finite number, infinity, or does not exist. (If the limit does not exist, write DNE)
\input{./Limits/2311-Compute-Limit-0009.HELP.tex}
\[\lim_{x\to{5}}{-4 \, {\left(x^{2} + 6 \, x + 8\right)} \sin\left(-\frac{2}{3} \, \pi x\right)}=\answer{-14 \, \left(9 \, \sqrt{3}\right)}\]

\end{problem}}%}

%%%%%%%%%%%%%%%%%%%%%%




\latexProblemContent{
\begin{problem}
Determine if the limit approaches a finite number, infinity, or does not exist. (If the limit does not exist, write DNE)
\input{./Limits/2311-Compute-Limit-0009.HELP.tex}
\[\lim_{x\to{2}}{4 \, {\left(x^{2} - 3 \, x - 4\right)} \tan\left(-\pi x\right)}=\answer{0}\]

\end{problem}}%}

%%%%%%%%%%%%%%%%%%%%%%




\latexProblemContent{
\begin{problem}
Determine if the limit approaches a finite number, infinity, or does not exist. (If the limit does not exist, write DNE)
\input{./Limits/2311-Compute-Limit-0009.HELP.tex}
\[\lim_{x\to{-1}}{-5 \, {\left(x + 5\right)} {\left(x + 2\right)} \sin\left(-2 \, \pi x\right)}=\answer{0}\]

\end{problem}}%}

%%%%%%%%%%%%%%%%%%%%%%




\latexProblemContent{
\begin{problem}
Determine if the limit approaches a finite number, infinity, or does not exist. (If the limit does not exist, write DNE)
\input{./Limits/2311-Compute-Limit-0009.HELP.tex}
\[\lim_{x\to{-4}}{-2 \, {\left(x + 2\right)}^{2} {\left(x - 3\right)} \tan\left(\frac{3}{2} \, \pi x\right)}=\answer{0}\]

\end{problem}}%}

%%%%%%%%%%%%%%%%%%%%%%




\latexProblemContent{
\begin{problem}
Determine if the limit approaches a finite number, infinity, or does not exist. (If the limit does not exist, write DNE)
\input{./Limits/2311-Compute-Limit-0009.HELP.tex}
\[\lim_{x\to{1}}{5 \, {\left(x - 5\right)} \sin\left(-\frac{3}{2} \, \pi x\right)}=\answer{-20}\]

\end{problem}}%}

%%%%%%%%%%%%%%%%%%%%%%




\latexProblemContent{
\begin{problem}
Determine if the limit approaches a finite number, infinity, or does not exist. (If the limit does not exist, write DNE)
\input{./Limits/2311-Compute-Limit-0009.HELP.tex}
\[\lim_{x\to{4}}{-3 \, {\left(x + 3\right)} {\left(x + 2\right)} \sin\left(-\pi x\right)}=\answer{0}\]

\end{problem}}%}

%%%%%%%%%%%%%%%%%%%%%%




\latexProblemContent{
\begin{problem}
Determine if the limit approaches a finite number, infinity, or does not exist. (If the limit does not exist, write DNE)
\input{./Limits/2311-Compute-Limit-0009.HELP.tex}
\[\lim_{x\to{3}}{-4 \, {\left(x + 4\right)} \tan\left(\frac{3}{2} \, \pi x\right)}=\answer{\infty}\]

\end{problem}}%}

%%%%%%%%%%%%%%%%%%%%%%




\latexProblemContent{
\begin{problem}
Determine if the limit approaches a finite number, infinity, or does not exist. (If the limit does not exist, write DNE)
\input{./Limits/2311-Compute-Limit-0009.HELP.tex}
\[\lim_{x\to{4}}{2 \, {\left(x^{3} + x^{2} - 4 \, x - 4\right)} \tan\left(-\frac{1}{3} \, \pi x\right)}=\answer{-40 \, \left(3 \, \sqrt{3}\right)}\]

\end{problem}}%}

%%%%%%%%%%%%%%%%%%%%%%




\latexProblemContent{
\begin{problem}
Determine if the limit approaches a finite number, infinity, or does not exist. (If the limit does not exist, write DNE)
\input{./Limits/2311-Compute-Limit-0009.HELP.tex}
\[\lim_{x\to{2}}{-5 \, {\left(x^{2} + x - 20\right)} \cos\left(4 \, \pi x\right)}=\answer{70}\]

\end{problem}}%}

%%%%%%%%%%%%%%%%%%%%%%




\latexProblemContent{
\begin{problem}
Determine if the limit approaches a finite number, infinity, or does not exist. (If the limit does not exist, write DNE)
\input{./Limits/2311-Compute-Limit-0009.HELP.tex}
\[\lim_{x\to{5}}{-2 \, {\left(x + 5\right)} {\left(x + 2\right)} \sin\left(-\frac{5}{2} \, \pi x\right)}=\answer{140}\]

\end{problem}}%}

%%%%%%%%%%%%%%%%%%%%%%




\latexProblemContent{
\begin{problem}
Determine if the limit approaches a finite number, infinity, or does not exist. (If the limit does not exist, write DNE)
\input{./Limits/2311-Compute-Limit-0009.HELP.tex}
\[\lim_{x\to{-3}}{-5 \, {\left(x + 5\right)} {\left(x - 1\right)}^{2} \cos\left(\frac{1}{2} \, \pi x\right)}=\answer{0}\]

\end{problem}}%}

%%%%%%%%%%%%%%%%%%%%%%




\latexProblemContent{
\begin{problem}
Determine if the limit approaches a finite number, infinity, or does not exist. (If the limit does not exist, write DNE)
\input{./Limits/2311-Compute-Limit-0009.HELP.tex}
\[\lim_{x\to{-1}}{-2 \, {\left(x^{2} - x - 6\right)} \sin\left(3 \, \pi x\right)}=\answer{0}\]

\end{problem}}%}

%%%%%%%%%%%%%%%%%%%%%%




\latexProblemContent{
\begin{problem}
Determine if the limit approaches a finite number, infinity, or does not exist. (If the limit does not exist, write DNE)
\input{./Limits/2311-Compute-Limit-0009.HELP.tex}
\[\lim_{x\to{-2}}{4 \, {\left(x^{3} - 5 \, x^{2} - 8 \, x + 48\right)} \sin\left(-\pi x\right)}=\answer{0}\]

\end{problem}}%}

%%%%%%%%%%%%%%%%%%%%%%




\latexProblemContent{
\begin{problem}
Determine if the limit approaches a finite number, infinity, or does not exist. (If the limit does not exist, write DNE)
\input{./Limits/2311-Compute-Limit-0009.HELP.tex}
\[\lim_{x\to{3}}{2 \, {\left(x - 1\right)} {\left(x - 2\right)} {\left(x - 3\right)} \cos\left(\pi x\right)}=\answer{0}\]

\end{problem}}%}

%%%%%%%%%%%%%%%%%%%%%%




\latexProblemContent{
\begin{problem}
Determine if the limit approaches a finite number, infinity, or does not exist. (If the limit does not exist, write DNE)
\input{./Limits/2311-Compute-Limit-0009.HELP.tex}
\[\lim_{x\to{-4}}{4 \, {\left(x^{3} - 8 \, x^{2} + 20 \, x - 16\right)} \cos\left(\frac{2}{3} \, \pi x\right)}=\answer{576}\]

\end{problem}}%}

%%%%%%%%%%%%%%%%%%%%%%




\latexProblemContent{
\begin{problem}
Determine if the limit approaches a finite number, infinity, or does not exist. (If the limit does not exist, write DNE)
\input{./Limits/2311-Compute-Limit-0009.HELP.tex}
\[\lim_{x\to{-3}}{3 \, {\left(x^{3} + x^{2} - 8 \, x - 12\right)} \sin\left(-\frac{1}{3} \, \pi x\right)}=\answer{0}\]

\end{problem}}%}

%%%%%%%%%%%%%%%%%%%%%%




\latexProblemContent{
\begin{problem}
Determine if the limit approaches a finite number, infinity, or does not exist. (If the limit does not exist, write DNE)
\input{./Limits/2311-Compute-Limit-0009.HELP.tex}
\[\lim_{x\to{-4}}{-{\left(x + 1\right)} {\left(x - 1\right)} {\left(x - 5\right)} \tan\left(\frac{5}{2} \, \pi x\right)}=\answer{0}\]

\end{problem}}%}

%%%%%%%%%%%%%%%%%%%%%%




\latexProblemContent{
\begin{problem}
Determine if the limit approaches a finite number, infinity, or does not exist. (If the limit does not exist, write DNE)
\input{./Limits/2311-Compute-Limit-0009.HELP.tex}
\[\lim_{x\to{2}}{-{\left(x + 1\right)} \sin\left(\frac{4}{3} \, \pi x\right)}=\answer{-\frac{1}{2} \, \left(3 \, \sqrt{3}\right)}\]

\end{problem}}%}

%%%%%%%%%%%%%%%%%%%%%%




\latexProblemContent{
\begin{problem}
Determine if the limit approaches a finite number, infinity, or does not exist. (If the limit does not exist, write DNE)
\input{./Limits/2311-Compute-Limit-0009.HELP.tex}
\[\lim_{x\to{1}}{-3 \, {\left(x + 3\right)} \tan\left(-\frac{1}{3} \, \pi x\right)}=\answer{4 \, \left(3 \, \sqrt{3}\right)}\]

\end{problem}}%}

%%%%%%%%%%%%%%%%%%%%%%




\latexProblemContent{
\begin{problem}
Determine if the limit approaches a finite number, infinity, or does not exist. (If the limit does not exist, write DNE)
\input{./Limits/2311-Compute-Limit-0009.HELP.tex}
\[\lim_{x\to{2}}{-5 \, {\left(x^{2} + x - 20\right)} \sin\left(2 \, \pi x\right)}=\answer{0}\]

\end{problem}}%}

%%%%%%%%%%%%%%%%%%%%%%




\latexProblemContent{
\begin{problem}
Determine if the limit approaches a finite number, infinity, or does not exist. (If the limit does not exist, write DNE)
\input{./Limits/2311-Compute-Limit-0009.HELP.tex}
\[\lim_{x\to{-2}}{2 \, {\left(x^{3} - x^{2} - 4 \, x + 4\right)} \tan\left(\pi x\right)}=\answer{0}\]

\end{problem}}%}

%%%%%%%%%%%%%%%%%%%%%%




\latexProblemContent{
\begin{problem}
Determine if the limit approaches a finite number, infinity, or does not exist. (If the limit does not exist, write DNE)
\input{./Limits/2311-Compute-Limit-0009.HELP.tex}
\[\lim_{x\to{2}}{-4 \, {\left(x^{3} + 5 \, x^{2} - 16 \, x - 80\right)} \sin\left(\frac{2}{3} \, \pi x\right)}=\answer{-56 \, \left(3 \, \sqrt{3}\right)}\]

\end{problem}}%}

%%%%%%%%%%%%%%%%%%%%%%




\latexProblemContent{
\begin{problem}
Determine if the limit approaches a finite number, infinity, or does not exist. (If the limit does not exist, write DNE)
\input{./Limits/2311-Compute-Limit-0009.HELP.tex}
\[\lim_{x\to{1}}{-{\left(x^{3} + 7 \, x^{2} + 14 \, x + 8\right)} \tan\left(-\frac{2}{3} \, \pi x\right)}=\answer{-10 \, \left(3 \, \sqrt{3}\right)}\]

\end{problem}}%}

%%%%%%%%%%%%%%%%%%%%%%




\latexProblemContent{
\begin{problem}
Determine if the limit approaches a finite number, infinity, or does not exist. (If the limit does not exist, write DNE)
\input{./Limits/2311-Compute-Limit-0009.HELP.tex}
\[\lim_{x\to{-1}}{-4 \, {\left(x + 4\right)} {\left(x + 3\right)} {\left(x + 2\right)} \tan\left(-\frac{3}{2} \, \pi x\right)}=\answer{\infty}\]

\end{problem}}%}

%%%%%%%%%%%%%%%%%%%%%%




\latexProblemContent{
\begin{problem}
Determine if the limit approaches a finite number, infinity, or does not exist. (If the limit does not exist, write DNE)
\input{./Limits/2311-Compute-Limit-0009.HELP.tex}
\[\lim_{x\to{-1}}{-3 \, {\left(x + 3\right)} {\left(x - 1\right)} \sin\left(\frac{1}{2} \, \pi x\right)}=\answer{-12}\]

\end{problem}}%}

%%%%%%%%%%%%%%%%%%%%%%




\latexProblemContent{
\begin{problem}
Determine if the limit approaches a finite number, infinity, or does not exist. (If the limit does not exist, write DNE)
\input{./Limits/2311-Compute-Limit-0009.HELP.tex}
\[\lim_{x\to{-4}}{2 \, {\left(x - 2\right)} {\left(x - 5\right)} \tan\left(\frac{5}{3} \, \pi x\right)}=\answer{4 \, \left(27 \, \sqrt{3}\right)}\]

\end{problem}}%}

%%%%%%%%%%%%%%%%%%%%%%




\latexProblemContent{
\begin{problem}
Determine if the limit approaches a finite number, infinity, or does not exist. (If the limit does not exist, write DNE)
\input{./Limits/2311-Compute-Limit-0009.HELP.tex}
\[\lim_{x\to{3}}{-4 \, {\left(x + 4\right)} \tan\left(-3 \, \pi x\right)}=\answer{0}\]

\end{problem}}%}

%%%%%%%%%%%%%%%%%%%%%%




\latexProblemContent{
\begin{problem}
Determine if the limit approaches a finite number, infinity, or does not exist. (If the limit does not exist, write DNE)
\input{./Limits/2311-Compute-Limit-0009.HELP.tex}
\[\lim_{x\to{1}}{4 \, {\left(x^{3} - 7 \, x^{2} + 2 \, x + 40\right)} \tan\left(-\pi x\right)}=\answer{0}\]

\end{problem}}%}

%%%%%%%%%%%%%%%%%%%%%%




\latexProblemContent{
\begin{problem}
Determine if the limit approaches a finite number, infinity, or does not exist. (If the limit does not exist, write DNE)
\input{./Limits/2311-Compute-Limit-0009.HELP.tex}
\[\lim_{x\to{3}}{-5 \, {\left(x^{2} + 3 \, x - 10\right)} \cos\left(\frac{2}{3} \, \pi x\right)}=\answer{-40}\]

\end{problem}}%}

%%%%%%%%%%%%%%%%%%%%%%




\latexProblemContent{
\begin{problem}
Determine if the limit approaches a finite number, infinity, or does not exist. (If the limit does not exist, write DNE)
\input{./Limits/2311-Compute-Limit-0009.HELP.tex}
\[\lim_{x\to{-4}}{{\left(x - 1\right)} \tan\left(-\frac{3}{2} \, \pi x\right)}=\answer{0}\]

\end{problem}}%}

%%%%%%%%%%%%%%%%%%%%%%




\latexProblemContent{
\begin{problem}
Determine if the limit approaches a finite number, infinity, or does not exist. (If the limit does not exist, write DNE)
\input{./Limits/2311-Compute-Limit-0009.HELP.tex}
\[\lim_{x\to{5}}{-{\left(x + 1\right)} \cos\left(\frac{2}{3} \, \pi x\right)}=\answer{3}\]

\end{problem}}%}

%%%%%%%%%%%%%%%%%%%%%%




\latexProblemContent{
\begin{problem}
Determine if the limit approaches a finite number, infinity, or does not exist. (If the limit does not exist, write DNE)
\input{./Limits/2311-Compute-Limit-0009.HELP.tex}
\[\lim_{x\to{-5}}{{\left(x^{2} - 5 \, x + 4\right)} \sin\left(\frac{4}{3} \, \pi x\right)}=\answer{-\left(27 \, \sqrt{3}\right)}\]

\end{problem}}%}

%%%%%%%%%%%%%%%%%%%%%%




\latexProblemContent{
\begin{problem}
Determine if the limit approaches a finite number, infinity, or does not exist. (If the limit does not exist, write DNE)
\input{./Limits/2311-Compute-Limit-0009.HELP.tex}
\[\lim_{x\to{1}}{-3 \, {\left(x^{3} + x^{2} - 9 \, x - 9\right)} \cos\left(-\pi x\right)}=\answer{-48}\]

\end{problem}}%}

%%%%%%%%%%%%%%%%%%%%%%




\latexProblemContent{
\begin{problem}
Determine if the limit approaches a finite number, infinity, or does not exist. (If the limit does not exist, write DNE)
\input{./Limits/2311-Compute-Limit-0009.HELP.tex}
\[\lim_{x\to{-1}}{-4 \, {\left(x + 4\right)} {\left(x - 4\right)} \tan\left(2 \, \pi x\right)}=\answer{0}\]

\end{problem}}%}

%%%%%%%%%%%%%%%%%%%%%%




\latexProblemContent{
\begin{problem}
Determine if the limit approaches a finite number, infinity, or does not exist. (If the limit does not exist, write DNE)
\input{./Limits/2311-Compute-Limit-0009.HELP.tex}
\[\lim_{x\to{-3}}{-4 \, {\left(x + 4\right)} {\left(x - 3\right)} \sin\left(\pi x\right)}=\answer{0}\]

\end{problem}}%}

%%%%%%%%%%%%%%%%%%%%%%




\latexProblemContent{
\begin{problem}
Determine if the limit approaches a finite number, infinity, or does not exist. (If the limit does not exist, write DNE)
\input{./Limits/2311-Compute-Limit-0009.HELP.tex}
\[\lim_{x\to{-4}}{5 \, {\left(x - 5\right)} \cos\left(-\frac{5}{3} \, \pi x\right)}=\answer{\frac{45}{2}}\]

\end{problem}}%}

%%%%%%%%%%%%%%%%%%%%%%




\latexProblemContent{
\begin{problem}
Determine if the limit approaches a finite number, infinity, or does not exist. (If the limit does not exist, write DNE)
\input{./Limits/2311-Compute-Limit-0009.HELP.tex}
\[\lim_{x\to{3}}{2 \, {\left(x - 2\right)} \tan\left(-3 \, \pi x\right)}=\answer{0}\]

\end{problem}}%}

%%%%%%%%%%%%%%%%%%%%%%




\latexProblemContent{
\begin{problem}
Determine if the limit approaches a finite number, infinity, or does not exist. (If the limit does not exist, write DNE)
\input{./Limits/2311-Compute-Limit-0009.HELP.tex}
\[\lim_{x\to{-4}}{2 \, {\left(x^{2} - 6 \, x + 8\right)} \sin\left(\frac{4}{3} \, \pi x\right)}=\answer{16 \, \left(3 \, \sqrt{3}\right)}\]

\end{problem}}%}

%%%%%%%%%%%%%%%%%%%%%%




\latexProblemContent{
\begin{problem}
Determine if the limit approaches a finite number, infinity, or does not exist. (If the limit does not exist, write DNE)
\input{./Limits/2311-Compute-Limit-0009.HELP.tex}
\[\lim_{x\to{2}}{-2 \, {\left(x + 2\right)} {\left(x - 2\right)} \tan\left(\pi x\right)}=\answer{0}\]

\end{problem}}%}

%%%%%%%%%%%%%%%%%%%%%%




\latexProblemContent{
\begin{problem}
Determine if the limit approaches a finite number, infinity, or does not exist. (If the limit does not exist, write DNE)
\input{./Limits/2311-Compute-Limit-0009.HELP.tex}
\[\lim_{x\to{5}}{-4 \, {\left(x^{2} - 16\right)} \tan\left(\frac{4}{3} \, \pi x\right)}=\answer{4 \, \left(9 \, \sqrt{3}\right)}\]

\end{problem}}%}

%%%%%%%%%%%%%%%%%%%%%%




\latexProblemContent{
\begin{problem}
Determine if the limit approaches a finite number, infinity, or does not exist. (If the limit does not exist, write DNE)
\input{./Limits/2311-Compute-Limit-0009.HELP.tex}
\[\lim_{x\to{-2}}{4 \, {\left(x^{3} - x^{2} - 16 \, x + 16\right)} \cos\left(-4 \, \pi x\right)}=\answer{144}\]

\end{problem}}%}

%%%%%%%%%%%%%%%%%%%%%%




\latexProblemContent{
\begin{problem}
Determine if the limit approaches a finite number, infinity, or does not exist. (If the limit does not exist, write DNE)
\input{./Limits/2311-Compute-Limit-0009.HELP.tex}
\[\lim_{x\to{-1}}{-4 \, {\left(x^{3} + 5 \, x^{2} - 16 \, x - 80\right)} \cos\left(2 \, \pi x\right)}=\answer{240}\]

\end{problem}}%}

%%%%%%%%%%%%%%%%%%%%%%




\latexProblemContent{
\begin{problem}
Determine if the limit approaches a finite number, infinity, or does not exist. (If the limit does not exist, write DNE)
\input{./Limits/2311-Compute-Limit-0009.HELP.tex}
\[\lim_{x\to{5}}{-{\left(x + 4\right)} {\left(x + 2\right)} {\left(x + 1\right)} \cos\left(-4 \, \pi x\right)}=\answer{-378}\]

\end{problem}}%}

%%%%%%%%%%%%%%%%%%%%%%




\latexProblemContent{
\begin{problem}
Determine if the limit approaches a finite number, infinity, or does not exist. (If the limit does not exist, write DNE)
\input{./Limits/2311-Compute-Limit-0009.HELP.tex}
\[\lim_{x\to{-2}}{5 \, {\left(x^{3} - x^{2} - 17 \, x - 15\right)} \sin\left(-\pi x\right)}=\answer{0}\]

\end{problem}}%}

%%%%%%%%%%%%%%%%%%%%%%




\latexProblemContent{
\begin{problem}
Determine if the limit approaches a finite number, infinity, or does not exist. (If the limit does not exist, write DNE)
\input{./Limits/2311-Compute-Limit-0009.HELP.tex}
\[\lim_{x\to{5}}{-2 \, {\left(x^{3} + x^{2} - 4 \, x - 4\right)} \sin\left(-\frac{1}{2} \, \pi x\right)}=\answer{252}\]

\end{problem}}%}

%%%%%%%%%%%%%%%%%%%%%%




\latexProblemContent{
\begin{problem}
Determine if the limit approaches a finite number, infinity, or does not exist. (If the limit does not exist, write DNE)
\input{./Limits/2311-Compute-Limit-0009.HELP.tex}
\[\lim_{x\to{-2}}{-5 \, {\left(x + 5\right)} {\left(x + 2\right)} \tan\left(-\frac{1}{3} \, \pi x\right)}=\answer{0}\]

\end{problem}}%}

%%%%%%%%%%%%%%%%%%%%%%




\latexProblemContent{
\begin{problem}
Determine if the limit approaches a finite number, infinity, or does not exist. (If the limit does not exist, write DNE)
\input{./Limits/2311-Compute-Limit-0009.HELP.tex}
\[\lim_{x\to{-4}}{-5 \, {\left(x^{3} + 2 \, x^{2} - 13 \, x + 10\right)} \cos\left(\frac{2}{3} \, \pi x\right)}=\answer{75}\]

\end{problem}}%}

%%%%%%%%%%%%%%%%%%%%%%




\latexProblemContent{
\begin{problem}
Determine if the limit approaches a finite number, infinity, or does not exist. (If the limit does not exist, write DNE)
\input{./Limits/2311-Compute-Limit-0009.HELP.tex}
\[\lim_{x\to{-1}}{-4 \, {\left(x^{2} + 6 \, x + 8\right)} \tan\left(-\frac{2}{3} \, \pi x\right)}=\answer{4 \, \left(3 \, \sqrt{3}\right)}\]

\end{problem}}%}

%%%%%%%%%%%%%%%%%%%%%%




\latexProblemContent{
\begin{problem}
Determine if the limit approaches a finite number, infinity, or does not exist. (If the limit does not exist, write DNE)
\input{./Limits/2311-Compute-Limit-0009.HELP.tex}
\[\lim_{x\to{-5}}{-4 \, {\left(x + 4\right)} {\left(x - 1\right)} \tan\left(\pi x\right)}=\answer{0}\]

\end{problem}}%}

%%%%%%%%%%%%%%%%%%%%%%




\latexProblemContent{
\begin{problem}
Determine if the limit approaches a finite number, infinity, or does not exist. (If the limit does not exist, write DNE)
\input{./Limits/2311-Compute-Limit-0009.HELP.tex}
\[\lim_{x\to{3}}{-3 \, {\left(x + 3\right)} \sin\left(\frac{3}{2} \, \pi x\right)}=\answer{-18}\]

\end{problem}}%}

%%%%%%%%%%%%%%%%%%%%%%




\latexProblemContent{
\begin{problem}
Determine if the limit approaches a finite number, infinity, or does not exist. (If the limit does not exist, write DNE)
\input{./Limits/2311-Compute-Limit-0009.HELP.tex}
\[\lim_{x\to{-1}}{{\left(x + 3\right)} {\left(x - 1\right)} \tan\left(-\pi x\right)}=\answer{0}\]

\end{problem}}%}

%%%%%%%%%%%%%%%%%%%%%%




\latexProblemContent{
\begin{problem}
Determine if the limit approaches a finite number, infinity, or does not exist. (If the limit does not exist, write DNE)
\input{./Limits/2311-Compute-Limit-0009.HELP.tex}
\[\lim_{x\to{3}}{-3 \, {\left(x + 3\right)} {\left(x - 2\right)} \tan\left(\frac{2}{3} \, \pi x\right)}=\answer{0}\]

\end{problem}}%}

%%%%%%%%%%%%%%%%%%%%%%




\latexProblemContent{
\begin{problem}
Determine if the limit approaches a finite number, infinity, or does not exist. (If the limit does not exist, write DNE)
\input{./Limits/2311-Compute-Limit-0009.HELP.tex}
\[\lim_{x\to{-5}}{-3 \, {\left(x + 4\right)} {\left(x + 3\right)} \sin\left(-\frac{4}{3} \, \pi x\right)}=\answer{-\left(3 \, \sqrt{3}\right)}\]

\end{problem}}%}

%%%%%%%%%%%%%%%%%%%%%%




%%%%%%%%%%%%%%%%%%%%%%




\latexProblemContent{
\begin{problem}
Determine if the limit approaches a finite number, infinity, or does not exist. (If the limit does not exist, write DNE)
\input{./Limits/2311-Compute-Limit-0009.HELP.tex}
\[\lim_{x\to{-4}}{{\left(x^{3} + 2 \, x^{2} - x - 2\right)} \tan\left(-\frac{1}{2} \, \pi x\right)}=\answer{0}\]

\end{problem}}%}

%%%%%%%%%%%%%%%%%%%%%%




\latexProblemContent{
\begin{problem}
Determine if the limit approaches a finite number, infinity, or does not exist. (If the limit does not exist, write DNE)
\input{./Limits/2311-Compute-Limit-0009.HELP.tex}
\[\lim_{x\to{5}}{3 \, {\left(x - 3\right)} \sin\left(-\pi x\right)}=\answer{0}\]

\end{problem}}%}

%%%%%%%%%%%%%%%%%%%%%%




\latexProblemContent{
\begin{problem}
Determine if the limit approaches a finite number, infinity, or does not exist. (If the limit does not exist, write DNE)
\input{./Limits/2311-Compute-Limit-0009.HELP.tex}
\[\lim_{x\to{5}}{-4 \, {\left(x + 5\right)} {\left(x + 4\right)} \tan\left(-\frac{5}{6} \, \pi x\right)}=\answer{40 \, \left(3 \, \sqrt{3}\right)}\]

\end{problem}}%}

%%%%%%%%%%%%%%%%%%%%%%




\latexProblemContent{
\begin{problem}
Determine if the limit approaches a finite number, infinity, or does not exist. (If the limit does not exist, write DNE)
\input{./Limits/2311-Compute-Limit-0009.HELP.tex}
\[\lim_{x\to{2}}{4 \, {\left(x^{3} - 2 \, x^{2} - 23 \, x + 60\right)} \sin\left(3 \, \pi x\right)}=\answer{0}\]

\end{problem}}%}

%%%%%%%%%%%%%%%%%%%%%%




\latexProblemContent{
\begin{problem}
Determine if the limit approaches a finite number, infinity, or does not exist. (If the limit does not exist, write DNE)
\input{./Limits/2311-Compute-Limit-0009.HELP.tex}
\[\lim_{x\to{-2}}{5 \, {\left(x + 2\right)} {\left(x - 1\right)} {\left(x - 5\right)} \tan\left(-\frac{1}{3} \, \pi x\right)}=\answer{0}\]

\end{problem}}%}

%%%%%%%%%%%%%%%%%%%%%%




\latexProblemContent{
\begin{problem}
Determine if the limit approaches a finite number, infinity, or does not exist. (If the limit does not exist, write DNE)
\input{./Limits/2311-Compute-Limit-0009.HELP.tex}
\[\lim_{x\to{3}}{-5 \, {\left(x + 5\right)} {\left(x + 4\right)} {\left(x - 3\right)} \sin\left(\frac{3}{2} \, \pi x\right)}=\answer{0}\]

\end{problem}}%}

%%%%%%%%%%%%%%%%%%%%%%




\latexProblemContent{
\begin{problem}
Determine if the limit approaches a finite number, infinity, or does not exist. (If the limit does not exist, write DNE)
\input{./Limits/2311-Compute-Limit-0009.HELP.tex}
\[\lim_{x\to{-1}}{{\left(x^{2} + 3 \, x - 4\right)} \sin\left(-4 \, \pi x\right)}=\answer{0}\]

\end{problem}}%}

%%%%%%%%%%%%%%%%%%%%%%




\latexProblemContent{
\begin{problem}
Determine if the limit approaches a finite number, infinity, or does not exist. (If the limit does not exist, write DNE)
\input{./Limits/2311-Compute-Limit-0009.HELP.tex}
\[\lim_{x\to{2}}{3 \, {\left(x + 2\right)} {\left(x - 3\right)} \sin\left(-\frac{2}{3} \, \pi x\right)}=\answer{-2 \, \left(3 \, \sqrt{3}\right)}\]

\end{problem}}%}

%%%%%%%%%%%%%%%%%%%%%%




\latexProblemContent{
\begin{problem}
Determine if the limit approaches a finite number, infinity, or does not exist. (If the limit does not exist, write DNE)
\input{./Limits/2311-Compute-Limit-0009.HELP.tex}
\[\lim_{x\to{3}}{-5 \, {\left(x^{3} + 2 \, x^{2} - 13 \, x + 10\right)} \sin\left(\frac{2}{3} \, \pi x\right)}=\answer{0}\]

\end{problem}}%}

%%%%%%%%%%%%%%%%%%%%%%




\latexProblemContent{
\begin{problem}
Determine if the limit approaches a finite number, infinity, or does not exist. (If the limit does not exist, write DNE)
\input{./Limits/2311-Compute-Limit-0009.HELP.tex}
\[\lim_{x\to{-3}}{4 \, {\left(x^{3} - 11 \, x^{2} + 40 \, x - 48\right)} \cos\left(\frac{3}{2} \, \pi x\right)}=\answer{0}\]

\end{problem}}%}

%%%%%%%%%%%%%%%%%%%%%%




\latexProblemContent{
\begin{problem}
Determine if the limit approaches a finite number, infinity, or does not exist. (If the limit does not exist, write DNE)
\input{./Limits/2311-Compute-Limit-0009.HELP.tex}
\[\lim_{x\to{-5}}{5 \, {\left(x + 5\right)} {\left(x - 5\right)} \tan\left(-\frac{5}{3} \, \pi x\right)}=\answer{0}\]

\end{problem}}%}

%%%%%%%%%%%%%%%%%%%%%%




\latexProblemContent{
\begin{problem}
Determine if the limit approaches a finite number, infinity, or does not exist. (If the limit does not exist, write DNE)
\input{./Limits/2311-Compute-Limit-0009.HELP.tex}
\[\lim_{x\to{-1}}{3 \, {\left(x + 2\right)} {\left(x - 3\right)} \cos\left(-2 \, \pi x\right)}=\answer{-12}\]

\end{problem}}%}

%%%%%%%%%%%%%%%%%%%%%%




\latexProblemContent{
\begin{problem}
Determine if the limit approaches a finite number, infinity, or does not exist. (If the limit does not exist, write DNE)
\input{./Limits/2311-Compute-Limit-0009.HELP.tex}
\[\lim_{x\to{-3}}{-{\left(x + 1\right)} {\left(x - 2\right)} \sin\left(\frac{2}{3} \, \pi x\right)}=\answer{0}\]

\end{problem}}%}

%%%%%%%%%%%%%%%%%%%%%%




\latexProblemContent{
\begin{problem}
Determine if the limit approaches a finite number, infinity, or does not exist. (If the limit does not exist, write DNE)
\input{./Limits/2311-Compute-Limit-0009.HELP.tex}
\[\lim_{x\to{-5}}{5 \, {\left(x + 5\right)} {\left(x - 5\right)} \sin\left(-\frac{5}{6} \, \pi x\right)}=\answer{0}\]

\end{problem}}%}

%%%%%%%%%%%%%%%%%%%%%%




\latexProblemContent{
\begin{problem}
Determine if the limit approaches a finite number, infinity, or does not exist. (If the limit does not exist, write DNE)
\input{./Limits/2311-Compute-Limit-0009.HELP.tex}
\[\lim_{x\to{-2}}{5 \, {\left(x - 5\right)} \tan\left(-\frac{1}{6} \, \pi x\right)}=\answer{-35 \, \sqrt{3}}\]

\end{problem}}%}

%%%%%%%%%%%%%%%%%%%%%%




\latexProblemContent{
\begin{problem}
Determine if the limit approaches a finite number, infinity, or does not exist. (If the limit does not exist, write DNE)
\input{./Limits/2311-Compute-Limit-0009.HELP.tex}
\[\lim_{x\to{-1}}{-2 \, {\left(x^{2} + 3 \, x + 2\right)} \tan\left(-\frac{1}{2} \, \pi x\right)}=\answer{-\frac{4}{\pi}}\]

\end{problem}}%}

%%%%%%%%%%%%%%%%%%%%%%




\latexProblemContent{
\begin{problem}
Determine if the limit approaches a finite number, infinity, or does not exist. (If the limit does not exist, write DNE)
\input{./Limits/2311-Compute-Limit-0009.HELP.tex}
\[\lim_{x\to{-3}}{2 \, {\left(x + 4\right)} {\left(x - 2\right)} {\left(x - 4\right)} \tan\left(-\frac{4}{3} \, \pi x\right)}=\answer{0}\]

\end{problem}}%}

%%%%%%%%%%%%%%%%%%%%%%




\latexProblemContent{
\begin{problem}
Determine if the limit approaches a finite number, infinity, or does not exist. (If the limit does not exist, write DNE)
\input{./Limits/2311-Compute-Limit-0009.HELP.tex}
\[\lim_{x\to{-2}}{-{\left(x + 1\right)} \tan\left(\frac{3}{2} \, \pi x\right)}=\answer{0}\]

\end{problem}}%}

%%%%%%%%%%%%%%%%%%%%%%




\latexProblemContent{
\begin{problem}
Determine if the limit approaches a finite number, infinity, or does not exist. (If the limit does not exist, write DNE)
\input{./Limits/2311-Compute-Limit-0009.HELP.tex}
\[\lim_{x\to{-3}}{-{\left(x + 1\right)} {\left(x - 1\right)} \tan\left(\frac{1}{3} \, \pi x\right)}=\answer{0}\]

\end{problem}}%}

%%%%%%%%%%%%%%%%%%%%%%




\latexProblemContent{
\begin{problem}
Determine if the limit approaches a finite number, infinity, or does not exist. (If the limit does not exist, write DNE)
\input{./Limits/2311-Compute-Limit-0009.HELP.tex}
\[\lim_{x\to{-3}}{5 \, {\left(x^{2} - 7 \, x + 10\right)} \sin\left(\frac{2}{3} \, \pi x\right)}=\answer{0}\]

\end{problem}}%}

%%%%%%%%%%%%%%%%%%%%%%




\latexProblemContent{
\begin{problem}
Determine if the limit approaches a finite number, infinity, or does not exist. (If the limit does not exist, write DNE)
\input{./Limits/2311-Compute-Limit-0009.HELP.tex}
\[\lim_{x\to{5}}{-3 \, {\left(x + 3\right)} {\left(x - 4\right)} {\left(x - 5\right)} \tan\left(\frac{4}{3} \, \pi x\right)}=\answer{0}\]

\end{problem}}%}

%%%%%%%%%%%%%%%%%%%%%%




\latexProblemContent{
\begin{problem}
Determine if the limit approaches a finite number, infinity, or does not exist. (If the limit does not exist, write DNE)
\input{./Limits/2311-Compute-Limit-0009.HELP.tex}
\[\lim_{x\to{5}}{-2 \, {\left(x + 5\right)} {\left(x + 4\right)} {\left(x + 2\right)} \cos\left(-2 \, \pi x\right)}=\answer{-1260}\]

\end{problem}}%}

%%%%%%%%%%%%%%%%%%%%%%




\latexProblemContent{
\begin{problem}
Determine if the limit approaches a finite number, infinity, or does not exist. (If the limit does not exist, write DNE)
\input{./Limits/2311-Compute-Limit-0009.HELP.tex}
\[\lim_{x\to{5}}{3 \, {\left(x^{2} + 2 \, x - 15\right)} \sin\left(-5 \, \pi x\right)}=\answer{0}\]

\end{problem}}%}

%%%%%%%%%%%%%%%%%%%%%%




\latexProblemContent{
\begin{problem}
Determine if the limit approaches a finite number, infinity, or does not exist. (If the limit does not exist, write DNE)
\input{./Limits/2311-Compute-Limit-0009.HELP.tex}
\[\lim_{x\to{4}}{{\left(x + 1\right)} {\left(x - 1\right)} \cos\left(-\pi x\right)}=\answer{15}\]

\end{problem}}%}

%%%%%%%%%%%%%%%%%%%%%%




\latexProblemContent{
\begin{problem}
Determine if the limit approaches a finite number, infinity, or does not exist. (If the limit does not exist, write DNE)
\input{./Limits/2311-Compute-Limit-0009.HELP.tex}
\[\lim_{x\to{2}}{{\left(x + 4\right)} {\left(x + 3\right)} {\left(x - 1\right)} \sin\left(-\frac{3}{2} \, \pi x\right)}=\answer{0}\]

\end{problem}}%}

%%%%%%%%%%%%%%%%%%%%%%




\latexProblemContent{
\begin{problem}
Determine if the limit approaches a finite number, infinity, or does not exist. (If the limit does not exist, write DNE)
\input{./Limits/2311-Compute-Limit-0009.HELP.tex}
\[\lim_{x\to{1}}{2 \, {\left(x^{3} - 8 \, x^{2} + 20 \, x - 16\right)} \tan\left(\frac{4}{3} \, \pi x\right)}=\answer{-2 \, \left(3 \, \sqrt{3}\right)}\]

\end{problem}}%}

%%%%%%%%%%%%%%%%%%%%%%




\latexProblemContent{
\begin{problem}
Determine if the limit approaches a finite number, infinity, or does not exist. (If the limit does not exist, write DNE)
\input{./Limits/2311-Compute-Limit-0009.HELP.tex}
\[\lim_{x\to{-5}}{4 \, {\left(x + 4\right)} {\left(x - 4\right)} \tan\left(-\frac{4}{3} \, \pi x\right)}=\answer{-4 \, \left(9 \, \sqrt{3}\right)}\]

\end{problem}}%}

%%%%%%%%%%%%%%%%%%%%%%




\latexProblemContent{
\begin{problem}
Determine if the limit approaches a finite number, infinity, or does not exist. (If the limit does not exist, write DNE)
\input{./Limits/2311-Compute-Limit-0009.HELP.tex}
\[\lim_{x\to{4}}{3 \, {\left(x - 1\right)} {\left(x - 3\right)} {\left(x - 4\right)} \cos\left(\frac{1}{6} \, \pi x\right)}=\answer{0}\]

\end{problem}}%}

%%%%%%%%%%%%%%%%%%%%%%




\latexProblemContent{
\begin{problem}
Determine if the limit approaches a finite number, infinity, or does not exist. (If the limit does not exist, write DNE)
\input{./Limits/2311-Compute-Limit-0009.HELP.tex}
\[\lim_{x\to{-1}}{2 \, {\left(x + 4\right)} {\left(x - 2\right)} \cos\left(-\frac{2}{3} \, \pi x\right)}=\answer{9}\]

\end{problem}}%}

%%%%%%%%%%%%%%%%%%%%%%




\latexProblemContent{
\begin{problem}
Determine if the limit approaches a finite number, infinity, or does not exist. (If the limit does not exist, write DNE)
\input{./Limits/2311-Compute-Limit-0009.HELP.tex}
\[\lim_{x\to{-4}}{5 \, {\left(x + 4\right)} {\left(x - 5\right)} \sin\left(-\frac{4}{3} \, \pi x\right)}=\answer{0}\]

\end{problem}}%}

%%%%%%%%%%%%%%%%%%%%%%




\latexProblemContent{
\begin{problem}
Determine if the limit approaches a finite number, infinity, or does not exist. (If the limit does not exist, write DNE)
\input{./Limits/2311-Compute-Limit-0009.HELP.tex}
\[\lim_{x\to{-4}}{4 \, {\left(x - 4\right)} {\left(x - 5\right)} \sin\left(\frac{5}{6} \, \pi x\right)}=\answer{16 \, \left(9 \, \sqrt{3}\right)}\]

\end{problem}}%}

%%%%%%%%%%%%%%%%%%%%%%




\latexProblemContent{
\begin{problem}
Determine if the limit approaches a finite number, infinity, or does not exist. (If the limit does not exist, write DNE)
\input{./Limits/2311-Compute-Limit-0009.HELP.tex}
\[\lim_{x\to{-2}}{{\left(x + 2\right)} {\left(x - 1\right)} \cos\left(-\frac{2}{3} \, \pi x\right)}=\answer{0}\]

\end{problem}}%}

%%%%%%%%%%%%%%%%%%%%%%




\latexProblemContent{
\begin{problem}
Determine if the limit approaches a finite number, infinity, or does not exist. (If the limit does not exist, write DNE)
\input{./Limits/2311-Compute-Limit-0009.HELP.tex}
\[\lim_{x\to{4}}{3 \, {\left(x^{3} - 3 \, x^{2} - 16 \, x + 48\right)} \tan\left(\frac{2}{3} \, \pi x\right)}=\answer{0}\]

\end{problem}}%}

%%%%%%%%%%%%%%%%%%%%%%




\latexProblemContent{
\begin{problem}
Determine if the limit approaches a finite number, infinity, or does not exist. (If the limit does not exist, write DNE)
\input{./Limits/2311-Compute-Limit-0009.HELP.tex}
\[\lim_{x\to{5}}{-4 \, {\left(x + 4\right)} \tan\left(-\pi x\right)}=\answer{0}\]

\end{problem}}%}

%%%%%%%%%%%%%%%%%%%%%%




\latexProblemContent{
\begin{problem}
Determine if the limit approaches a finite number, infinity, or does not exist. (If the limit does not exist, write DNE)
\input{./Limits/2311-Compute-Limit-0009.HELP.tex}
\[\lim_{x\to{-1}}{-5 \, {\left(x + 5\right)} {\left(x + 1\right)} {\left(x - 2\right)} \tan\left(-\pi x\right)}=\answer{0}\]

\end{problem}}%}

%%%%%%%%%%%%%%%%%%%%%%




\latexProblemContent{
\begin{problem}
Determine if the limit approaches a finite number, infinity, or does not exist. (If the limit does not exist, write DNE)
\input{./Limits/2311-Compute-Limit-0009.HELP.tex}
\[\lim_{x\to{3}}{5 \, {\left(x^{2} - 2 \, x - 15\right)} \tan\left(-\frac{3}{2} \, \pi x\right)}=\answer{\infty}\]

\end{problem}}%}

%%%%%%%%%%%%%%%%%%%%%%




\latexProblemContent{
\begin{problem}
Determine if the limit approaches a finite number, infinity, or does not exist. (If the limit does not exist, write DNE)
\input{./Limits/2311-Compute-Limit-0009.HELP.tex}
\[\lim_{x\to{-1}}{5 \, {\left(x^{3} - 5 \, x^{2} - 9 \, x + 45\right)} \cos\left(-\frac{3}{2} \, \pi x\right)}=\answer{0}\]

\end{problem}}%}

%%%%%%%%%%%%%%%%%%%%%%




\latexProblemContent{
\begin{problem}
Determine if the limit approaches a finite number, infinity, or does not exist. (If the limit does not exist, write DNE)
\input{./Limits/2311-Compute-Limit-0009.HELP.tex}
\[\lim_{x\to{-1}}{-5 \, {\left(x + 5\right)} {\left(x + 3\right)} {\left(x - 4\right)} \tan\left(2 \, \pi x\right)}=\answer{0}\]

\end{problem}}%}

%%%%%%%%%%%%%%%%%%%%%%




\latexProblemContent{
\begin{problem}
Determine if the limit approaches a finite number, infinity, or does not exist. (If the limit does not exist, write DNE)
\input{./Limits/2311-Compute-Limit-0009.HELP.tex}
\[\lim_{x\to{-4}}{-{\left(x + 1\right)} {\left(x - 2\right)} {\left(x - 3\right)} \sin\left(\frac{1}{3} \, \pi x\right)}=\answer{7 \, \left(9 \, \sqrt{3}\right)}\]

\end{problem}}%}

%%%%%%%%%%%%%%%%%%%%%%




\latexProblemContent{
\begin{problem}
Determine if the limit approaches a finite number, infinity, or does not exist. (If the limit does not exist, write DNE)
\input{./Limits/2311-Compute-Limit-0009.HELP.tex}
\[\lim_{x\to{1}}{-2 \, {\left(x + 5\right)} {\left(x + 2\right)} {\left(x - 2\right)} \tan\left(\frac{2}{3} \, \pi x\right)}=\answer{-4 \, \left(9 \, \sqrt{3}\right)}\]

\end{problem}}%}

%%%%%%%%%%%%%%%%%%%%%%




\latexProblemContent{
\begin{problem}
Determine if the limit approaches a finite number, infinity, or does not exist. (If the limit does not exist, write DNE)
\input{./Limits/2311-Compute-Limit-0009.HELP.tex}
\[\lim_{x\to{-2}}{-4 \, {\left(x + 4\right)}^{2} {\left(x - 2\right)} \sin\left(2 \, \pi x\right)}=\answer{0}\]

\end{problem}}%}

%%%%%%%%%%%%%%%%%%%%%%




\latexProblemContent{
\begin{problem}
Determine if the limit approaches a finite number, infinity, or does not exist. (If the limit does not exist, write DNE)
\input{./Limits/2311-Compute-Limit-0009.HELP.tex}
\[\lim_{x\to{-3}}{3 \, {\left(x^{3} + 4 \, x^{2} - 9 \, x - 36\right)} \cos\left(-4 \, \pi x\right)}=\answer{0}\]

\end{problem}}%}

%%%%%%%%%%%%%%%%%%%%%%




\latexProblemContent{
\begin{problem}
Determine if the limit approaches a finite number, infinity, or does not exist. (If the limit does not exist, write DNE)
\input{./Limits/2311-Compute-Limit-0009.HELP.tex}
\[\lim_{x\to{3}}{-4 \, {\left(x^{3} + 7 \, x^{2} + 14 \, x + 8\right)} \cos\left(-\frac{1}{3} \, \pi x\right)}=\answer{560}\]

\end{problem}}%}

%%%%%%%%%%%%%%%%%%%%%%




\latexProblemContent{
\begin{problem}
Determine if the limit approaches a finite number, infinity, or does not exist. (If the limit does not exist, write DNE)
\input{./Limits/2311-Compute-Limit-0009.HELP.tex}
\[\lim_{x\to{2}}{5 \, {\left(x - 2\right)} {\left(x - 5\right)} \sin\left(\frac{2}{3} \, \pi x\right)}=\answer{0}\]

\end{problem}}%}

%%%%%%%%%%%%%%%%%%%%%%




\latexProblemContent{
\begin{problem}
Determine if the limit approaches a finite number, infinity, or does not exist. (If the limit does not exist, write DNE)
\input{./Limits/2311-Compute-Limit-0009.HELP.tex}
\[\lim_{x\to{-4}}{4 \, {\left(x^{2} + x - 20\right)} \cos\left(-\frac{5}{6} \, \pi x\right)}=\answer{16}\]

\end{problem}}%}

%%%%%%%%%%%%%%%%%%%%%%




\latexProblemContent{
\begin{problem}
Determine if the limit approaches a finite number, infinity, or does not exist. (If the limit does not exist, write DNE)
\input{./Limits/2311-Compute-Limit-0009.HELP.tex}
\[\lim_{x\to{-3}}{-2 \, {\left(x + 2\right)} {\left(x - 1\right)} \cos\left(\pi x\right)}=\answer{8}\]

\end{problem}}%}

%%%%%%%%%%%%%%%%%%%%%%




\latexProblemContent{
\begin{problem}
Determine if the limit approaches a finite number, infinity, or does not exist. (If the limit does not exist, write DNE)
\input{./Limits/2311-Compute-Limit-0009.HELP.tex}
\[\lim_{x\to{4}}{-5 \, {\left(x + 5\right)} {\left(x + 4\right)} {\left(x - 4\right)} \sin\left(4 \, \pi x\right)}=\answer{0}\]

\end{problem}}%}

%%%%%%%%%%%%%%%%%%%%%%




\latexProblemContent{
\begin{problem}
Determine if the limit approaches a finite number, infinity, or does not exist. (If the limit does not exist, write DNE)
\input{./Limits/2311-Compute-Limit-0009.HELP.tex}
\[\lim_{x\to{1}}{5 \, {\left(x - 5\right)} \tan\left(\pi x\right)}=\answer{0}\]

\end{problem}}%}

%%%%%%%%%%%%%%%%%%%%%%




\latexProblemContent{
\begin{problem}
Determine if the limit approaches a finite number, infinity, or does not exist. (If the limit does not exist, write DNE)
\input{./Limits/2311-Compute-Limit-0009.HELP.tex}
\[\lim_{x\to{2}}{-2 \, {\left(x + 3\right)} {\left(x + 2\right)} \cos\left(-\frac{1}{2} \, \pi x\right)}=\answer{40}\]

\end{problem}}%}

%%%%%%%%%%%%%%%%%%%%%%




\latexProblemContent{
\begin{problem}
Determine if the limit approaches a finite number, infinity, or does not exist. (If the limit does not exist, write DNE)
\input{./Limits/2311-Compute-Limit-0009.HELP.tex}
\[\lim_{x\to{-3}}{-4 \, {\left(x^{2} + 3 \, x - 4\right)} \cos\left(\frac{1}{3} \, \pi x\right)}=\answer{-16}\]

\end{problem}}%}

%%%%%%%%%%%%%%%%%%%%%%




\latexProblemContent{
\begin{problem}
Determine if the limit approaches a finite number, infinity, or does not exist. (If the limit does not exist, write DNE)
\input{./Limits/2311-Compute-Limit-0009.HELP.tex}
\[\lim_{x\to{3}}{2 \, {\left(x - 1\right)} {\left(x - 2\right)} {\left(x - 3\right)} \tan\left(\pi x\right)}=\answer{0}\]

\end{problem}}%}

%%%%%%%%%%%%%%%%%%%%%%




\latexProblemContent{
\begin{problem}
Determine if the limit approaches a finite number, infinity, or does not exist. (If the limit does not exist, write DNE)
\input{./Limits/2311-Compute-Limit-0009.HELP.tex}
\[\lim_{x\to{4}}{-2 \, {\left(x^{3} + x^{2} - 22 \, x - 40\right)} \sin\left(-4 \, \pi x\right)}=\answer{0}\]

\end{problem}}%}

%%%%%%%%%%%%%%%%%%%%%%




\latexProblemContent{
\begin{problem}
Determine if the limit approaches a finite number, infinity, or does not exist. (If the limit does not exist, write DNE)
\input{./Limits/2311-Compute-Limit-0009.HELP.tex}
\[\lim_{x\to{-1}}{5 \, {\left(x - 5\right)} \sin\left(\frac{1}{3} \, \pi x\right)}=\answer{5 \, \left(3 \, \sqrt{3}\right)}\]

\end{problem}}%}

%%%%%%%%%%%%%%%%%%%%%%




\latexProblemContent{
\begin{problem}
Determine if the limit approaches a finite number, infinity, or does not exist. (If the limit does not exist, write DNE)
\input{./Limits/2311-Compute-Limit-0009.HELP.tex}
\[\lim_{x\to{5}}{4 \, {\left(x^{3} - 5 \, x^{2} - 8 \, x + 48\right)} \tan\left(-\frac{1}{2} \, \pi x\right)}=\answer{\infty}\]

\end{problem}}%}

%%%%%%%%%%%%%%%%%%%%%%




\latexProblemContent{
\begin{problem}
Determine if the limit approaches a finite number, infinity, or does not exist. (If the limit does not exist, write DNE)
\input{./Limits/2311-Compute-Limit-0009.HELP.tex}
\[\lim_{x\to{-5}}{-3 \, {\left(x + 3\right)} {\left(x - 1\right)} \sin\left(\frac{1}{2} \, \pi x\right)}=\answer{36}\]

\end{problem}}%}

%%%%%%%%%%%%%%%%%%%%%%




\latexProblemContent{
\begin{problem}
Determine if the limit approaches a finite number, infinity, or does not exist. (If the limit does not exist, write DNE)
\input{./Limits/2311-Compute-Limit-0009.HELP.tex}
\[\lim_{x\to{-2}}{{\left(x^{3} + 6 \, x^{2} + 5 \, x - 12\right)} \cos\left(-\frac{4}{3} \, \pi x\right)}=\answer{3}\]

\end{problem}}%}

%%%%%%%%%%%%%%%%%%%%%%




\latexProblemContent{
\begin{problem}
Determine if the limit approaches a finite number, infinity, or does not exist. (If the limit does not exist, write DNE)
\input{./Limits/2311-Compute-Limit-0009.HELP.tex}
\[\lim_{x\to{3}}{5 \, {\left(x - 5\right)} \cos\left(-\frac{1}{2} \, \pi x\right)}=\answer{0}\]

\end{problem}}%}

%%%%%%%%%%%%%%%%%%%%%%




\latexProblemContent{
\begin{problem}
Determine if the limit approaches a finite number, infinity, or does not exist. (If the limit does not exist, write DNE)
\input{./Limits/2311-Compute-Limit-0009.HELP.tex}
\[\lim_{x\to{-2}}{-4 \, {\left(x + 4\right)} {\left(x - 4\right)} {\left(x - 5\right)} \tan\left(\frac{4}{3} \, \pi x\right)}=\answer{-112 \, \left(3 \, \sqrt{3}\right)}\]

\end{problem}}%}

%%%%%%%%%%%%%%%%%%%%%%




\latexProblemContent{
\begin{problem}
Determine if the limit approaches a finite number, infinity, or does not exist. (If the limit does not exist, write DNE)
\input{./Limits/2311-Compute-Limit-0009.HELP.tex}
\[\lim_{x\to{-4}}{2 \, {\left(x + 1\right)} {\left(x - 2\right)} \tan\left(-\pi x\right)}=\answer{0}\]

\end{problem}}%}

%%%%%%%%%%%%%%%%%%%%%%




\latexProblemContent{
\begin{problem}
Determine if the limit approaches a finite number, infinity, or does not exist. (If the limit does not exist, write DNE)
\input{./Limits/2311-Compute-Limit-0009.HELP.tex}
\[\lim_{x\to{-4}}{2 \, {\left(x + 4\right)} {\left(x - 2\right)}^{2} \sin\left(-2 \, \pi x\right)}=\answer{0}\]

\end{problem}}%}

%%%%%%%%%%%%%%%%%%%%%%




\latexProblemContent{
\begin{problem}
Determine if the limit approaches a finite number, infinity, or does not exist. (If the limit does not exist, write DNE)
\input{./Limits/2311-Compute-Limit-0009.HELP.tex}
\[\lim_{x\to{4}}{-4 \, {\left(x^{2} + 9 \, x + 20\right)} \tan\left(-\frac{5}{3} \, \pi x\right)}=\answer{-32 \, \left(9 \, \sqrt{3}\right)}\]

\end{problem}}%}

%%%%%%%%%%%%%%%%%%%%%%




\latexProblemContent{
\begin{problem}
Determine if the limit approaches a finite number, infinity, or does not exist. (If the limit does not exist, write DNE)
\input{./Limits/2311-Compute-Limit-0009.HELP.tex}
\[\lim_{x\to{-3}}{-5 \, {\left(x + 5\right)} {\left(x + 3\right)} \cos\left(-3 \, \pi x\right)}=\answer{0}\]

\end{problem}}%}

%%%%%%%%%%%%%%%%%%%%%%




\latexProblemContent{
\begin{problem}
Determine if the limit approaches a finite number, infinity, or does not exist. (If the limit does not exist, write DNE)
\input{./Limits/2311-Compute-Limit-0009.HELP.tex}
\[\lim_{x\to{3}}{-3 \, {\left(x + 3\right)} {\left(x + 2\right)} \tan\left(-2 \, \pi x\right)}=\answer{0}\]

\end{problem}}%}

%%%%%%%%%%%%%%%%%%%%%%




\latexProblemContent{
\begin{problem}
Determine if the limit approaches a finite number, infinity, or does not exist. (If the limit does not exist, write DNE)
\input{./Limits/2311-Compute-Limit-0009.HELP.tex}
\[\lim_{x\to{-4}}{-3 \, {\left(x^{2} + 5 \, x + 6\right)} \cos\left(-2 \, \pi x\right)}=\answer{-6}\]

\end{problem}}%}

%%%%%%%%%%%%%%%%%%%%%%




\latexProblemContent{
\begin{problem}
Determine if the limit approaches a finite number, infinity, or does not exist. (If the limit does not exist, write DNE)
\input{./Limits/2311-Compute-Limit-0009.HELP.tex}
\[\lim_{x\to{1}}{-5 \, {\left(x + 5\right)} {\left(x - 2\right)} \cos\left(\frac{2}{3} \, \pi x\right)}=\answer{-15}\]

\end{problem}}%}

%%%%%%%%%%%%%%%%%%%%%%




\latexProblemContent{
\begin{problem}
Determine if the limit approaches a finite number, infinity, or does not exist. (If the limit does not exist, write DNE)
\input{./Limits/2311-Compute-Limit-0009.HELP.tex}
\[\lim_{x\to{-1}}{-4 \, {\left(x^{2} - 16\right)} \tan\left(\frac{2}{3} \, \pi x\right)}=\answer{20 \, \left(3 \, \sqrt{3}\right)}\]

\end{problem}}%}

%%%%%%%%%%%%%%%%%%%%%%




\latexProblemContent{
\begin{problem}
Determine if the limit approaches a finite number, infinity, or does not exist. (If the limit does not exist, write DNE)
\input{./Limits/2311-Compute-Limit-0009.HELP.tex}
\[\lim_{x\to{-2}}{{\left(x - 1\right)} \tan\left(-\frac{1}{3} \, \pi x\right)}=\answer{3^{\frac{3}{2}}}\]

\end{problem}}%}

%%%%%%%%%%%%%%%%%%%%%%




\latexProblemContent{
\begin{problem}
Determine if the limit approaches a finite number, infinity, or does not exist. (If the limit does not exist, write DNE)
\input{./Limits/2311-Compute-Limit-0009.HELP.tex}
\[\lim_{x\to{-5}}{-{\left(x^{3} + 5 \, x^{2} - x - 5\right)} \sin\left(-\frac{5}{3} \, \pi x\right)}=\answer{0}\]

\end{problem}}%}

%%%%%%%%%%%%%%%%%%%%%%




\latexProblemContent{
\begin{problem}
Determine if the limit approaches a finite number, infinity, or does not exist. (If the limit does not exist, write DNE)
\input{./Limits/2311-Compute-Limit-0009.HELP.tex}
\[\lim_{x\to{2}}{-5 \, {\left(x^{2} + 8 \, x + 15\right)} \sin\left(-\pi x\right)}=\answer{0}\]

\end{problem}}%}

%%%%%%%%%%%%%%%%%%%%%%




\latexProblemContent{
\begin{problem}
Determine if the limit approaches a finite number, infinity, or does not exist. (If the limit does not exist, write DNE)
\input{./Limits/2311-Compute-Limit-0009.HELP.tex}
\[\lim_{x\to{-3}}{3 \, {\left(x^{3} - x^{2} - 5 \, x - 3\right)} \tan\left(-\pi x\right)}=\answer{0}\]

\end{problem}}%}

%%%%%%%%%%%%%%%%%%%%%%




\latexProblemContent{
\begin{problem}
Determine if the limit approaches a finite number, infinity, or does not exist. (If the limit does not exist, write DNE)
\input{./Limits/2311-Compute-Limit-0009.HELP.tex}
\[\lim_{x\to{2}}{-2 \, {\left(x + 2\right)} {\left(x - 3\right)} \cos\left(3 \, \pi x\right)}=\answer{8}\]

\end{problem}}%}

%%%%%%%%%%%%%%%%%%%%%%




\latexProblemContent{
\begin{problem}
Determine if the limit approaches a finite number, infinity, or does not exist. (If the limit does not exist, write DNE)
\input{./Limits/2311-Compute-Limit-0009.HELP.tex}
\[\lim_{x\to{-1}}{-4 \, {\left(x + 4\right)} {\left(x + 2\right)} {\left(x - 2\right)} \sin\left(-\frac{2}{3} \, \pi x\right)}=\answer{2 \, \left(9 \, \sqrt{3}\right)}\]

\end{problem}}%}

%%%%%%%%%%%%%%%%%%%%%%




\latexProblemContent{
\begin{problem}
Determine if the limit approaches a finite number, infinity, or does not exist. (If the limit does not exist, write DNE)
\input{./Limits/2311-Compute-Limit-0009.HELP.tex}
\[\lim_{x\to{-1}}{2 \, {\left(x^{2} - 3 \, x + 2\right)} \tan\left(\frac{1}{6} \, \pi x\right)}=\answer{-4 \, \sqrt{3}}\]

\end{problem}}%}

%%%%%%%%%%%%%%%%%%%%%%




\latexProblemContent{
\begin{problem}
Determine if the limit approaches a finite number, infinity, or does not exist. (If the limit does not exist, write DNE)
\input{./Limits/2311-Compute-Limit-0009.HELP.tex}
\[\lim_{x\to{-2}}{5 \, {\left(x + 1\right)} {\left(x - 1\right)} {\left(x - 5\right)} \sin\left(\pi x\right)}=\answer{0}\]

\end{problem}}%}

%%%%%%%%%%%%%%%%%%%%%%




\latexProblemContent{
\begin{problem}
Determine if the limit approaches a finite number, infinity, or does not exist. (If the limit does not exist, write DNE)
\input{./Limits/2311-Compute-Limit-0009.HELP.tex}
\[\lim_{x\to{5}}{-5 \, {\left(x + 5\right)} {\left(x - 3\right)} \tan\left(3 \, \pi x\right)}=\answer{0}\]

\end{problem}}%}

%%%%%%%%%%%%%%%%%%%%%%




\latexProblemContent{
\begin{problem}
Determine if the limit approaches a finite number, infinity, or does not exist. (If the limit does not exist, write DNE)
\input{./Limits/2311-Compute-Limit-0009.HELP.tex}
\[\lim_{x\to{-3}}{-2 \, {\left(x + 4\right)} {\left(x + 2\right)} \tan\left(-2 \, \pi x\right)}=\answer{0}\]

\end{problem}}%}

%%%%%%%%%%%%%%%%%%%%%%




\latexProblemContent{
\begin{problem}
Determine if the limit approaches a finite number, infinity, or does not exist. (If the limit does not exist, write DNE)
\input{./Limits/2311-Compute-Limit-0009.HELP.tex}
\[\lim_{x\to{5}}{3 \, {\left(x + 5\right)} {\left(x + 4\right)} {\left(x - 3\right)} \tan\left(-\frac{5}{6} \, \pi x\right)}=\answer{-20 \, \left(9 \, \sqrt{3}\right)}\]

\end{problem}}%}

%%%%%%%%%%%%%%%%%%%%%%




\latexProblemContent{
\begin{problem}
Determine if the limit approaches a finite number, infinity, or does not exist. (If the limit does not exist, write DNE)
\input{./Limits/2311-Compute-Limit-0009.HELP.tex}
\[\lim_{x\to{1}}{-4 \, {\left(x^{3} + 7 \, x^{2} + 14 \, x + 8\right)} \sin\left(-\frac{1}{3} \, \pi x\right)}=\answer{20 \, \left(3 \, \sqrt{3}\right)}\]

\end{problem}}%}

%%%%%%%%%%%%%%%%%%%%%%




\latexProblemContent{
\begin{problem}
Determine if the limit approaches a finite number, infinity, or does not exist. (If the limit does not exist, write DNE)
\input{./Limits/2311-Compute-Limit-0009.HELP.tex}
\[\lim_{x\to{3}}{-3 \, {\left(x^{3} + 6 \, x^{2} + 11 \, x + 6\right)} \tan\left(-\frac{1}{3} \, \pi x\right)}=\answer{0}\]

\end{problem}}%}

%%%%%%%%%%%%%%%%%%%%%%




\latexProblemContent{
\begin{problem}
Determine if the limit approaches a finite number, infinity, or does not exist. (If the limit does not exist, write DNE)
\input{./Limits/2311-Compute-Limit-0009.HELP.tex}
\[\lim_{x\to{1}}{-4 \, {\left(x + 4\right)} {\left(x - 3\right)} \sin\left(\pi x\right)}=\answer{0}\]

\end{problem}}%}

%%%%%%%%%%%%%%%%%%%%%%




\latexProblemContent{
\begin{problem}
Determine if the limit approaches a finite number, infinity, or does not exist. (If the limit does not exist, write DNE)
\input{./Limits/2311-Compute-Limit-0009.HELP.tex}
\[\lim_{x\to{-1}}{3 \, {\left(x - 3\right)} {\left(x - 5\right)} \sin\left(\frac{5}{6} \, \pi x\right)}=\answer{-36}\]

\end{problem}}%}

%%%%%%%%%%%%%%%%%%%%%%




\latexProblemContent{
\begin{problem}
Determine if the limit approaches a finite number, infinity, or does not exist. (If the limit does not exist, write DNE)
\input{./Limits/2311-Compute-Limit-0009.HELP.tex}
\[\lim_{x\to{2}}{-4 \, {\left(x^{3} + 4 \, x^{2} - x - 4\right)} \tan\left(-\frac{1}{2} \, \pi x\right)}=\answer{0}\]

\end{problem}}%}

%%%%%%%%%%%%%%%%%%%%%%




\latexProblemContent{
\begin{problem}
Determine if the limit approaches a finite number, infinity, or does not exist. (If the limit does not exist, write DNE)
\input{./Limits/2311-Compute-Limit-0009.HELP.tex}
\[\lim_{x\to{-3}}{2 \, {\left(x - 2\right)} \cos\left(-\frac{1}{6} \, \pi x\right)}=\answer{0}\]

\end{problem}}%}

%%%%%%%%%%%%%%%%%%%%%%




\latexProblemContent{
\begin{problem}
Determine if the limit approaches a finite number, infinity, or does not exist. (If the limit does not exist, write DNE)
\input{./Limits/2311-Compute-Limit-0009.HELP.tex}
\[\lim_{x\to{3}}{2 \, {\left(x - 1\right)} {\left(x - 2\right)} {\left(x - 4\right)} \tan\left(\frac{1}{3} \, \pi x\right)}=\answer{0}\]

\end{problem}}%}

%%%%%%%%%%%%%%%%%%%%%%




\latexProblemContent{
\begin{problem}
Determine if the limit approaches a finite number, infinity, or does not exist. (If the limit does not exist, write DNE)
\input{./Limits/2311-Compute-Limit-0009.HELP.tex}
\[\lim_{x\to{4}}{3 \, {\left(x^{2} - 7 \, x + 12\right)} \cos\left(2 \, \pi x\right)}=\answer{0}\]

\end{problem}}%}

%%%%%%%%%%%%%%%%%%%%%%




\latexProblemContent{
\begin{problem}
Determine if the limit approaches a finite number, infinity, or does not exist. (If the limit does not exist, write DNE)
\input{./Limits/2311-Compute-Limit-0009.HELP.tex}
\[\lim_{x\to{-4}}{-2 \, {\left(x + 2\right)} \sin\left(-\frac{5}{3} \, \pi x\right)}=\answer{2 \, \sqrt{3}}\]

\end{problem}}%}

%%%%%%%%%%%%%%%%%%%%%%




\latexProblemContent{
\begin{problem}
Determine if the limit approaches a finite number, infinity, or does not exist. (If the limit does not exist, write DNE)
\input{./Limits/2311-Compute-Limit-0009.HELP.tex}
\[\lim_{x\to{2}}{-{\left(x + 2\right)} {\left(x + 1\right)} \tan\left(-2 \, \pi x\right)}=\answer{0}\]

\end{problem}}%}

%%%%%%%%%%%%%%%%%%%%%%




\latexProblemContent{
\begin{problem}
Determine if the limit approaches a finite number, infinity, or does not exist. (If the limit does not exist, write DNE)
\input{./Limits/2311-Compute-Limit-0009.HELP.tex}
\[\lim_{x\to{-2}}{-4 \, {\left(x^{2} + 6 \, x + 8\right)} \cos\left(-\frac{1}{3} \, \pi x\right)}=\answer{0}\]

\end{problem}}%}

%%%%%%%%%%%%%%%%%%%%%%




\latexProblemContent{
\begin{problem}
Determine if the limit approaches a finite number, infinity, or does not exist. (If the limit does not exist, write DNE)
\input{./Limits/2311-Compute-Limit-0009.HELP.tex}
\[\lim_{x\to{4}}{-4 \, {\left(x^{2} + 6 \, x + 8\right)} \sin\left(-2 \, \pi x\right)}=\answer{0}\]

\end{problem}}%}

%%%%%%%%%%%%%%%%%%%%%%




\latexProblemContent{
\begin{problem}
Determine if the limit approaches a finite number, infinity, or does not exist. (If the limit does not exist, write DNE)
\input{./Limits/2311-Compute-Limit-0009.HELP.tex}
\[\lim_{x\to{-3}}{2 \, {\left(x^{3} - 2 \, x^{2} - 16 \, x + 32\right)} \tan\left(-\frac{2}{3} \, \pi x\right)}=\answer{0}\]

\end{problem}}%}

%%%%%%%%%%%%%%%%%%%%%%




\latexProblemContent{
\begin{problem}
Determine if the limit approaches a finite number, infinity, or does not exist. (If the limit does not exist, write DNE)
\input{./Limits/2311-Compute-Limit-0009.HELP.tex}
\[\lim_{x\to{3}}{-{\left(x + 1\right)} \sin\left(\frac{1}{6} \, \pi x\right)}=\answer{-4}\]

\end{problem}}%}

%%%%%%%%%%%%%%%%%%%%%%




\latexProblemContent{
\begin{problem}
Determine if the limit approaches a finite number, infinity, or does not exist. (If the limit does not exist, write DNE)
\input{./Limits/2311-Compute-Limit-0009.HELP.tex}
\[\lim_{x\to{2}}{-5 \, {\left(x + 5\right)} {\left(x - 2\right)} \sin\left(\frac{1}{3} \, \pi x\right)}=\answer{0}\]

\end{problem}}%}

%%%%%%%%%%%%%%%%%%%%%%




\latexProblemContent{
\begin{problem}
Determine if the limit approaches a finite number, infinity, or does not exist. (If the limit does not exist, write DNE)
\input{./Limits/2311-Compute-Limit-0009.HELP.tex}
\[\lim_{x\to{-4}}{{\left(x - 1\right)} {\left(x - 3\right)} \sin\left(3 \, \pi x\right)}=\answer{0}\]

\end{problem}}%}

%%%%%%%%%%%%%%%%%%%%%%




\latexProblemContent{
\begin{problem}
Determine if the limit approaches a finite number, infinity, or does not exist. (If the limit does not exist, write DNE)
\input{./Limits/2311-Compute-Limit-0009.HELP.tex}
\[\lim_{x\to{4}}{-2 \, {\left(x^{3} - 2 \, x^{2} - 13 \, x - 10\right)} \tan\left(\frac{5}{3} \, \pi x\right)}=\answer{-20 \, \left(3 \, \sqrt{3}\right)}\]

\end{problem}}%}

%%%%%%%%%%%%%%%%%%%%%%




\latexProblemContent{
\begin{problem}
Determine if the limit approaches a finite number, infinity, or does not exist. (If the limit does not exist, write DNE)
\input{./Limits/2311-Compute-Limit-0009.HELP.tex}
\[\lim_{x\to{-5}}{2 \, {\left(x^{2} + x - 6\right)} \sin\left(-\pi x\right)}=\answer{0}\]

\end{problem}}%}

%%%%%%%%%%%%%%%%%%%%%%




\latexProblemContent{
\begin{problem}
Determine if the limit approaches a finite number, infinity, or does not exist. (If the limit does not exist, write DNE)
\input{./Limits/2311-Compute-Limit-0009.HELP.tex}
\[\lim_{x\to{-3}}{3 \, {\left(x^{3} - 2 \, x^{2} - 23 \, x + 60\right)} \cos\left(-\frac{5}{3} \, \pi x\right)}=\answer{-252}\]

\end{problem}}%}

%%%%%%%%%%%%%%%%%%%%%%




\latexProblemContent{
\begin{problem}
Determine if the limit approaches a finite number, infinity, or does not exist. (If the limit does not exist, write DNE)
\input{./Limits/2311-Compute-Limit-0009.HELP.tex}
\[\lim_{x\to{4}}{{\left(x^{2} - 1\right)} \tan\left(-\frac{1}{3} \, \pi x\right)}=\answer{-5 \, \left(3 \, \sqrt{3}\right)}\]

\end{problem}}%}

%%%%%%%%%%%%%%%%%%%%%%




\latexProblemContent{
\begin{problem}
Determine if the limit approaches a finite number, infinity, or does not exist. (If the limit does not exist, write DNE)
\input{./Limits/2311-Compute-Limit-0009.HELP.tex}
\[\lim_{x\to{-5}}{-2 \, {\left(x + 2\right)} {\left(x - 4\right)} \cos\left(4 \, \pi x\right)}=\answer{-54}\]

\end{problem}}%}

%%%%%%%%%%%%%%%%%%%%%%




\latexProblemContent{
\begin{problem}
Determine if the limit approaches a finite number, infinity, or does not exist. (If the limit does not exist, write DNE)
\input{./Limits/2311-Compute-Limit-0009.HELP.tex}
\[\lim_{x\to{-1}}{{\left(x + 4\right)} {\left(x + 3\right)} {\left(x - 1\right)} \tan\left(-\pi x\right)}=\answer{0}\]

\end{problem}}%}

%%%%%%%%%%%%%%%%%%%%%%




\latexProblemContent{
\begin{problem}
Determine if the limit approaches a finite number, infinity, or does not exist. (If the limit does not exist, write DNE)
\input{./Limits/2311-Compute-Limit-0009.HELP.tex}
\[\lim_{x\to{-1}}{5 \, {\left(x + 3\right)} {\left(x + 1\right)} {\left(x - 5\right)} \sin\left(-\frac{1}{6} \, \pi x\right)}=\answer{0}\]

\end{problem}}%}

%%%%%%%%%%%%%%%%%%%%%%




\latexProblemContent{
\begin{problem}
Determine if the limit approaches a finite number, infinity, or does not exist. (If the limit does not exist, write DNE)
\input{./Limits/2311-Compute-Limit-0009.HELP.tex}
\[\lim_{x\to{-5}}{-2 \, {\left(x + 2\right)}^{2} {\left(x - 5\right)} \sin\left(\frac{5}{2} \, \pi x\right)}=\answer{-180}\]

\end{problem}}%}

%%%%%%%%%%%%%%%%%%%%%%



