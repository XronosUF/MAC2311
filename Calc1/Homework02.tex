\documentclass[•]{ximera}

%\usepackage{amsmath}

\title{Homework 2}
\author{people}

\begin{abstract}
Location here
\end{abstract}

\usepackage{tikz}
\usepackage{ifthen}

\newcommand{\xyplane}[2]{
\draw[->, style=thick] (-#1,0)--(#1+0.3,0);
\draw[->, style=thick] (0,-#2)--(0,#2+0.3);
\foreach \x in {-#1,...,#1}{\ifthenelse{\x=0}{}{
    {\draw (\x ,-0.1)--(\x ,0.1);
     \draw (\x,0) node[below, font=\footnotesize] {$\x$};
    }}}
\foreach \y in {-#2,...,#2}{\ifthenelse{\y=0}{}{
    {\draw (-0.1,\y)--(0.1,\y);
     \draw (0,\y) node[left, font=\footnotesize] {$\y$};    
    }}}
}



\begin{document}
\maketitle

%\shuffletrue
%%%%%%%%%%%%%%%%%%%%%%%
%%\tagged{Cat@One, Cat@Two, Cat@Three, Cat@Four, Cat@Five, Ans@FRQ, Type@Compute, Topic@Limit, Sub@Rational, Sub@Continuity}{
\begin{shuffle}%{2}
\begin{exercise}

Use the function to answer the following questions.
\[f(x)=\left\{\begin{array}{ll}{{\sqrt{x^{2} - 16} + 8}}&,x<{{4}}\\[5pt]
{{\frac{x^{2} - 16}{x + 4}}}&,x\geq{{4}}
\end{array}\right.\]
\begin{shuffle}%{4}
\begin{problem}
Compute $\lim\limits_{x\to{{4}}^-}f(x)=\answer{{{8}}}$\\[1in]
\end{problem}
\begin{problem}
Compute $\lim\limits_{x\to{{4}}^+}f(x)=\answer{{{0}}}$\\[1in]
\end{problem}
\begin{problem}
Compute $f({{4}})=\answer{{{0}}}$\\[1in]
\end{problem}
\begin{problem}
The function is ...
\begin{multipleChoice}
\choice{continuous at $x={{4}}$.}
\choice[correct]{discontinuous at $x={{4}}$.}
\end{multipleChoice}
\end{problem}
\end{shuffle}
\end{exercise}%}
%%%%%%%%%%%%%%%%%%%%%%




%%%%%%%%%%%%%%%%%%%%%%%
%%\tagged{Cat@One, Cat@Two, Cat@Three, Cat@Four, Cat@Five, Ans@ShortAns, Type@Compute, Topic@Limit, Sub@Poly, Sub@Trig, Sub@Continuity}{

\begin{exercise}
Determine if the limit approaches a finite number, infinity, or does not exist. (If the limit does not exist, write DNE)
\[\lim_{x\to{{1}}}{{-2 \, {\left(x^{3} - 3 \, x^{2} - 4 \, x + 12\right)} \sin\left(\frac{2}{3} \, \pi x\right)}}=\answer{{{\left(-6 \, \sqrt{3}\right)}}}\]

\end{exercise}%}
%%%%%%%%%%%%%%%%%%%%%%
\end{shuffle}


Limit definition of derivative:\\
\geogebra{tHTMSYDj}{906}{552}

\vspace{0.5in}

Generating Sinusoids\\
\desmos{absl7xylcv}{}{}

\vspace{1in}

%Include Tikz Pictures:\\
%\includegraphics{Homework02-1.svg}
%
%\begin{image}
%\begin{tikzpicture}
%\xyplane{5}{5}
%\draw[thick, color=blue, smooth, samples=200, domain=-4:2, <->] plot (\x, {\x*\x+2*\x-3});
%\draw[thick, color=orange, smooth, samples=200, domain=-3:3, <->] plot (\x, \x-1);
%\end{tikzpicture}
%\end{image}











\end{document}
