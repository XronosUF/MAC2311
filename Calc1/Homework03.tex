\documentclass[handout]{ximera}

%\usepackage{amsmath}
\usepackage{PackageLoader}
\title{Homework 1}
\author{people}

\begin{abstract}
Location here
\end{abstract}

\begin{document}
\maketitle

\begin{shuffle}
%%%%%%%%%%%%%%%%%%%%%%
%\tagged{Cat@One, Cat@Two, Cat@Three, Cat@Four, Cat@Five, Ans@ShortAns, Type@Compute, Topic@Limit, Sub@Rational}{

\begin{problem}
WHATATAKKNKLASJDLKSHAKLDHASLDHLAHSFD  This Worked????

\answer{Oh YEAH!!!}
\end{problem}%}
%%%%%%%%%%%%%%%%%%%%%%



%%%%%%%%%%%%%%%%%%%%%%
%\tagged{Cat@One, Cat@Two, Cat@Three, Cat@Four, Cat@Five, Ans@ShortAns, Type@Compute, Topic@Limit, Sub@Rational}{

\begin{problem}
Determine if the limit approaches a finite number, infinity, or does not exist. (If the limit does not exist, write DNE)
\[\lim_{x\to{{-2}}}\dfrac{{{x^{2} + 3 \, x + 2}}}{{{x + 1}}}=\answer[tolerance=0.1]{0}\]
\end{problem}%}
%%%%%%%%%%%%%%%%%%%%%


%%%%%%%%%%%%%%%%%%%%%%
%\tagged{Cat@One, Cat@Two, Cat@Three, Cat@Four, Cat@Five, Ans@ShortAns, Type@Compute, Topic@Limit, Sub@Rational}{

\begin{problem}
Determine if the limit approaches a finite number, infinity, or does not exist. (If the limit does not exist, write DNE)  \vspace{5pt}
\begin{hint}
For this problem, consider factoring the numerator and denominator.  Then simplify the fraction to make computing the limit easier.  For more information, check out the following video:
\end{hint}
\begin{hint}
\youtube{GGQngIp0YGI}\\
The above is below\\
{link: \url{https://mediasite.video.ufl.edu/Mediasite/Play/a91d316017564ce09dd4392b3d4cc2031d}}
%Link: https://mediasite.video.ufl.edu/Mediasite/Play/a91d316017564ce09dd4392b3d4cc2031d
%\begin{html}
%https://mediasite.video.ufl.edu/Mediasite/Play/a91d316017564ce09dd4392b3d4cc2031d
%\end{html}
%\begin{html}
%<iframe width="320" height="240" frameborder="0" scrolling="auto" marginheight="0" marginwidth="0" src="https://mediasite.video.ufl.edu/Mediasite/Play/a91d316017564ce09dd4392b3d4cc2031d"></iframe>
%\end{html}
\end{hint}
%\end{expandable}


\[\lim_{x\to{{-3}}}\dfrac{{{x^{2} + x - 6}}}{{{x^{2} + 5 \, x + 6}}}=\answer{{{5}}}\]
\end{problem}%}

\end{shuffle}
%%%%%%%%%%%%%%%%%%%%%



%%%%%%%%%%%%%%%%%%%%%5
%%%%%%%%%%%%%%%%%%%%%%

\pgfmathrandominteger{\bob}{2}{42}
\bob

BLAHHH






\end{document}
