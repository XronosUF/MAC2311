%%%%%%%%%%%%%%%%%%%%%%%
%%\tagged{Cat@One, Cat@Two, Cat@Three, Cat@Four, Cat@Five, Ans@ShortAns, Type@Compute, Topic@Integral, Sub@Definite, Sub@Theorems_FTC, Sub@Trig}{

\latexProblemContent{
\begin{problem}

Use the Fundamental Theorem of Calculus to evaluate the integral.

\expandafter\input{\file@loc Integrals/2311-Compute-Integral-0012.HELP.tex}

\[\int_{\frac{2}{3} \, \pi}^{\frac{7}{4} \, \pi} {-10 \, \sin\left(x\right)}\;dx=\answer{5 \, \sqrt{2} + 5}\]
\end{problem}}%}

%%%%%%%%%%%%%%%%%%%%%%


\latexProblemContent{
\begin{problem}

Use the Fundamental Theorem of Calculus to evaluate the integral.

\expandafter\input{\file@loc Integrals/2311-Compute-Integral-0012.HELP.tex}

\[\int_{\frac{2}{3} \, \pi}^{\frac{5}{3} \, \pi} {5 \, \sin\left(x\right)}\;dx=\answer{-5}\]
\end{problem}}%}

%%%%%%%%%%%%%%%%%%%%%%


\latexProblemContent{
\begin{problem}

Use the Fundamental Theorem of Calculus to evaluate the integral.

\expandafter\input{\file@loc Integrals/2311-Compute-Integral-0012.HELP.tex}

\[\int_{\frac{3}{4} \, \pi}^{\frac{5}{6} \, \pi} {-10 \, \sin\left(x\right)}\;dx=\answer{-5 \, \sqrt{3} + 5 \, \sqrt{2}}\]
\end{problem}}%}

%%%%%%%%%%%%%%%%%%%%%%


\latexProblemContent{
\begin{problem}

Use the Fundamental Theorem of Calculus to evaluate the integral.

\expandafter\input{\file@loc Integrals/2311-Compute-Integral-0012.HELP.tex}

\[\int_{\frac{2}{3} \, \pi}^{\frac{3}{2} \, \pi} {2 \, \sin\left(x\right)}\;dx=\answer{-1}\]
\end{problem}}%}

%%%%%%%%%%%%%%%%%%%%%%


\latexProblemContent{
\begin{problem}

Use the Fundamental Theorem of Calculus to evaluate the integral.

\expandafter\input{\file@loc Integrals/2311-Compute-Integral-0012.HELP.tex}

\[\int_{\frac{1}{3} \, \pi}^{\pi} {9 \, \sin\left(x\right)}\;dx=\answer{\frac{27}{2}}\]
\end{problem}}%}

%%%%%%%%%%%%%%%%%%%%%%


\latexProblemContent{
\begin{problem}

Use the Fundamental Theorem of Calculus to evaluate the integral.

\expandafter\input{\file@loc Integrals/2311-Compute-Integral-0012.HELP.tex}

\[\int_{\frac{2}{3} \, \pi}^{\frac{3}{2} \, \pi} {8 \, \cos\left(x\right)}\;dx=\answer{-4 \, \sqrt{3} - 8}\]
\end{problem}}%}

%%%%%%%%%%%%%%%%%%%%%%


\latexProblemContent{
\begin{problem}

Use the Fundamental Theorem of Calculus to evaluate the integral.

\expandafter\input{\file@loc Integrals/2311-Compute-Integral-0012.HELP.tex}

\[\int_{\frac{1}{2} \, \pi}^{\frac{4}{3} \, \pi} {-10 \, \sin\left(x\right)}\;dx=\answer{-5}\]
\end{problem}}%}

%%%%%%%%%%%%%%%%%%%%%%


\latexProblemContent{
\begin{problem}

Use the Fundamental Theorem of Calculus to evaluate the integral.

\expandafter\input{\file@loc Integrals/2311-Compute-Integral-0012.HELP.tex}

\[\int_{\frac{1}{2} \, \pi}^{\frac{5}{4} \, \pi} {-2 \, \cos\left(x\right)}\;dx=\answer{\sqrt{2} + 2}\]
\end{problem}}%}

%%%%%%%%%%%%%%%%%%%%%%


\latexProblemContent{
\begin{problem}

Use the Fundamental Theorem of Calculus to evaluate the integral.

\expandafter\input{\file@loc Integrals/2311-Compute-Integral-0012.HELP.tex}

\[\int_{\frac{1}{4} \, \pi}^{\frac{2}{3} \, \pi} {-9 \, \sin\left(x\right)}\;dx=\answer{-\frac{9}{2} \, \sqrt{2} - \frac{9}{2}}\]
\end{problem}}%}

%%%%%%%%%%%%%%%%%%%%%%


\latexProblemContent{
\begin{problem}

Use the Fundamental Theorem of Calculus to evaluate the integral.

\expandafter\input{\file@loc Integrals/2311-Compute-Integral-0012.HELP.tex}

\[\int_{\frac{1}{6} \, \pi}^{\frac{7}{6} \, \pi} {-\sin\left(x\right)}\;dx=\answer{-\sqrt{3}}\]
\end{problem}}%}

%%%%%%%%%%%%%%%%%%%%%%


\latexProblemContent{
\begin{problem}

Use the Fundamental Theorem of Calculus to evaluate the integral.

\expandafter\input{\file@loc Integrals/2311-Compute-Integral-0012.HELP.tex}

\[\int_{\frac{5}{6} \, \pi}^{\frac{3}{2} \, \pi} {-2 \, \sin\left(x\right)}\;dx=\answer{\sqrt{3}}\]
\end{problem}}%}

%%%%%%%%%%%%%%%%%%%%%%


\latexProblemContent{
\begin{problem}

Use the Fundamental Theorem of Calculus to evaluate the integral.

\expandafter\input{\file@loc Integrals/2311-Compute-Integral-0012.HELP.tex}

\[\int_{\frac{3}{4} \, \pi}^{\frac{7}{6} \, \pi} {5 \, \sin\left(x\right)}\;dx=\answer{\frac{5}{2} \, \sqrt{3} - \frac{5}{2} \, \sqrt{2}}\]
\end{problem}}%}

%%%%%%%%%%%%%%%%%%%%%%


\latexProblemContent{
\begin{problem}

Use the Fundamental Theorem of Calculus to evaluate the integral.

\expandafter\input{\file@loc Integrals/2311-Compute-Integral-0012.HELP.tex}

\[\int_{\frac{1}{6} \, \pi}^{\frac{3}{4} \, \pi} {-9 \, \cos\left(x\right)}\;dx=\answer{-\frac{9}{2} \, \sqrt{2} + \frac{9}{2}}\]
\end{problem}}%}

%%%%%%%%%%%%%%%%%%%%%%


\latexProblemContent{
\begin{problem}

Use the Fundamental Theorem of Calculus to evaluate the integral.

\expandafter\input{\file@loc Integrals/2311-Compute-Integral-0012.HELP.tex}

\[\int_{\frac{2}{3} \, \pi}^{\frac{11}{6} \, \pi} {-2 \, \sin\left(x\right)}\;dx=\answer{\sqrt{3} + 1}\]
\end{problem}}%}

%%%%%%%%%%%%%%%%%%%%%%


\latexProblemContent{
\begin{problem}

Use the Fundamental Theorem of Calculus to evaluate the integral.

\expandafter\input{\file@loc Integrals/2311-Compute-Integral-0012.HELP.tex}

\[\int_{\frac{2}{3} \, \pi}^{\frac{4}{3} \, \pi} {-4 \, \cos\left(x\right)}\;dx=\answer{4 \, \sqrt{3}}\]
\end{problem}}%}

%%%%%%%%%%%%%%%%%%%%%%


\latexProblemContent{
\begin{problem}

Use the Fundamental Theorem of Calculus to evaluate the integral.

\expandafter\input{\file@loc Integrals/2311-Compute-Integral-0012.HELP.tex}

\[\int_{\frac{1}{3} \, \pi}^{\frac{4}{3} \, \pi} {-8 \, \sin\left(x\right)}\;dx=\answer{-8}\]
\end{problem}}%}

%%%%%%%%%%%%%%%%%%%%%%


\latexProblemContent{
\begin{problem}

Use the Fundamental Theorem of Calculus to evaluate the integral.

\expandafter\input{\file@loc Integrals/2311-Compute-Integral-0012.HELP.tex}

\[\int_{\frac{2}{3} \, \pi}^{\frac{3}{2} \, \pi} {3 \, \cos\left(x\right)}\;dx=\answer{-\frac{3}{2} \, \sqrt{3} - 3}\]
\end{problem}}%}

%%%%%%%%%%%%%%%%%%%%%%


\latexProblemContent{
\begin{problem}

Use the Fundamental Theorem of Calculus to evaluate the integral.

\expandafter\input{\file@loc Integrals/2311-Compute-Integral-0012.HELP.tex}

\[\int_{\frac{2}{3} \, \pi}^{\frac{7}{6} \, \pi} {3 \, \cos\left(x\right)}\;dx=\answer{-\frac{3}{2} \, \sqrt{3} - \frac{3}{2}}\]
\end{problem}}%}

%%%%%%%%%%%%%%%%%%%%%%


\latexProblemContent{
\begin{problem}

Use the Fundamental Theorem of Calculus to evaluate the integral.

\expandafter\input{\file@loc Integrals/2311-Compute-Integral-0012.HELP.tex}

\[\int_{\frac{1}{6} \, \pi}^{\frac{3}{2} \, \pi} {-10 \, \cos\left(x\right)}\;dx=\answer{15}\]
\end{problem}}%}

%%%%%%%%%%%%%%%%%%%%%%


\latexProblemContent{
\begin{problem}

Use the Fundamental Theorem of Calculus to evaluate the integral.

\expandafter\input{\file@loc Integrals/2311-Compute-Integral-0012.HELP.tex}

\[\int_{\frac{5}{6} \, \pi}^{\frac{3}{2} \, \pi} {2 \, \cos\left(x\right)}\;dx=\answer{-3}\]
\end{problem}}%}

%%%%%%%%%%%%%%%%%%%%%%


\latexProblemContent{
\begin{problem}

Use the Fundamental Theorem of Calculus to evaluate the integral.

\expandafter\input{\file@loc Integrals/2311-Compute-Integral-0012.HELP.tex}

\[\int_{\frac{1}{3} \, \pi}^{\frac{5}{3} \, \pi} {-3 \, \sin\left(x\right)}\;dx=\answer{0}\]
\end{problem}}%}

%%%%%%%%%%%%%%%%%%%%%%


\latexProblemContent{
\begin{problem}

Use the Fundamental Theorem of Calculus to evaluate the integral.

\expandafter\input{\file@loc Integrals/2311-Compute-Integral-0012.HELP.tex}

\[\int_{\frac{5}{6} \, \pi}^{\frac{11}{6} \, \pi} {-8 \, \cos\left(x\right)}\;dx=\answer{8}\]
\end{problem}}%}

%%%%%%%%%%%%%%%%%%%%%%


\latexProblemContent{
\begin{problem}

Use the Fundamental Theorem of Calculus to evaluate the integral.

\expandafter\input{\file@loc Integrals/2311-Compute-Integral-0012.HELP.tex}

\[\int_{\frac{5}{6} \, \pi}^{\frac{11}{6} \, \pi} {6 \, \cos\left(x\right)}\;dx=\answer{-6}\]
\end{problem}}%}

%%%%%%%%%%%%%%%%%%%%%%


\latexProblemContent{
\begin{problem}

Use the Fundamental Theorem of Calculus to evaluate the integral.

\expandafter\input{\file@loc Integrals/2311-Compute-Integral-0012.HELP.tex}

\[\int_{\frac{1}{6} \, \pi}^{\frac{1}{2} \, \pi} {-8 \, \sin\left(x\right)}\;dx=\answer{-4 \, \sqrt{3}}\]
\end{problem}}%}

%%%%%%%%%%%%%%%%%%%%%%


\latexProblemContent{
\begin{problem}

Use the Fundamental Theorem of Calculus to evaluate the integral.

\expandafter\input{\file@loc Integrals/2311-Compute-Integral-0012.HELP.tex}

\[\int_{\frac{1}{6} \, \pi}^{\frac{5}{6} \, \pi} {10 \, \sin\left(x\right)}\;dx=\answer{10 \, \sqrt{3}}\]
\end{problem}}%}

%%%%%%%%%%%%%%%%%%%%%%


\latexProblemContent{
\begin{problem}

Use the Fundamental Theorem of Calculus to evaluate the integral.

\expandafter\input{\file@loc Integrals/2311-Compute-Integral-0012.HELP.tex}

\[\int_{\frac{1}{3} \, \pi}^{\pi} {\cos\left(x\right)}\;dx=\answer{-\frac{1}{2} \, \sqrt{3}}\]
\end{problem}}%}

%%%%%%%%%%%%%%%%%%%%%%


\latexProblemContent{
\begin{problem}

Use the Fundamental Theorem of Calculus to evaluate the integral.

\expandafter\input{\file@loc Integrals/2311-Compute-Integral-0012.HELP.tex}

\[\int_{\frac{5}{6} \, \pi}^{\frac{5}{4} \, \pi} {-3 \, \cos\left(x\right)}\;dx=\answer{\frac{3}{2} \, \sqrt{2} + \frac{3}{2}}\]
\end{problem}}%}

%%%%%%%%%%%%%%%%%%%%%%


\latexProblemContent{
\begin{problem}

Use the Fundamental Theorem of Calculus to evaluate the integral.

\expandafter\input{\file@loc Integrals/2311-Compute-Integral-0012.HELP.tex}

\[\int_{\frac{3}{4} \, \pi}^{\frac{5}{4} \, \pi} {6 \, \cos\left(x\right)}\;dx=\answer{-3 \, \left(2 \, \sqrt{2}\right)}\]
\end{problem}}%}

%%%%%%%%%%%%%%%%%%%%%%


\latexProblemContent{
\begin{problem}

Use the Fundamental Theorem of Calculus to evaluate the integral.

\expandafter\input{\file@loc Integrals/2311-Compute-Integral-0012.HELP.tex}

\[\int_{\frac{2}{3} \, \pi}^{\frac{5}{4} \, \pi} {-9 \, \cos\left(x\right)}\;dx=\answer{\frac{9}{2} \, \sqrt{3} + \frac{9}{2} \, \sqrt{2}}\]
\end{problem}}%}

%%%%%%%%%%%%%%%%%%%%%%


\latexProblemContent{
\begin{problem}

Use the Fundamental Theorem of Calculus to evaluate the integral.

\expandafter\input{\file@loc Integrals/2311-Compute-Integral-0012.HELP.tex}

\[\int_{\frac{1}{4} \, \pi}^{\frac{3}{4} \, \pi} {9 \, \cos\left(x\right)}\;dx=\answer{0}\]
\end{problem}}%}

%%%%%%%%%%%%%%%%%%%%%%


\latexProblemContent{
\begin{problem}

Use the Fundamental Theorem of Calculus to evaluate the integral.

\expandafter\input{\file@loc Integrals/2311-Compute-Integral-0012.HELP.tex}

\[\int_{\frac{1}{2} \, \pi}^{\frac{5}{4} \, \pi} {-5 \, \cos\left(x\right)}\;dx=\answer{\frac{5}{2} \, \sqrt{2} + 5}\]
\end{problem}}%}

%%%%%%%%%%%%%%%%%%%%%%


\latexProblemContent{
\begin{problem}

Use the Fundamental Theorem of Calculus to evaluate the integral.

\expandafter\input{\file@loc Integrals/2311-Compute-Integral-0012.HELP.tex}

\[\int_{\frac{1}{6} \, \pi}^{\frac{4}{3} \, \pi} {-8 \, \sin\left(x\right)}\;dx=\answer{-4 \, \sqrt{3} - 4}\]
\end{problem}}%}

%%%%%%%%%%%%%%%%%%%%%%


\latexProblemContent{
\begin{problem}

Use the Fundamental Theorem of Calculus to evaluate the integral.

\expandafter\input{\file@loc Integrals/2311-Compute-Integral-0012.HELP.tex}

\[\int_{\frac{1}{3} \, \pi}^{\frac{7}{4} \, \pi} {10 \, \sin\left(x\right)}\;dx=\answer{-5 \, \sqrt{2} + 5}\]
\end{problem}}%}

%%%%%%%%%%%%%%%%%%%%%%


\latexProblemContent{
\begin{problem}

Use the Fundamental Theorem of Calculus to evaluate the integral.

\expandafter\input{\file@loc Integrals/2311-Compute-Integral-0012.HELP.tex}

\[\int_{\frac{1}{2} \, \pi}^{\frac{3}{2} \, \pi} {-6 \, \sin\left(x\right)}\;dx=\answer{0}\]
\end{problem}}%}

%%%%%%%%%%%%%%%%%%%%%%


\latexProblemContent{
\begin{problem}

Use the Fundamental Theorem of Calculus to evaluate the integral.

\expandafter\input{\file@loc Integrals/2311-Compute-Integral-0012.HELP.tex}

\[\int_{\frac{2}{3} \, \pi}^{\frac{5}{6} \, \pi} {-7 \, \cos\left(x\right)}\;dx=\answer{\frac{7}{2} \, \sqrt{3} - \frac{7}{2}}\]
\end{problem}}%}

%%%%%%%%%%%%%%%%%%%%%%


\latexProblemContent{
\begin{problem}

Use the Fundamental Theorem of Calculus to evaluate the integral.

\expandafter\input{\file@loc Integrals/2311-Compute-Integral-0012.HELP.tex}

\[\int_{\frac{1}{6} \, \pi}^{\frac{11}{6} \, \pi} {6 \, \sin\left(x\right)}\;dx=\answer{0}\]
\end{problem}}%}

%%%%%%%%%%%%%%%%%%%%%%


\latexProblemContent{
\begin{problem}

Use the Fundamental Theorem of Calculus to evaluate the integral.

\expandafter\input{\file@loc Integrals/2311-Compute-Integral-0012.HELP.tex}

\[\int_{\frac{3}{4} \, \pi}^{\frac{5}{3} \, \pi} {-\cos\left(x\right)}\;dx=\answer{\frac{1}{2} \, \sqrt{3} + \frac{1}{2} \, \sqrt{2}}\]
\end{problem}}%}

%%%%%%%%%%%%%%%%%%%%%%


\latexProblemContent{
\begin{problem}

Use the Fundamental Theorem of Calculus to evaluate the integral.

\expandafter\input{\file@loc Integrals/2311-Compute-Integral-0012.HELP.tex}

\[\int_{\frac{5}{6} \, \pi}^{\frac{5}{3} \, \pi} {-2 \, \cos\left(x\right)}\;dx=\answer{\sqrt{3} + 1}\]
\end{problem}}%}

%%%%%%%%%%%%%%%%%%%%%%


\latexProblemContent{
\begin{problem}

Use the Fundamental Theorem of Calculus to evaluate the integral.

\expandafter\input{\file@loc Integrals/2311-Compute-Integral-0012.HELP.tex}

\[\int_{\frac{5}{6} \, \pi}^{\frac{4}{3} \, \pi} {4 \, \sin\left(x\right)}\;dx=\answer{-2 \, \sqrt{3} + 2}\]
\end{problem}}%}

%%%%%%%%%%%%%%%%%%%%%%


\latexProblemContent{
\begin{problem}

Use the Fundamental Theorem of Calculus to evaluate the integral.

\expandafter\input{\file@loc Integrals/2311-Compute-Integral-0012.HELP.tex}

\[\int_{\frac{1}{2} \, \pi}^{\frac{11}{6} \, \pi} {2 \, \sin\left(x\right)}\;dx=\answer{-\sqrt{3}}\]
\end{problem}}%}

%%%%%%%%%%%%%%%%%%%%%%


\latexProblemContent{
\begin{problem}

Use the Fundamental Theorem of Calculus to evaluate the integral.

\expandafter\input{\file@loc Integrals/2311-Compute-Integral-0012.HELP.tex}

\[\int_{\frac{1}{6} \, \pi}^{\frac{7}{6} \, \pi} {9 \, \sin\left(x\right)}\;dx=\answer{3^{\frac{5}{2}}}\]
\end{problem}}%}

%%%%%%%%%%%%%%%%%%%%%%


\latexProblemContent{
\begin{problem}

Use the Fundamental Theorem of Calculus to evaluate the integral.

\expandafter\input{\file@loc Integrals/2311-Compute-Integral-0012.HELP.tex}

\[\int_{\frac{1}{3} \, \pi}^{\frac{7}{4} \, \pi} {9 \, \cos\left(x\right)}\;dx=\answer{-\frac{9}{2} \, \sqrt{3} - \frac{9}{2} \, \sqrt{2}}\]
\end{problem}}%}

%%%%%%%%%%%%%%%%%%%%%%


\latexProblemContent{
\begin{problem}

Use the Fundamental Theorem of Calculus to evaluate the integral.

\expandafter\input{\file@loc Integrals/2311-Compute-Integral-0012.HELP.tex}

\[\int_{\frac{3}{4} \, \pi}^{\frac{5}{6} \, \pi} {-3 \, \sin\left(x\right)}\;dx=\answer{-\frac{3}{2} \, \sqrt{3} + \frac{3}{2} \, \sqrt{2}}\]
\end{problem}}%}

%%%%%%%%%%%%%%%%%%%%%%


\latexProblemContent{
\begin{problem}

Use the Fundamental Theorem of Calculus to evaluate the integral.

\expandafter\input{\file@loc Integrals/2311-Compute-Integral-0012.HELP.tex}

\[\int_{\frac{1}{6} \, \pi}^{\frac{4}{3} \, \pi} {-\sin\left(x\right)}\;dx=\answer{-\frac{1}{2} \, \sqrt{3} - \frac{1}{2}}\]
\end{problem}}%}

%%%%%%%%%%%%%%%%%%%%%%


\latexProblemContent{
\begin{problem}

Use the Fundamental Theorem of Calculus to evaluate the integral.

\expandafter\input{\file@loc Integrals/2311-Compute-Integral-0012.HELP.tex}

\[\int_{\frac{1}{6} \, \pi}^{\frac{4}{3} \, \pi} {4 \, \cos\left(x\right)}\;dx=\answer{-2 \, \sqrt{3} - 2}\]
\end{problem}}%}

%%%%%%%%%%%%%%%%%%%%%%


\latexProblemContent{
\begin{problem}

Use the Fundamental Theorem of Calculus to evaluate the integral.

\expandafter\input{\file@loc Integrals/2311-Compute-Integral-0012.HELP.tex}

\[\int_{\frac{2}{3} \, \pi}^{\frac{5}{6} \, \pi} {10 \, \cos\left(x\right)}\;dx=\answer{-5 \, \sqrt{3} + 5}\]
\end{problem}}%}

%%%%%%%%%%%%%%%%%%%%%%


\latexProblemContent{
\begin{problem}

Use the Fundamental Theorem of Calculus to evaluate the integral.

\expandafter\input{\file@loc Integrals/2311-Compute-Integral-0012.HELP.tex}

\[\int_{\frac{1}{3} \, \pi}^{\frac{2}{3} \, \pi} {4 \, \sin\left(x\right)}\;dx=\answer{4}\]
\end{problem}}%}

%%%%%%%%%%%%%%%%%%%%%%


\latexProblemContent{
\begin{problem}

Use the Fundamental Theorem of Calculus to evaluate the integral.

\expandafter\input{\file@loc Integrals/2311-Compute-Integral-0012.HELP.tex}

\[\int_{\frac{1}{6} \, \pi}^{\frac{5}{4} \, \pi} {-9 \, \cos\left(x\right)}\;dx=\answer{\frac{9}{2} \, \sqrt{2} + \frac{9}{2}}\]
\end{problem}}%}

%%%%%%%%%%%%%%%%%%%%%%


\latexProblemContent{
\begin{problem}

Use the Fundamental Theorem of Calculus to evaluate the integral.

\expandafter\input{\file@loc Integrals/2311-Compute-Integral-0012.HELP.tex}

\[\int_{\frac{1}{6} \, \pi}^{\frac{2}{3} \, \pi} {-4 \, \sin\left(x\right)}\;dx=\answer{-2 \, \sqrt{3} - 2}\]
\end{problem}}%}

%%%%%%%%%%%%%%%%%%%%%%


\latexProblemContent{
\begin{problem}

Use the Fundamental Theorem of Calculus to evaluate the integral.

\expandafter\input{\file@loc Integrals/2311-Compute-Integral-0012.HELP.tex}

\[\int_{\frac{1}{3} \, \pi}^{\frac{11}{6} \, \pi} {-6 \, \cos\left(x\right)}\;dx=\answer{3 \, \sqrt{3} + 3}\]
\end{problem}}%}

%%%%%%%%%%%%%%%%%%%%%%


\latexProblemContent{
\begin{problem}

Use the Fundamental Theorem of Calculus to evaluate the integral.

\expandafter\input{\file@loc Integrals/2311-Compute-Integral-0012.HELP.tex}

\[\int_{\frac{1}{6} \, \pi}^{\frac{7}{4} \, \pi} {-2 \, \cos\left(x\right)}\;dx=\answer{\sqrt{2} + 1}\]
\end{problem}}%}

%%%%%%%%%%%%%%%%%%%%%%


\latexProblemContent{
\begin{problem}

Use the Fundamental Theorem of Calculus to evaluate the integral.

\expandafter\input{\file@loc Integrals/2311-Compute-Integral-0012.HELP.tex}

\[\int_{\frac{1}{4} \, \pi}^{\pi} {\cos\left(x\right)}\;dx=\answer{-\frac{1}{2} \, \sqrt{2}}\]
\end{problem}}%}

%%%%%%%%%%%%%%%%%%%%%%


\latexProblemContent{
\begin{problem}

Use the Fundamental Theorem of Calculus to evaluate the integral.

\expandafter\input{\file@loc Integrals/2311-Compute-Integral-0012.HELP.tex}

\[\int_{\frac{1}{4} \, \pi}^{\pi} {-\cos\left(x\right)}\;dx=\answer{\frac{1}{2} \, \sqrt{2}}\]
\end{problem}}%}

%%%%%%%%%%%%%%%%%%%%%%


\latexProblemContent{
\begin{problem}

Use the Fundamental Theorem of Calculus to evaluate the integral.

\expandafter\input{\file@loc Integrals/2311-Compute-Integral-0012.HELP.tex}

\[\int_{\frac{2}{3} \, \pi}^{\frac{5}{4} \, \pi} {10 \, \cos\left(x\right)}\;dx=\answer{-5 \, \sqrt{3} - 5 \, \sqrt{2}}\]
\end{problem}}%}

%%%%%%%%%%%%%%%%%%%%%%


\latexProblemContent{
\begin{problem}

Use the Fundamental Theorem of Calculus to evaluate the integral.

\expandafter\input{\file@loc Integrals/2311-Compute-Integral-0012.HELP.tex}

\[\int_{\frac{1}{4} \, \pi}^{\frac{11}{6} \, \pi} {-\cos\left(x\right)}\;dx=\answer{\frac{1}{2} \, \sqrt{2} + \frac{1}{2}}\]
\end{problem}}%}

%%%%%%%%%%%%%%%%%%%%%%


\latexProblemContent{
\begin{problem}

Use the Fundamental Theorem of Calculus to evaluate the integral.

\expandafter\input{\file@loc Integrals/2311-Compute-Integral-0012.HELP.tex}

\[\int_{\frac{3}{4} \, \pi}^{\frac{5}{4} \, \pi} {\sin\left(x\right)}\;dx=\answer{0}\]
\end{problem}}%}

%%%%%%%%%%%%%%%%%%%%%%


\latexProblemContent{
\begin{problem}

Use the Fundamental Theorem of Calculus to evaluate the integral.

\expandafter\input{\file@loc Integrals/2311-Compute-Integral-0012.HELP.tex}

\[\int_{\frac{1}{4} \, \pi}^{\frac{5}{3} \, \pi} {3 \, \cos\left(x\right)}\;dx=\answer{-\frac{3}{2} \, \sqrt{3} - \frac{3}{2} \, \sqrt{2}}\]
\end{problem}}%}

%%%%%%%%%%%%%%%%%%%%%%


\latexProblemContent{
\begin{problem}

Use the Fundamental Theorem of Calculus to evaluate the integral.

\expandafter\input{\file@loc Integrals/2311-Compute-Integral-0012.HELP.tex}

\[\int_{\frac{1}{6} \, \pi}^{\frac{1}{4} \, \pi} {\sin\left(x\right)}\;dx=\answer{\frac{1}{2} \, \sqrt{3} - \frac{1}{2} \, \sqrt{2}}\]
\end{problem}}%}

%%%%%%%%%%%%%%%%%%%%%%


\latexProblemContent{
\begin{problem}

Use the Fundamental Theorem of Calculus to evaluate the integral.

\expandafter\input{\file@loc Integrals/2311-Compute-Integral-0012.HELP.tex}

\[\int_{\frac{1}{6} \, \pi}^{\frac{5}{3} \, \pi} {5 \, \cos\left(x\right)}\;dx=\answer{-\frac{5}{2} \, \sqrt{3} - \frac{5}{2}}\]
\end{problem}}%}

%%%%%%%%%%%%%%%%%%%%%%


\latexProblemContent{
\begin{problem}

Use the Fundamental Theorem of Calculus to evaluate the integral.

\expandafter\input{\file@loc Integrals/2311-Compute-Integral-0012.HELP.tex}

\[\int_{\frac{2}{3} \, \pi}^{\frac{5}{6} \, \pi} {-4 \, \sin\left(x\right)}\;dx=\answer{-2 \, \sqrt{3} + 2}\]
\end{problem}}%}

%%%%%%%%%%%%%%%%%%%%%%


\latexProblemContent{
\begin{problem}

Use the Fundamental Theorem of Calculus to evaluate the integral.

\expandafter\input{\file@loc Integrals/2311-Compute-Integral-0012.HELP.tex}

\[\int_{\frac{1}{4} \, \pi}^{\frac{1}{3} \, \pi} {2 \, \sin\left(x\right)}\;dx=\answer{\sqrt{2} - 1}\]
\end{problem}}%}

%%%%%%%%%%%%%%%%%%%%%%


\latexProblemContent{
\begin{problem}

Use the Fundamental Theorem of Calculus to evaluate the integral.

\expandafter\input{\file@loc Integrals/2311-Compute-Integral-0012.HELP.tex}

\[\int_{\frac{2}{3} \, \pi}^{\frac{11}{6} \, \pi} {9 \, \sin\left(x\right)}\;dx=\answer{-\frac{9}{2} \, \sqrt{3} - \frac{9}{2}}\]
\end{problem}}%}

%%%%%%%%%%%%%%%%%%%%%%


\latexProblemContent{
\begin{problem}

Use the Fundamental Theorem of Calculus to evaluate the integral.

\expandafter\input{\file@loc Integrals/2311-Compute-Integral-0012.HELP.tex}

\[\int_{\frac{1}{3} \, \pi}^{\frac{11}{6} \, \pi} {9 \, \sin\left(x\right)}\;dx=\answer{-\frac{9}{2} \, \sqrt{3} + \frac{9}{2}}\]
\end{problem}}%}

%%%%%%%%%%%%%%%%%%%%%%


\latexProblemContent{
\begin{problem}

Use the Fundamental Theorem of Calculus to evaluate the integral.

\expandafter\input{\file@loc Integrals/2311-Compute-Integral-0012.HELP.tex}

\[\int_{\frac{1}{4} \, \pi}^{\frac{1}{3} \, \pi} {-3 \, \cos\left(x\right)}\;dx=\answer{-\frac{3}{2} \, \sqrt{3} + \frac{3}{2} \, \sqrt{2}}\]
\end{problem}}%}

%%%%%%%%%%%%%%%%%%%%%%


\latexProblemContent{
\begin{problem}

Use the Fundamental Theorem of Calculus to evaluate the integral.

\expandafter\input{\file@loc Integrals/2311-Compute-Integral-0012.HELP.tex}

\[\int_{\frac{1}{3} \, \pi}^{\frac{1}{2} \, \pi} {2 \, \sin\left(x\right)}\;dx=\answer{1}\]
\end{problem}}%}

%%%%%%%%%%%%%%%%%%%%%%


\latexProblemContent{
\begin{problem}

Use the Fundamental Theorem of Calculus to evaluate the integral.

\expandafter\input{\file@loc Integrals/2311-Compute-Integral-0012.HELP.tex}

\[\int_{\frac{1}{2} \, \pi}^{\frac{3}{2} \, \pi} {\cos\left(x\right)}\;dx=\answer{-2}\]
\end{problem}}%}

%%%%%%%%%%%%%%%%%%%%%%


\latexProblemContent{
\begin{problem}

Use the Fundamental Theorem of Calculus to evaluate the integral.

\expandafter\input{\file@loc Integrals/2311-Compute-Integral-0012.HELP.tex}

\[\int_{\frac{3}{4} \, \pi}^{\frac{5}{3} \, \pi} {5 \, \cos\left(x\right)}\;dx=\answer{-\frac{5}{2} \, \sqrt{3} - \frac{5}{2} \, \sqrt{2}}\]
\end{problem}}%}

%%%%%%%%%%%%%%%%%%%%%%


\latexProblemContent{
\begin{problem}

Use the Fundamental Theorem of Calculus to evaluate the integral.

\expandafter\input{\file@loc Integrals/2311-Compute-Integral-0012.HELP.tex}

\[\int_{\frac{1}{6} \, \pi}^{\frac{7}{4} \, \pi} {\cos\left(x\right)}\;dx=\answer{-\frac{1}{2} \, \sqrt{2} - \frac{1}{2}}\]
\end{problem}}%}

%%%%%%%%%%%%%%%%%%%%%%


\latexProblemContent{
\begin{problem}

Use the Fundamental Theorem of Calculus to evaluate the integral.

\expandafter\input{\file@loc Integrals/2311-Compute-Integral-0012.HELP.tex}

\[\int_{\frac{1}{2} \, \pi}^{\frac{7}{6} \, \pi} {6 \, \sin\left(x\right)}\;dx=\answer{3^{\frac{3}{2}}}\]
\end{problem}}%}

%%%%%%%%%%%%%%%%%%%%%%


\latexProblemContent{
\begin{problem}

Use the Fundamental Theorem of Calculus to evaluate the integral.

\expandafter\input{\file@loc Integrals/2311-Compute-Integral-0012.HELP.tex}

\[\int_{\frac{1}{3} \, \pi}^{\pi} {-\cos\left(x\right)}\;dx=\answer{\frac{1}{2} \, \sqrt{3}}\]
\end{problem}}%}

%%%%%%%%%%%%%%%%%%%%%%


\latexProblemContent{
\begin{problem}

Use the Fundamental Theorem of Calculus to evaluate the integral.

\expandafter\input{\file@loc Integrals/2311-Compute-Integral-0012.HELP.tex}

\[\int_{\frac{2}{3} \, \pi}^{\frac{11}{6} \, \pi} {-6 \, \sin\left(x\right)}\;dx=\answer{3 \, \sqrt{3} + 3}\]
\end{problem}}%}

%%%%%%%%%%%%%%%%%%%%%%


\latexProblemContent{
\begin{problem}

Use the Fundamental Theorem of Calculus to evaluate the integral.

\expandafter\input{\file@loc Integrals/2311-Compute-Integral-0012.HELP.tex}

\[\int_{\frac{1}{6} \, \pi}^{\frac{3}{4} \, \pi} {8 \, \cos\left(x\right)}\;dx=\answer{4 \, \sqrt{2} - 4}\]
\end{problem}}%}

%%%%%%%%%%%%%%%%%%%%%%


\latexProblemContent{
\begin{problem}

Use the Fundamental Theorem of Calculus to evaluate the integral.

\expandafter\input{\file@loc Integrals/2311-Compute-Integral-0012.HELP.tex}

\[\int_{\frac{1}{4} \, \pi}^{\frac{7}{4} \, \pi} {-10 \, \cos\left(x\right)}\;dx=\answer{5 \, \left(2 \, \sqrt{2}\right)}\]
\end{problem}}%}

%%%%%%%%%%%%%%%%%%%%%%


\latexProblemContent{
\begin{problem}

Use the Fundamental Theorem of Calculus to evaluate the integral.

\expandafter\input{\file@loc Integrals/2311-Compute-Integral-0012.HELP.tex}

\[\int_{\frac{3}{4} \, \pi}^{\frac{4}{3} \, \pi} {-9 \, \cos\left(x\right)}\;dx=\answer{\frac{9}{2} \, \sqrt{3} + \frac{9}{2} \, \sqrt{2}}\]
\end{problem}}%}

%%%%%%%%%%%%%%%%%%%%%%


\latexProblemContent{
\begin{problem}

Use the Fundamental Theorem of Calculus to evaluate the integral.

\expandafter\input{\file@loc Integrals/2311-Compute-Integral-0012.HELP.tex}

\[\int_{\frac{1}{6} \, \pi}^{\frac{3}{4} \, \pi} {9 \, \sin\left(x\right)}\;dx=\answer{\frac{9}{2} \, \sqrt{3} + \frac{9}{2} \, \sqrt{2}}\]
\end{problem}}%}

%%%%%%%%%%%%%%%%%%%%%%


\latexProblemContent{
\begin{problem}

Use the Fundamental Theorem of Calculus to evaluate the integral.

\expandafter\input{\file@loc Integrals/2311-Compute-Integral-0012.HELP.tex}

\[\int_{\frac{3}{4} \, \pi}^{\pi} {-10 \, \cos\left(x\right)}\;dx=\answer{5 \, \sqrt{2}}\]
\end{problem}}%}

%%%%%%%%%%%%%%%%%%%%%%


\latexProblemContent{
\begin{problem}

Use the Fundamental Theorem of Calculus to evaluate the integral.

\expandafter\input{\file@loc Integrals/2311-Compute-Integral-0012.HELP.tex}

\[\int_{\frac{1}{6} \, \pi}^{\frac{2}{3} \, \pi} {9 \, \sin\left(x\right)}\;dx=\answer{\frac{9}{2} \, \sqrt{3} + \frac{9}{2}}\]
\end{problem}}%}

%%%%%%%%%%%%%%%%%%%%%%


\latexProblemContent{
\begin{problem}

Use the Fundamental Theorem of Calculus to evaluate the integral.

\expandafter\input{\file@loc Integrals/2311-Compute-Integral-0012.HELP.tex}

\[\int_{\frac{1}{4} \, \pi}^{\pi} {-8 \, \sin\left(x\right)}\;dx=\answer{-4 \, \sqrt{2} - 8}\]
\end{problem}}%}

%%%%%%%%%%%%%%%%%%%%%%


\latexProblemContent{
\begin{problem}

Use the Fundamental Theorem of Calculus to evaluate the integral.

\expandafter\input{\file@loc Integrals/2311-Compute-Integral-0012.HELP.tex}

\[\int_{\frac{1}{4} \, \pi}^{\frac{7}{6} \, \pi} {-4 \, \cos\left(x\right)}\;dx=\answer{2 \, \sqrt{2} + 2}\]
\end{problem}}%}

%%%%%%%%%%%%%%%%%%%%%%


\latexProblemContent{
\begin{problem}

Use the Fundamental Theorem of Calculus to evaluate the integral.

\expandafter\input{\file@loc Integrals/2311-Compute-Integral-0012.HELP.tex}

\[\int_{\frac{5}{6} \, \pi}^{\frac{3}{2} \, \pi} {\sin\left(x\right)}\;dx=\answer{-\frac{1}{2} \, \sqrt{3}}\]
\end{problem}}%}

%%%%%%%%%%%%%%%%%%%%%%


\latexProblemContent{
\begin{problem}

Use the Fundamental Theorem of Calculus to evaluate the integral.

\expandafter\input{\file@loc Integrals/2311-Compute-Integral-0012.HELP.tex}

\[\int_{\frac{1}{6} \, \pi}^{\frac{4}{3} \, \pi} {10 \, \cos\left(x\right)}\;dx=\answer{-5 \, \sqrt{3} - 5}\]
\end{problem}}%}

%%%%%%%%%%%%%%%%%%%%%%


\latexProblemContent{
\begin{problem}

Use the Fundamental Theorem of Calculus to evaluate the integral.

\expandafter\input{\file@loc Integrals/2311-Compute-Integral-0012.HELP.tex}

\[\int_{\frac{1}{3} \, \pi}^{\frac{11}{6} \, \pi} {8 \, \sin\left(x\right)}\;dx=\answer{-4 \, \sqrt{3} + 4}\]
\end{problem}}%}

%%%%%%%%%%%%%%%%%%%%%%


%%%%%%%%%%%%%%%%%%%%%%


\latexProblemContent{
\begin{problem}

Use the Fundamental Theorem of Calculus to evaluate the integral.

\expandafter\input{\file@loc Integrals/2311-Compute-Integral-0012.HELP.tex}

\[\int_{\frac{1}{4} \, \pi}^{\frac{7}{4} \, \pi} {2 \, \cos\left(x\right)}\;dx=\answer{-\left(2 \, \sqrt{2}\right)}\]
\end{problem}}%}

%%%%%%%%%%%%%%%%%%%%%%


\latexProblemContent{
\begin{problem}

Use the Fundamental Theorem of Calculus to evaluate the integral.

\expandafter\input{\file@loc Integrals/2311-Compute-Integral-0012.HELP.tex}

\[\int_{\frac{2}{3} \, \pi}^{\frac{7}{6} \, \pi} {-3 \, \cos\left(x\right)}\;dx=\answer{\frac{3}{2} \, \sqrt{3} + \frac{3}{2}}\]
\end{problem}}%}

%%%%%%%%%%%%%%%%%%%%%%


\latexProblemContent{
\begin{problem}

Use the Fundamental Theorem of Calculus to evaluate the integral.

\expandafter\input{\file@loc Integrals/2311-Compute-Integral-0012.HELP.tex}

\[\int_{\frac{3}{4} \, \pi}^{\frac{3}{2} \, \pi} {-3 \, \sin\left(x\right)}\;dx=\answer{\frac{3}{2} \, \sqrt{2}}\]
\end{problem}}%}

%%%%%%%%%%%%%%%%%%%%%%


\latexProblemContent{
\begin{problem}

Use the Fundamental Theorem of Calculus to evaluate the integral.

\expandafter\input{\file@loc Integrals/2311-Compute-Integral-0012.HELP.tex}

\[\int_{\frac{1}{2} \, \pi}^{\frac{2}{3} \, \pi} {-10 \, \cos\left(x\right)}\;dx=\answer{-5 \, \sqrt{3} + 10}\]
\end{problem}}%}

%%%%%%%%%%%%%%%%%%%%%%


\latexProblemContent{
\begin{problem}

Use the Fundamental Theorem of Calculus to evaluate the integral.

\expandafter\input{\file@loc Integrals/2311-Compute-Integral-0012.HELP.tex}

\[\int_{\frac{1}{3} \, \pi}^{\frac{5}{6} \, \pi} {-5 \, \sin\left(x\right)}\;dx=\answer{-\frac{5}{2} \, \sqrt{3} - \frac{5}{2}}\]
\end{problem}}%}

%%%%%%%%%%%%%%%%%%%%%%


\latexProblemContent{
\begin{problem}

Use the Fundamental Theorem of Calculus to evaluate the integral.

\expandafter\input{\file@loc Integrals/2311-Compute-Integral-0012.HELP.tex}

\[\int_{\frac{1}{3} \, \pi}^{\frac{1}{2} \, \pi} {-10 \, \sin\left(x\right)}\;dx=\answer{-5}\]
\end{problem}}%}

%%%%%%%%%%%%%%%%%%%%%%


\latexProblemContent{
\begin{problem}

Use the Fundamental Theorem of Calculus to evaluate the integral.

\expandafter\input{\file@loc Integrals/2311-Compute-Integral-0012.HELP.tex}

\[\int_{\frac{1}{4} \, \pi}^{\frac{11}{6} \, \pi} {-10 \, \sin\left(x\right)}\;dx=\answer{5 \, \sqrt{3} - 5 \, \sqrt{2}}\]
\end{problem}}%}

%%%%%%%%%%%%%%%%%%%%%%


\latexProblemContent{
\begin{problem}

Use the Fundamental Theorem of Calculus to evaluate the integral.

\expandafter\input{\file@loc Integrals/2311-Compute-Integral-0012.HELP.tex}

\[\int_{\frac{1}{6} \, \pi}^{\frac{11}{6} \, \pi} {9 \, \sin\left(x\right)}\;dx=\answer{0}\]
\end{problem}}%}

%%%%%%%%%%%%%%%%%%%%%%


\latexProblemContent{
\begin{problem}

Use the Fundamental Theorem of Calculus to evaluate the integral.

\expandafter\input{\file@loc Integrals/2311-Compute-Integral-0012.HELP.tex}

\[\int_{\frac{3}{4} \, \pi}^{\frac{4}{3} \, \pi} {-2 \, \cos\left(x\right)}\;dx=\answer{\sqrt{3} + \sqrt{2}}\]
\end{problem}}%}

%%%%%%%%%%%%%%%%%%%%%%


\latexProblemContent{
\begin{problem}

Use the Fundamental Theorem of Calculus to evaluate the integral.

\expandafter\input{\file@loc Integrals/2311-Compute-Integral-0012.HELP.tex}

\[\int_{\frac{1}{2} \, \pi}^{\frac{7}{6} \, \pi} {3 \, \sin\left(x\right)}\;dx=\answer{\frac{1}{2} \, \left(3 \, \sqrt{3}\right)}\]
\end{problem}}%}

%%%%%%%%%%%%%%%%%%%%%%


\latexProblemContent{
\begin{problem}

Use the Fundamental Theorem of Calculus to evaluate the integral.

\expandafter\input{\file@loc Integrals/2311-Compute-Integral-0012.HELP.tex}

\[\int_{\frac{1}{6} \, \pi}^{\frac{3}{4} \, \pi} {-3 \, \cos\left(x\right)}\;dx=\answer{-\frac{3}{2} \, \sqrt{2} + \frac{3}{2}}\]
\end{problem}}%}

%%%%%%%%%%%%%%%%%%%%%%


\latexProblemContent{
\begin{problem}

Use the Fundamental Theorem of Calculus to evaluate the integral.

\expandafter\input{\file@loc Integrals/2311-Compute-Integral-0012.HELP.tex}

\[\int_{\frac{5}{6} \, \pi}^{\frac{4}{3} \, \pi} {10 \, \sin\left(x\right)}\;dx=\answer{-5 \, \sqrt{3} + 5}\]
\end{problem}}%}

%%%%%%%%%%%%%%%%%%%%%%


\latexProblemContent{
\begin{problem}

Use the Fundamental Theorem of Calculus to evaluate the integral.

\expandafter\input{\file@loc Integrals/2311-Compute-Integral-0012.HELP.tex}

\[\int_{\frac{1}{3} \, \pi}^{\frac{7}{6} \, \pi} {\sin\left(x\right)}\;dx=\answer{\frac{1}{2} \, \sqrt{3} + \frac{1}{2}}\]
\end{problem}}%}

%%%%%%%%%%%%%%%%%%%%%%


\latexProblemContent{
\begin{problem}

Use the Fundamental Theorem of Calculus to evaluate the integral.

\expandafter\input{\file@loc Integrals/2311-Compute-Integral-0012.HELP.tex}

\[\int_{\frac{1}{6} \, \pi}^{\frac{7}{4} \, \pi} {-\sin\left(x\right)}\;dx=\answer{-\frac{1}{2} \, \sqrt{3} + \frac{1}{2} \, \sqrt{2}}\]
\end{problem}}%}

%%%%%%%%%%%%%%%%%%%%%%


\latexProblemContent{
\begin{problem}

Use the Fundamental Theorem of Calculus to evaluate the integral.

\expandafter\input{\file@loc Integrals/2311-Compute-Integral-0012.HELP.tex}

\[\int_{\frac{1}{4} \, \pi}^{\frac{1}{2} \, \pi} {-8 \, \cos\left(x\right)}\;dx=\answer{4 \, \sqrt{2} - 8}\]
\end{problem}}%}

%%%%%%%%%%%%%%%%%%%%%%


\latexProblemContent{
\begin{problem}

Use the Fundamental Theorem of Calculus to evaluate the integral.

\expandafter\input{\file@loc Integrals/2311-Compute-Integral-0012.HELP.tex}

\[\int_{\frac{1}{4} \, \pi}^{\frac{2}{3} \, \pi} {\sin\left(x\right)}\;dx=\answer{\frac{1}{2} \, \sqrt{2} + \frac{1}{2}}\]
\end{problem}}%}

%%%%%%%%%%%%%%%%%%%%%%


\latexProblemContent{
\begin{problem}

Use the Fundamental Theorem of Calculus to evaluate the integral.

\expandafter\input{\file@loc Integrals/2311-Compute-Integral-0012.HELP.tex}

\[\int_{\frac{1}{2} \, \pi}^{\frac{5}{3} \, \pi} {-2 \, \sin\left(x\right)}\;dx=\answer{1}\]
\end{problem}}%}

%%%%%%%%%%%%%%%%%%%%%%


\latexProblemContent{
\begin{problem}

Use the Fundamental Theorem of Calculus to evaluate the integral.

\expandafter\input{\file@loc Integrals/2311-Compute-Integral-0012.HELP.tex}

\[\int_{\frac{1}{3} \, \pi}^{\frac{1}{2} \, \pi} {-7 \, \cos\left(x\right)}\;dx=\answer{\frac{7}{2} \, \sqrt{3} - 7}\]
\end{problem}}%}

%%%%%%%%%%%%%%%%%%%%%%


\latexProblemContent{
\begin{problem}

Use the Fundamental Theorem of Calculus to evaluate the integral.

\expandafter\input{\file@loc Integrals/2311-Compute-Integral-0012.HELP.tex}

\[\int_{\frac{5}{6} \, \pi}^{\frac{7}{6} \, \pi} {6 \, \cos\left(x\right)}\;dx=\answer{-6}\]
\end{problem}}%}

%%%%%%%%%%%%%%%%%%%%%%


\latexProblemContent{
\begin{problem}

Use the Fundamental Theorem of Calculus to evaluate the integral.

\expandafter\input{\file@loc Integrals/2311-Compute-Integral-0012.HELP.tex}

\[\int_{\frac{5}{6} \, \pi}^{\frac{5}{3} \, \pi} {9 \, \sin\left(x\right)}\;dx=\answer{-\frac{9}{2} \, \sqrt{3} - \frac{9}{2}}\]
\end{problem}}%}

%%%%%%%%%%%%%%%%%%%%%%


\latexProblemContent{
\begin{problem}

Use the Fundamental Theorem of Calculus to evaluate the integral.

\expandafter\input{\file@loc Integrals/2311-Compute-Integral-0012.HELP.tex}

\[\int_{\frac{2}{3} \, \pi}^{\frac{7}{6} \, \pi} {10 \, \cos\left(x\right)}\;dx=\answer{-5 \, \sqrt{3} - 5}\]
\end{problem}}%}

%%%%%%%%%%%%%%%%%%%%%%


\latexProblemContent{
\begin{problem}

Use the Fundamental Theorem of Calculus to evaluate the integral.

\expandafter\input{\file@loc Integrals/2311-Compute-Integral-0012.HELP.tex}

\[\int_{\frac{1}{6} \, \pi}^{\frac{4}{3} \, \pi} {2 \, \cos\left(x\right)}\;dx=\answer{-\sqrt{3} - 1}\]
\end{problem}}%}

%%%%%%%%%%%%%%%%%%%%%%


\latexProblemContent{
\begin{problem}

Use the Fundamental Theorem of Calculus to evaluate the integral.

\expandafter\input{\file@loc Integrals/2311-Compute-Integral-0012.HELP.tex}

\[\int_{\frac{5}{6} \, \pi}^{\frac{5}{4} \, \pi} {3 \, \cos\left(x\right)}\;dx=\answer{-\frac{3}{2} \, \sqrt{2} - \frac{3}{2}}\]
\end{problem}}%}

%%%%%%%%%%%%%%%%%%%%%%


\latexProblemContent{
\begin{problem}

Use the Fundamental Theorem of Calculus to evaluate the integral.

\expandafter\input{\file@loc Integrals/2311-Compute-Integral-0012.HELP.tex}

\[\int_{\frac{1}{6} \, \pi}^{\frac{7}{4} \, \pi} {10 \, \sin\left(x\right)}\;dx=\answer{5 \, \sqrt{3} - 5 \, \sqrt{2}}\]
\end{problem}}%}

%%%%%%%%%%%%%%%%%%%%%%


\latexProblemContent{
\begin{problem}

Use the Fundamental Theorem of Calculus to evaluate the integral.

\expandafter\input{\file@loc Integrals/2311-Compute-Integral-0012.HELP.tex}

\[\int_{\frac{1}{3} \, \pi}^{\frac{3}{4} \, \pi} {-4 \, \sin\left(x\right)}\;dx=\answer{-2 \, \sqrt{2} - 2}\]
\end{problem}}%}

%%%%%%%%%%%%%%%%%%%%%%


\latexProblemContent{
\begin{problem}

Use the Fundamental Theorem of Calculus to evaluate the integral.

\expandafter\input{\file@loc Integrals/2311-Compute-Integral-0012.HELP.tex}

\[\int_{\frac{2}{3} \, \pi}^{\frac{5}{3} \, \pi} {-6 \, \cos\left(x\right)}\;dx=\answer{2 \, \left(3 \, \sqrt{3}\right)}\]
\end{problem}}%}

%%%%%%%%%%%%%%%%%%%%%%


\latexProblemContent{
\begin{problem}

Use the Fundamental Theorem of Calculus to evaluate the integral.

\expandafter\input{\file@loc Integrals/2311-Compute-Integral-0012.HELP.tex}

\[\int_{\frac{1}{4} \, \pi}^{\frac{4}{3} \, \pi} {-8 \, \cos\left(x\right)}\;dx=\answer{4 \, \sqrt{3} + 4 \, \sqrt{2}}\]
\end{problem}}%}

%%%%%%%%%%%%%%%%%%%%%%


\latexProblemContent{
\begin{problem}

Use the Fundamental Theorem of Calculus to evaluate the integral.

\expandafter\input{\file@loc Integrals/2311-Compute-Integral-0012.HELP.tex}

\[\int_{\frac{1}{2} \, \pi}^{\frac{7}{6} \, \pi} {6 \, \cos\left(x\right)}\;dx=\answer{-9}\]
\end{problem}}%}

%%%%%%%%%%%%%%%%%%%%%%


\latexProblemContent{
\begin{problem}

Use the Fundamental Theorem of Calculus to evaluate the integral.

\expandafter\input{\file@loc Integrals/2311-Compute-Integral-0012.HELP.tex}

\[\int_{\frac{1}{6} \, \pi}^{\frac{11}{6} \, \pi} {-3 \, \cos\left(x\right)}\;dx=\answer{3}\]
\end{problem}}%}

%%%%%%%%%%%%%%%%%%%%%%


\latexProblemContent{
\begin{problem}

Use the Fundamental Theorem of Calculus to evaluate the integral.

\expandafter\input{\file@loc Integrals/2311-Compute-Integral-0012.HELP.tex}

\[\int_{\frac{1}{6} \, \pi}^{\frac{3}{2} \, \pi} {-3 \, \sin\left(x\right)}\;dx=\answer{-\frac{1}{2} \, \left(3 \, \sqrt{3}\right)}\]
\end{problem}}%}

%%%%%%%%%%%%%%%%%%%%%%


\latexProblemContent{
\begin{problem}

Use the Fundamental Theorem of Calculus to evaluate the integral.

\expandafter\input{\file@loc Integrals/2311-Compute-Integral-0012.HELP.tex}

\[\int_{\frac{2}{3} \, \pi}^{\frac{5}{4} \, \pi} {10 \, \sin\left(x\right)}\;dx=\answer{5 \, \sqrt{2} - 5}\]
\end{problem}}%}

%%%%%%%%%%%%%%%%%%%%%%


\latexProblemContent{
\begin{problem}

Use the Fundamental Theorem of Calculus to evaluate the integral.

\expandafter\input{\file@loc Integrals/2311-Compute-Integral-0012.HELP.tex}

\[\int_{\frac{1}{3} \, \pi}^{\frac{3}{4} \, \pi} {7 \, \cos\left(x\right)}\;dx=\answer{-\frac{7}{2} \, \sqrt{3} + \frac{7}{2} \, \sqrt{2}}\]
\end{problem}}%}

%%%%%%%%%%%%%%%%%%%%%%


\latexProblemContent{
\begin{problem}

Use the Fundamental Theorem of Calculus to evaluate the integral.

\expandafter\input{\file@loc Integrals/2311-Compute-Integral-0012.HELP.tex}

\[\int_{\frac{1}{3} \, \pi}^{\frac{7}{6} \, \pi} {\cos\left(x\right)}\;dx=\answer{-\frac{1}{2} \, \sqrt{3} - \frac{1}{2}}\]
\end{problem}}%}

%%%%%%%%%%%%%%%%%%%%%%


\latexProblemContent{
\begin{problem}

Use the Fundamental Theorem of Calculus to evaluate the integral.

\expandafter\input{\file@loc Integrals/2311-Compute-Integral-0012.HELP.tex}

\[\int_{\frac{1}{4} \, \pi}^{\frac{5}{4} \, \pi} {-2 \, \sin\left(x\right)}\;dx=\answer{-\left(2 \, \sqrt{2}\right)}\]
\end{problem}}%}

%%%%%%%%%%%%%%%%%%%%%%


\latexProblemContent{
\begin{problem}

Use the Fundamental Theorem of Calculus to evaluate the integral.

\expandafter\input{\file@loc Integrals/2311-Compute-Integral-0012.HELP.tex}

\[\int_{\frac{1}{4} \, \pi}^{\pi} {3 \, \sin\left(x\right)}\;dx=\answer{\frac{3}{2} \, \sqrt{2} + 3}\]
\end{problem}}%}

%%%%%%%%%%%%%%%%%%%%%%


\latexProblemContent{
\begin{problem}

Use the Fundamental Theorem of Calculus to evaluate the integral.

\expandafter\input{\file@loc Integrals/2311-Compute-Integral-0012.HELP.tex}

\[\int_{\frac{2}{3} \, \pi}^{\frac{7}{4} \, \pi} {8 \, \cos\left(x\right)}\;dx=\answer{-4 \, \sqrt{3} - 4 \, \sqrt{2}}\]
\end{problem}}%}

%%%%%%%%%%%%%%%%%%%%%%


\latexProblemContent{
\begin{problem}

Use the Fundamental Theorem of Calculus to evaluate the integral.

\expandafter\input{\file@loc Integrals/2311-Compute-Integral-0012.HELP.tex}

\[\int_{\frac{5}{6} \, \pi}^{\frac{5}{3} \, \pi} {\cos\left(x\right)}\;dx=\answer{-\frac{1}{2} \, \sqrt{3} - \frac{1}{2}}\]
\end{problem}}%}

%%%%%%%%%%%%%%%%%%%%%%


\latexProblemContent{
\begin{problem}

Use the Fundamental Theorem of Calculus to evaluate the integral.

\expandafter\input{\file@loc Integrals/2311-Compute-Integral-0012.HELP.tex}

\[\int_{\frac{5}{6} \, \pi}^{\frac{7}{4} \, \pi} {\cos\left(x\right)}\;dx=\answer{-\frac{1}{2} \, \sqrt{2} - \frac{1}{2}}\]
\end{problem}}%}

%%%%%%%%%%%%%%%%%%%%%%


\latexProblemContent{
\begin{problem}

Use the Fundamental Theorem of Calculus to evaluate the integral.

\expandafter\input{\file@loc Integrals/2311-Compute-Integral-0012.HELP.tex}

\[\int_{\frac{3}{4} \, \pi}^{\frac{7}{6} \, \pi} {-10 \, \cos\left(x\right)}\;dx=\answer{5 \, \sqrt{2} + 5}\]
\end{problem}}%}

%%%%%%%%%%%%%%%%%%%%%%


\latexProblemContent{
\begin{problem}

Use the Fundamental Theorem of Calculus to evaluate the integral.

\expandafter\input{\file@loc Integrals/2311-Compute-Integral-0012.HELP.tex}

\[\int_{\frac{5}{6} \, \pi}^{\frac{3}{2} \, \pi} {5 \, \cos\left(x\right)}\;dx=\answer{-\frac{15}{2}}\]
\end{problem}}%}

%%%%%%%%%%%%%%%%%%%%%%


\latexProblemContent{
\begin{problem}

Use the Fundamental Theorem of Calculus to evaluate the integral.

\expandafter\input{\file@loc Integrals/2311-Compute-Integral-0012.HELP.tex}

\[\int_{\frac{5}{6} \, \pi}^{\frac{4}{3} \, \pi} {-10 \, \cos\left(x\right)}\;dx=\answer{5 \, \sqrt{3} + 5}\]
\end{problem}}%}

%%%%%%%%%%%%%%%%%%%%%%


\latexProblemContent{
\begin{problem}

Use the Fundamental Theorem of Calculus to evaluate the integral.

\expandafter\input{\file@loc Integrals/2311-Compute-Integral-0012.HELP.tex}

\[\int_{\frac{1}{6} \, \pi}^{\frac{5}{4} \, \pi} {3 \, \cos\left(x\right)}\;dx=\answer{-\frac{3}{2} \, \sqrt{2} - \frac{3}{2}}\]
\end{problem}}%}

%%%%%%%%%%%%%%%%%%%%%%


\latexProblemContent{
\begin{problem}

Use the Fundamental Theorem of Calculus to evaluate the integral.

\expandafter\input{\file@loc Integrals/2311-Compute-Integral-0012.HELP.tex}

\[\int_{\frac{2}{3} \, \pi}^{\frac{5}{3} \, \pi} {7 \, \cos\left(x\right)}\;dx=\answer{-7 \, \sqrt{3}}\]
\end{problem}}%}

%%%%%%%%%%%%%%%%%%%%%%


\latexProblemContent{
\begin{problem}

Use the Fundamental Theorem of Calculus to evaluate the integral.

\expandafter\input{\file@loc Integrals/2311-Compute-Integral-0012.HELP.tex}

\[\int_{\frac{5}{6} \, \pi}^{\frac{5}{3} \, \pi} {4 \, \sin\left(x\right)}\;dx=\answer{-2 \, \sqrt{3} - 2}\]
\end{problem}}%}

%%%%%%%%%%%%%%%%%%%%%%


\latexProblemContent{
\begin{problem}

Use the Fundamental Theorem of Calculus to evaluate the integral.

\expandafter\input{\file@loc Integrals/2311-Compute-Integral-0012.HELP.tex}

\[\int_{\frac{5}{6} \, \pi}^{\frac{5}{3} \, \pi} {3 \, \cos\left(x\right)}\;dx=\answer{-\frac{3}{2} \, \sqrt{3} - \frac{3}{2}}\]
\end{problem}}%}

%%%%%%%%%%%%%%%%%%%%%%


\latexProblemContent{
\begin{problem}

Use the Fundamental Theorem of Calculus to evaluate the integral.

\expandafter\input{\file@loc Integrals/2311-Compute-Integral-0012.HELP.tex}

\[\int_{\frac{1}{3} \, \pi}^{\frac{3}{4} \, \pi} {-2 \, \sin\left(x\right)}\;dx=\answer{-\sqrt{2} - 1}\]
\end{problem}}%}

%%%%%%%%%%%%%%%%%%%%%%


\latexProblemContent{
\begin{problem}

Use the Fundamental Theorem of Calculus to evaluate the integral.

\expandafter\input{\file@loc Integrals/2311-Compute-Integral-0012.HELP.tex}

\[\int_{\frac{1}{3} \, \pi}^{\frac{2}{3} \, \pi} {10 \, \cos\left(x\right)}\;dx=\answer{0}\]
\end{problem}}%}

%%%%%%%%%%%%%%%%%%%%%%


\latexProblemContent{
\begin{problem}

Use the Fundamental Theorem of Calculus to evaluate the integral.

\expandafter\input{\file@loc Integrals/2311-Compute-Integral-0012.HELP.tex}

\[\int_{\frac{3}{4} \, \pi}^{\frac{5}{3} \, \pi} {-3 \, \cos\left(x\right)}\;dx=\answer{\frac{3}{2} \, \sqrt{3} + \frac{3}{2} \, \sqrt{2}}\]
\end{problem}}%}

%%%%%%%%%%%%%%%%%%%%%%


\latexProblemContent{
\begin{problem}

Use the Fundamental Theorem of Calculus to evaluate the integral.

\expandafter\input{\file@loc Integrals/2311-Compute-Integral-0012.HELP.tex}

\[\int_{\frac{5}{6} \, \pi}^{\frac{5}{4} \, \pi} {-9 \, \sin\left(x\right)}\;dx=\answer{\frac{9}{2} \, \sqrt{3} - \frac{9}{2} \, \sqrt{2}}\]
\end{problem}}%}

%%%%%%%%%%%%%%%%%%%%%%


\latexProblemContent{
\begin{problem}

Use the Fundamental Theorem of Calculus to evaluate the integral.

\expandafter\input{\file@loc Integrals/2311-Compute-Integral-0012.HELP.tex}

\[\int_{\frac{5}{6} \, \pi}^{\frac{11}{6} \, \pi} {4 \, \cos\left(x\right)}\;dx=\answer{-4}\]
\end{problem}}%}

%%%%%%%%%%%%%%%%%%%%%%


\latexProblemContent{
\begin{problem}

Use the Fundamental Theorem of Calculus to evaluate the integral.

\expandafter\input{\file@loc Integrals/2311-Compute-Integral-0012.HELP.tex}

\[\int_{\frac{1}{2} \, \pi}^{\frac{5}{6} \, \pi} {-5 \, \cos\left(x\right)}\;dx=\answer{\frac{5}{2}}\]
\end{problem}}%}

%%%%%%%%%%%%%%%%%%%%%%


\latexProblemContent{
\begin{problem}

Use the Fundamental Theorem of Calculus to evaluate the integral.

\expandafter\input{\file@loc Integrals/2311-Compute-Integral-0012.HELP.tex}

\[\int_{\frac{1}{2} \, \pi}^{\frac{3}{4} \, \pi} {2 \, \sin\left(x\right)}\;dx=\answer{\sqrt{2}}\]
\end{problem}}%}

%%%%%%%%%%%%%%%%%%%%%%


\latexProblemContent{
\begin{problem}

Use the Fundamental Theorem of Calculus to evaluate the integral.

\expandafter\input{\file@loc Integrals/2311-Compute-Integral-0012.HELP.tex}

\[\int_{\frac{1}{2} \, \pi}^{\frac{2}{3} \, \pi} {2 \, \sin\left(x\right)}\;dx=\answer{1}\]
\end{problem}}%}

%%%%%%%%%%%%%%%%%%%%%%


\latexProblemContent{
\begin{problem}

Use the Fundamental Theorem of Calculus to evaluate the integral.

\expandafter\input{\file@loc Integrals/2311-Compute-Integral-0012.HELP.tex}

\[\int_{\frac{1}{2} \, \pi}^{\frac{11}{6} \, \pi} {-3 \, \sin\left(x\right)}\;dx=\answer{\frac{1}{2} \, \left(3 \, \sqrt{3}\right)}\]
\end{problem}}%}

%%%%%%%%%%%%%%%%%%%%%%


\latexProblemContent{
\begin{problem}

Use the Fundamental Theorem of Calculus to evaluate the integral.

\expandafter\input{\file@loc Integrals/2311-Compute-Integral-0012.HELP.tex}

\[\int_{\frac{1}{3} \, \pi}^{\frac{5}{6} \, \pi} {8 \, \sin\left(x\right)}\;dx=\answer{4 \, \sqrt{3} + 4}\]
\end{problem}}%}

%%%%%%%%%%%%%%%%%%%%%%


\latexProblemContent{
\begin{problem}

Use the Fundamental Theorem of Calculus to evaluate the integral.

\expandafter\input{\file@loc Integrals/2311-Compute-Integral-0012.HELP.tex}

\[\int_{\frac{5}{6} \, \pi}^{\frac{7}{6} \, \pi} {-2 \, \sin\left(x\right)}\;dx=\answer{0}\]
\end{problem}}%}

%%%%%%%%%%%%%%%%%%%%%%


%%%%%%%%%%%%%%%%%%%%%%


\latexProblemContent{
\begin{problem}

Use the Fundamental Theorem of Calculus to evaluate the integral.

\expandafter\input{\file@loc Integrals/2311-Compute-Integral-0012.HELP.tex}

\[\int_{\frac{1}{4} \, \pi}^{\frac{3}{2} \, \pi} {-8 \, \sin\left(x\right)}\;dx=\answer{-\left(4 \, \sqrt{2}\right)}\]
\end{problem}}%}

%%%%%%%%%%%%%%%%%%%%%%


\latexProblemContent{
\begin{problem}

Use the Fundamental Theorem of Calculus to evaluate the integral.

\expandafter\input{\file@loc Integrals/2311-Compute-Integral-0012.HELP.tex}

\[\int_{\frac{2}{3} \, \pi}^{\frac{5}{3} \, \pi} {\sin\left(x\right)}\;dx=\answer{-1}\]
\end{problem}}%}

%%%%%%%%%%%%%%%%%%%%%%


\latexProblemContent{
\begin{problem}

Use the Fundamental Theorem of Calculus to evaluate the integral.

\expandafter\input{\file@loc Integrals/2311-Compute-Integral-0012.HELP.tex}

\[\int_{\frac{1}{3} \, \pi}^{\frac{11}{6} \, \pi} {3 \, \sin\left(x\right)}\;dx=\answer{-\frac{3}{2} \, \sqrt{3} + \frac{3}{2}}\]
\end{problem}}%}

%%%%%%%%%%%%%%%%%%%%%%


\latexProblemContent{
\begin{problem}

Use the Fundamental Theorem of Calculus to evaluate the integral.

\expandafter\input{\file@loc Integrals/2311-Compute-Integral-0012.HELP.tex}

\[\int_{\frac{1}{2} \, \pi}^{\frac{5}{4} \, \pi} {9 \, \cos\left(x\right)}\;dx=\answer{-\frac{9}{2} \, \sqrt{2} - 9}\]
\end{problem}}%}

%%%%%%%%%%%%%%%%%%%%%%


\latexProblemContent{
\begin{problem}

Use the Fundamental Theorem of Calculus to evaluate the integral.

\expandafter\input{\file@loc Integrals/2311-Compute-Integral-0012.HELP.tex}

\[\int_{\frac{1}{2} \, \pi}^{\frac{3}{2} \, \pi} {-5 \, \sin\left(x\right)}\;dx=\answer{0}\]
\end{problem}}%}

%%%%%%%%%%%%%%%%%%%%%%


\latexProblemContent{
\begin{problem}

Use the Fundamental Theorem of Calculus to evaluate the integral.

\expandafter\input{\file@loc Integrals/2311-Compute-Integral-0012.HELP.tex}

\[\int_{\frac{1}{6} \, \pi}^{\frac{4}{3} \, \pi} {\sin\left(x\right)}\;dx=\answer{\frac{1}{2} \, \sqrt{3} + \frac{1}{2}}\]
\end{problem}}%}

%%%%%%%%%%%%%%%%%%%%%%


\latexProblemContent{
\begin{problem}

Use the Fundamental Theorem of Calculus to evaluate the integral.

\expandafter\input{\file@loc Integrals/2311-Compute-Integral-0012.HELP.tex}

\[\int_{\frac{2}{3} \, \pi}^{\frac{7}{6} \, \pi} {3 \, \sin\left(x\right)}\;dx=\answer{\frac{3}{2} \, \sqrt{3} - \frac{3}{2}}\]
\end{problem}}%}

%%%%%%%%%%%%%%%%%%%%%%


\latexProblemContent{
\begin{problem}

Use the Fundamental Theorem of Calculus to evaluate the integral.

\expandafter\input{\file@loc Integrals/2311-Compute-Integral-0012.HELP.tex}

\[\int_{\frac{1}{4} \, \pi}^{\frac{1}{2} \, \pi} {5 \, \sin\left(x\right)}\;dx=\answer{\frac{5}{2} \, \sqrt{2}}\]
\end{problem}}%}

%%%%%%%%%%%%%%%%%%%%%%


\latexProblemContent{
\begin{problem}

Use the Fundamental Theorem of Calculus to evaluate the integral.

\expandafter\input{\file@loc Integrals/2311-Compute-Integral-0012.HELP.tex}

\[\int_{\frac{1}{6} \, \pi}^{\frac{5}{3} \, \pi} {\cos\left(x\right)}\;dx=\answer{-\frac{1}{2} \, \sqrt{3} - \frac{1}{2}}\]
\end{problem}}%}

%%%%%%%%%%%%%%%%%%%%%%


\latexProblemContent{
\begin{problem}

Use the Fundamental Theorem of Calculus to evaluate the integral.

\expandafter\input{\file@loc Integrals/2311-Compute-Integral-0012.HELP.tex}

\[\int_{\frac{1}{4} \, \pi}^{\frac{1}{2} \, \pi} {-9 \, \sin\left(x\right)}\;dx=\answer{-\frac{9}{2} \, \sqrt{2}}\]
\end{problem}}%}

%%%%%%%%%%%%%%%%%%%%%%


\latexProblemContent{
\begin{problem}

Use the Fundamental Theorem of Calculus to evaluate the integral.

\expandafter\input{\file@loc Integrals/2311-Compute-Integral-0012.HELP.tex}

\[\int_{\frac{3}{4} \, \pi}^{\frac{7}{6} \, \pi} {-5 \, \sin\left(x\right)}\;dx=\answer{-\frac{5}{2} \, \sqrt{3} + \frac{5}{2} \, \sqrt{2}}\]
\end{problem}}%}

%%%%%%%%%%%%%%%%%%%%%%


\latexProblemContent{
\begin{problem}

Use the Fundamental Theorem of Calculus to evaluate the integral.

\expandafter\input{\file@loc Integrals/2311-Compute-Integral-0012.HELP.tex}

\[\int_{\frac{1}{4} \, \pi}^{\frac{11}{6} \, \pi} {5 \, \cos\left(x\right)}\;dx=\answer{-\frac{5}{2} \, \sqrt{2} - \frac{5}{2}}\]
\end{problem}}%}

%%%%%%%%%%%%%%%%%%%%%%


\latexProblemContent{
\begin{problem}

Use the Fundamental Theorem of Calculus to evaluate the integral.

\expandafter\input{\file@loc Integrals/2311-Compute-Integral-0012.HELP.tex}

\[\int_{\frac{1}{4} \, \pi}^{\frac{4}{3} \, \pi} {6 \, \cos\left(x\right)}\;dx=\answer{-3 \, \sqrt{3} - 3 \, \sqrt{2}}\]
\end{problem}}%}

%%%%%%%%%%%%%%%%%%%%%%


\latexProblemContent{
\begin{problem}

Use the Fundamental Theorem of Calculus to evaluate the integral.

\expandafter\input{\file@loc Integrals/2311-Compute-Integral-0012.HELP.tex}

\[\int_{\frac{2}{3} \, \pi}^{\pi} {-10 \, \cos\left(x\right)}\;dx=\answer{5 \, \sqrt{3}}\]
\end{problem}}%}

%%%%%%%%%%%%%%%%%%%%%%


\latexProblemContent{
\begin{problem}

Use the Fundamental Theorem of Calculus to evaluate the integral.

\expandafter\input{\file@loc Integrals/2311-Compute-Integral-0012.HELP.tex}

\[\int_{\frac{1}{3} \, \pi}^{\frac{5}{4} \, \pi} {2 \, \cos\left(x\right)}\;dx=\answer{-\sqrt{3} - \sqrt{2}}\]
\end{problem}}%}

%%%%%%%%%%%%%%%%%%%%%%


\latexProblemContent{
\begin{problem}

Use the Fundamental Theorem of Calculus to evaluate the integral.

\expandafter\input{\file@loc Integrals/2311-Compute-Integral-0012.HELP.tex}

\[\int_{\frac{2}{3} \, \pi}^{\frac{5}{4} \, \pi} {-6 \, \sin\left(x\right)}\;dx=\answer{-3 \, \sqrt{2} + 3}\]
\end{problem}}%}

%%%%%%%%%%%%%%%%%%%%%%


\latexProblemContent{
\begin{problem}

Use the Fundamental Theorem of Calculus to evaluate the integral.

\expandafter\input{\file@loc Integrals/2311-Compute-Integral-0012.HELP.tex}

\[\int_{\frac{1}{4} \, \pi}^{\frac{7}{4} \, \pi} {-10 \, \sin\left(x\right)}\;dx=\answer{0}\]
\end{problem}}%}

%%%%%%%%%%%%%%%%%%%%%%


\latexProblemContent{
\begin{problem}

Use the Fundamental Theorem of Calculus to evaluate the integral.

\expandafter\input{\file@loc Integrals/2311-Compute-Integral-0012.HELP.tex}

\[\int_{\frac{1}{3} \, \pi}^{\pi} {-8 \, \sin\left(x\right)}\;dx=\answer{-12}\]
\end{problem}}%}

%%%%%%%%%%%%%%%%%%%%%%


\latexProblemContent{
\begin{problem}

Use the Fundamental Theorem of Calculus to evaluate the integral.

\expandafter\input{\file@loc Integrals/2311-Compute-Integral-0012.HELP.tex}

\[\int_{\frac{5}{6} \, \pi}^{\frac{4}{3} \, \pi} {-8 \, \cos\left(x\right)}\;dx=\answer{4 \, \sqrt{3} + 4}\]
\end{problem}}%}

%%%%%%%%%%%%%%%%%%%%%%


\latexProblemContent{
\begin{problem}

Use the Fundamental Theorem of Calculus to evaluate the integral.

\expandafter\input{\file@loc Integrals/2311-Compute-Integral-0012.HELP.tex}

\[\int_{\frac{1}{6} \, \pi}^{\frac{7}{4} \, \pi} {-6 \, \cos\left(x\right)}\;dx=\answer{3 \, \sqrt{2} + 3}\]
\end{problem}}%}

%%%%%%%%%%%%%%%%%%%%%%


\latexProblemContent{
\begin{problem}

Use the Fundamental Theorem of Calculus to evaluate the integral.

\expandafter\input{\file@loc Integrals/2311-Compute-Integral-0012.HELP.tex}

\[\int_{\frac{5}{6} \, \pi}^{\frac{7}{4} \, \pi} {6 \, \sin\left(x\right)}\;dx=\answer{-3 \, \sqrt{3} - 3 \, \sqrt{2}}\]
\end{problem}}%}

%%%%%%%%%%%%%%%%%%%%%%


\latexProblemContent{
\begin{problem}

Use the Fundamental Theorem of Calculus to evaluate the integral.

\expandafter\input{\file@loc Integrals/2311-Compute-Integral-0012.HELP.tex}

\[\int_{\frac{1}{4} \, \pi}^{\frac{1}{2} \, \pi} {3 \, \sin\left(x\right)}\;dx=\answer{\frac{3}{2} \, \sqrt{2}}\]
\end{problem}}%}

%%%%%%%%%%%%%%%%%%%%%%


%%%%%%%%%%%%%%%%%%%%%%


\latexProblemContent{
\begin{problem}

Use the Fundamental Theorem of Calculus to evaluate the integral.

\expandafter\input{\file@loc Integrals/2311-Compute-Integral-0012.HELP.tex}

\[\int_{\frac{1}{4} \, \pi}^{\frac{5}{3} \, \pi} {8 \, \sin\left(x\right)}\;dx=\answer{4 \, \sqrt{2} - 4}\]
\end{problem}}%}

%%%%%%%%%%%%%%%%%%%%%%


\latexProblemContent{
\begin{problem}

Use the Fundamental Theorem of Calculus to evaluate the integral.

\expandafter\input{\file@loc Integrals/2311-Compute-Integral-0012.HELP.tex}

\[\int_{\frac{1}{4} \, \pi}^{\frac{2}{3} \, \pi} {-7 \, \cos\left(x\right)}\;dx=\answer{-\frac{7}{2} \, \sqrt{3} + \frac{7}{2} \, \sqrt{2}}\]
\end{problem}}%}

%%%%%%%%%%%%%%%%%%%%%%


\latexProblemContent{
\begin{problem}

Use the Fundamental Theorem of Calculus to evaluate the integral.

\expandafter\input{\file@loc Integrals/2311-Compute-Integral-0012.HELP.tex}

\[\int_{\frac{3}{4} \, \pi}^{\frac{5}{6} \, \pi} {-9 \, \sin\left(x\right)}\;dx=\answer{-\frac{9}{2} \, \sqrt{3} + \frac{9}{2} \, \sqrt{2}}\]
\end{problem}}%}

%%%%%%%%%%%%%%%%%%%%%%


\latexProblemContent{
\begin{problem}

Use the Fundamental Theorem of Calculus to evaluate the integral.

\expandafter\input{\file@loc Integrals/2311-Compute-Integral-0012.HELP.tex}

\[\int_{\frac{1}{6} \, \pi}^{\frac{5}{4} \, \pi} {2 \, \sin\left(x\right)}\;dx=\answer{\sqrt{3} + \sqrt{2}}\]
\end{problem}}%}

%%%%%%%%%%%%%%%%%%%%%%


\latexProblemContent{
\begin{problem}

Use the Fundamental Theorem of Calculus to evaluate the integral.

\expandafter\input{\file@loc Integrals/2311-Compute-Integral-0012.HELP.tex}

\[\int_{\frac{5}{6} \, \pi}^{\frac{3}{2} \, \pi} {-10 \, \sin\left(x\right)}\;dx=\answer{5 \, \sqrt{3}}\]
\end{problem}}%}

%%%%%%%%%%%%%%%%%%%%%%


\latexProblemContent{
\begin{problem}

Use the Fundamental Theorem of Calculus to evaluate the integral.

\expandafter\input{\file@loc Integrals/2311-Compute-Integral-0012.HELP.tex}

\[\int_{\frac{1}{2} \, \pi}^{\frac{11}{6} \, \pi} {4 \, \cos\left(x\right)}\;dx=\answer{-6}\]
\end{problem}}%}

%%%%%%%%%%%%%%%%%%%%%%


\latexProblemContent{
\begin{problem}

Use the Fundamental Theorem of Calculus to evaluate the integral.

\expandafter\input{\file@loc Integrals/2311-Compute-Integral-0012.HELP.tex}

\[\int_{\frac{1}{2} \, \pi}^{\frac{4}{3} \, \pi} {-9 \, \sin\left(x\right)}\;dx=\answer{-\frac{9}{2}}\]
\end{problem}}%}

%%%%%%%%%%%%%%%%%%%%%%


\latexProblemContent{
\begin{problem}

Use the Fundamental Theorem of Calculus to evaluate the integral.

\expandafter\input{\file@loc Integrals/2311-Compute-Integral-0012.HELP.tex}

\[\int_{\frac{1}{6} \, \pi}^{\frac{2}{3} \, \pi} {-2 \, \sin\left(x\right)}\;dx=\answer{-\sqrt{3} - 1}\]
\end{problem}}%}

%%%%%%%%%%%%%%%%%%%%%%


\latexProblemContent{
\begin{problem}

Use the Fundamental Theorem of Calculus to evaluate the integral.

\expandafter\input{\file@loc Integrals/2311-Compute-Integral-0012.HELP.tex}

\[\int_{\frac{2}{3} \, \pi}^{\frac{7}{4} \, \pi} {-6 \, \cos\left(x\right)}\;dx=\answer{3 \, \sqrt{3} + 3 \, \sqrt{2}}\]
\end{problem}}%}

%%%%%%%%%%%%%%%%%%%%%%


\latexProblemContent{
\begin{problem}

Use the Fundamental Theorem of Calculus to evaluate the integral.

\expandafter\input{\file@loc Integrals/2311-Compute-Integral-0012.HELP.tex}

\[\int_{\frac{1}{3} \, \pi}^{\pi} {-9 \, \sin\left(x\right)}\;dx=\answer{-\frac{27}{2}}\]
\end{problem}}%}

%%%%%%%%%%%%%%%%%%%%%%


\latexProblemContent{
\begin{problem}

Use the Fundamental Theorem of Calculus to evaluate the integral.

\expandafter\input{\file@loc Integrals/2311-Compute-Integral-0012.HELP.tex}

\[\int_{\frac{1}{4} \, \pi}^{\frac{5}{6} \, \pi} {-10 \, \cos\left(x\right)}\;dx=\answer{5 \, \sqrt{2} - 5}\]
\end{problem}}%}

%%%%%%%%%%%%%%%%%%%%%%


\latexProblemContent{
\begin{problem}

Use the Fundamental Theorem of Calculus to evaluate the integral.

\expandafter\input{\file@loc Integrals/2311-Compute-Integral-0012.HELP.tex}

\[\int_{\frac{1}{2} \, \pi}^{\frac{2}{3} \, \pi} {-8 \, \sin\left(x\right)}\;dx=\answer{-4}\]
\end{problem}}%}

%%%%%%%%%%%%%%%%%%%%%%


\latexProblemContent{
\begin{problem}

Use the Fundamental Theorem of Calculus to evaluate the integral.

\expandafter\input{\file@loc Integrals/2311-Compute-Integral-0012.HELP.tex}

\[\int_{\frac{1}{6} \, \pi}^{\pi} {8 \, \sin\left(x\right)}\;dx=\answer{4 \, \sqrt{3} + 8}\]
\end{problem}}%}

%%%%%%%%%%%%%%%%%%%%%%


\latexProblemContent{
\begin{problem}

Use the Fundamental Theorem of Calculus to evaluate the integral.

\expandafter\input{\file@loc Integrals/2311-Compute-Integral-0012.HELP.tex}

\[\int_{\frac{1}{2} \, \pi}^{\frac{7}{4} \, \pi} {7 \, \sin\left(x\right)}\;dx=\answer{-\frac{7}{2} \, \sqrt{2}}\]
\end{problem}}%}

%%%%%%%%%%%%%%%%%%%%%%


\latexProblemContent{
\begin{problem}

Use the Fundamental Theorem of Calculus to evaluate the integral.

\expandafter\input{\file@loc Integrals/2311-Compute-Integral-0012.HELP.tex}

\[\int_{\frac{2}{3} \, \pi}^{\frac{5}{3} \, \pi} {-2 \, \cos\left(x\right)}\;dx=\answer{2 \, \sqrt{3}}\]
\end{problem}}%}

%%%%%%%%%%%%%%%%%%%%%%


\latexProblemContent{
\begin{problem}

Use the Fundamental Theorem of Calculus to evaluate the integral.

\expandafter\input{\file@loc Integrals/2311-Compute-Integral-0012.HELP.tex}

\[\int_{\frac{1}{6} \, \pi}^{\frac{3}{4} \, \pi} {-\cos\left(x\right)}\;dx=\answer{-\frac{1}{2} \, \sqrt{2} + \frac{1}{2}}\]
\end{problem}}%}

%%%%%%%%%%%%%%%%%%%%%%


\latexProblemContent{
\begin{problem}

Use the Fundamental Theorem of Calculus to evaluate the integral.

\expandafter\input{\file@loc Integrals/2311-Compute-Integral-0012.HELP.tex}

\[\int_{\frac{2}{3} \, \pi}^{\frac{5}{4} \, \pi} {3 \, \sin\left(x\right)}\;dx=\answer{\frac{3}{2} \, \sqrt{2} - \frac{3}{2}}\]
\end{problem}}%}

%%%%%%%%%%%%%%%%%%%%%%


\latexProblemContent{
\begin{problem}

Use the Fundamental Theorem of Calculus to evaluate the integral.

\expandafter\input{\file@loc Integrals/2311-Compute-Integral-0012.HELP.tex}

\[\int_{\frac{5}{6} \, \pi}^{\frac{5}{4} \, \pi} {-7 \, \cos\left(x\right)}\;dx=\answer{\frac{7}{2} \, \sqrt{2} + \frac{7}{2}}\]
\end{problem}}%}

%%%%%%%%%%%%%%%%%%%%%%


\latexProblemContent{
\begin{problem}

Use the Fundamental Theorem of Calculus to evaluate the integral.

\expandafter\input{\file@loc Integrals/2311-Compute-Integral-0012.HELP.tex}

\[\int_{\frac{5}{6} \, \pi}^{\frac{4}{3} \, \pi} {-7 \, \cos\left(x\right)}\;dx=\answer{\frac{7}{2} \, \sqrt{3} + \frac{7}{2}}\]
\end{problem}}%}

%%%%%%%%%%%%%%%%%%%%%%


\latexProblemContent{
\begin{problem}

Use the Fundamental Theorem of Calculus to evaluate the integral.

\expandafter\input{\file@loc Integrals/2311-Compute-Integral-0012.HELP.tex}

\[\int_{\frac{5}{6} \, \pi}^{\frac{5}{4} \, \pi} {2 \, \cos\left(x\right)}\;dx=\answer{-\sqrt{2} - 1}\]
\end{problem}}%}

%%%%%%%%%%%%%%%%%%%%%%


\latexProblemContent{
\begin{problem}

Use the Fundamental Theorem of Calculus to evaluate the integral.

\expandafter\input{\file@loc Integrals/2311-Compute-Integral-0012.HELP.tex}

\[\int_{\frac{1}{3} \, \pi}^{\frac{5}{4} \, \pi} {\sin\left(x\right)}\;dx=\answer{\frac{1}{2} \, \sqrt{2} + \frac{1}{2}}\]
\end{problem}}%}

%%%%%%%%%%%%%%%%%%%%%%


\latexProblemContent{
\begin{problem}

Use the Fundamental Theorem of Calculus to evaluate the integral.

\expandafter\input{\file@loc Integrals/2311-Compute-Integral-0012.HELP.tex}

\[\int_{\frac{1}{6} \, \pi}^{\frac{11}{6} \, \pi} {-5 \, \cos\left(x\right)}\;dx=\answer{5}\]
\end{problem}}%}

%%%%%%%%%%%%%%%%%%%%%%


\latexProblemContent{
\begin{problem}

Use the Fundamental Theorem of Calculus to evaluate the integral.

\expandafter\input{\file@loc Integrals/2311-Compute-Integral-0012.HELP.tex}

\[\int_{\frac{1}{2} \, \pi}^{\frac{5}{6} \, \pi} {7 \, \cos\left(x\right)}\;dx=\answer{-\frac{7}{2}}\]
\end{problem}}%}

%%%%%%%%%%%%%%%%%%%%%%


\latexProblemContent{
\begin{problem}

Use the Fundamental Theorem of Calculus to evaluate the integral.

\expandafter\input{\file@loc Integrals/2311-Compute-Integral-0012.HELP.tex}

\[\int_{\frac{3}{4} \, \pi}^{\frac{5}{6} \, \pi} {4 \, \cos\left(x\right)}\;dx=\answer{-2 \, \sqrt{2} + 2}\]
\end{problem}}%}

%%%%%%%%%%%%%%%%%%%%%%


\latexProblemContent{
\begin{problem}

Use the Fundamental Theorem of Calculus to evaluate the integral.

\expandafter\input{\file@loc Integrals/2311-Compute-Integral-0012.HELP.tex}

\[\int_{\frac{1}{6} \, \pi}^{\frac{1}{4} \, \pi} {8 \, \cos\left(x\right)}\;dx=\answer{4 \, \sqrt{2} - 4}\]
\end{problem}}%}

%%%%%%%%%%%%%%%%%%%%%%


\latexProblemContent{
\begin{problem}

Use the Fundamental Theorem of Calculus to evaluate the integral.

\expandafter\input{\file@loc Integrals/2311-Compute-Integral-0012.HELP.tex}

\[\int_{\frac{3}{4} \, \pi}^{\frac{5}{3} \, \pi} {-10 \, \cos\left(x\right)}\;dx=\answer{5 \, \sqrt{3} + 5 \, \sqrt{2}}\]
\end{problem}}%}

%%%%%%%%%%%%%%%%%%%%%%


\latexProblemContent{
\begin{problem}

Use the Fundamental Theorem of Calculus to evaluate the integral.

\expandafter\input{\file@loc Integrals/2311-Compute-Integral-0012.HELP.tex}

\[\int_{\frac{1}{6} \, \pi}^{\frac{5}{3} \, \pi} {-6 \, \cos\left(x\right)}\;dx=\answer{3 \, \sqrt{3} + 3}\]
\end{problem}}%}

%%%%%%%%%%%%%%%%%%%%%%


\latexProblemContent{
\begin{problem}

Use the Fundamental Theorem of Calculus to evaluate the integral.

\expandafter\input{\file@loc Integrals/2311-Compute-Integral-0012.HELP.tex}

\[\int_{\frac{1}{3} \, \pi}^{\frac{7}{4} \, \pi} {2 \, \cos\left(x\right)}\;dx=\answer{-\sqrt{3} - \sqrt{2}}\]
\end{problem}}%}

%%%%%%%%%%%%%%%%%%%%%%


\latexProblemContent{
\begin{problem}

Use the Fundamental Theorem of Calculus to evaluate the integral.

\expandafter\input{\file@loc Integrals/2311-Compute-Integral-0012.HELP.tex}

\[\int_{\frac{1}{6} \, \pi}^{\frac{7}{6} \, \pi} {-\cos\left(x\right)}\;dx=\answer{1}\]
\end{problem}}%}

%%%%%%%%%%%%%%%%%%%%%%


\latexProblemContent{
\begin{problem}

Use the Fundamental Theorem of Calculus to evaluate the integral.

\expandafter\input{\file@loc Integrals/2311-Compute-Integral-0012.HELP.tex}

\[\int_{\frac{1}{3} \, \pi}^{\frac{3}{2} \, \pi} {-4 \, \cos\left(x\right)}\;dx=\answer{2 \, \sqrt{3} + 4}\]
\end{problem}}%}

%%%%%%%%%%%%%%%%%%%%%%


\latexProblemContent{
\begin{problem}

Use the Fundamental Theorem of Calculus to evaluate the integral.

\expandafter\input{\file@loc Integrals/2311-Compute-Integral-0012.HELP.tex}

\[\int_{\frac{1}{3} \, \pi}^{\frac{3}{4} \, \pi} {8 \, \sin\left(x\right)}\;dx=\answer{4 \, \sqrt{2} + 4}\]
\end{problem}}%}

%%%%%%%%%%%%%%%%%%%%%%


\latexProblemContent{
\begin{problem}

Use the Fundamental Theorem of Calculus to evaluate the integral.

\expandafter\input{\file@loc Integrals/2311-Compute-Integral-0012.HELP.tex}

\[\int_{\frac{2}{3} \, \pi}^{\frac{11}{6} \, \pi} {10 \, \sin\left(x\right)}\;dx=\answer{-5 \, \sqrt{3} - 5}\]
\end{problem}}%}

%%%%%%%%%%%%%%%%%%%%%%


\latexProblemContent{
\begin{problem}

Use the Fundamental Theorem of Calculus to evaluate the integral.

\expandafter\input{\file@loc Integrals/2311-Compute-Integral-0012.HELP.tex}

\[\int_{\frac{5}{6} \, \pi}^{\frac{3}{2} \, \pi} {-7 \, \sin\left(x\right)}\;dx=\answer{\frac{7}{2} \, \sqrt{3}}\]
\end{problem}}%}

%%%%%%%%%%%%%%%%%%%%%%


\latexProblemContent{
\begin{problem}

Use the Fundamental Theorem of Calculus to evaluate the integral.

\expandafter\input{\file@loc Integrals/2311-Compute-Integral-0012.HELP.tex}

\[\int_{\frac{3}{4} \, \pi}^{\frac{7}{4} \, \pi} {5 \, \cos\left(x\right)}\;dx=\answer{-5 \, \sqrt{2}}\]
\end{problem}}%}

%%%%%%%%%%%%%%%%%%%%%%


\latexProblemContent{
\begin{problem}

Use the Fundamental Theorem of Calculus to evaluate the integral.

\expandafter\input{\file@loc Integrals/2311-Compute-Integral-0012.HELP.tex}

\[\int_{\frac{5}{6} \, \pi}^{\frac{4}{3} \, \pi} {5 \, \cos\left(x\right)}\;dx=\answer{-\frac{5}{2} \, \sqrt{3} - \frac{5}{2}}\]
\end{problem}}%}

%%%%%%%%%%%%%%%%%%%%%%


\latexProblemContent{
\begin{problem}

Use the Fundamental Theorem of Calculus to evaluate the integral.

\expandafter\input{\file@loc Integrals/2311-Compute-Integral-0012.HELP.tex}

\[\int_{\frac{3}{4} \, \pi}^{\frac{5}{3} \, \pi} {6 \, \cos\left(x\right)}\;dx=\answer{-3 \, \sqrt{3} - 3 \, \sqrt{2}}\]
\end{problem}}%}

%%%%%%%%%%%%%%%%%%%%%%


\latexProblemContent{
\begin{problem}

Use the Fundamental Theorem of Calculus to evaluate the integral.

\expandafter\input{\file@loc Integrals/2311-Compute-Integral-0012.HELP.tex}

\[\int_{\frac{1}{4} \, \pi}^{\frac{7}{4} \, \pi} {4 \, \sin\left(x\right)}\;dx=\answer{0}\]
\end{problem}}%}

%%%%%%%%%%%%%%%%%%%%%%


\latexProblemContent{
\begin{problem}

Use the Fundamental Theorem of Calculus to evaluate the integral.

\expandafter\input{\file@loc Integrals/2311-Compute-Integral-0012.HELP.tex}

\[\int_{\frac{5}{6} \, \pi}^{\frac{7}{6} \, \pi} {-8 \, \sin\left(x\right)}\;dx=\answer{0}\]
\end{problem}}%}

%%%%%%%%%%%%%%%%%%%%%%


\latexProblemContent{
\begin{problem}

Use the Fundamental Theorem of Calculus to evaluate the integral.

\expandafter\input{\file@loc Integrals/2311-Compute-Integral-0012.HELP.tex}

\[\int_{\frac{3}{4} \, \pi}^{\frac{5}{4} \, \pi} {10 \, \sin\left(x\right)}\;dx=\answer{0}\]
\end{problem}}%}

%%%%%%%%%%%%%%%%%%%%%%


\latexProblemContent{
\begin{problem}

Use the Fundamental Theorem of Calculus to evaluate the integral.

\expandafter\input{\file@loc Integrals/2311-Compute-Integral-0012.HELP.tex}

\[\int_{\frac{1}{6} \, \pi}^{\frac{5}{4} \, \pi} {8 \, \cos\left(x\right)}\;dx=\answer{-4 \, \sqrt{2} - 4}\]
\end{problem}}%}

%%%%%%%%%%%%%%%%%%%%%%


\latexProblemContent{
\begin{problem}

Use the Fundamental Theorem of Calculus to evaluate the integral.

\expandafter\input{\file@loc Integrals/2311-Compute-Integral-0012.HELP.tex}

\[\int_{\frac{1}{6} \, \pi}^{\frac{3}{2} \, \pi} {-6 \, \cos\left(x\right)}\;dx=\answer{9}\]
\end{problem}}%}

%%%%%%%%%%%%%%%%%%%%%%


\latexProblemContent{
\begin{problem}

Use the Fundamental Theorem of Calculus to evaluate the integral.

\expandafter\input{\file@loc Integrals/2311-Compute-Integral-0012.HELP.tex}

\[\int_{\frac{1}{2} \, \pi}^{\frac{3}{4} \, \pi} {6 \, \sin\left(x\right)}\;dx=\answer{3 \, \sqrt{2}}\]
\end{problem}}%}

%%%%%%%%%%%%%%%%%%%%%%


\latexProblemContent{
\begin{problem}

Use the Fundamental Theorem of Calculus to evaluate the integral.

\expandafter\input{\file@loc Integrals/2311-Compute-Integral-0012.HELP.tex}

\[\int_{\frac{1}{3} \, \pi}^{\frac{5}{6} \, \pi} {-7 \, \cos\left(x\right)}\;dx=\answer{\frac{7}{2} \, \sqrt{3} - \frac{7}{2}}\]
\end{problem}}%}

%%%%%%%%%%%%%%%%%%%%%%


\latexProblemContent{
\begin{problem}

Use the Fundamental Theorem of Calculus to evaluate the integral.

\expandafter\input{\file@loc Integrals/2311-Compute-Integral-0012.HELP.tex}

\[\int_{\frac{1}{3} \, \pi}^{\frac{1}{2} \, \pi} {2 \, \cos\left(x\right)}\;dx=\answer{-\sqrt{3} + 2}\]
\end{problem}}%}

%%%%%%%%%%%%%%%%%%%%%%


\latexProblemContent{
\begin{problem}

Use the Fundamental Theorem of Calculus to evaluate the integral.

\expandafter\input{\file@loc Integrals/2311-Compute-Integral-0012.HELP.tex}

\[\int_{\frac{2}{3} \, \pi}^{\frac{11}{6} \, \pi} {-2 \, \cos\left(x\right)}\;dx=\answer{\sqrt{3} + 1}\]
\end{problem}}%}

%%%%%%%%%%%%%%%%%%%%%%


\latexProblemContent{
\begin{problem}

Use the Fundamental Theorem of Calculus to evaluate the integral.

\expandafter\input{\file@loc Integrals/2311-Compute-Integral-0012.HELP.tex}

\[\int_{\frac{3}{4} \, \pi}^{\frac{7}{6} \, \pi} {\sin\left(x\right)}\;dx=\answer{\frac{1}{2} \, \sqrt{3} - \frac{1}{2} \, \sqrt{2}}\]
\end{problem}}%}

%%%%%%%%%%%%%%%%%%%%%%


\latexProblemContent{
\begin{problem}

Use the Fundamental Theorem of Calculus to evaluate the integral.

\expandafter\input{\file@loc Integrals/2311-Compute-Integral-0012.HELP.tex}

\[\int_{\frac{1}{4} \, \pi}^{\frac{4}{3} \, \pi} {-8 \, \sin\left(x\right)}\;dx=\answer{-4 \, \sqrt{2} - 4}\]
\end{problem}}%}

%%%%%%%%%%%%%%%%%%%%%%


\latexProblemContent{
\begin{problem}

Use the Fundamental Theorem of Calculus to evaluate the integral.

\expandafter\input{\file@loc Integrals/2311-Compute-Integral-0012.HELP.tex}

\[\int_{\frac{1}{2} \, \pi}^{\frac{5}{4} \, \pi} {-5 \, \sin\left(x\right)}\;dx=\answer{-\frac{5}{2} \, \sqrt{2}}\]
\end{problem}}%}

%%%%%%%%%%%%%%%%%%%%%%


\latexProblemContent{
\begin{problem}

Use the Fundamental Theorem of Calculus to evaluate the integral.

\expandafter\input{\file@loc Integrals/2311-Compute-Integral-0012.HELP.tex}

\[\int_{\frac{2}{3} \, \pi}^{\frac{7}{6} \, \pi} {-\sin\left(x\right)}\;dx=\answer{-\frac{1}{2} \, \sqrt{3} + \frac{1}{2}}\]
\end{problem}}%}

%%%%%%%%%%%%%%%%%%%%%%


\latexProblemContent{
\begin{problem}

Use the Fundamental Theorem of Calculus to evaluate the integral.

\expandafter\input{\file@loc Integrals/2311-Compute-Integral-0012.HELP.tex}

\[\int_{\frac{2}{3} \, \pi}^{\frac{3}{2} \, \pi} {-7 \, \sin\left(x\right)}\;dx=\answer{\frac{7}{2}}\]
\end{problem}}%}

%%%%%%%%%%%%%%%%%%%%%%


\latexProblemContent{
\begin{problem}

Use the Fundamental Theorem of Calculus to evaluate the integral.

\expandafter\input{\file@loc Integrals/2311-Compute-Integral-0012.HELP.tex}

\[\int_{\frac{1}{2} \, \pi}^{\pi} {-5 \, \sin\left(x\right)}\;dx=\answer{-5}\]
\end{problem}}%}

%%%%%%%%%%%%%%%%%%%%%%


\latexProblemContent{
\begin{problem}

Use the Fundamental Theorem of Calculus to evaluate the integral.

\expandafter\input{\file@loc Integrals/2311-Compute-Integral-0012.HELP.tex}

\[\int_{\frac{1}{3} \, \pi}^{\frac{2}{3} \, \pi} {7 \, \sin\left(x\right)}\;dx=\answer{7}\]
\end{problem}}%}

%%%%%%%%%%%%%%%%%%%%%%


\latexProblemContent{
\begin{problem}

Use the Fundamental Theorem of Calculus to evaluate the integral.

\expandafter\input{\file@loc Integrals/2311-Compute-Integral-0012.HELP.tex}

\[\int_{\frac{1}{6} \, \pi}^{\frac{1}{2} \, \pi} {-\cos\left(x\right)}\;dx=\answer{-\frac{1}{2}}\]
\end{problem}}%}

%%%%%%%%%%%%%%%%%%%%%%


\latexProblemContent{
\begin{problem}

Use the Fundamental Theorem of Calculus to evaluate the integral.

\expandafter\input{\file@loc Integrals/2311-Compute-Integral-0012.HELP.tex}

\[\int_{\frac{3}{4} \, \pi}^{\frac{4}{3} \, \pi} {-6 \, \sin\left(x\right)}\;dx=\answer{3 \, \sqrt{2} - 3}\]
\end{problem}}%}

%%%%%%%%%%%%%%%%%%%%%%


\latexProblemContent{
\begin{problem}

Use the Fundamental Theorem of Calculus to evaluate the integral.

\expandafter\input{\file@loc Integrals/2311-Compute-Integral-0012.HELP.tex}

\[\int_{\frac{3}{4} \, \pi}^{\frac{5}{3} \, \pi} {9 \, \sin\left(x\right)}\;dx=\answer{-\frac{9}{2} \, \sqrt{2} - \frac{9}{2}}\]
\end{problem}}%}

%%%%%%%%%%%%%%%%%%%%%%


\latexProblemContent{
\begin{problem}

Use the Fundamental Theorem of Calculus to evaluate the integral.

\expandafter\input{\file@loc Integrals/2311-Compute-Integral-0012.HELP.tex}

\[\int_{\frac{3}{4} \, \pi}^{\frac{5}{4} \, \pi} {-6 \, \cos\left(x\right)}\;dx=\answer{3 \, \left(2 \, \sqrt{2}\right)}\]
\end{problem}}%}

%%%%%%%%%%%%%%%%%%%%%%


\latexProblemContent{
\begin{problem}

Use the Fundamental Theorem of Calculus to evaluate the integral.

\expandafter\input{\file@loc Integrals/2311-Compute-Integral-0012.HELP.tex}

\[\int_{\frac{1}{2} \, \pi}^{\frac{5}{3} \, \pi} {\cos\left(x\right)}\;dx=\answer{-\frac{1}{2} \, \sqrt{3} - 1}\]
\end{problem}}%}

%%%%%%%%%%%%%%%%%%%%%%


\latexProblemContent{
\begin{problem}

Use the Fundamental Theorem of Calculus to evaluate the integral.

\expandafter\input{\file@loc Integrals/2311-Compute-Integral-0012.HELP.tex}

\[\int_{\frac{1}{4} \, \pi}^{\pi} {6 \, \sin\left(x\right)}\;dx=\answer{3 \, \sqrt{2} + 6}\]
\end{problem}}%}

%%%%%%%%%%%%%%%%%%%%%%


\latexProblemContent{
\begin{problem}

Use the Fundamental Theorem of Calculus to evaluate the integral.

\expandafter\input{\file@loc Integrals/2311-Compute-Integral-0012.HELP.tex}

\[\int_{\frac{1}{3} \, \pi}^{\frac{3}{2} \, \pi} {-7 \, \sin\left(x\right)}\;dx=\answer{-\frac{7}{2}}\]
\end{problem}}%}

%%%%%%%%%%%%%%%%%%%%%%


%%%%%%%%%%%%%%%%%%%%%%


\latexProblemContent{
\begin{problem}

Use the Fundamental Theorem of Calculus to evaluate the integral.

\expandafter\input{\file@loc Integrals/2311-Compute-Integral-0012.HELP.tex}

\[\int_{\frac{5}{6} \, \pi}^{\pi} {2 \, \sin\left(x\right)}\;dx=\answer{-\sqrt{3} + 2}\]
\end{problem}}%}

%%%%%%%%%%%%%%%%%%%%%%


\latexProblemContent{
\begin{problem}

Use the Fundamental Theorem of Calculus to evaluate the integral.

\expandafter\input{\file@loc Integrals/2311-Compute-Integral-0012.HELP.tex}

\[\int_{\frac{5}{6} \, \pi}^{\frac{7}{6} \, \pi} {-7 \, \cos\left(x\right)}\;dx=\answer{7}\]
\end{problem}}%}

%%%%%%%%%%%%%%%%%%%%%%


\latexProblemContent{
\begin{problem}

Use the Fundamental Theorem of Calculus to evaluate the integral.

\expandafter\input{\file@loc Integrals/2311-Compute-Integral-0012.HELP.tex}

\[\int_{\frac{1}{3} \, \pi}^{\frac{7}{4} \, \pi} {-6 \, \cos\left(x\right)}\;dx=\answer{3 \, \sqrt{3} + 3 \, \sqrt{2}}\]
\end{problem}}%}

%%%%%%%%%%%%%%%%%%%%%%


\latexProblemContent{
\begin{problem}

Use the Fundamental Theorem of Calculus to evaluate the integral.

\expandafter\input{\file@loc Integrals/2311-Compute-Integral-0012.HELP.tex}

\[\int_{\frac{1}{6} \, \pi}^{\frac{1}{2} \, \pi} {-4 \, \cos\left(x\right)}\;dx=\answer{-2}\]
\end{problem}}%}

%%%%%%%%%%%%%%%%%%%%%%


%%%%%%%%%%%%%%%%%%%%%%


\latexProblemContent{
\begin{problem}

Use the Fundamental Theorem of Calculus to evaluate the integral.

\expandafter\input{\file@loc Integrals/2311-Compute-Integral-0012.HELP.tex}

\[\int_{\frac{5}{6} \, \pi}^{\frac{4}{3} \, \pi} {7 \, \sin\left(x\right)}\;dx=\answer{-\frac{7}{2} \, \sqrt{3} + \frac{7}{2}}\]
\end{problem}}%}

%%%%%%%%%%%%%%%%%%%%%%


\latexProblemContent{
\begin{problem}

Use the Fundamental Theorem of Calculus to evaluate the integral.

\expandafter\input{\file@loc Integrals/2311-Compute-Integral-0012.HELP.tex}

\[\int_{\frac{5}{6} \, \pi}^{\frac{3}{2} \, \pi} {-8 \, \cos\left(x\right)}\;dx=\answer{12}\]
\end{problem}}%}

%%%%%%%%%%%%%%%%%%%%%%


\latexProblemContent{
\begin{problem}

Use the Fundamental Theorem of Calculus to evaluate the integral.

\expandafter\input{\file@loc Integrals/2311-Compute-Integral-0012.HELP.tex}

\[\int_{\frac{3}{4} \, \pi}^{\frac{5}{4} \, \pi} {3 \, \sin\left(x\right)}\;dx=\answer{0}\]
\end{problem}}%}

%%%%%%%%%%%%%%%%%%%%%%


\latexProblemContent{
\begin{problem}

Use the Fundamental Theorem of Calculus to evaluate the integral.

\expandafter\input{\file@loc Integrals/2311-Compute-Integral-0012.HELP.tex}

\[\int_{\frac{1}{2} \, \pi}^{\frac{5}{4} \, \pi} {4 \, \sin\left(x\right)}\;dx=\answer{2^{\frac{3}{2}}}\]
\end{problem}}%}

%%%%%%%%%%%%%%%%%%%%%%


\latexProblemContent{
\begin{problem}

Use the Fundamental Theorem of Calculus to evaluate the integral.

\expandafter\input{\file@loc Integrals/2311-Compute-Integral-0012.HELP.tex}

\[\int_{\frac{1}{6} \, \pi}^{\frac{1}{2} \, \pi} {-9 \, \cos\left(x\right)}\;dx=\answer{-\frac{9}{2}}\]
\end{problem}}%}

%%%%%%%%%%%%%%%%%%%%%%


\latexProblemContent{
\begin{problem}

Use the Fundamental Theorem of Calculus to evaluate the integral.

\expandafter\input{\file@loc Integrals/2311-Compute-Integral-0012.HELP.tex}

\[\int_{\frac{1}{6} \, \pi}^{\frac{5}{6} \, \pi} {-2 \, \cos\left(x\right)}\;dx=\answer{0}\]
\end{problem}}%}

%%%%%%%%%%%%%%%%%%%%%%


\latexProblemContent{
\begin{problem}

Use the Fundamental Theorem of Calculus to evaluate the integral.

\expandafter\input{\file@loc Integrals/2311-Compute-Integral-0012.HELP.tex}

\[\int_{\frac{1}{4} \, \pi}^{\frac{5}{4} \, \pi} {-6 \, \sin\left(x\right)}\;dx=\answer{-3 \, \left(2 \, \sqrt{2}\right)}\]
\end{problem}}%}

%%%%%%%%%%%%%%%%%%%%%%


\latexProblemContent{
\begin{problem}

Use the Fundamental Theorem of Calculus to evaluate the integral.

\expandafter\input{\file@loc Integrals/2311-Compute-Integral-0012.HELP.tex}

\[\int_{\frac{3}{4} \, \pi}^{\frac{4}{3} \, \pi} {-9 \, \sin\left(x\right)}\;dx=\answer{\frac{9}{2} \, \sqrt{2} - \frac{9}{2}}\]
\end{problem}}%}

%%%%%%%%%%%%%%%%%%%%%%


\latexProblemContent{
\begin{problem}

Use the Fundamental Theorem of Calculus to evaluate the integral.

\expandafter\input{\file@loc Integrals/2311-Compute-Integral-0012.HELP.tex}

\[\int_{\frac{1}{6} \, \pi}^{\frac{3}{2} \, \pi} {4 \, \sin\left(x\right)}\;dx=\answer{2 \, \sqrt{3}}\]
\end{problem}}%}

%%%%%%%%%%%%%%%%%%%%%%


\latexProblemContent{
\begin{problem}

Use the Fundamental Theorem of Calculus to evaluate the integral.

\expandafter\input{\file@loc Integrals/2311-Compute-Integral-0012.HELP.tex}

\[\int_{\frac{5}{6} \, \pi}^{\frac{7}{6} \, \pi} {-6 \, \cos\left(x\right)}\;dx=\answer{6}\]
\end{problem}}%}

%%%%%%%%%%%%%%%%%%%%%%


\latexProblemContent{
\begin{problem}

Use the Fundamental Theorem of Calculus to evaluate the integral.

\expandafter\input{\file@loc Integrals/2311-Compute-Integral-0012.HELP.tex}

\[\int_{\frac{1}{3} \, \pi}^{\frac{3}{2} \, \pi} {-2 \, \sin\left(x\right)}\;dx=\answer{-1}\]
\end{problem}}%}

%%%%%%%%%%%%%%%%%%%%%%


\latexProblemContent{
\begin{problem}

Use the Fundamental Theorem of Calculus to evaluate the integral.

\expandafter\input{\file@loc Integrals/2311-Compute-Integral-0012.HELP.tex}

\[\int_{\frac{1}{2} \, \pi}^{\frac{3}{2} \, \pi} {-10 \, \sin\left(x\right)}\;dx=\answer{0}\]
\end{problem}}%}

%%%%%%%%%%%%%%%%%%%%%%


\latexProblemContent{
\begin{problem}

Use the Fundamental Theorem of Calculus to evaluate the integral.

\expandafter\input{\file@loc Integrals/2311-Compute-Integral-0012.HELP.tex}

\[\int_{\frac{2}{3} \, \pi}^{\frac{3}{2} \, \pi} {-8 \, \sin\left(x\right)}\;dx=\answer{4}\]
\end{problem}}%}

%%%%%%%%%%%%%%%%%%%%%%


\latexProblemContent{
\begin{problem}

Use the Fundamental Theorem of Calculus to evaluate the integral.

\expandafter\input{\file@loc Integrals/2311-Compute-Integral-0012.HELP.tex}

\[\int_{\frac{2}{3} \, \pi}^{\frac{3}{4} \, \pi} {\cos\left(x\right)}\;dx=\answer{-\frac{1}{2} \, \sqrt{3} + \frac{1}{2} \, \sqrt{2}}\]
\end{problem}}%}

%%%%%%%%%%%%%%%%%%%%%%


\latexProblemContent{
\begin{problem}

Use the Fundamental Theorem of Calculus to evaluate the integral.

\expandafter\input{\file@loc Integrals/2311-Compute-Integral-0012.HELP.tex}

\[\int_{\frac{1}{2} \, \pi}^{\frac{4}{3} \, \pi} {-9 \, \cos\left(x\right)}\;dx=\answer{\frac{9}{2} \, \sqrt{3} + 9}\]
\end{problem}}%}

%%%%%%%%%%%%%%%%%%%%%%


\latexProblemContent{
\begin{problem}

Use the Fundamental Theorem of Calculus to evaluate the integral.

\expandafter\input{\file@loc Integrals/2311-Compute-Integral-0012.HELP.tex}

\[\int_{\frac{1}{4} \, \pi}^{\frac{4}{3} \, \pi} {-6 \, \cos\left(x\right)}\;dx=\answer{3 \, \sqrt{3} + 3 \, \sqrt{2}}\]
\end{problem}}%}

%%%%%%%%%%%%%%%%%%%%%%


\latexProblemContent{
\begin{problem}

Use the Fundamental Theorem of Calculus to evaluate the integral.

\expandafter\input{\file@loc Integrals/2311-Compute-Integral-0012.HELP.tex}

\[\int_{\frac{1}{2} \, \pi}^{\frac{5}{4} \, \pi} {3 \, \cos\left(x\right)}\;dx=\answer{-\frac{3}{2} \, \sqrt{2} - 3}\]
\end{problem}}%}

%%%%%%%%%%%%%%%%%%%%%%


\latexProblemContent{
\begin{problem}

Use the Fundamental Theorem of Calculus to evaluate the integral.

\expandafter\input{\file@loc Integrals/2311-Compute-Integral-0012.HELP.tex}

\[\int_{\frac{1}{6} \, \pi}^{\frac{1}{2} \, \pi} {7 \, \cos\left(x\right)}\;dx=\answer{\frac{7}{2}}\]
\end{problem}}%}

%%%%%%%%%%%%%%%%%%%%%%


\latexProblemContent{
\begin{problem}

Use the Fundamental Theorem of Calculus to evaluate the integral.

\expandafter\input{\file@loc Integrals/2311-Compute-Integral-0012.HELP.tex}

\[\int_{\frac{1}{2} \, \pi}^{\frac{7}{4} \, \pi} {7 \, \cos\left(x\right)}\;dx=\answer{-\frac{7}{2} \, \sqrt{2} - 7}\]
\end{problem}}%}

%%%%%%%%%%%%%%%%%%%%%%


%%%%%%%%%%%%%%%%%%%%%%


\latexProblemContent{
\begin{problem}

Use the Fundamental Theorem of Calculus to evaluate the integral.

\expandafter\input{\file@loc Integrals/2311-Compute-Integral-0012.HELP.tex}

\[\int_{\frac{1}{6} \, \pi}^{\frac{1}{4} \, \pi} {-2 \, \cos\left(x\right)}\;dx=\answer{-\sqrt{2} + 1}\]
\end{problem}}%}

%%%%%%%%%%%%%%%%%%%%%%


\latexProblemContent{
\begin{problem}

Use the Fundamental Theorem of Calculus to evaluate the integral.

\expandafter\input{\file@loc Integrals/2311-Compute-Integral-0012.HELP.tex}

\[\int_{\frac{3}{4} \, \pi}^{\frac{5}{3} \, \pi} {2 \, \sin\left(x\right)}\;dx=\answer{-\sqrt{2} - 1}\]
\end{problem}}%}

%%%%%%%%%%%%%%%%%%%%%%


\latexProblemContent{
\begin{problem}

Use the Fundamental Theorem of Calculus to evaluate the integral.

\expandafter\input{\file@loc Integrals/2311-Compute-Integral-0012.HELP.tex}

\[\int_{\frac{1}{3} \, \pi}^{\frac{5}{6} \, \pi} {2 \, \cos\left(x\right)}\;dx=\answer{-\sqrt{3} + 1}\]
\end{problem}}%}

%%%%%%%%%%%%%%%%%%%%%%


\latexProblemContent{
\begin{problem}

Use the Fundamental Theorem of Calculus to evaluate the integral.

\expandafter\input{\file@loc Integrals/2311-Compute-Integral-0012.HELP.tex}

\[\int_{\frac{1}{2} \, \pi}^{\frac{7}{6} \, \pi} {10 \, \cos\left(x\right)}\;dx=\answer{-15}\]
\end{problem}}%}

%%%%%%%%%%%%%%%%%%%%%%


\latexProblemContent{
\begin{problem}

Use the Fundamental Theorem of Calculus to evaluate the integral.

\expandafter\input{\file@loc Integrals/2311-Compute-Integral-0012.HELP.tex}

\[\int_{\frac{1}{3} \, \pi}^{\frac{4}{3} \, \pi} {9 \, \sin\left(x\right)}\;dx=\answer{9}\]
\end{problem}}%}

%%%%%%%%%%%%%%%%%%%%%%


\latexProblemContent{
\begin{problem}

Use the Fundamental Theorem of Calculus to evaluate the integral.

\expandafter\input{\file@loc Integrals/2311-Compute-Integral-0012.HELP.tex}

\[\int_{\frac{1}{6} \, \pi}^{\frac{5}{3} \, \pi} {-2 \, \cos\left(x\right)}\;dx=\answer{\sqrt{3} + 1}\]
\end{problem}}%}

%%%%%%%%%%%%%%%%%%%%%%


\latexProblemContent{
\begin{problem}

Use the Fundamental Theorem of Calculus to evaluate the integral.

\expandafter\input{\file@loc Integrals/2311-Compute-Integral-0012.HELP.tex}

\[\int_{\frac{1}{6} \, \pi}^{\frac{5}{3} \, \pi} {-3 \, \sin\left(x\right)}\;dx=\answer{-\frac{3}{2} \, \sqrt{3} + \frac{3}{2}}\]
\end{problem}}%}

%%%%%%%%%%%%%%%%%%%%%%


\latexProblemContent{
\begin{problem}

Use the Fundamental Theorem of Calculus to evaluate the integral.

\expandafter\input{\file@loc Integrals/2311-Compute-Integral-0012.HELP.tex}

\[\int_{\frac{1}{2} \, \pi}^{\frac{4}{3} \, \pi} {4 \, \cos\left(x\right)}\;dx=\answer{-2 \, \sqrt{3} - 4}\]
\end{problem}}%}

%%%%%%%%%%%%%%%%%%%%%%


\latexProblemContent{
\begin{problem}

Use the Fundamental Theorem of Calculus to evaluate the integral.

\expandafter\input{\file@loc Integrals/2311-Compute-Integral-0012.HELP.tex}

\[\int_{\frac{5}{6} \, \pi}^{\frac{11}{6} \, \pi} {-3 \, \sin\left(x\right)}\;dx=\answer{3^{\frac{3}{2}}}\]
\end{problem}}%}

%%%%%%%%%%%%%%%%%%%%%%


\latexProblemContent{
\begin{problem}

Use the Fundamental Theorem of Calculus to evaluate the integral.

\expandafter\input{\file@loc Integrals/2311-Compute-Integral-0012.HELP.tex}

\[\int_{\frac{5}{6} \, \pi}^{\frac{5}{4} \, \pi} {-6 \, \cos\left(x\right)}\;dx=\answer{3 \, \sqrt{2} + 3}\]
\end{problem}}%}

%%%%%%%%%%%%%%%%%%%%%%


\latexProblemContent{
\begin{problem}

Use the Fundamental Theorem of Calculus to evaluate the integral.

\expandafter\input{\file@loc Integrals/2311-Compute-Integral-0012.HELP.tex}

\[\int_{\frac{1}{2} \, \pi}^{\frac{4}{3} \, \pi} {8 \, \sin\left(x\right)}\;dx=\answer{4}\]
\end{problem}}%}

%%%%%%%%%%%%%%%%%%%%%%


%%%%%%%%%%%%%%%%%%%%%%


\latexProblemContent{
\begin{problem}

Use the Fundamental Theorem of Calculus to evaluate the integral.

\expandafter\input{\file@loc Integrals/2311-Compute-Integral-0012.HELP.tex}

\[\int_{\frac{1}{6} \, \pi}^{\frac{3}{2} \, \pi} {5 \, \cos\left(x\right)}\;dx=\answer{-\frac{15}{2}}\]
\end{problem}}%}

%%%%%%%%%%%%%%%%%%%%%%


\latexProblemContent{
\begin{problem}

Use the Fundamental Theorem of Calculus to evaluate the integral.

\expandafter\input{\file@loc Integrals/2311-Compute-Integral-0012.HELP.tex}

\[\int_{\frac{1}{2} \, \pi}^{\frac{4}{3} \, \pi} {10 \, \cos\left(x\right)}\;dx=\answer{-5 \, \sqrt{3} - 10}\]
\end{problem}}%}

%%%%%%%%%%%%%%%%%%%%%%


\latexProblemContent{
\begin{problem}

Use the Fundamental Theorem of Calculus to evaluate the integral.

\expandafter\input{\file@loc Integrals/2311-Compute-Integral-0012.HELP.tex}

\[\int_{\frac{2}{3} \, \pi}^{\frac{4}{3} \, \pi} {-4 \, \sin\left(x\right)}\;dx=\answer{0}\]
\end{problem}}%}

%%%%%%%%%%%%%%%%%%%%%%


\latexProblemContent{
\begin{problem}

Use the Fundamental Theorem of Calculus to evaluate the integral.

\expandafter\input{\file@loc Integrals/2311-Compute-Integral-0012.HELP.tex}

\[\int_{\frac{1}{2} \, \pi}^{\frac{2}{3} \, \pi} {8 \, \sin\left(x\right)}\;dx=\answer{4}\]
\end{problem}}%}

%%%%%%%%%%%%%%%%%%%%%%


\latexProblemContent{
\begin{problem}

Use the Fundamental Theorem of Calculus to evaluate the integral.

\expandafter\input{\file@loc Integrals/2311-Compute-Integral-0012.HELP.tex}

\[\int_{\frac{1}{6} \, \pi}^{\frac{1}{4} \, \pi} {2 \, \cos\left(x\right)}\;dx=\answer{\sqrt{2} - 1}\]
\end{problem}}%}

%%%%%%%%%%%%%%%%%%%%%%


\latexProblemContent{
\begin{problem}

Use the Fundamental Theorem of Calculus to evaluate the integral.

\expandafter\input{\file@loc Integrals/2311-Compute-Integral-0012.HELP.tex}

\[\int_{\frac{1}{4} \, \pi}^{\frac{11}{6} \, \pi} {-7 \, \cos\left(x\right)}\;dx=\answer{\frac{7}{2} \, \sqrt{2} + \frac{7}{2}}\]
\end{problem}}%}

%%%%%%%%%%%%%%%%%%%%%%


\latexProblemContent{
\begin{problem}

Use the Fundamental Theorem of Calculus to evaluate the integral.

\expandafter\input{\file@loc Integrals/2311-Compute-Integral-0012.HELP.tex}

\[\int_{\frac{1}{3} \, \pi}^{\pi} {-\sin\left(x\right)}\;dx=\answer{-\frac{3}{2}}\]
\end{problem}}%}

%%%%%%%%%%%%%%%%%%%%%%


\latexProblemContent{
\begin{problem}

Use the Fundamental Theorem of Calculus to evaluate the integral.

\expandafter\input{\file@loc Integrals/2311-Compute-Integral-0012.HELP.tex}

\[\int_{\frac{1}{4} \, \pi}^{\frac{1}{2} \, \pi} {8 \, \cos\left(x\right)}\;dx=\answer{-4 \, \sqrt{2} + 8}\]
\end{problem}}%}

%%%%%%%%%%%%%%%%%%%%%%


\latexProblemContent{
\begin{problem}

Use the Fundamental Theorem of Calculus to evaluate the integral.

\expandafter\input{\file@loc Integrals/2311-Compute-Integral-0012.HELP.tex}

\[\int_{\frac{1}{4} \, \pi}^{\frac{3}{4} \, \pi} {-3 \, \cos\left(x\right)}\;dx=\answer{0}\]
\end{problem}}%}

%%%%%%%%%%%%%%%%%%%%%%


\latexProblemContent{
\begin{problem}

Use the Fundamental Theorem of Calculus to evaluate the integral.

\expandafter\input{\file@loc Integrals/2311-Compute-Integral-0012.HELP.tex}

\[\int_{\frac{1}{2} \, \pi}^{\pi} {8 \, \cos\left(x\right)}\;dx=\answer{-8}\]
\end{problem}}%}

%%%%%%%%%%%%%%%%%%%%%%


\latexProblemContent{
\begin{problem}

Use the Fundamental Theorem of Calculus to evaluate the integral.

\expandafter\input{\file@loc Integrals/2311-Compute-Integral-0012.HELP.tex}

\[\int_{\frac{1}{4} \, \pi}^{\frac{4}{3} \, \pi} {3 \, \sin\left(x\right)}\;dx=\answer{\frac{3}{2} \, \sqrt{2} + \frac{3}{2}}\]
\end{problem}}%}

%%%%%%%%%%%%%%%%%%%%%%


\latexProblemContent{
\begin{problem}

Use the Fundamental Theorem of Calculus to evaluate the integral.

\expandafter\input{\file@loc Integrals/2311-Compute-Integral-0012.HELP.tex}

\[\int_{\frac{1}{4} \, \pi}^{\frac{2}{3} \, \pi} {4 \, \cos\left(x\right)}\;dx=\answer{2 \, \sqrt{3} - 2 \, \sqrt{2}}\]
\end{problem}}%}

%%%%%%%%%%%%%%%%%%%%%%


\latexProblemContent{
\begin{problem}

Use the Fundamental Theorem of Calculus to evaluate the integral.

\expandafter\input{\file@loc Integrals/2311-Compute-Integral-0012.HELP.tex}

\[\int_{\frac{1}{3} \, \pi}^{\frac{3}{4} \, \pi} {3 \, \sin\left(x\right)}\;dx=\answer{\frac{3}{2} \, \sqrt{2} + \frac{3}{2}}\]
\end{problem}}%}

%%%%%%%%%%%%%%%%%%%%%%


\latexProblemContent{
\begin{problem}

Use the Fundamental Theorem of Calculus to evaluate the integral.

\expandafter\input{\file@loc Integrals/2311-Compute-Integral-0012.HELP.tex}

\[\int_{\frac{3}{4} \, \pi}^{\frac{4}{3} \, \pi} {8 \, \sin\left(x\right)}\;dx=\answer{-4 \, \sqrt{2} + 4}\]
\end{problem}}%}

%%%%%%%%%%%%%%%%%%%%%%


\latexProblemContent{
\begin{problem}

Use the Fundamental Theorem of Calculus to evaluate the integral.

\expandafter\input{\file@loc Integrals/2311-Compute-Integral-0012.HELP.tex}

\[\int_{\frac{1}{6} \, \pi}^{\frac{5}{4} \, \pi} {-3 \, \cos\left(x\right)}\;dx=\answer{\frac{3}{2} \, \sqrt{2} + \frac{3}{2}}\]
\end{problem}}%}

%%%%%%%%%%%%%%%%%%%%%%


\latexProblemContent{
\begin{problem}

Use the Fundamental Theorem of Calculus to evaluate the integral.

\expandafter\input{\file@loc Integrals/2311-Compute-Integral-0012.HELP.tex}

\[\int_{\frac{1}{6} \, \pi}^{\frac{5}{3} \, \pi} {6 \, \sin\left(x\right)}\;dx=\answer{3 \, \sqrt{3} - 3}\]
\end{problem}}%}

%%%%%%%%%%%%%%%%%%%%%%


\latexProblemContent{
\begin{problem}

Use the Fundamental Theorem of Calculus to evaluate the integral.

\expandafter\input{\file@loc Integrals/2311-Compute-Integral-0012.HELP.tex}

\[\int_{\frac{2}{3} \, \pi}^{\frac{4}{3} \, \pi} {5 \, \cos\left(x\right)}\;dx=\answer{-5 \, \sqrt{3}}\]
\end{problem}}%}

%%%%%%%%%%%%%%%%%%%%%%


\latexProblemContent{
\begin{problem}

Use the Fundamental Theorem of Calculus to evaluate the integral.

\expandafter\input{\file@loc Integrals/2311-Compute-Integral-0012.HELP.tex}

\[\int_{\frac{2}{3} \, \pi}^{\pi} {-8 \, \sin\left(x\right)}\;dx=\answer{-4}\]
\end{problem}}%}

%%%%%%%%%%%%%%%%%%%%%%


\latexProblemContent{
\begin{problem}

Use the Fundamental Theorem of Calculus to evaluate the integral.

\expandafter\input{\file@loc Integrals/2311-Compute-Integral-0012.HELP.tex}

\[\int_{\frac{2}{3} \, \pi}^{\frac{11}{6} \, \pi} {8 \, \cos\left(x\right)}\;dx=\answer{-4 \, \sqrt{3} - 4}\]
\end{problem}}%}

%%%%%%%%%%%%%%%%%%%%%%


\latexProblemContent{
\begin{problem}

Use the Fundamental Theorem of Calculus to evaluate the integral.

\expandafter\input{\file@loc Integrals/2311-Compute-Integral-0012.HELP.tex}

\[\int_{\frac{5}{6} \, \pi}^{\frac{5}{3} \, \pi} {2 \, \sin\left(x\right)}\;dx=\answer{-\sqrt{3} - 1}\]
\end{problem}}%}

%%%%%%%%%%%%%%%%%%%%%%


\latexProblemContent{
\begin{problem}

Use the Fundamental Theorem of Calculus to evaluate the integral.

\expandafter\input{\file@loc Integrals/2311-Compute-Integral-0012.HELP.tex}

\[\int_{\frac{1}{6} \, \pi}^{\frac{11}{6} \, \pi} {3 \, \cos\left(x\right)}\;dx=\answer{-3}\]
\end{problem}}%}

%%%%%%%%%%%%%%%%%%%%%%


\latexProblemContent{
\begin{problem}

Use the Fundamental Theorem of Calculus to evaluate the integral.

\expandafter\input{\file@loc Integrals/2311-Compute-Integral-0012.HELP.tex}

\[\int_{\frac{1}{4} \, \pi}^{\frac{5}{4} \, \pi} {-5 \, \cos\left(x\right)}\;dx=\answer{5 \, \sqrt{2}}\]
\end{problem}}%}

%%%%%%%%%%%%%%%%%%%%%%


\latexProblemContent{
\begin{problem}

Use the Fundamental Theorem of Calculus to evaluate the integral.

\expandafter\input{\file@loc Integrals/2311-Compute-Integral-0012.HELP.tex}

\[\int_{\frac{5}{6} \, \pi}^{\frac{7}{4} \, \pi} {-8 \, \sin\left(x\right)}\;dx=\answer{4 \, \sqrt{3} + 4 \, \sqrt{2}}\]
\end{problem}}%}

%%%%%%%%%%%%%%%%%%%%%%


\latexProblemContent{
\begin{problem}

Use the Fundamental Theorem of Calculus to evaluate the integral.

\expandafter\input{\file@loc Integrals/2311-Compute-Integral-0012.HELP.tex}

\[\int_{\frac{3}{4} \, \pi}^{\pi} {-9 \, \cos\left(x\right)}\;dx=\answer{\frac{9}{2} \, \sqrt{2}}\]
\end{problem}}%}

%%%%%%%%%%%%%%%%%%%%%%


\latexProblemContent{
\begin{problem}

Use the Fundamental Theorem of Calculus to evaluate the integral.

\expandafter\input{\file@loc Integrals/2311-Compute-Integral-0012.HELP.tex}

\[\int_{\frac{1}{4} \, \pi}^{\frac{7}{4} \, \pi} {-2 \, \sin\left(x\right)}\;dx=\answer{0}\]
\end{problem}}%}

%%%%%%%%%%%%%%%%%%%%%%


%%%%%%%%%%%%%%%%%%%%%%


\latexProblemContent{
\begin{problem}

Use the Fundamental Theorem of Calculus to evaluate the integral.

\expandafter\input{\file@loc Integrals/2311-Compute-Integral-0012.HELP.tex}

\[\int_{\frac{1}{3} \, \pi}^{\frac{4}{3} \, \pi} {-10 \, \cos\left(x\right)}\;dx=\answer{10 \, \sqrt{3}}\]
\end{problem}}%}

%%%%%%%%%%%%%%%%%%%%%%


\latexProblemContent{
\begin{problem}

Use the Fundamental Theorem of Calculus to evaluate the integral.

\expandafter\input{\file@loc Integrals/2311-Compute-Integral-0012.HELP.tex}

\[\int_{\frac{1}{3} \, \pi}^{\frac{7}{4} \, \pi} {10 \, \cos\left(x\right)}\;dx=\answer{-5 \, \sqrt{3} - 5 \, \sqrt{2}}\]
\end{problem}}%}

%%%%%%%%%%%%%%%%%%%%%%


\latexProblemContent{
\begin{problem}

Use the Fundamental Theorem of Calculus to evaluate the integral.

\expandafter\input{\file@loc Integrals/2311-Compute-Integral-0012.HELP.tex}

\[\int_{\frac{1}{2} \, \pi}^{\pi} {-9 \, \sin\left(x\right)}\;dx=\answer{-9}\]
\end{problem}}%}

%%%%%%%%%%%%%%%%%%%%%%


\latexProblemContent{
\begin{problem}

Use the Fundamental Theorem of Calculus to evaluate the integral.

\expandafter\input{\file@loc Integrals/2311-Compute-Integral-0012.HELP.tex}

\[\int_{\frac{2}{3} \, \pi}^{\frac{3}{4} \, \pi} {-\sin\left(x\right)}\;dx=\answer{-\frac{1}{2} \, \sqrt{2} + \frac{1}{2}}\]
\end{problem}}%}

%%%%%%%%%%%%%%%%%%%%%%


\latexProblemContent{
\begin{problem}

Use the Fundamental Theorem of Calculus to evaluate the integral.

\expandafter\input{\file@loc Integrals/2311-Compute-Integral-0012.HELP.tex}

\[\int_{\frac{1}{2} \, \pi}^{\frac{3}{2} \, \pi} {10 \, \sin\left(x\right)}\;dx=\answer{0}\]
\end{problem}}%}

%%%%%%%%%%%%%%%%%%%%%%


\latexProblemContent{
\begin{problem}

Use the Fundamental Theorem of Calculus to evaluate the integral.

\expandafter\input{\file@loc Integrals/2311-Compute-Integral-0012.HELP.tex}

\[\int_{\frac{1}{3} \, \pi}^{\frac{3}{4} \, \pi} {-4 \, \cos\left(x\right)}\;dx=\answer{2 \, \sqrt{3} - 2 \, \sqrt{2}}\]
\end{problem}}%}

%%%%%%%%%%%%%%%%%%%%%%


\latexProblemContent{
\begin{problem}

Use the Fundamental Theorem of Calculus to evaluate the integral.

\expandafter\input{\file@loc Integrals/2311-Compute-Integral-0012.HELP.tex}

\[\int_{\frac{2}{3} \, \pi}^{\pi} {5 \, \sin\left(x\right)}\;dx=\answer{\frac{5}{2}}\]
\end{problem}}%}

%%%%%%%%%%%%%%%%%%%%%%


\latexProblemContent{
\begin{problem}

Use the Fundamental Theorem of Calculus to evaluate the integral.

\expandafter\input{\file@loc Integrals/2311-Compute-Integral-0012.HELP.tex}

\[\int_{\frac{1}{2} \, \pi}^{\frac{4}{3} \, \pi} {\cos\left(x\right)}\;dx=\answer{-\frac{1}{2} \, \sqrt{3} - 1}\]
\end{problem}}%}

%%%%%%%%%%%%%%%%%%%%%%


\latexProblemContent{
\begin{problem}

Use the Fundamental Theorem of Calculus to evaluate the integral.

\expandafter\input{\file@loc Integrals/2311-Compute-Integral-0012.HELP.tex}

\[\int_{\frac{1}{4} \, \pi}^{\pi} {7 \, \sin\left(x\right)}\;dx=\answer{\frac{7}{2} \, \sqrt{2} + 7}\]
\end{problem}}%}

%%%%%%%%%%%%%%%%%%%%%%


\latexProblemContent{
\begin{problem}

Use the Fundamental Theorem of Calculus to evaluate the integral.

\expandafter\input{\file@loc Integrals/2311-Compute-Integral-0012.HELP.tex}

\[\int_{\frac{1}{3} \, \pi}^{\frac{5}{4} \, \pi} {-8 \, \cos\left(x\right)}\;dx=\answer{4 \, \sqrt{3} + 4 \, \sqrt{2}}\]
\end{problem}}%}

%%%%%%%%%%%%%%%%%%%%%%


\latexProblemContent{
\begin{problem}

Use the Fundamental Theorem of Calculus to evaluate the integral.

\expandafter\input{\file@loc Integrals/2311-Compute-Integral-0012.HELP.tex}

\[\int_{\frac{2}{3} \, \pi}^{\frac{11}{6} \, \pi} {7 \, \cos\left(x\right)}\;dx=\answer{-\frac{7}{2} \, \sqrt{3} - \frac{7}{2}}\]
\end{problem}}%}

%%%%%%%%%%%%%%%%%%%%%%


\latexProblemContent{
\begin{problem}

Use the Fundamental Theorem of Calculus to evaluate the integral.

\expandafter\input{\file@loc Integrals/2311-Compute-Integral-0012.HELP.tex}

\[\int_{\frac{1}{6} \, \pi}^{\frac{5}{6} \, \pi} {3 \, \cos\left(x\right)}\;dx=\answer{0}\]
\end{problem}}%}

%%%%%%%%%%%%%%%%%%%%%%


\latexProblemContent{
\begin{problem}

Use the Fundamental Theorem of Calculus to evaluate the integral.

\expandafter\input{\file@loc Integrals/2311-Compute-Integral-0012.HELP.tex}

\[\int_{\frac{1}{2} \, \pi}^{\frac{7}{4} \, \pi} {8 \, \sin\left(x\right)}\;dx=\answer{-\left(4 \, \sqrt{2}\right)}\]
\end{problem}}%}

%%%%%%%%%%%%%%%%%%%%%%


\latexProblemContent{
\begin{problem}

Use the Fundamental Theorem of Calculus to evaluate the integral.

\expandafter\input{\file@loc Integrals/2311-Compute-Integral-0012.HELP.tex}

\[\int_{\frac{1}{6} \, \pi}^{\frac{3}{2} \, \pi} {10 \, \sin\left(x\right)}\;dx=\answer{5 \, \sqrt{3}}\]
\end{problem}}%}

%%%%%%%%%%%%%%%%%%%%%%


\latexProblemContent{
\begin{problem}

Use the Fundamental Theorem of Calculus to evaluate the integral.

\expandafter\input{\file@loc Integrals/2311-Compute-Integral-0012.HELP.tex}

\[\int_{\frac{3}{4} \, \pi}^{\frac{3}{2} \, \pi} {-6 \, \cos\left(x\right)}\;dx=\answer{3 \, \sqrt{2} + 6}\]
\end{problem}}%}

%%%%%%%%%%%%%%%%%%%%%%


\latexProblemContent{
\begin{problem}

Use the Fundamental Theorem of Calculus to evaluate the integral.

\expandafter\input{\file@loc Integrals/2311-Compute-Integral-0012.HELP.tex}

\[\int_{\frac{5}{6} \, \pi}^{\frac{7}{6} \, \pi} {-9 \, \sin\left(x\right)}\;dx=\answer{0}\]
\end{problem}}%}

%%%%%%%%%%%%%%%%%%%%%%


%%%%%%%%%%%%%%%%%%%%%%


\latexProblemContent{
\begin{problem}

Use the Fundamental Theorem of Calculus to evaluate the integral.

\expandafter\input{\file@loc Integrals/2311-Compute-Integral-0012.HELP.tex}

\[\int_{\frac{1}{6} \, \pi}^{\pi} {6 \, \sin\left(x\right)}\;dx=\answer{3 \, \sqrt{3} + 6}\]
\end{problem}}%}

%%%%%%%%%%%%%%%%%%%%%%


%%%%%%%%%%%%%%%%%%%%%%


\latexProblemContent{
\begin{problem}

Use the Fundamental Theorem of Calculus to evaluate the integral.

\expandafter\input{\file@loc Integrals/2311-Compute-Integral-0012.HELP.tex}

\[\int_{\frac{1}{4} \, \pi}^{\frac{11}{6} \, \pi} {-6 \, \cos\left(x\right)}\;dx=\answer{3 \, \sqrt{2} + 3}\]
\end{problem}}%}

%%%%%%%%%%%%%%%%%%%%%%


\latexProblemContent{
\begin{problem}

Use the Fundamental Theorem of Calculus to evaluate the integral.

\expandafter\input{\file@loc Integrals/2311-Compute-Integral-0012.HELP.tex}

\[\int_{\frac{3}{4} \, \pi}^{\frac{4}{3} \, \pi} {4 \, \cos\left(x\right)}\;dx=\answer{-2 \, \sqrt{3} - 2 \, \sqrt{2}}\]
\end{problem}}%}

%%%%%%%%%%%%%%%%%%%%%%


%%%%%%%%%%%%%%%%%%%%%%


\latexProblemContent{
\begin{problem}

Use the Fundamental Theorem of Calculus to evaluate the integral.

\expandafter\input{\file@loc Integrals/2311-Compute-Integral-0012.HELP.tex}

\[\int_{\frac{1}{3} \, \pi}^{\frac{7}{6} \, \pi} {-6 \, \sin\left(x\right)}\;dx=\answer{-3 \, \sqrt{3} - 3}\]
\end{problem}}%}

%%%%%%%%%%%%%%%%%%%%%%


\latexProblemContent{
\begin{problem}

Use the Fundamental Theorem of Calculus to evaluate the integral.

\expandafter\input{\file@loc Integrals/2311-Compute-Integral-0012.HELP.tex}

\[\int_{\frac{1}{3} \, \pi}^{\frac{5}{3} \, \pi} {7 \, \cos\left(x\right)}\;dx=\answer{-7 \, \sqrt{3}}\]
\end{problem}}%}

%%%%%%%%%%%%%%%%%%%%%%


\latexProblemContent{
\begin{problem}

Use the Fundamental Theorem of Calculus to evaluate the integral.

\expandafter\input{\file@loc Integrals/2311-Compute-Integral-0012.HELP.tex}

\[\int_{\frac{1}{6} \, \pi}^{\frac{2}{3} \, \pi} {3 \, \cos\left(x\right)}\;dx=\answer{\frac{3}{2} \, \sqrt{3} - \frac{3}{2}}\]
\end{problem}}%}

%%%%%%%%%%%%%%%%%%%%%%


\latexProblemContent{
\begin{problem}

Use the Fundamental Theorem of Calculus to evaluate the integral.

\expandafter\input{\file@loc Integrals/2311-Compute-Integral-0012.HELP.tex}

\[\int_{\frac{5}{6} \, \pi}^{\frac{11}{6} \, \pi} {7 \, \sin\left(x\right)}\;dx=\answer{-7 \, \sqrt{3}}\]
\end{problem}}%}

%%%%%%%%%%%%%%%%%%%%%%


\latexProblemContent{
\begin{problem}

Use the Fundamental Theorem of Calculus to evaluate the integral.

\expandafter\input{\file@loc Integrals/2311-Compute-Integral-0012.HELP.tex}

\[\int_{\frac{2}{3} \, \pi}^{\frac{7}{4} \, \pi} {3 \, \cos\left(x\right)}\;dx=\answer{-\frac{3}{2} \, \sqrt{3} - \frac{3}{2} \, \sqrt{2}}\]
\end{problem}}%}

%%%%%%%%%%%%%%%%%%%%%%


\latexProblemContent{
\begin{problem}

Use the Fundamental Theorem of Calculus to evaluate the integral.

\expandafter\input{\file@loc Integrals/2311-Compute-Integral-0012.HELP.tex}

\[\int_{\frac{1}{2} \, \pi}^{\pi} {2 \, \cos\left(x\right)}\;dx=\answer{-2}\]
\end{problem}}%}

%%%%%%%%%%%%%%%%%%%%%%


\latexProblemContent{
\begin{problem}

Use the Fundamental Theorem of Calculus to evaluate the integral.

\expandafter\input{\file@loc Integrals/2311-Compute-Integral-0012.HELP.tex}

\[\int_{\frac{1}{6} \, \pi}^{\frac{7}{4} \, \pi} {-3 \, \sin\left(x\right)}\;dx=\answer{-\frac{3}{2} \, \sqrt{3} + \frac{3}{2} \, \sqrt{2}}\]
\end{problem}}%}

%%%%%%%%%%%%%%%%%%%%%%


\latexProblemContent{
\begin{problem}

Use the Fundamental Theorem of Calculus to evaluate the integral.

\expandafter\input{\file@loc Integrals/2311-Compute-Integral-0012.HELP.tex}

\[\int_{\frac{2}{3} \, \pi}^{\frac{5}{4} \, \pi} {5 \, \sin\left(x\right)}\;dx=\answer{\frac{5}{2} \, \sqrt{2} - \frac{5}{2}}\]
\end{problem}}%}

%%%%%%%%%%%%%%%%%%%%%%


\latexProblemContent{
\begin{problem}

Use the Fundamental Theorem of Calculus to evaluate the integral.

\expandafter\input{\file@loc Integrals/2311-Compute-Integral-0012.HELP.tex}

\[\int_{\frac{3}{4} \, \pi}^{\frac{5}{6} \, \pi} {-7 \, \cos\left(x\right)}\;dx=\answer{\frac{7}{2} \, \sqrt{2} - \frac{7}{2}}\]
\end{problem}}%}

%%%%%%%%%%%%%%%%%%%%%%


\latexProblemContent{
\begin{problem}

Use the Fundamental Theorem of Calculus to evaluate the integral.

\expandafter\input{\file@loc Integrals/2311-Compute-Integral-0012.HELP.tex}

\[\int_{\frac{1}{6} \, \pi}^{\frac{7}{6} \, \pi} {8 \, \sin\left(x\right)}\;dx=\answer{8 \, \sqrt{3}}\]
\end{problem}}%}

%%%%%%%%%%%%%%%%%%%%%%


\latexProblemContent{
\begin{problem}

Use the Fundamental Theorem of Calculus to evaluate the integral.

\expandafter\input{\file@loc Integrals/2311-Compute-Integral-0012.HELP.tex}

\[\int_{\frac{3}{4} \, \pi}^{\pi} {4 \, \sin\left(x\right)}\;dx=\answer{-2 \, \sqrt{2} + 4}\]
\end{problem}}%}

%%%%%%%%%%%%%%%%%%%%%%


\latexProblemContent{
\begin{problem}

Use the Fundamental Theorem of Calculus to evaluate the integral.

\expandafter\input{\file@loc Integrals/2311-Compute-Integral-0012.HELP.tex}

\[\int_{\frac{1}{6} \, \pi}^{\frac{11}{6} \, \pi} {-7 \, \cos\left(x\right)}\;dx=\answer{7}\]
\end{problem}}%}

%%%%%%%%%%%%%%%%%%%%%%


\latexProblemContent{
\begin{problem}

Use the Fundamental Theorem of Calculus to evaluate the integral.

\expandafter\input{\file@loc Integrals/2311-Compute-Integral-0012.HELP.tex}

\[\int_{\frac{1}{6} \, \pi}^{\frac{1}{3} \, \pi} {6 \, \cos\left(x\right)}\;dx=\answer{3 \, \sqrt{3} - 3}\]
\end{problem}}%}

%%%%%%%%%%%%%%%%%%%%%%


\latexProblemContent{
\begin{problem}

Use the Fundamental Theorem of Calculus to evaluate the integral.

\expandafter\input{\file@loc Integrals/2311-Compute-Integral-0012.HELP.tex}

\[\int_{\frac{1}{6} \, \pi}^{\frac{5}{3} \, \pi} {2 \, \cos\left(x\right)}\;dx=\answer{-\sqrt{3} - 1}\]
\end{problem}}%}

%%%%%%%%%%%%%%%%%%%%%%


%%%%%%%%%%%%%%%%%%%%%%


\latexProblemContent{
\begin{problem}

Use the Fundamental Theorem of Calculus to evaluate the integral.

\expandafter\input{\file@loc Integrals/2311-Compute-Integral-0012.HELP.tex}

\[\int_{\frac{1}{2} \, \pi}^{\frac{7}{4} \, \pi} {-9 \, \sin\left(x\right)}\;dx=\answer{\frac{9}{2} \, \sqrt{2}}\]
\end{problem}}%}

%%%%%%%%%%%%%%%%%%%%%%


\latexProblemContent{
\begin{problem}

Use the Fundamental Theorem of Calculus to evaluate the integral.

\expandafter\input{\file@loc Integrals/2311-Compute-Integral-0012.HELP.tex}

\[\int_{\frac{1}{2} \, \pi}^{\frac{11}{6} \, \pi} {-2 \, \sin\left(x\right)}\;dx=\answer{\sqrt{3}}\]
\end{problem}}%}

%%%%%%%%%%%%%%%%%%%%%%


\latexProblemContent{
\begin{problem}

Use the Fundamental Theorem of Calculus to evaluate the integral.

\expandafter\input{\file@loc Integrals/2311-Compute-Integral-0012.HELP.tex}

\[\int_{\frac{1}{6} \, \pi}^{\frac{4}{3} \, \pi} {8 \, \cos\left(x\right)}\;dx=\answer{-4 \, \sqrt{3} - 4}\]
\end{problem}}%}

%%%%%%%%%%%%%%%%%%%%%%


\latexProblemContent{
\begin{problem}

Use the Fundamental Theorem of Calculus to evaluate the integral.

\expandafter\input{\file@loc Integrals/2311-Compute-Integral-0012.HELP.tex}

\[\int_{\frac{5}{6} \, \pi}^{\frac{5}{3} \, \pi} {2 \, \cos\left(x\right)}\;dx=\answer{-\sqrt{3} - 1}\]
\end{problem}}%}

%%%%%%%%%%%%%%%%%%%%%%


\latexProblemContent{
\begin{problem}

Use the Fundamental Theorem of Calculus to evaluate the integral.

\expandafter\input{\file@loc Integrals/2311-Compute-Integral-0012.HELP.tex}

\[\int_{\frac{1}{3} \, \pi}^{\frac{3}{4} \, \pi} {9 \, \cos\left(x\right)}\;dx=\answer{-\frac{9}{2} \, \sqrt{3} + \frac{9}{2} \, \sqrt{2}}\]
\end{problem}}%}

%%%%%%%%%%%%%%%%%%%%%%


\latexProblemContent{
\begin{problem}

Use the Fundamental Theorem of Calculus to evaluate the integral.

\expandafter\input{\file@loc Integrals/2311-Compute-Integral-0012.HELP.tex}

\[\int_{\frac{1}{3} \, \pi}^{\frac{4}{3} \, \pi} {2 \, \sin\left(x\right)}\;dx=\answer{2}\]
\end{problem}}%}

%%%%%%%%%%%%%%%%%%%%%%


\latexProblemContent{
\begin{problem}

Use the Fundamental Theorem of Calculus to evaluate the integral.

\expandafter\input{\file@loc Integrals/2311-Compute-Integral-0012.HELP.tex}

\[\int_{\frac{3}{4} \, \pi}^{\frac{3}{2} \, \pi} {-9 \, \sin\left(x\right)}\;dx=\answer{\frac{9}{2} \, \sqrt{2}}\]
\end{problem}}%}

%%%%%%%%%%%%%%%%%%%%%%


\latexProblemContent{
\begin{problem}

Use the Fundamental Theorem of Calculus to evaluate the integral.

\expandafter\input{\file@loc Integrals/2311-Compute-Integral-0012.HELP.tex}

\[\int_{\frac{1}{2} \, \pi}^{\frac{3}{2} \, \pi} {10 \, \cos\left(x\right)}\;dx=\answer{-20}\]
\end{problem}}%}

%%%%%%%%%%%%%%%%%%%%%%


\latexProblemContent{
\begin{problem}

Use the Fundamental Theorem of Calculus to evaluate the integral.

\expandafter\input{\file@loc Integrals/2311-Compute-Integral-0012.HELP.tex}

\[\int_{\frac{3}{4} \, \pi}^{\frac{5}{6} \, \pi} {-2 \, \sin\left(x\right)}\;dx=\answer{-\sqrt{3} + \sqrt{2}}\]
\end{problem}}%}

%%%%%%%%%%%%%%%%%%%%%%


\latexProblemContent{
\begin{problem}

Use the Fundamental Theorem of Calculus to evaluate the integral.

\expandafter\input{\file@loc Integrals/2311-Compute-Integral-0012.HELP.tex}

\[\int_{\frac{3}{4} \, \pi}^{\frac{4}{3} \, \pi} {6 \, \cos\left(x\right)}\;dx=\answer{-3 \, \sqrt{3} - 3 \, \sqrt{2}}\]
\end{problem}}%}

%%%%%%%%%%%%%%%%%%%%%%


\latexProblemContent{
\begin{problem}

Use the Fundamental Theorem of Calculus to evaluate the integral.

\expandafter\input{\file@loc Integrals/2311-Compute-Integral-0012.HELP.tex}

\[\int_{\frac{1}{4} \, \pi}^{\frac{1}{2} \, \pi} {-3 \, \cos\left(x\right)}\;dx=\answer{\frac{3}{2} \, \sqrt{2} - 3}\]
\end{problem}}%}

%%%%%%%%%%%%%%%%%%%%%%


%%%%%%%%%%%%%%%%%%%%%%


\latexProblemContent{
\begin{problem}

Use the Fundamental Theorem of Calculus to evaluate the integral.

\expandafter\input{\file@loc Integrals/2311-Compute-Integral-0012.HELP.tex}

\[\int_{\frac{3}{4} \, \pi}^{\pi} {8 \, \cos\left(x\right)}\;dx=\answer{-\left(4 \, \sqrt{2}\right)}\]
\end{problem}}%}

%%%%%%%%%%%%%%%%%%%%%%


\latexProblemContent{
\begin{problem}

Use the Fundamental Theorem of Calculus to evaluate the integral.

\expandafter\input{\file@loc Integrals/2311-Compute-Integral-0012.HELP.tex}

\[\int_{\frac{3}{4} \, \pi}^{\frac{4}{3} \, \pi} {10 \, \cos\left(x\right)}\;dx=\answer{-5 \, \sqrt{3} - 5 \, \sqrt{2}}\]
\end{problem}}%}

%%%%%%%%%%%%%%%%%%%%%%


\latexProblemContent{
\begin{problem}

Use the Fundamental Theorem of Calculus to evaluate the integral.

\expandafter\input{\file@loc Integrals/2311-Compute-Integral-0012.HELP.tex}

\[\int_{\frac{1}{6} \, \pi}^{\frac{5}{6} \, \pi} {8 \, \sin\left(x\right)}\;dx=\answer{8 \, \sqrt{3}}\]
\end{problem}}%}

%%%%%%%%%%%%%%%%%%%%%%


\latexProblemContent{
\begin{problem}

Use the Fundamental Theorem of Calculus to evaluate the integral.

\expandafter\input{\file@loc Integrals/2311-Compute-Integral-0012.HELP.tex}

\[\int_{\frac{5}{6} \, \pi}^{\frac{5}{3} \, \pi} {-10 \, \cos\left(x\right)}\;dx=\answer{5 \, \sqrt{3} + 5}\]
\end{problem}}%}

%%%%%%%%%%%%%%%%%%%%%%


%%%%%%%%%%%%%%%%%%%%%%


\latexProblemContent{
\begin{problem}

Use the Fundamental Theorem of Calculus to evaluate the integral.

\expandafter\input{\file@loc Integrals/2311-Compute-Integral-0012.HELP.tex}

\[\int_{\frac{2}{3} \, \pi}^{\frac{5}{6} \, \pi} {3 \, \sin\left(x\right)}\;dx=\answer{\frac{3}{2} \, \sqrt{3} - \frac{3}{2}}\]
\end{problem}}%}

%%%%%%%%%%%%%%%%%%%%%%


\latexProblemContent{
\begin{problem}

Use the Fundamental Theorem of Calculus to evaluate the integral.

\expandafter\input{\file@loc Integrals/2311-Compute-Integral-0012.HELP.tex}

\[\int_{\frac{1}{6} \, \pi}^{\frac{1}{3} \, \pi} {-4 \, \sin\left(x\right)}\;dx=\answer{-2 \, \sqrt{3} + 2}\]
\end{problem}}%}

%%%%%%%%%%%%%%%%%%%%%%


\latexProblemContent{
\begin{problem}

Use the Fundamental Theorem of Calculus to evaluate the integral.

\expandafter\input{\file@loc Integrals/2311-Compute-Integral-0012.HELP.tex}

\[\int_{\frac{5}{6} \, \pi}^{\frac{3}{2} \, \pi} {-5 \, \cos\left(x\right)}\;dx=\answer{\frac{15}{2}}\]
\end{problem}}%}

%%%%%%%%%%%%%%%%%%%%%%


\latexProblemContent{
\begin{problem}

Use the Fundamental Theorem of Calculus to evaluate the integral.

\expandafter\input{\file@loc Integrals/2311-Compute-Integral-0012.HELP.tex}

\[\int_{\frac{1}{4} \, \pi}^{\frac{3}{4} \, \pi} {-7 \, \sin\left(x\right)}\;dx=\answer{-7 \, \sqrt{2}}\]
\end{problem}}%}

%%%%%%%%%%%%%%%%%%%%%%


\latexProblemContent{
\begin{problem}

Use the Fundamental Theorem of Calculus to evaluate the integral.

\expandafter\input{\file@loc Integrals/2311-Compute-Integral-0012.HELP.tex}

\[\int_{\frac{1}{2} \, \pi}^{\frac{5}{6} \, \pi} {-\cos\left(x\right)}\;dx=\answer{\frac{1}{2}}\]
\end{problem}}%}

%%%%%%%%%%%%%%%%%%%%%%


\latexProblemContent{
\begin{problem}

Use the Fundamental Theorem of Calculus to evaluate the integral.

\expandafter\input{\file@loc Integrals/2311-Compute-Integral-0012.HELP.tex}

\[\int_{\frac{1}{6} \, \pi}^{\frac{1}{4} \, \pi} {3 \, \cos\left(x\right)}\;dx=\answer{\frac{3}{2} \, \sqrt{2} - \frac{3}{2}}\]
\end{problem}}%}

%%%%%%%%%%%%%%%%%%%%%%


\latexProblemContent{
\begin{problem}

Use the Fundamental Theorem of Calculus to evaluate the integral.

\expandafter\input{\file@loc Integrals/2311-Compute-Integral-0012.HELP.tex}

\[\int_{\frac{1}{2} \, \pi}^{\frac{3}{4} \, \pi} {3 \, \sin\left(x\right)}\;dx=\answer{\frac{3}{2} \, \sqrt{2}}\]
\end{problem}}%}

%%%%%%%%%%%%%%%%%%%%%%


\latexProblemContent{
\begin{problem}

Use the Fundamental Theorem of Calculus to evaluate the integral.

\expandafter\input{\file@loc Integrals/2311-Compute-Integral-0012.HELP.tex}

\[\int_{\frac{5}{6} \, \pi}^{\frac{4}{3} \, \pi} {3 \, \cos\left(x\right)}\;dx=\answer{-\frac{3}{2} \, \sqrt{3} - \frac{3}{2}}\]
\end{problem}}%}

%%%%%%%%%%%%%%%%%%%%%%


\latexProblemContent{
\begin{problem}

Use the Fundamental Theorem of Calculus to evaluate the integral.

\expandafter\input{\file@loc Integrals/2311-Compute-Integral-0012.HELP.tex}

\[\int_{\frac{2}{3} \, \pi}^{\frac{5}{3} \, \pi} {6 \, \sin\left(x\right)}\;dx=\answer{-6}\]
\end{problem}}%}

%%%%%%%%%%%%%%%%%%%%%%


\latexProblemContent{
\begin{problem}

Use the Fundamental Theorem of Calculus to evaluate the integral.

\expandafter\input{\file@loc Integrals/2311-Compute-Integral-0012.HELP.tex}

\[\int_{\frac{1}{4} \, \pi}^{\frac{3}{2} \, \pi} {-10 \, \sin\left(x\right)}\;dx=\answer{-5 \, \sqrt{2}}\]
\end{problem}}%}

%%%%%%%%%%%%%%%%%%%%%%


\latexProblemContent{
\begin{problem}

Use the Fundamental Theorem of Calculus to evaluate the integral.

\expandafter\input{\file@loc Integrals/2311-Compute-Integral-0012.HELP.tex}

\[\int_{\frac{1}{4} \, \pi}^{\frac{1}{2} \, \pi} {9 \, \cos\left(x\right)}\;dx=\answer{-\frac{9}{2} \, \sqrt{2} + 9}\]
\end{problem}}%}

%%%%%%%%%%%%%%%%%%%%%%


\latexProblemContent{
\begin{problem}

Use the Fundamental Theorem of Calculus to evaluate the integral.

\expandafter\input{\file@loc Integrals/2311-Compute-Integral-0012.HELP.tex}

\[\int_{\frac{2}{3} \, \pi}^{\frac{5}{4} \, \pi} {\sin\left(x\right)}\;dx=\answer{\frac{1}{2} \, \sqrt{2} - \frac{1}{2}}\]
\end{problem}}%}

%%%%%%%%%%%%%%%%%%%%%%


\latexProblemContent{
\begin{problem}

Use the Fundamental Theorem of Calculus to evaluate the integral.

\expandafter\input{\file@loc Integrals/2311-Compute-Integral-0012.HELP.tex}

\[\int_{\frac{2}{3} \, \pi}^{\frac{11}{6} \, \pi} {-5 \, \cos\left(x\right)}\;dx=\answer{\frac{5}{2} \, \sqrt{3} + \frac{5}{2}}\]
\end{problem}}%}

%%%%%%%%%%%%%%%%%%%%%%


%%%%%%%%%%%%%%%%%%%%%%


\latexProblemContent{
\begin{problem}

Use the Fundamental Theorem of Calculus to evaluate the integral.

\expandafter\input{\file@loc Integrals/2311-Compute-Integral-0012.HELP.tex}

\[\int_{\frac{5}{6} \, \pi}^{\frac{5}{4} \, \pi} {\sin\left(x\right)}\;dx=\answer{-\frac{1}{2} \, \sqrt{3} + \frac{1}{2} \, \sqrt{2}}\]
\end{problem}}%}

%%%%%%%%%%%%%%%%%%%%%%


\latexProblemContent{
\begin{problem}

Use the Fundamental Theorem of Calculus to evaluate the integral.

\expandafter\input{\file@loc Integrals/2311-Compute-Integral-0012.HELP.tex}

\[\int_{\frac{1}{2} \, \pi}^{\frac{7}{4} \, \pi} {3 \, \cos\left(x\right)}\;dx=\answer{-\frac{3}{2} \, \sqrt{2} - 3}\]
\end{problem}}%}

%%%%%%%%%%%%%%%%%%%%%%


\latexProblemContent{
\begin{problem}

Use the Fundamental Theorem of Calculus to evaluate the integral.

\expandafter\input{\file@loc Integrals/2311-Compute-Integral-0012.HELP.tex}

\[\int_{\frac{1}{3} \, \pi}^{\frac{1}{2} \, \pi} {-3 \, \cos\left(x\right)}\;dx=\answer{\frac{3}{2} \, \sqrt{3} - 3}\]
\end{problem}}%}

%%%%%%%%%%%%%%%%%%%%%%


\latexProblemContent{
\begin{problem}

Use the Fundamental Theorem of Calculus to evaluate the integral.

\expandafter\input{\file@loc Integrals/2311-Compute-Integral-0012.HELP.tex}

\[\int_{\frac{1}{2} \, \pi}^{\frac{5}{3} \, \pi} {7 \, \sin\left(x\right)}\;dx=\answer{-\frac{7}{2}}\]
\end{problem}}%}

%%%%%%%%%%%%%%%%%%%%%%


%%%%%%%%%%%%%%%%%%%%%%


\latexProblemContent{
\begin{problem}

Use the Fundamental Theorem of Calculus to evaluate the integral.

\expandafter\input{\file@loc Integrals/2311-Compute-Integral-0012.HELP.tex}

\[\int_{\frac{1}{6} \, \pi}^{\frac{1}{2} \, \pi} {4 \, \sin\left(x\right)}\;dx=\answer{2 \, \sqrt{3}}\]
\end{problem}}%}

%%%%%%%%%%%%%%%%%%%%%%


\latexProblemContent{
\begin{problem}

Use the Fundamental Theorem of Calculus to evaluate the integral.

\expandafter\input{\file@loc Integrals/2311-Compute-Integral-0012.HELP.tex}

\[\int_{\frac{3}{4} \, \pi}^{\frac{7}{6} \, \pi} {-4 \, \cos\left(x\right)}\;dx=\answer{2 \, \sqrt{2} + 2}\]
\end{problem}}%}

%%%%%%%%%%%%%%%%%%%%%%


\latexProblemContent{
\begin{problem}

Use the Fundamental Theorem of Calculus to evaluate the integral.

\expandafter\input{\file@loc Integrals/2311-Compute-Integral-0012.HELP.tex}

\[\int_{\frac{2}{3} \, \pi}^{\frac{3}{2} \, \pi} {-4 \, \sin\left(x\right)}\;dx=\answer{2}\]
\end{problem}}%}

%%%%%%%%%%%%%%%%%%%%%%


\latexProblemContent{
\begin{problem}

Use the Fundamental Theorem of Calculus to evaluate the integral.

\expandafter\input{\file@loc Integrals/2311-Compute-Integral-0012.HELP.tex}

\[\int_{\frac{1}{2} \, \pi}^{\frac{5}{3} \, \pi} {-2 \, \cos\left(x\right)}\;dx=\answer{\sqrt{3} + 2}\]
\end{problem}}%}

%%%%%%%%%%%%%%%%%%%%%%


\latexProblemContent{
\begin{problem}

Use the Fundamental Theorem of Calculus to evaluate the integral.

\expandafter\input{\file@loc Integrals/2311-Compute-Integral-0012.HELP.tex}

\[\int_{\frac{2}{3} \, \pi}^{\frac{4}{3} \, \pi} {-3 \, \sin\left(x\right)}\;dx=\answer{0}\]
\end{problem}}%}

%%%%%%%%%%%%%%%%%%%%%%


%%%%%%%%%%%%%%%%%%%%%%


\latexProblemContent{
\begin{problem}

Use the Fundamental Theorem of Calculus to evaluate the integral.

\expandafter\input{\file@loc Integrals/2311-Compute-Integral-0012.HELP.tex}

\[\int_{\frac{1}{4} \, \pi}^{\frac{3}{2} \, \pi} {6 \, \sin\left(x\right)}\;dx=\answer{3 \, \sqrt{2}}\]
\end{problem}}%}

%%%%%%%%%%%%%%%%%%%%%%


\latexProblemContent{
\begin{problem}

Use the Fundamental Theorem of Calculus to evaluate the integral.

\expandafter\input{\file@loc Integrals/2311-Compute-Integral-0012.HELP.tex}

\[\int_{\frac{1}{4} \, \pi}^{\frac{5}{6} \, \pi} {-4 \, \sin\left(x\right)}\;dx=\answer{-2 \, \sqrt{3} - 2 \, \sqrt{2}}\]
\end{problem}}%}

%%%%%%%%%%%%%%%%%%%%%%


\latexProblemContent{
\begin{problem}

Use the Fundamental Theorem of Calculus to evaluate the integral.

\expandafter\input{\file@loc Integrals/2311-Compute-Integral-0012.HELP.tex}

\[\int_{\frac{5}{6} \, \pi}^{\frac{11}{6} \, \pi} {6 \, \sin\left(x\right)}\;dx=\answer{-2 \, \left(3 \, \sqrt{3}\right)}\]
\end{problem}}%}

%%%%%%%%%%%%%%%%%%%%%%


\latexProblemContent{
\begin{problem}

Use the Fundamental Theorem of Calculus to evaluate the integral.

\expandafter\input{\file@loc Integrals/2311-Compute-Integral-0012.HELP.tex}

\[\int_{\frac{3}{4} \, \pi}^{\frac{5}{4} \, \pi} {-10 \, \cos\left(x\right)}\;dx=\answer{5 \, \left(2 \, \sqrt{2}\right)}\]
\end{problem}}%}

%%%%%%%%%%%%%%%%%%%%%%


\latexProblemContent{
\begin{problem}

Use the Fundamental Theorem of Calculus to evaluate the integral.

\expandafter\input{\file@loc Integrals/2311-Compute-Integral-0012.HELP.tex}

\[\int_{\frac{1}{6} \, \pi}^{\frac{5}{6} \, \pi} {-5 \, \sin\left(x\right)}\;dx=\answer{-5 \, \sqrt{3}}\]
\end{problem}}%}

%%%%%%%%%%%%%%%%%%%%%%


\latexProblemContent{
\begin{problem}

Use the Fundamental Theorem of Calculus to evaluate the integral.

\expandafter\input{\file@loc Integrals/2311-Compute-Integral-0012.HELP.tex}

\[\int_{\frac{1}{6} \, \pi}^{\frac{11}{6} \, \pi} {-6 \, \sin\left(x\right)}\;dx=\answer{0}\]
\end{problem}}%}

%%%%%%%%%%%%%%%%%%%%%%


\latexProblemContent{
\begin{problem}

Use the Fundamental Theorem of Calculus to evaluate the integral.

\expandafter\input{\file@loc Integrals/2311-Compute-Integral-0012.HELP.tex}

\[\int_{\frac{1}{4} \, \pi}^{\frac{1}{2} \, \pi} {-6 \, \cos\left(x\right)}\;dx=\answer{3 \, \sqrt{2} - 6}\]
\end{problem}}%}

%%%%%%%%%%%%%%%%%%%%%%


\latexProblemContent{
\begin{problem}

Use the Fundamental Theorem of Calculus to evaluate the integral.

\expandafter\input{\file@loc Integrals/2311-Compute-Integral-0012.HELP.tex}

\[\int_{\frac{5}{6} \, \pi}^{\frac{11}{6} \, \pi} {-\cos\left(x\right)}\;dx=\answer{1}\]
\end{problem}}%}

%%%%%%%%%%%%%%%%%%%%%%


\latexProblemContent{
\begin{problem}

Use the Fundamental Theorem of Calculus to evaluate the integral.

\expandafter\input{\file@loc Integrals/2311-Compute-Integral-0012.HELP.tex}

\[\int_{\frac{1}{3} \, \pi}^{\frac{3}{4} \, \pi} {-10 \, \sin\left(x\right)}\;dx=\answer{-5 \, \sqrt{2} - 5}\]
\end{problem}}%}

%%%%%%%%%%%%%%%%%%%%%%


\latexProblemContent{
\begin{problem}

Use the Fundamental Theorem of Calculus to evaluate the integral.

\expandafter\input{\file@loc Integrals/2311-Compute-Integral-0012.HELP.tex}

\[\int_{\frac{3}{4} \, \pi}^{\frac{3}{2} \, \pi} {3 \, \sin\left(x\right)}\;dx=\answer{-\frac{3}{2} \, \sqrt{2}}\]
\end{problem}}%}

%%%%%%%%%%%%%%%%%%%%%%


\latexProblemContent{
\begin{problem}

Use the Fundamental Theorem of Calculus to evaluate the integral.

\expandafter\input{\file@loc Integrals/2311-Compute-Integral-0012.HELP.tex}

\[\int_{\frac{1}{3} \, \pi}^{\frac{1}{2} \, \pi} {10 \, \cos\left(x\right)}\;dx=\answer{-5 \, \sqrt{3} + 10}\]
\end{problem}}%}

%%%%%%%%%%%%%%%%%%%%%%


\latexProblemContent{
\begin{problem}

Use the Fundamental Theorem of Calculus to evaluate the integral.

\expandafter\input{\file@loc Integrals/2311-Compute-Integral-0012.HELP.tex}

\[\int_{\frac{1}{6} \, \pi}^{\frac{1}{2} \, \pi} {-2 \, \cos\left(x\right)}\;dx=\answer{-1}\]
\end{problem}}%}

%%%%%%%%%%%%%%%%%%%%%%


\latexProblemContent{
\begin{problem}

Use the Fundamental Theorem of Calculus to evaluate the integral.

\expandafter\input{\file@loc Integrals/2311-Compute-Integral-0012.HELP.tex}

\[\int_{\frac{1}{6} \, \pi}^{\frac{7}{6} \, \pi} {-6 \, \cos\left(x\right)}\;dx=\answer{6}\]
\end{problem}}%}

%%%%%%%%%%%%%%%%%%%%%%


\latexProblemContent{
\begin{problem}

Use the Fundamental Theorem of Calculus to evaluate the integral.

\expandafter\input{\file@loc Integrals/2311-Compute-Integral-0012.HELP.tex}

\[\int_{\frac{1}{6} \, \pi}^{\frac{5}{3} \, \pi} {4 \, \cos\left(x\right)}\;dx=\answer{-2 \, \sqrt{3} - 2}\]
\end{problem}}%}

%%%%%%%%%%%%%%%%%%%%%%


\latexProblemContent{
\begin{problem}

Use the Fundamental Theorem of Calculus to evaluate the integral.

\expandafter\input{\file@loc Integrals/2311-Compute-Integral-0012.HELP.tex}

\[\int_{\frac{3}{4} \, \pi}^{\frac{5}{3} \, \pi} {-2 \, \sin\left(x\right)}\;dx=\answer{\sqrt{2} + 1}\]
\end{problem}}%}

%%%%%%%%%%%%%%%%%%%%%%


\latexProblemContent{
\begin{problem}

Use the Fundamental Theorem of Calculus to evaluate the integral.

\expandafter\input{\file@loc Integrals/2311-Compute-Integral-0012.HELP.tex}

\[\int_{\frac{3}{4} \, \pi}^{\frac{3}{2} \, \pi} {8 \, \cos\left(x\right)}\;dx=\answer{-4 \, \sqrt{2} - 8}\]
\end{problem}}%}

%%%%%%%%%%%%%%%%%%%%%%


\latexProblemContent{
\begin{problem}

Use the Fundamental Theorem of Calculus to evaluate the integral.

\expandafter\input{\file@loc Integrals/2311-Compute-Integral-0012.HELP.tex}

\[\int_{\frac{1}{6} \, \pi}^{\frac{4}{3} \, \pi} {9 \, \cos\left(x\right)}\;dx=\answer{-\frac{9}{2} \, \sqrt{3} - \frac{9}{2}}\]
\end{problem}}%}

%%%%%%%%%%%%%%%%%%%%%%


\latexProblemContent{
\begin{problem}

Use the Fundamental Theorem of Calculus to evaluate the integral.

\expandafter\input{\file@loc Integrals/2311-Compute-Integral-0012.HELP.tex}

\[\int_{\frac{1}{6} \, \pi}^{\frac{3}{4} \, \pi} {-5 \, \sin\left(x\right)}\;dx=\answer{-\frac{5}{2} \, \sqrt{3} - \frac{5}{2} \, \sqrt{2}}\]
\end{problem}}%}

%%%%%%%%%%%%%%%%%%%%%%


\latexProblemContent{
\begin{problem}

Use the Fundamental Theorem of Calculus to evaluate the integral.

\expandafter\input{\file@loc Integrals/2311-Compute-Integral-0012.HELP.tex}

\[\int_{\frac{1}{4} \, \pi}^{\frac{7}{4} \, \pi} {-6 \, \sin\left(x\right)}\;dx=\answer{0}\]
\end{problem}}%}

%%%%%%%%%%%%%%%%%%%%%%


\latexProblemContent{
\begin{problem}

Use the Fundamental Theorem of Calculus to evaluate the integral.

\expandafter\input{\file@loc Integrals/2311-Compute-Integral-0012.HELP.tex}

\[\int_{\frac{1}{4} \, \pi}^{\frac{3}{2} \, \pi} {\cos\left(x\right)}\;dx=\answer{-\frac{1}{2} \, \sqrt{2} - 1}\]
\end{problem}}%}

%%%%%%%%%%%%%%%%%%%%%%


\latexProblemContent{
\begin{problem}

Use the Fundamental Theorem of Calculus to evaluate the integral.

\expandafter\input{\file@loc Integrals/2311-Compute-Integral-0012.HELP.tex}

\[\int_{\frac{1}{2} \, \pi}^{\pi} {-\cos\left(x\right)}\;dx=\answer{1}\]
\end{problem}}%}

%%%%%%%%%%%%%%%%%%%%%%


\latexProblemContent{
\begin{problem}

Use the Fundamental Theorem of Calculus to evaluate the integral.

\expandafter\input{\file@loc Integrals/2311-Compute-Integral-0012.HELP.tex}

\[\int_{\frac{2}{3} \, \pi}^{\frac{11}{6} \, \pi} {3 \, \sin\left(x\right)}\;dx=\answer{-\frac{3}{2} \, \sqrt{3} - \frac{3}{2}}\]
\end{problem}}%}

%%%%%%%%%%%%%%%%%%%%%%


\latexProblemContent{
\begin{problem}

Use the Fundamental Theorem of Calculus to evaluate the integral.

\expandafter\input{\file@loc Integrals/2311-Compute-Integral-0012.HELP.tex}

\[\int_{\frac{5}{6} \, \pi}^{\frac{7}{4} \, \pi} {-8 \, \cos\left(x\right)}\;dx=\answer{4 \, \sqrt{2} + 4}\]
\end{problem}}%}

%%%%%%%%%%%%%%%%%%%%%%


\latexProblemContent{
\begin{problem}

Use the Fundamental Theorem of Calculus to evaluate the integral.

\expandafter\input{\file@loc Integrals/2311-Compute-Integral-0012.HELP.tex}

\[\int_{\frac{1}{6} \, \pi}^{\frac{1}{4} \, \pi} {4 \, \sin\left(x\right)}\;dx=\answer{2 \, \sqrt{3} - 2 \, \sqrt{2}}\]
\end{problem}}%}

%%%%%%%%%%%%%%%%%%%%%%


\latexProblemContent{
\begin{problem}

Use the Fundamental Theorem of Calculus to evaluate the integral.

\expandafter\input{\file@loc Integrals/2311-Compute-Integral-0012.HELP.tex}

\[\int_{\frac{1}{2} \, \pi}^{\frac{4}{3} \, \pi} {-4 \, \cos\left(x\right)}\;dx=\answer{2 \, \sqrt{3} + 4}\]
\end{problem}}%}

%%%%%%%%%%%%%%%%%%%%%%


\latexProblemContent{
\begin{problem}

Use the Fundamental Theorem of Calculus to evaluate the integral.

\expandafter\input{\file@loc Integrals/2311-Compute-Integral-0012.HELP.tex}

\[\int_{\frac{1}{4} \, \pi}^{\frac{3}{4} \, \pi} {4 \, \sin\left(x\right)}\;dx=\answer{2^{\frac{5}{2}}}\]
\end{problem}}%}

%%%%%%%%%%%%%%%%%%%%%%


\latexProblemContent{
\begin{problem}

Use the Fundamental Theorem of Calculus to evaluate the integral.

\expandafter\input{\file@loc Integrals/2311-Compute-Integral-0012.HELP.tex}

\[\int_{\frac{1}{3} \, \pi}^{\pi} {-2 \, \cos\left(x\right)}\;dx=\answer{\sqrt{3}}\]
\end{problem}}%}

%%%%%%%%%%%%%%%%%%%%%%


\latexProblemContent{
\begin{problem}

Use the Fundamental Theorem of Calculus to evaluate the integral.

\expandafter\input{\file@loc Integrals/2311-Compute-Integral-0012.HELP.tex}

\[\int_{\frac{3}{4} \, \pi}^{\frac{7}{4} \, \pi} {-4 \, \cos\left(x\right)}\;dx=\answer{2^{\frac{5}{2}}}\]
\end{problem}}%}

%%%%%%%%%%%%%%%%%%%%%%


\latexProblemContent{
\begin{problem}

Use the Fundamental Theorem of Calculus to evaluate the integral.

\expandafter\input{\file@loc Integrals/2311-Compute-Integral-0012.HELP.tex}

\[\int_{\frac{1}{4} \, \pi}^{\frac{5}{3} \, \pi} {9 \, \cos\left(x\right)}\;dx=\answer{-\frac{9}{2} \, \sqrt{3} - \frac{9}{2} \, \sqrt{2}}\]
\end{problem}}%}

%%%%%%%%%%%%%%%%%%%%%%


\latexProblemContent{
\begin{problem}

Use the Fundamental Theorem of Calculus to evaluate the integral.

\expandafter\input{\file@loc Integrals/2311-Compute-Integral-0012.HELP.tex}

\[\int_{\frac{5}{6} \, \pi}^{\frac{7}{4} \, \pi} {6 \, \cos\left(x\right)}\;dx=\answer{-3 \, \sqrt{2} - 3}\]
\end{problem}}%}

%%%%%%%%%%%%%%%%%%%%%%


\latexProblemContent{
\begin{problem}

Use the Fundamental Theorem of Calculus to evaluate the integral.

\expandafter\input{\file@loc Integrals/2311-Compute-Integral-0012.HELP.tex}

\[\int_{\frac{1}{3} \, \pi}^{\frac{3}{4} \, \pi} {-8 \, \sin\left(x\right)}\;dx=\answer{-4 \, \sqrt{2} - 4}\]
\end{problem}}%}

%%%%%%%%%%%%%%%%%%%%%%


%%%%%%%%%%%%%%%%%%%%%%


\latexProblemContent{
\begin{problem}

Use the Fundamental Theorem of Calculus to evaluate the integral.

\expandafter\input{\file@loc Integrals/2311-Compute-Integral-0012.HELP.tex}

\[\int_{\frac{2}{3} \, \pi}^{\frac{7}{4} \, \pi} {-4 \, \sin\left(x\right)}\;dx=\answer{2 \, \sqrt{2} + 2}\]
\end{problem}}%}

%%%%%%%%%%%%%%%%%%%%%%


\latexProblemContent{
\begin{problem}

Use the Fundamental Theorem of Calculus to evaluate the integral.

\expandafter\input{\file@loc Integrals/2311-Compute-Integral-0012.HELP.tex}

\[\int_{\frac{2}{3} \, \pi}^{\frac{5}{4} \, \pi} {-8 \, \cos\left(x\right)}\;dx=\answer{4 \, \sqrt{3} + 4 \, \sqrt{2}}\]
\end{problem}}%}

%%%%%%%%%%%%%%%%%%%%%%


\latexProblemContent{
\begin{problem}

Use the Fundamental Theorem of Calculus to evaluate the integral.

\expandafter\input{\file@loc Integrals/2311-Compute-Integral-0012.HELP.tex}

\[\int_{\frac{1}{4} \, \pi}^{\frac{7}{4} \, \pi} {\sin\left(x\right)}\;dx=\answer{0}\]
\end{problem}}%}

%%%%%%%%%%%%%%%%%%%%%%


\latexProblemContent{
\begin{problem}

Use the Fundamental Theorem of Calculus to evaluate the integral.

\expandafter\input{\file@loc Integrals/2311-Compute-Integral-0012.HELP.tex}

\[\int_{\frac{1}{4} \, \pi}^{\frac{5}{3} \, \pi} {\sin\left(x\right)}\;dx=\answer{\frac{1}{2} \, \sqrt{2} - \frac{1}{2}}\]
\end{problem}}%}

%%%%%%%%%%%%%%%%%%%%%%


\latexProblemContent{
\begin{problem}

Use the Fundamental Theorem of Calculus to evaluate the integral.

\expandafter\input{\file@loc Integrals/2311-Compute-Integral-0012.HELP.tex}

\[\int_{\frac{1}{3} \, \pi}^{\frac{7}{6} \, \pi} {2 \, \sin\left(x\right)}\;dx=\answer{\sqrt{3} + 1}\]
\end{problem}}%}

%%%%%%%%%%%%%%%%%%%%%%


\latexProblemContent{
\begin{problem}

Use the Fundamental Theorem of Calculus to evaluate the integral.

\expandafter\input{\file@loc Integrals/2311-Compute-Integral-0012.HELP.tex}

\[\int_{\frac{3}{4} \, \pi}^{\frac{5}{6} \, \pi} {-8 \, \sin\left(x\right)}\;dx=\answer{-4 \, \sqrt{3} + 4 \, \sqrt{2}}\]
\end{problem}}%}

%%%%%%%%%%%%%%%%%%%%%%


\latexProblemContent{
\begin{problem}

Use the Fundamental Theorem of Calculus to evaluate the integral.

\expandafter\input{\file@loc Integrals/2311-Compute-Integral-0012.HELP.tex}

\[\int_{\frac{1}{6} \, \pi}^{\frac{1}{3} \, \pi} {-5 \, \sin\left(x\right)}\;dx=\answer{-\frac{5}{2} \, \sqrt{3} + \frac{5}{2}}\]
\end{problem}}%}

%%%%%%%%%%%%%%%%%%%%%%


\latexProblemContent{
\begin{problem}

Use the Fundamental Theorem of Calculus to evaluate the integral.

\expandafter\input{\file@loc Integrals/2311-Compute-Integral-0012.HELP.tex}

\[\int_{\frac{1}{6} \, \pi}^{\frac{11}{6} \, \pi} {4 \, \sin\left(x\right)}\;dx=\answer{0}\]
\end{problem}}%}

%%%%%%%%%%%%%%%%%%%%%%


\latexProblemContent{
\begin{problem}

Use the Fundamental Theorem of Calculus to evaluate the integral.

\expandafter\input{\file@loc Integrals/2311-Compute-Integral-0012.HELP.tex}

\[\int_{\frac{1}{3} \, \pi}^{\pi} {6 \, \cos\left(x\right)}\;dx=\answer{-\left(3 \, \sqrt{3}\right)}\]
\end{problem}}%}

%%%%%%%%%%%%%%%%%%%%%%


\latexProblemContent{
\begin{problem}

Use the Fundamental Theorem of Calculus to evaluate the integral.

\expandafter\input{\file@loc Integrals/2311-Compute-Integral-0012.HELP.tex}

\[\int_{\frac{5}{6} \, \pi}^{\frac{11}{6} \, \pi} {-5 \, \sin\left(x\right)}\;dx=\answer{5 \, \sqrt{3}}\]
\end{problem}}%}

%%%%%%%%%%%%%%%%%%%%%%


\latexProblemContent{
\begin{problem}

Use the Fundamental Theorem of Calculus to evaluate the integral.

\expandafter\input{\file@loc Integrals/2311-Compute-Integral-0012.HELP.tex}

\[\int_{\frac{5}{6} \, \pi}^{\pi} {6 \, \cos\left(x\right)}\;dx=\answer{-3}\]
\end{problem}}%}

%%%%%%%%%%%%%%%%%%%%%%


\latexProblemContent{
\begin{problem}

Use the Fundamental Theorem of Calculus to evaluate the integral.

\expandafter\input{\file@loc Integrals/2311-Compute-Integral-0012.HELP.tex}

\[\int_{\frac{1}{6} \, \pi}^{\frac{3}{2} \, \pi} {-7 \, \sin\left(x\right)}\;dx=\answer{-\frac{7}{2} \, \sqrt{3}}\]
\end{problem}}%}

%%%%%%%%%%%%%%%%%%%%%%


\latexProblemContent{
\begin{problem}

Use the Fundamental Theorem of Calculus to evaluate the integral.

\expandafter\input{\file@loc Integrals/2311-Compute-Integral-0012.HELP.tex}

\[\int_{\frac{1}{3} \, \pi}^{\pi} {6 \, \sin\left(x\right)}\;dx=\answer{9}\]
\end{problem}}%}

%%%%%%%%%%%%%%%%%%%%%%


\latexProblemContent{
\begin{problem}

Use the Fundamental Theorem of Calculus to evaluate the integral.

\expandafter\input{\file@loc Integrals/2311-Compute-Integral-0012.HELP.tex}

\[\int_{\frac{1}{4} \, \pi}^{\frac{7}{4} \, \pi} {10 \, \sin\left(x\right)}\;dx=\answer{0}\]
\end{problem}}%}

%%%%%%%%%%%%%%%%%%%%%%


\latexProblemContent{
\begin{problem}

Use the Fundamental Theorem of Calculus to evaluate the integral.

\expandafter\input{\file@loc Integrals/2311-Compute-Integral-0012.HELP.tex}

\[\int_{\frac{1}{2} \, \pi}^{\frac{5}{4} \, \pi} {-7 \, \sin\left(x\right)}\;dx=\answer{-\frac{7}{2} \, \sqrt{2}}\]
\end{problem}}%}

%%%%%%%%%%%%%%%%%%%%%%


\latexProblemContent{
\begin{problem}

Use the Fundamental Theorem of Calculus to evaluate the integral.

\expandafter\input{\file@loc Integrals/2311-Compute-Integral-0012.HELP.tex}

\[\int_{\frac{1}{3} \, \pi}^{\frac{7}{6} \, \pi} {9 \, \cos\left(x\right)}\;dx=\answer{-\frac{9}{2} \, \sqrt{3} - \frac{9}{2}}\]
\end{problem}}%}

%%%%%%%%%%%%%%%%%%%%%%


\latexProblemContent{
\begin{problem}

Use the Fundamental Theorem of Calculus to evaluate the integral.

\expandafter\input{\file@loc Integrals/2311-Compute-Integral-0012.HELP.tex}

\[\int_{\frac{1}{4} \, \pi}^{\frac{5}{6} \, \pi} {-\cos\left(x\right)}\;dx=\answer{\frac{1}{2} \, \sqrt{2} - \frac{1}{2}}\]
\end{problem}}%}

%%%%%%%%%%%%%%%%%%%%%%


\latexProblemContent{
\begin{problem}

Use the Fundamental Theorem of Calculus to evaluate the integral.

\expandafter\input{\file@loc Integrals/2311-Compute-Integral-0012.HELP.tex}

\[\int_{\frac{1}{2} \, \pi}^{\frac{11}{6} \, \pi} {10 \, \cos\left(x\right)}\;dx=\answer{-15}\]
\end{problem}}%}

%%%%%%%%%%%%%%%%%%%%%%


%%%%%%%%%%%%%%%%%%%%%%


\latexProblemContent{
\begin{problem}

Use the Fundamental Theorem of Calculus to evaluate the integral.

\expandafter\input{\file@loc Integrals/2311-Compute-Integral-0012.HELP.tex}

\[\int_{\frac{1}{4} \, \pi}^{\pi} {-2 \, \sin\left(x\right)}\;dx=\answer{-\sqrt{2} - 2}\]
\end{problem}}%}

%%%%%%%%%%%%%%%%%%%%%%


\latexProblemContent{
\begin{problem}

Use the Fundamental Theorem of Calculus to evaluate the integral.

\expandafter\input{\file@loc Integrals/2311-Compute-Integral-0012.HELP.tex}

\[\int_{\frac{5}{6} \, \pi}^{\frac{4}{3} \, \pi} {-2 \, \sin\left(x\right)}\;dx=\answer{\sqrt{3} - 1}\]
\end{problem}}%}

%%%%%%%%%%%%%%%%%%%%%%


%%%%%%%%%%%%%%%%%%%%%%


\latexProblemContent{
\begin{problem}

Use the Fundamental Theorem of Calculus to evaluate the integral.

\expandafter\input{\file@loc Integrals/2311-Compute-Integral-0012.HELP.tex}

\[\int_{\frac{1}{2} \, \pi}^{\frac{5}{3} \, \pi} {-9 \, \sin\left(x\right)}\;dx=\answer{\frac{9}{2}}\]
\end{problem}}%}

%%%%%%%%%%%%%%%%%%%%%%


\latexProblemContent{
\begin{problem}

Use the Fundamental Theorem of Calculus to evaluate the integral.

\expandafter\input{\file@loc Integrals/2311-Compute-Integral-0012.HELP.tex}

\[\int_{\frac{1}{3} \, \pi}^{\frac{4}{3} \, \pi} {10 \, \cos\left(x\right)}\;dx=\answer{-10 \, \sqrt{3}}\]
\end{problem}}%}

%%%%%%%%%%%%%%%%%%%%%%


\latexProblemContent{
\begin{problem}

Use the Fundamental Theorem of Calculus to evaluate the integral.

\expandafter\input{\file@loc Integrals/2311-Compute-Integral-0012.HELP.tex}

\[\int_{\frac{1}{4} \, \pi}^{\frac{3}{4} \, \pi} {-2 \, \cos\left(x\right)}\;dx=\answer{0}\]
\end{problem}}%}

%%%%%%%%%%%%%%%%%%%%%%


\latexProblemContent{
\begin{problem}

Use the Fundamental Theorem of Calculus to evaluate the integral.

\expandafter\input{\file@loc Integrals/2311-Compute-Integral-0012.HELP.tex}

\[\int_{\frac{2}{3} \, \pi}^{\frac{11}{6} \, \pi} {-9 \, \cos\left(x\right)}\;dx=\answer{\frac{9}{2} \, \sqrt{3} + \frac{9}{2}}\]
\end{problem}}%}

%%%%%%%%%%%%%%%%%%%%%%


\latexProblemContent{
\begin{problem}

Use the Fundamental Theorem of Calculus to evaluate the integral.

\expandafter\input{\file@loc Integrals/2311-Compute-Integral-0012.HELP.tex}

\[\int_{\frac{2}{3} \, \pi}^{\frac{3}{2} \, \pi} {4 \, \sin\left(x\right)}\;dx=\answer{-2}\]
\end{problem}}%}

%%%%%%%%%%%%%%%%%%%%%%


%%%%%%%%%%%%%%%%%%%%%%


%%%%%%%%%%%%%%%%%%%%%%


\latexProblemContent{
\begin{problem}

Use the Fundamental Theorem of Calculus to evaluate the integral.

\expandafter\input{\file@loc Integrals/2311-Compute-Integral-0012.HELP.tex}

\[\int_{\frac{1}{4} \, \pi}^{\frac{7}{4} \, \pi} {-5 \, \cos\left(x\right)}\;dx=\answer{5 \, \sqrt{2}}\]
\end{problem}}%}

%%%%%%%%%%%%%%%%%%%%%%


\latexProblemContent{
\begin{problem}

Use the Fundamental Theorem of Calculus to evaluate the integral.

\expandafter\input{\file@loc Integrals/2311-Compute-Integral-0012.HELP.tex}

\[\int_{\frac{1}{3} \, \pi}^{\frac{5}{3} \, \pi} {-5 \, \sin\left(x\right)}\;dx=\answer{0}\]
\end{problem}}%}

%%%%%%%%%%%%%%%%%%%%%%


%%%%%%%%%%%%%%%%%%%%%%


\latexProblemContent{
\begin{problem}

Use the Fundamental Theorem of Calculus to evaluate the integral.

\expandafter\input{\file@loc Integrals/2311-Compute-Integral-0012.HELP.tex}

\[\int_{\frac{5}{6} \, \pi}^{\frac{7}{6} \, \pi} {4 \, \cos\left(x\right)}\;dx=\answer{-4}\]
\end{problem}}%}

%%%%%%%%%%%%%%%%%%%%%%


\latexProblemContent{
\begin{problem}

Use the Fundamental Theorem of Calculus to evaluate the integral.

\expandafter\input{\file@loc Integrals/2311-Compute-Integral-0012.HELP.tex}

\[\int_{\frac{1}{6} \, \pi}^{\frac{5}{6} \, \pi} {-3 \, \cos\left(x\right)}\;dx=\answer{0}\]
\end{problem}}%}

%%%%%%%%%%%%%%%%%%%%%%


%%%%%%%%%%%%%%%%%%%%%%


\latexProblemContent{
\begin{problem}

Use the Fundamental Theorem of Calculus to evaluate the integral.

\expandafter\input{\file@loc Integrals/2311-Compute-Integral-0012.HELP.tex}

\[\int_{\frac{1}{6} \, \pi}^{\frac{5}{6} \, \pi} {-\cos\left(x\right)}\;dx=\answer{0}\]
\end{problem}}%}

%%%%%%%%%%%%%%%%%%%%%%


%%%%%%%%%%%%%%%%%%%%%%


\latexProblemContent{
\begin{problem}

Use the Fundamental Theorem of Calculus to evaluate the integral.

\expandafter\input{\file@loc Integrals/2311-Compute-Integral-0012.HELP.tex}

\[\int_{\frac{1}{4} \, \pi}^{\frac{5}{6} \, \pi} {8 \, \cos\left(x\right)}\;dx=\answer{-4 \, \sqrt{2} + 4}\]
\end{problem}}%}

%%%%%%%%%%%%%%%%%%%%%%


\latexProblemContent{
\begin{problem}

Use the Fundamental Theorem of Calculus to evaluate the integral.

\expandafter\input{\file@loc Integrals/2311-Compute-Integral-0012.HELP.tex}

\[\int_{\frac{2}{3} \, \pi}^{\frac{7}{4} \, \pi} {-9 \, \cos\left(x\right)}\;dx=\answer{\frac{9}{2} \, \sqrt{3} + \frac{9}{2} \, \sqrt{2}}\]
\end{problem}}%}

%%%%%%%%%%%%%%%%%%%%%%


%%%%%%%%%%%%%%%%%%%%%%


\latexProblemContent{
\begin{problem}

Use the Fundamental Theorem of Calculus to evaluate the integral.

\expandafter\input{\file@loc Integrals/2311-Compute-Integral-0012.HELP.tex}

\[\int_{\frac{1}{6} \, \pi}^{\frac{7}{6} \, \pi} {3 \, \cos\left(x\right)}\;dx=\answer{-3}\]
\end{problem}}%}

%%%%%%%%%%%%%%%%%%%%%%


\latexProblemContent{
\begin{problem}

Use the Fundamental Theorem of Calculus to evaluate the integral.

\expandafter\input{\file@loc Integrals/2311-Compute-Integral-0012.HELP.tex}

\[\int_{\frac{1}{4} \, \pi}^{\frac{11}{6} \, \pi} {-2 \, \cos\left(x\right)}\;dx=\answer{\sqrt{2} + 1}\]
\end{problem}}%}

%%%%%%%%%%%%%%%%%%%%%%


%%%%%%%%%%%%%%%%%%%%%%


%%%%%%%%%%%%%%%%%%%%%%


\latexProblemContent{
\begin{problem}

Use the Fundamental Theorem of Calculus to evaluate the integral.

\expandafter\input{\file@loc Integrals/2311-Compute-Integral-0012.HELP.tex}

\[\int_{\frac{5}{6} \, \pi}^{\frac{7}{6} \, \pi} {-10 \, \sin\left(x\right)}\;dx=\answer{0}\]
\end{problem}}%}

%%%%%%%%%%%%%%%%%%%%%%


\latexProblemContent{
\begin{problem}

Use the Fundamental Theorem of Calculus to evaluate the integral.

\expandafter\input{\file@loc Integrals/2311-Compute-Integral-0012.HELP.tex}

\[\int_{\frac{2}{3} \, \pi}^{\frac{11}{6} \, \pi} {-5 \, \sin\left(x\right)}\;dx=\answer{\frac{5}{2} \, \sqrt{3} + \frac{5}{2}}\]
\end{problem}}%}

%%%%%%%%%%%%%%%%%%%%%%


\latexProblemContent{
\begin{problem}

Use the Fundamental Theorem of Calculus to evaluate the integral.

\expandafter\input{\file@loc Integrals/2311-Compute-Integral-0012.HELP.tex}

\[\int_{\frac{3}{4} \, \pi}^{\pi} {4 \, \cos\left(x\right)}\;dx=\answer{-\left(2 \, \sqrt{2}\right)}\]
\end{problem}}%}

%%%%%%%%%%%%%%%%%%%%%%


\latexProblemContent{
\begin{problem}

Use the Fundamental Theorem of Calculus to evaluate the integral.

\expandafter\input{\file@loc Integrals/2311-Compute-Integral-0012.HELP.tex}

\[\int_{\frac{1}{6} \, \pi}^{\frac{11}{6} \, \pi} {8 \, \cos\left(x\right)}\;dx=\answer{-8}\]
\end{problem}}%}

%%%%%%%%%%%%%%%%%%%%%%


\latexProblemContent{
\begin{problem}

Use the Fundamental Theorem of Calculus to evaluate the integral.

\expandafter\input{\file@loc Integrals/2311-Compute-Integral-0012.HELP.tex}

\[\int_{\frac{2}{3} \, \pi}^{\frac{7}{4} \, \pi} {-9 \, \sin\left(x\right)}\;dx=\answer{\frac{9}{2} \, \sqrt{2} + \frac{9}{2}}\]
\end{problem}}%}

%%%%%%%%%%%%%%%%%%%%%%


\latexProblemContent{
\begin{problem}

Use the Fundamental Theorem of Calculus to evaluate the integral.

\expandafter\input{\file@loc Integrals/2311-Compute-Integral-0012.HELP.tex}

\[\int_{\frac{3}{4} \, \pi}^{\frac{7}{6} \, \pi} {4 \, \cos\left(x\right)}\;dx=\answer{-2 \, \sqrt{2} - 2}\]
\end{problem}}%}

%%%%%%%%%%%%%%%%%%%%%%


\latexProblemContent{
\begin{problem}

Use the Fundamental Theorem of Calculus to evaluate the integral.

\expandafter\input{\file@loc Integrals/2311-Compute-Integral-0012.HELP.tex}

\[\int_{\frac{1}{6} \, \pi}^{\frac{5}{6} \, \pi} {7 \, \sin\left(x\right)}\;dx=\answer{7 \, \sqrt{3}}\]
\end{problem}}%}

%%%%%%%%%%%%%%%%%%%%%%


\latexProblemContent{
\begin{problem}

Use the Fundamental Theorem of Calculus to evaluate the integral.

\expandafter\input{\file@loc Integrals/2311-Compute-Integral-0012.HELP.tex}

\[\int_{\frac{2}{3} \, \pi}^{\pi} {6 \, \cos\left(x\right)}\;dx=\answer{-\left(3 \, \sqrt{3}\right)}\]
\end{problem}}%}

%%%%%%%%%%%%%%%%%%%%%%


\latexProblemContent{
\begin{problem}

Use the Fundamental Theorem of Calculus to evaluate the integral.

\expandafter\input{\file@loc Integrals/2311-Compute-Integral-0012.HELP.tex}

\[\int_{\frac{1}{4} \, \pi}^{\frac{1}{3} \, \pi} {-5 \, \sin\left(x\right)}\;dx=\answer{-\frac{5}{2} \, \sqrt{2} + \frac{5}{2}}\]
\end{problem}}%}

%%%%%%%%%%%%%%%%%%%%%%


\latexProblemContent{
\begin{problem}

Use the Fundamental Theorem of Calculus to evaluate the integral.

\expandafter\input{\file@loc Integrals/2311-Compute-Integral-0012.HELP.tex}

\[\int_{\frac{1}{6} \, \pi}^{\frac{5}{3} \, \pi} {8 \, \cos\left(x\right)}\;dx=\answer{-4 \, \sqrt{3} - 4}\]
\end{problem}}%}

%%%%%%%%%%%%%%%%%%%%%%


\latexProblemContent{
\begin{problem}

Use the Fundamental Theorem of Calculus to evaluate the integral.

\expandafter\input{\file@loc Integrals/2311-Compute-Integral-0012.HELP.tex}

\[\int_{\frac{1}{2} \, \pi}^{\frac{7}{6} \, \pi} {2 \, \sin\left(x\right)}\;dx=\answer{\sqrt{3}}\]
\end{problem}}%}

%%%%%%%%%%%%%%%%%%%%%%


%%%%%%%%%%%%%%%%%%%%%%


\latexProblemContent{
\begin{problem}

Use the Fundamental Theorem of Calculus to evaluate the integral.

\expandafter\input{\file@loc Integrals/2311-Compute-Integral-0012.HELP.tex}

\[\int_{\frac{5}{6} \, \pi}^{\frac{5}{3} \, \pi} {-\sin\left(x\right)}\;dx=\answer{\frac{1}{2} \, \sqrt{3} + \frac{1}{2}}\]
\end{problem}}%}

%%%%%%%%%%%%%%%%%%%%%%


\latexProblemContent{
\begin{problem}

Use the Fundamental Theorem of Calculus to evaluate the integral.

\expandafter\input{\file@loc Integrals/2311-Compute-Integral-0012.HELP.tex}

\[\int_{\frac{1}{4} \, \pi}^{\frac{7}{4} \, \pi} {5 \, \cos\left(x\right)}\;dx=\answer{-5 \, \sqrt{2}}\]
\end{problem}}%}

%%%%%%%%%%%%%%%%%%%%%%


\latexProblemContent{
\begin{problem}

Use the Fundamental Theorem of Calculus to evaluate the integral.

\expandafter\input{\file@loc Integrals/2311-Compute-Integral-0012.HELP.tex}

\[\int_{\frac{1}{3} \, \pi}^{\frac{5}{3} \, \pi} {2 \, \sin\left(x\right)}\;dx=\answer{0}\]
\end{problem}}%}

%%%%%%%%%%%%%%%%%%%%%%


\latexProblemContent{
\begin{problem}

Use the Fundamental Theorem of Calculus to evaluate the integral.

\expandafter\input{\file@loc Integrals/2311-Compute-Integral-0012.HELP.tex}

\[\int_{\frac{1}{2} \, \pi}^{\frac{5}{4} \, \pi} {3 \, \sin\left(x\right)}\;dx=\answer{\frac{3}{2} \, \sqrt{2}}\]
\end{problem}}%}

%%%%%%%%%%%%%%%%%%%%%%


\latexProblemContent{
\begin{problem}

Use the Fundamental Theorem of Calculus to evaluate the integral.

\expandafter\input{\file@loc Integrals/2311-Compute-Integral-0012.HELP.tex}

\[\int_{\frac{1}{3} \, \pi}^{\frac{5}{6} \, \pi} {5 \, \sin\left(x\right)}\;dx=\answer{\frac{5}{2} \, \sqrt{3} + \frac{5}{2}}\]
\end{problem}}%}

%%%%%%%%%%%%%%%%%%%%%%


\latexProblemContent{
\begin{problem}

Use the Fundamental Theorem of Calculus to evaluate the integral.

\expandafter\input{\file@loc Integrals/2311-Compute-Integral-0012.HELP.tex}

\[\int_{\frac{1}{3} \, \pi}^{\frac{5}{4} \, \pi} {3 \, \cos\left(x\right)}\;dx=\answer{-\frac{3}{2} \, \sqrt{3} - \frac{3}{2} \, \sqrt{2}}\]
\end{problem}}%}

%%%%%%%%%%%%%%%%%%%%%%


\latexProblemContent{
\begin{problem}

Use the Fundamental Theorem of Calculus to evaluate the integral.

\expandafter\input{\file@loc Integrals/2311-Compute-Integral-0012.HELP.tex}

\[\int_{\frac{1}{4} \, \pi}^{\frac{1}{3} \, \pi} {-5 \, \cos\left(x\right)}\;dx=\answer{-\frac{5}{2} \, \sqrt{3} + \frac{5}{2} \, \sqrt{2}}\]
\end{problem}}%}

%%%%%%%%%%%%%%%%%%%%%%


\latexProblemContent{
\begin{problem}

Use the Fundamental Theorem of Calculus to evaluate the integral.

\expandafter\input{\file@loc Integrals/2311-Compute-Integral-0012.HELP.tex}

\[\int_{\frac{1}{4} \, \pi}^{\frac{7}{6} \, \pi} {-10 \, \cos\left(x\right)}\;dx=\answer{5 \, \sqrt{2} + 5}\]
\end{problem}}%}

%%%%%%%%%%%%%%%%%%%%%%


\latexProblemContent{
\begin{problem}

Use the Fundamental Theorem of Calculus to evaluate the integral.

\expandafter\input{\file@loc Integrals/2311-Compute-Integral-0012.HELP.tex}

\[\int_{\frac{5}{6} \, \pi}^{\frac{7}{4} \, \pi} {-10 \, \cos\left(x\right)}\;dx=\answer{5 \, \sqrt{2} + 5}\]
\end{problem}}%}

%%%%%%%%%%%%%%%%%%%%%%


\latexProblemContent{
\begin{problem}

Use the Fundamental Theorem of Calculus to evaluate the integral.

\expandafter\input{\file@loc Integrals/2311-Compute-Integral-0012.HELP.tex}

\[\int_{\frac{5}{6} \, \pi}^{\frac{7}{4} \, \pi} {4 \, \cos\left(x\right)}\;dx=\answer{-2 \, \sqrt{2} - 2}\]
\end{problem}}%}

%%%%%%%%%%%%%%%%%%%%%%


\latexProblemContent{
\begin{problem}

Use the Fundamental Theorem of Calculus to evaluate the integral.

\expandafter\input{\file@loc Integrals/2311-Compute-Integral-0012.HELP.tex}

\[\int_{\frac{1}{2} \, \pi}^{\frac{11}{6} \, \pi} {-2 \, \cos\left(x\right)}\;dx=\answer{3}\]
\end{problem}}%}

%%%%%%%%%%%%%%%%%%%%%%


\latexProblemContent{
\begin{problem}

Use the Fundamental Theorem of Calculus to evaluate the integral.

\expandafter\input{\file@loc Integrals/2311-Compute-Integral-0012.HELP.tex}

\[\int_{\frac{2}{3} \, \pi}^{\frac{5}{4} \, \pi} {-9 \, \sin\left(x\right)}\;dx=\answer{-\frac{9}{2} \, \sqrt{2} + \frac{9}{2}}\]
\end{problem}}%}

%%%%%%%%%%%%%%%%%%%%%%


\latexProblemContent{
\begin{problem}

Use the Fundamental Theorem of Calculus to evaluate the integral.

\expandafter\input{\file@loc Integrals/2311-Compute-Integral-0012.HELP.tex}

\[\int_{\frac{1}{2} \, \pi}^{\frac{4}{3} \, \pi} {9 \, \cos\left(x\right)}\;dx=\answer{-\frac{9}{2} \, \sqrt{3} - 9}\]
\end{problem}}%}

%%%%%%%%%%%%%%%%%%%%%%


\latexProblemContent{
\begin{problem}

Use the Fundamental Theorem of Calculus to evaluate the integral.

\expandafter\input{\file@loc Integrals/2311-Compute-Integral-0012.HELP.tex}

\[\int_{\frac{1}{2} \, \pi}^{\frac{5}{4} \, \pi} {2 \, \sin\left(x\right)}\;dx=\answer{\sqrt{2}}\]
\end{problem}}%}

%%%%%%%%%%%%%%%%%%%%%%


\latexProblemContent{
\begin{problem}

Use the Fundamental Theorem of Calculus to evaluate the integral.

\expandafter\input{\file@loc Integrals/2311-Compute-Integral-0012.HELP.tex}

\[\int_{\frac{2}{3} \, \pi}^{\frac{4}{3} \, \pi} {-\cos\left(x\right)}\;dx=\answer{\sqrt{3}}\]
\end{problem}}%}

%%%%%%%%%%%%%%%%%%%%%%


\latexProblemContent{
\begin{problem}

Use the Fundamental Theorem of Calculus to evaluate the integral.

\expandafter\input{\file@loc Integrals/2311-Compute-Integral-0012.HELP.tex}

\[\int_{\frac{1}{4} \, \pi}^{\frac{1}{2} \, \pi} {7 \, \cos\left(x\right)}\;dx=\answer{-\frac{7}{2} \, \sqrt{2} + 7}\]
\end{problem}}%}

%%%%%%%%%%%%%%%%%%%%%%


%%%%%%%%%%%%%%%%%%%%%%


\latexProblemContent{
\begin{problem}

Use the Fundamental Theorem of Calculus to evaluate the integral.

\expandafter\input{\file@loc Integrals/2311-Compute-Integral-0012.HELP.tex}

\[\int_{\frac{1}{3} \, \pi}^{\frac{7}{6} \, \pi} {-4 \, \cos\left(x\right)}\;dx=\answer{2 \, \sqrt{3} + 2}\]
\end{problem}}%}

%%%%%%%%%%%%%%%%%%%%%%


%%%%%%%%%%%%%%%%%%%%%%


\latexProblemContent{
\begin{problem}

Use the Fundamental Theorem of Calculus to evaluate the integral.

\expandafter\input{\file@loc Integrals/2311-Compute-Integral-0012.HELP.tex}

\[\int_{\frac{5}{6} \, \pi}^{\frac{4}{3} \, \pi} {10 \, \cos\left(x\right)}\;dx=\answer{-5 \, \sqrt{3} - 5}\]
\end{problem}}%}

%%%%%%%%%%%%%%%%%%%%%%


\latexProblemContent{
\begin{problem}

Use the Fundamental Theorem of Calculus to evaluate the integral.

\expandafter\input{\file@loc Integrals/2311-Compute-Integral-0012.HELP.tex}

\[\int_{\frac{2}{3} \, \pi}^{\pi} {5 \, \cos\left(x\right)}\;dx=\answer{-\frac{5}{2} \, \sqrt{3}}\]
\end{problem}}%}

%%%%%%%%%%%%%%%%%%%%%%


\latexProblemContent{
\begin{problem}

Use the Fundamental Theorem of Calculus to evaluate the integral.

\expandafter\input{\file@loc Integrals/2311-Compute-Integral-0012.HELP.tex}

\[\int_{\frac{1}{3} \, \pi}^{\frac{5}{4} \, \pi} {10 \, \sin\left(x\right)}\;dx=\answer{5 \, \sqrt{2} + 5}\]
\end{problem}}%}

%%%%%%%%%%%%%%%%%%%%%%


\latexProblemContent{
\begin{problem}

Use the Fundamental Theorem of Calculus to evaluate the integral.

\expandafter\input{\file@loc Integrals/2311-Compute-Integral-0012.HELP.tex}

\[\int_{\frac{1}{3} \, \pi}^{\frac{5}{3} \, \pi} {-\cos\left(x\right)}\;dx=\answer{\sqrt{3}}\]
\end{problem}}%}

%%%%%%%%%%%%%%%%%%%%%%


\latexProblemContent{
\begin{problem}

Use the Fundamental Theorem of Calculus to evaluate the integral.

\expandafter\input{\file@loc Integrals/2311-Compute-Integral-0012.HELP.tex}

\[\int_{\frac{3}{4} \, \pi}^{\frac{11}{6} \, \pi} {-\cos\left(x\right)}\;dx=\answer{\frac{1}{2} \, \sqrt{2} + \frac{1}{2}}\]
\end{problem}}%}

%%%%%%%%%%%%%%%%%%%%%%


%%%%%%%%%%%%%%%%%%%%%%


\latexProblemContent{
\begin{problem}

Use the Fundamental Theorem of Calculus to evaluate the integral.

\expandafter\input{\file@loc Integrals/2311-Compute-Integral-0012.HELP.tex}

\[\int_{\frac{1}{4} \, \pi}^{\pi} {5 \, \cos\left(x\right)}\;dx=\answer{-\frac{5}{2} \, \sqrt{2}}\]
\end{problem}}%}

%%%%%%%%%%%%%%%%%%%%%%


%%%%%%%%%%%%%%%%%%%%%%


\latexProblemContent{
\begin{problem}

Use the Fundamental Theorem of Calculus to evaluate the integral.

\expandafter\input{\file@loc Integrals/2311-Compute-Integral-0012.HELP.tex}

\[\int_{\frac{3}{4} \, \pi}^{\pi} {-3 \, \cos\left(x\right)}\;dx=\answer{\frac{3}{2} \, \sqrt{2}}\]
\end{problem}}%}

%%%%%%%%%%%%%%%%%%%%%%


\latexProblemContent{
\begin{problem}

Use the Fundamental Theorem of Calculus to evaluate the integral.

\expandafter\input{\file@loc Integrals/2311-Compute-Integral-0012.HELP.tex}

\[\int_{\frac{5}{6} \, \pi}^{\frac{3}{2} \, \pi} {\cos\left(x\right)}\;dx=\answer{-\frac{3}{2}}\]
\end{problem}}%}

%%%%%%%%%%%%%%%%%%%%%%


\latexProblemContent{
\begin{problem}

Use the Fundamental Theorem of Calculus to evaluate the integral.

\expandafter\input{\file@loc Integrals/2311-Compute-Integral-0012.HELP.tex}

\[\int_{\frac{5}{6} \, \pi}^{\frac{7}{4} \, \pi} {-5 \, \sin\left(x\right)}\;dx=\answer{\frac{5}{2} \, \sqrt{3} + \frac{5}{2} \, \sqrt{2}}\]
\end{problem}}%}

%%%%%%%%%%%%%%%%%%%%%%


%%%%%%%%%%%%%%%%%%%%%%


\latexProblemContent{
\begin{problem}

Use the Fundamental Theorem of Calculus to evaluate the integral.

\expandafter\input{\file@loc Integrals/2311-Compute-Integral-0012.HELP.tex}

\[\int_{\frac{3}{4} \, \pi}^{\frac{5}{3} \, \pi} {10 \, \sin\left(x\right)}\;dx=\answer{-5 \, \sqrt{2} - 5}\]
\end{problem}}%}

%%%%%%%%%%%%%%%%%%%%%%


\latexProblemContent{
\begin{problem}

Use the Fundamental Theorem of Calculus to evaluate the integral.

\expandafter\input{\file@loc Integrals/2311-Compute-Integral-0012.HELP.tex}

\[\int_{\frac{1}{6} \, \pi}^{\frac{4}{3} \, \pi} {7 \, \sin\left(x\right)}\;dx=\answer{\frac{7}{2} \, \sqrt{3} + \frac{7}{2}}\]
\end{problem}}%}

%%%%%%%%%%%%%%%%%%%%%%


\latexProblemContent{
\begin{problem}

Use the Fundamental Theorem of Calculus to evaluate the integral.

\expandafter\input{\file@loc Integrals/2311-Compute-Integral-0012.HELP.tex}

\[\int_{\frac{1}{2} \, \pi}^{\frac{7}{6} \, \pi} {-5 \, \sin\left(x\right)}\;dx=\answer{-\frac{5}{2} \, \sqrt{3}}\]
\end{problem}}%}

%%%%%%%%%%%%%%%%%%%%%%


%%%%%%%%%%%%%%%%%%%%%%


\latexProblemContent{
\begin{problem}

Use the Fundamental Theorem of Calculus to evaluate the integral.

\expandafter\input{\file@loc Integrals/2311-Compute-Integral-0012.HELP.tex}

\[\int_{\frac{1}{2} \, \pi}^{\frac{3}{4} \, \pi} {-8 \, \cos\left(x\right)}\;dx=\answer{-4 \, \sqrt{2} + 8}\]
\end{problem}}%}

%%%%%%%%%%%%%%%%%%%%%%


\latexProblemContent{
\begin{problem}

Use the Fundamental Theorem of Calculus to evaluate the integral.

\expandafter\input{\file@loc Integrals/2311-Compute-Integral-0012.HELP.tex}

\[\int_{\frac{3}{4} \, \pi}^{\pi} {-8 \, \sin\left(x\right)}\;dx=\answer{4 \, \sqrt{2} - 8}\]
\end{problem}}%}

%%%%%%%%%%%%%%%%%%%%%%


\latexProblemContent{
\begin{problem}

Use the Fundamental Theorem of Calculus to evaluate the integral.

\expandafter\input{\file@loc Integrals/2311-Compute-Integral-0012.HELP.tex}

\[\int_{\frac{5}{6} \, \pi}^{\frac{5}{3} \, \pi} {6 \, \cos\left(x\right)}\;dx=\answer{-3 \, \sqrt{3} - 3}\]
\end{problem}}%}

%%%%%%%%%%%%%%%%%%%%%%


\latexProblemContent{
\begin{problem}

Use the Fundamental Theorem of Calculus to evaluate the integral.

\expandafter\input{\file@loc Integrals/2311-Compute-Integral-0012.HELP.tex}

\[\int_{\frac{5}{6} \, \pi}^{\frac{5}{4} \, \pi} {7 \, \sin\left(x\right)}\;dx=\answer{-\frac{7}{2} \, \sqrt{3} + \frac{7}{2} \, \sqrt{2}}\]
\end{problem}}%}

%%%%%%%%%%%%%%%%%%%%%%


\latexProblemContent{
\begin{problem}

Use the Fundamental Theorem of Calculus to evaluate the integral.

\expandafter\input{\file@loc Integrals/2311-Compute-Integral-0012.HELP.tex}

\[\int_{\frac{2}{3} \, \pi}^{\pi} {-9 \, \sin\left(x\right)}\;dx=\answer{-\frac{9}{2}}\]
\end{problem}}%}

%%%%%%%%%%%%%%%%%%%%%%


\latexProblemContent{
\begin{problem}

Use the Fundamental Theorem of Calculus to evaluate the integral.

\expandafter\input{\file@loc Integrals/2311-Compute-Integral-0012.HELP.tex}

\[\int_{\frac{1}{4} \, \pi}^{\frac{7}{4} \, \pi} {9 \, \cos\left(x\right)}\;dx=\answer{-9 \, \sqrt{2}}\]
\end{problem}}%}

%%%%%%%%%%%%%%%%%%%%%%


\latexProblemContent{
\begin{problem}

Use the Fundamental Theorem of Calculus to evaluate the integral.

\expandafter\input{\file@loc Integrals/2311-Compute-Integral-0012.HELP.tex}

\[\int_{\frac{1}{6} \, \pi}^{\frac{1}{2} \, \pi} {\cos\left(x\right)}\;dx=\answer{\frac{1}{2}}\]
\end{problem}}%}

%%%%%%%%%%%%%%%%%%%%%%


\latexProblemContent{
\begin{problem}

Use the Fundamental Theorem of Calculus to evaluate the integral.

\expandafter\input{\file@loc Integrals/2311-Compute-Integral-0012.HELP.tex}

\[\int_{\frac{5}{6} \, \pi}^{\pi} {9 \, \sin\left(x\right)}\;dx=\answer{-\frac{9}{2} \, \sqrt{3} + 9}\]
\end{problem}}%}

%%%%%%%%%%%%%%%%%%%%%%


\latexProblemContent{
\begin{problem}

Use the Fundamental Theorem of Calculus to evaluate the integral.

\expandafter\input{\file@loc Integrals/2311-Compute-Integral-0012.HELP.tex}

\[\int_{\frac{1}{2} \, \pi}^{\frac{5}{6} \, \pi} {9 \, \sin\left(x\right)}\;dx=\answer{\frac{1}{2} \, \left(9 \, \sqrt{3}\right)}\]
\end{problem}}%}

%%%%%%%%%%%%%%%%%%%%%%


\latexProblemContent{
\begin{problem}

Use the Fundamental Theorem of Calculus to evaluate the integral.

\expandafter\input{\file@loc Integrals/2311-Compute-Integral-0012.HELP.tex}

\[\int_{\frac{1}{2} \, \pi}^{\frac{5}{3} \, \pi} {9 \, \cos\left(x\right)}\;dx=\answer{-\frac{9}{2} \, \sqrt{3} - 9}\]
\end{problem}}%}

%%%%%%%%%%%%%%%%%%%%%%


\latexProblemContent{
\begin{problem}

Use the Fundamental Theorem of Calculus to evaluate the integral.

\expandafter\input{\file@loc Integrals/2311-Compute-Integral-0012.HELP.tex}

\[\int_{\frac{1}{4} \, \pi}^{\frac{3}{2} \, \pi} {-9 \, \sin\left(x\right)}\;dx=\answer{-\frac{9}{2} \, \sqrt{2}}\]
\end{problem}}%}

%%%%%%%%%%%%%%%%%%%%%%


\latexProblemContent{
\begin{problem}

Use the Fundamental Theorem of Calculus to evaluate the integral.

\expandafter\input{\file@loc Integrals/2311-Compute-Integral-0012.HELP.tex}

\[\int_{\frac{3}{4} \, \pi}^{\frac{11}{6} \, \pi} {7 \, \cos\left(x\right)}\;dx=\answer{-\frac{7}{2} \, \sqrt{2} - \frac{7}{2}}\]
\end{problem}}%}

%%%%%%%%%%%%%%%%%%%%%%




