%%%%%%%%%%%%%%%%%%%%%%%
%%\tagged{Cat@One, Cat@Two, Cat@Three, Cat@Four, Cat@Five, Ans@ShortAns, Type@Compute, Topic@Integral, Sub@Definite, Sub@Theorems_FTC}{

\latexProblemContent{
\begin{problem}

Use the Fundamental Theorem of Calculus to find the derivative of the function.
\[g(t)=\int_{4}^{t} {{\left(x - 2\right)}^{\frac{1}{4}}}\;dx\]

\expandafter\input{\file@loc Integrals/2311-Compute-Integral-0008.HELP.tex}

\[\dfrac{d}{dt}(g(t))=\answer{{\left(t - 2\right)}^{\frac{1}{4}}}\]
\end{problem}}%}

%%%%%%%%%%%%%%%%%%%%%%




\latexProblemContent{
\begin{problem}

Use the Fundamental Theorem of Calculus to find the derivative of the function.
\[g(t)=\int_{1}^{t} {{\left(x - 6\right)}^{\frac{1}{4}}}\;dx\]

\expandafter\input{\file@loc Integrals/2311-Compute-Integral-0008.HELP.tex}

\[\dfrac{d}{dt}(g(t))=\answer{{\left(t - 6\right)}^{\frac{1}{4}}}\]
\end{problem}}%}

%%%%%%%%%%%%%%%%%%%%%%




\latexProblemContent{
\begin{problem}

Use the Fundamental Theorem of Calculus to find the derivative of the function.
\[g(t)=\int_{4}^{t} {x - 16}\;dx\]

\expandafter\input{\file@loc Integrals/2311-Compute-Integral-0008.HELP.tex}

\[\dfrac{d}{dt}(g(t))=\answer{t - 16}\]
\end{problem}}%}

%%%%%%%%%%%%%%%%%%%%%%




\latexProblemContent{
\begin{problem}

Use the Fundamental Theorem of Calculus to find the derivative of the function.
\[g(t)=\int_{4}^{t} {e^{\left(e^{\left(x + 9\right)}\right)}}\;dx\]

\expandafter\input{\file@loc Integrals/2311-Compute-Integral-0008.HELP.tex}

\[\dfrac{d}{dt}(g(t))=\answer{e^{\left(e^{\left(t + 9\right)}\right)}}\]
\end{problem}}%}

%%%%%%%%%%%%%%%%%%%%%%




\latexProblemContent{
\begin{problem}

Use the Fundamental Theorem of Calculus to find the derivative of the function.
\[g(t)=\int_{2}^{t} {{\left(x - 8\right)}^{\frac{1}{4}}}\;dx\]

\expandafter\input{\file@loc Integrals/2311-Compute-Integral-0008.HELP.tex}

\[\dfrac{d}{dt}(g(t))=\answer{{\left(t - 8\right)}^{\frac{1}{4}}}\]
\end{problem}}%}

%%%%%%%%%%%%%%%%%%%%%%




\latexProblemContent{
\begin{problem}

Use the Fundamental Theorem of Calculus to find the derivative of the function.
\[g(t)=\int_{4}^{t} {{\left(x + 8\right)}^{4}}\;dx\]

\expandafter\input{\file@loc Integrals/2311-Compute-Integral-0008.HELP.tex}

\[\dfrac{d}{dt}(g(t))=\answer{{\left(t + 8\right)}^{4}}\]
\end{problem}}%}

%%%%%%%%%%%%%%%%%%%%%%




\latexProblemContent{
\begin{problem}

Use the Fundamental Theorem of Calculus to find the derivative of the function.
\[g(t)=\int_{2}^{t} {x - 4}\;dx\]

\expandafter\input{\file@loc Integrals/2311-Compute-Integral-0008.HELP.tex}

\[\dfrac{d}{dt}(g(t))=\answer{t - 4}\]
\end{problem}}%}

%%%%%%%%%%%%%%%%%%%%%%




\latexProblemContent{
\begin{problem}

Use the Fundamental Theorem of Calculus to find the derivative of the function.
\[g(t)=\int_{1}^{t} {x + 14}\;dx\]

\expandafter\input{\file@loc Integrals/2311-Compute-Integral-0008.HELP.tex}

\[\dfrac{d}{dt}(g(t))=\answer{t + 14}\]
\end{problem}}%}

%%%%%%%%%%%%%%%%%%%%%%




\latexProblemContent{
\begin{problem}

Use the Fundamental Theorem of Calculus to find the derivative of the function.
\[g(t)=\int_{3}^{t} {\cos\left(\cos\left(x - 4\right)\right)}\;dx\]

\expandafter\input{\file@loc Integrals/2311-Compute-Integral-0008.HELP.tex}

\[\dfrac{d}{dt}(g(t))=\answer{\cos\left(\cos\left(t - 4\right)\right)}\]
\end{problem}}%}

%%%%%%%%%%%%%%%%%%%%%%




\latexProblemContent{
\begin{problem}

Use the Fundamental Theorem of Calculus to find the derivative of the function.
\[g(t)=\int_{2}^{t} {{\left(x - 4\right)}^{9}}\;dx\]

\expandafter\input{\file@loc Integrals/2311-Compute-Integral-0008.HELP.tex}

\[\dfrac{d}{dt}(g(t))=\answer{{\left(t - 4\right)}^{9}}\]
\end{problem}}%}

%%%%%%%%%%%%%%%%%%%%%%




\latexProblemContent{
\begin{problem}

Use the Fundamental Theorem of Calculus to find the derivative of the function.
\[g(t)=\int_{2}^{t} {x + 8}\;dx\]

\expandafter\input{\file@loc Integrals/2311-Compute-Integral-0008.HELP.tex}

\[\dfrac{d}{dt}(g(t))=\answer{t + 8}\]
\end{problem}}%}

%%%%%%%%%%%%%%%%%%%%%%




\latexProblemContent{
\begin{problem}

Use the Fundamental Theorem of Calculus to find the derivative of the function.
\[g(t)=\int_{1}^{t} {\sin\left(\sin\left(x - 1\right)\right)}\;dx\]

\expandafter\input{\file@loc Integrals/2311-Compute-Integral-0008.HELP.tex}

\[\dfrac{d}{dt}(g(t))=\answer{\sin\left(\sin\left(t - 1\right)\right)}\]
\end{problem}}%}

%%%%%%%%%%%%%%%%%%%%%%




\latexProblemContent{
\begin{problem}

Use the Fundamental Theorem of Calculus to find the derivative of the function.
\[g(t)=\int_{1}^{t} {e^{\left(e^{\left(x - 1\right)}\right)}}\;dx\]

\expandafter\input{\file@loc Integrals/2311-Compute-Integral-0008.HELP.tex}

\[\dfrac{d}{dt}(g(t))=\answer{e^{\left(e^{\left(t - 1\right)}\right)}}\]
\end{problem}}%}

%%%%%%%%%%%%%%%%%%%%%%




\latexProblemContent{
\begin{problem}

Use the Fundamental Theorem of Calculus to find the derivative of the function.
\[g(t)=\int_{4}^{t} {\sin\left(\sin\left(x - 6\right)\right)}\;dx\]

\expandafter\input{\file@loc Integrals/2311-Compute-Integral-0008.HELP.tex}

\[\dfrac{d}{dt}(g(t))=\answer{\sin\left(\sin\left(t - 6\right)\right)}\]
\end{problem}}%}

%%%%%%%%%%%%%%%%%%%%%%




\latexProblemContent{
\begin{problem}

Use the Fundamental Theorem of Calculus to find the derivative of the function.
\[g(t)=\int_{3}^{t} {{\left(x + 9\right)}^{4}}\;dx\]

\expandafter\input{\file@loc Integrals/2311-Compute-Integral-0008.HELP.tex}

\[\dfrac{d}{dt}(g(t))=\answer{{\left(t + 9\right)}^{4}}\]
\end{problem}}%}

%%%%%%%%%%%%%%%%%%%%%%




\latexProblemContent{
\begin{problem}

Use the Fundamental Theorem of Calculus to find the derivative of the function.
\[g(t)=\int_{1}^{t} {{\left(x - 1\right)}^{\frac{1}{4}}}\;dx\]

\expandafter\input{\file@loc Integrals/2311-Compute-Integral-0008.HELP.tex}

\[\dfrac{d}{dt}(g(t))=\answer{{\left(t - 1\right)}^{\frac{1}{4}}}\]
\end{problem}}%}

%%%%%%%%%%%%%%%%%%%%%%




\latexProblemContent{
\begin{problem}

Use the Fundamental Theorem of Calculus to find the derivative of the function.
\[g(t)=\int_{5}^{t} {x + 5}\;dx\]

\expandafter\input{\file@loc Integrals/2311-Compute-Integral-0008.HELP.tex}

\[\dfrac{d}{dt}(g(t))=\answer{t + 5}\]
\end{problem}}%}

%%%%%%%%%%%%%%%%%%%%%%




\latexProblemContent{
\begin{problem}

Use the Fundamental Theorem of Calculus to find the derivative of the function.
\[g(t)=\int_{1}^{t} {{\left(x + 1\right)}^{16}}\;dx\]

\expandafter\input{\file@loc Integrals/2311-Compute-Integral-0008.HELP.tex}

\[\dfrac{d}{dt}(g(t))=\answer{{\left(t + 1\right)}^{16}}\]
\end{problem}}%}

%%%%%%%%%%%%%%%%%%%%%%




\latexProblemContent{
\begin{problem}

Use the Fundamental Theorem of Calculus to find the derivative of the function.
\[g(t)=\int_{2}^{t} {{\left(x + 4\right)}^{4}}\;dx\]

\expandafter\input{\file@loc Integrals/2311-Compute-Integral-0008.HELP.tex}

\[\dfrac{d}{dt}(g(t))=\answer{{\left(t + 4\right)}^{4}}\]
\end{problem}}%}

%%%%%%%%%%%%%%%%%%%%%%




\latexProblemContent{
\begin{problem}

Use the Fundamental Theorem of Calculus to find the derivative of the function.
\[g(t)=\int_{5}^{t} {{\left(x + 7\right)}^{9}}\;dx\]

\expandafter\input{\file@loc Integrals/2311-Compute-Integral-0008.HELP.tex}

\[\dfrac{d}{dt}(g(t))=\answer{{\left(t + 7\right)}^{9}}\]
\end{problem}}%}

%%%%%%%%%%%%%%%%%%%%%%




\latexProblemContent{
\begin{problem}

Use the Fundamental Theorem of Calculus to find the derivative of the function.
\[g(t)=\int_{1}^{t} {{\left(x + 8\right)}^{9}}\;dx\]

\expandafter\input{\file@loc Integrals/2311-Compute-Integral-0008.HELP.tex}

\[\dfrac{d}{dt}(g(t))=\answer{{\left(t + 8\right)}^{9}}\]
\end{problem}}%}

%%%%%%%%%%%%%%%%%%%%%%




\latexProblemContent{
\begin{problem}

Use the Fundamental Theorem of Calculus to find the derivative of the function.
\[g(t)=\int_{4}^{t} {{\left(x + 4\right)}^{16}}\;dx\]

\expandafter\input{\file@loc Integrals/2311-Compute-Integral-0008.HELP.tex}

\[\dfrac{d}{dt}(g(t))=\answer{{\left(t + 4\right)}^{16}}\]
\end{problem}}%}

%%%%%%%%%%%%%%%%%%%%%%




\latexProblemContent{
\begin{problem}

Use the Fundamental Theorem of Calculus to find the derivative of the function.
\[g(t)=\int_{2}^{t} {\sin\left(\sin\left(x - 9\right)\right)}\;dx\]

\expandafter\input{\file@loc Integrals/2311-Compute-Integral-0008.HELP.tex}

\[\dfrac{d}{dt}(g(t))=\answer{\sin\left(\sin\left(t - 9\right)\right)}\]
\end{problem}}%}

%%%%%%%%%%%%%%%%%%%%%%




\latexProblemContent{
\begin{problem}

Use the Fundamental Theorem of Calculus to find the derivative of the function.
\[g(t)=\int_{4}^{t} {{\left(x - 5\right)}^{4}}\;dx\]

\expandafter\input{\file@loc Integrals/2311-Compute-Integral-0008.HELP.tex}

\[\dfrac{d}{dt}(g(t))=\answer{{\left(t - 5\right)}^{4}}\]
\end{problem}}%}

%%%%%%%%%%%%%%%%%%%%%%




\latexProblemContent{
\begin{problem}

Use the Fundamental Theorem of Calculus to find the derivative of the function.
\[g(t)=\int_{1}^{t} {e^{\left(e^{\left(x + 5\right)}\right)}}\;dx\]

\expandafter\input{\file@loc Integrals/2311-Compute-Integral-0008.HELP.tex}

\[\dfrac{d}{dt}(g(t))=\answer{e^{\left(e^{\left(t + 5\right)}\right)}}\]
\end{problem}}%}

%%%%%%%%%%%%%%%%%%%%%%




\latexProblemContent{
\begin{problem}

Use the Fundamental Theorem of Calculus to find the derivative of the function.
\[g(t)=\int_{2}^{t} {{\left(x - 2\right)}^{16}}\;dx\]

\expandafter\input{\file@loc Integrals/2311-Compute-Integral-0008.HELP.tex}

\[\dfrac{d}{dt}(g(t))=\answer{{\left(t - 2\right)}^{16}}\]
\end{problem}}%}

%%%%%%%%%%%%%%%%%%%%%%




\latexProblemContent{
\begin{problem}

Use the Fundamental Theorem of Calculus to find the derivative of the function.
\[g(t)=\int_{2}^{t} {{\left(x + 2\right)}^{4}}\;dx\]

\expandafter\input{\file@loc Integrals/2311-Compute-Integral-0008.HELP.tex}

\[\dfrac{d}{dt}(g(t))=\answer{{\left(t + 2\right)}^{4}}\]
\end{problem}}%}

%%%%%%%%%%%%%%%%%%%%%%




\latexProblemContent{
\begin{problem}

Use the Fundamental Theorem of Calculus to find the derivative of the function.
\[g(t)=\int_{3}^{t} {{\left(x - 4\right)}^{\frac{1}{4}}}\;dx\]

\expandafter\input{\file@loc Integrals/2311-Compute-Integral-0008.HELP.tex}

\[\dfrac{d}{dt}(g(t))=\answer{{\left(t - 4\right)}^{\frac{1}{4}}}\]
\end{problem}}%}

%%%%%%%%%%%%%%%%%%%%%%




\latexProblemContent{
\begin{problem}

Use the Fundamental Theorem of Calculus to find the derivative of the function.
\[g(t)=\int_{1}^{t} {{\left(x + 4\right)}^{9}}\;dx\]

\expandafter\input{\file@loc Integrals/2311-Compute-Integral-0008.HELP.tex}

\[\dfrac{d}{dt}(g(t))=\answer{{\left(t + 4\right)}^{9}}\]
\end{problem}}%}

%%%%%%%%%%%%%%%%%%%%%%




\latexProblemContent{
\begin{problem}

Use the Fundamental Theorem of Calculus to find the derivative of the function.
\[g(t)=\int_{2}^{t} {{\left(x - 7\right)}^{4}}\;dx\]

\expandafter\input{\file@loc Integrals/2311-Compute-Integral-0008.HELP.tex}

\[\dfrac{d}{dt}(g(t))=\answer{{\left(t - 7\right)}^{4}}\]
\end{problem}}%}

%%%%%%%%%%%%%%%%%%%%%%




\latexProblemContent{
\begin{problem}

Use the Fundamental Theorem of Calculus to find the derivative of the function.
\[g(t)=\int_{4}^{t} {x + 9}\;dx\]

\expandafter\input{\file@loc Integrals/2311-Compute-Integral-0008.HELP.tex}

\[\dfrac{d}{dt}(g(t))=\answer{t + 9}\]
\end{problem}}%}

%%%%%%%%%%%%%%%%%%%%%%




\latexProblemContent{
\begin{problem}

Use the Fundamental Theorem of Calculus to find the derivative of the function.
\[g(t)=\int_{4}^{t} {{\left(x + 10\right)}^{9}}\;dx\]

\expandafter\input{\file@loc Integrals/2311-Compute-Integral-0008.HELP.tex}

\[\dfrac{d}{dt}(g(t))=\answer{{\left(t + 10\right)}^{9}}\]
\end{problem}}%}

%%%%%%%%%%%%%%%%%%%%%%




%%%%%%%%%%%%%%%%%%%%%%




\latexProblemContent{
\begin{problem}

Use the Fundamental Theorem of Calculus to find the derivative of the function.
\[g(t)=\int_{2}^{t} {{\left(x + 2\right)}^{9}}\;dx\]

\expandafter\input{\file@loc Integrals/2311-Compute-Integral-0008.HELP.tex}

\[\dfrac{d}{dt}(g(t))=\answer{{\left(t + 2\right)}^{9}}\]
\end{problem}}%}

%%%%%%%%%%%%%%%%%%%%%%




\latexProblemContent{
\begin{problem}

Use the Fundamental Theorem of Calculus to find the derivative of the function.
\[g(t)=\int_{2}^{t} {e^{\left(e^{\left(x + 10\right)}\right)}}\;dx\]

\expandafter\input{\file@loc Integrals/2311-Compute-Integral-0008.HELP.tex}

\[\dfrac{d}{dt}(g(t))=\answer{e^{\left(e^{\left(t + 10\right)}\right)}}\]
\end{problem}}%}

%%%%%%%%%%%%%%%%%%%%%%




\latexProblemContent{
\begin{problem}

Use the Fundamental Theorem of Calculus to find the derivative of the function.
\[g(t)=\int_{2}^{t} {{\left(x - 1\right)}^{4}}\;dx\]

\expandafter\input{\file@loc Integrals/2311-Compute-Integral-0008.HELP.tex}

\[\dfrac{d}{dt}(g(t))=\answer{{\left(t - 1\right)}^{4}}\]
\end{problem}}%}

%%%%%%%%%%%%%%%%%%%%%%




\latexProblemContent{
\begin{problem}

Use the Fundamental Theorem of Calculus to find the derivative of the function.
\[g(t)=\int_{3}^{t} {\sin\left(\sin\left(x - 6\right)\right)}\;dx\]

\expandafter\input{\file@loc Integrals/2311-Compute-Integral-0008.HELP.tex}

\[\dfrac{d}{dt}(g(t))=\answer{\sin\left(\sin\left(t - 6\right)\right)}\]
\end{problem}}%}

%%%%%%%%%%%%%%%%%%%%%%




\latexProblemContent{
\begin{problem}

Use the Fundamental Theorem of Calculus to find the derivative of the function.
\[g(t)=\int_{3}^{t} {{\left(x + 6\right)}^{9}}\;dx\]

\expandafter\input{\file@loc Integrals/2311-Compute-Integral-0008.HELP.tex}

\[\dfrac{d}{dt}(g(t))=\answer{{\left(t + 6\right)}^{9}}\]
\end{problem}}%}

%%%%%%%%%%%%%%%%%%%%%%




\latexProblemContent{
\begin{problem}

Use the Fundamental Theorem of Calculus to find the derivative of the function.
\[g(t)=\int_{4}^{t} {{\left(x + 8\right)}^{9}}\;dx\]

\expandafter\input{\file@loc Integrals/2311-Compute-Integral-0008.HELP.tex}

\[\dfrac{d}{dt}(g(t))=\answer{{\left(t + 8\right)}^{9}}\]
\end{problem}}%}

%%%%%%%%%%%%%%%%%%%%%%




\latexProblemContent{
\begin{problem}

Use the Fundamental Theorem of Calculus to find the derivative of the function.
\[g(t)=\int_{5}^{t} {{\left(x + 2\right)}^{9}}\;dx\]

\expandafter\input{\file@loc Integrals/2311-Compute-Integral-0008.HELP.tex}

\[\dfrac{d}{dt}(g(t))=\answer{{\left(t + 2\right)}^{9}}\]
\end{problem}}%}

%%%%%%%%%%%%%%%%%%%%%%




\latexProblemContent{
\begin{problem}

Use the Fundamental Theorem of Calculus to find the derivative of the function.
\[g(t)=\int_{5}^{t} {x + 6}\;dx\]

\expandafter\input{\file@loc Integrals/2311-Compute-Integral-0008.HELP.tex}

\[\dfrac{d}{dt}(g(t))=\answer{t + 6}\]
\end{problem}}%}

%%%%%%%%%%%%%%%%%%%%%%




\latexProblemContent{
\begin{problem}

Use the Fundamental Theorem of Calculus to find the derivative of the function.
\[g(t)=\int_{5}^{t} {x - 2}\;dx\]

\expandafter\input{\file@loc Integrals/2311-Compute-Integral-0008.HELP.tex}

\[\dfrac{d}{dt}(g(t))=\answer{t - 2}\]
\end{problem}}%}

%%%%%%%%%%%%%%%%%%%%%%




\latexProblemContent{
\begin{problem}

Use the Fundamental Theorem of Calculus to find the derivative of the function.
\[g(t)=\int_{1}^{t} {\log\left(\log\left(x - 2\right)\right)}\;dx\]

\expandafter\input{\file@loc Integrals/2311-Compute-Integral-0008.HELP.tex}

\[\dfrac{d}{dt}(g(t))=\answer{\log\left(\log\left(t - 2\right)\right)}\]
\end{problem}}%}

%%%%%%%%%%%%%%%%%%%%%%




\latexProblemContent{
\begin{problem}

Use the Fundamental Theorem of Calculus to find the derivative of the function.
\[g(t)=\int_{5}^{t} {{\left(x + 1\right)}^{4}}\;dx\]

\expandafter\input{\file@loc Integrals/2311-Compute-Integral-0008.HELP.tex}

\[\dfrac{d}{dt}(g(t))=\answer{{\left(t + 1\right)}^{4}}\]
\end{problem}}%}

%%%%%%%%%%%%%%%%%%%%%%




\latexProblemContent{
\begin{problem}

Use the Fundamental Theorem of Calculus to find the derivative of the function.
\[g(t)=\int_{3}^{t} {e^{\left(e^{\left(x - 3\right)}\right)}}\;dx\]

\expandafter\input{\file@loc Integrals/2311-Compute-Integral-0008.HELP.tex}

\[\dfrac{d}{dt}(g(t))=\answer{e^{\left(e^{\left(t - 3\right)}\right)}}\]
\end{problem}}%}

%%%%%%%%%%%%%%%%%%%%%%




\latexProblemContent{
\begin{problem}

Use the Fundamental Theorem of Calculus to find the derivative of the function.
\[g(t)=\int_{2}^{t} {{\left(x + 9\right)}^{\frac{1}{4}}}\;dx\]

\expandafter\input{\file@loc Integrals/2311-Compute-Integral-0008.HELP.tex}

\[\dfrac{d}{dt}(g(t))=\answer{{\left(t + 9\right)}^{\frac{1}{4}}}\]
\end{problem}}%}

%%%%%%%%%%%%%%%%%%%%%%




\latexProblemContent{
\begin{problem}

Use the Fundamental Theorem of Calculus to find the derivative of the function.
\[g(t)=\int_{3}^{t} {{\left(x + 7\right)}^{4}}\;dx\]

\expandafter\input{\file@loc Integrals/2311-Compute-Integral-0008.HELP.tex}

\[\dfrac{d}{dt}(g(t))=\answer{{\left(t + 7\right)}^{4}}\]
\end{problem}}%}

%%%%%%%%%%%%%%%%%%%%%%




\latexProblemContent{
\begin{problem}

Use the Fundamental Theorem of Calculus to find the derivative of the function.
\[g(t)=\int_{1}^{t} {{\left(x - 7\right)}^{9}}\;dx\]

\expandafter\input{\file@loc Integrals/2311-Compute-Integral-0008.HELP.tex}

\[\dfrac{d}{dt}(g(t))=\answer{{\left(t - 7\right)}^{9}}\]
\end{problem}}%}

%%%%%%%%%%%%%%%%%%%%%%




\latexProblemContent{
\begin{problem}

Use the Fundamental Theorem of Calculus to find the derivative of the function.
\[g(t)=\int_{4}^{t} {e^{\left(e^{\left(x + 7\right)}\right)}}\;dx\]

\expandafter\input{\file@loc Integrals/2311-Compute-Integral-0008.HELP.tex}

\[\dfrac{d}{dt}(g(t))=\answer{e^{\left(e^{\left(t + 7\right)}\right)}}\]
\end{problem}}%}

%%%%%%%%%%%%%%%%%%%%%%




\latexProblemContent{
\begin{problem}

Use the Fundamental Theorem of Calculus to find the derivative of the function.
\[g(t)=\int_{2}^{t} {{\left(x + 1\right)}^{4}}\;dx\]

\expandafter\input{\file@loc Integrals/2311-Compute-Integral-0008.HELP.tex}

\[\dfrac{d}{dt}(g(t))=\answer{{\left(t + 1\right)}^{4}}\]
\end{problem}}%}

%%%%%%%%%%%%%%%%%%%%%%




\latexProblemContent{
\begin{problem}

Use the Fundamental Theorem of Calculus to find the derivative of the function.
\[g(t)=\int_{4}^{t} {{\left(x + 9\right)}^{4}}\;dx\]

\expandafter\input{\file@loc Integrals/2311-Compute-Integral-0008.HELP.tex}

\[\dfrac{d}{dt}(g(t))=\answer{{\left(t + 9\right)}^{4}}\]
\end{problem}}%}

%%%%%%%%%%%%%%%%%%%%%%




\latexProblemContent{
\begin{problem}

Use the Fundamental Theorem of Calculus to find the derivative of the function.
\[g(t)=\int_{4}^{t} {\cos\left(\cos\left(x + 10\right)\right)}\;dx\]

\expandafter\input{\file@loc Integrals/2311-Compute-Integral-0008.HELP.tex}

\[\dfrac{d}{dt}(g(t))=\answer{\cos\left(\cos\left(t + 10\right)\right)}\]
\end{problem}}%}

%%%%%%%%%%%%%%%%%%%%%%




\latexProblemContent{
\begin{problem}

Use the Fundamental Theorem of Calculus to find the derivative of the function.
\[g(t)=\int_{3}^{t} {{\left(x + 8\right)}^{4}}\;dx\]

\expandafter\input{\file@loc Integrals/2311-Compute-Integral-0008.HELP.tex}

\[\dfrac{d}{dt}(g(t))=\answer{{\left(t + 8\right)}^{4}}\]
\end{problem}}%}

%%%%%%%%%%%%%%%%%%%%%%




\latexProblemContent{
\begin{problem}

Use the Fundamental Theorem of Calculus to find the derivative of the function.
\[g(t)=\int_{4}^{t} {\sin\left(\sin\left(x - 5\right)\right)}\;dx\]

\expandafter\input{\file@loc Integrals/2311-Compute-Integral-0008.HELP.tex}

\[\dfrac{d}{dt}(g(t))=\answer{\sin\left(\sin\left(t - 5\right)\right)}\]
\end{problem}}%}

%%%%%%%%%%%%%%%%%%%%%%




\latexProblemContent{
\begin{problem}

Use the Fundamental Theorem of Calculus to find the derivative of the function.
\[g(t)=\int_{5}^{t} {\sin\left(\sin\left(x + 7\right)\right)}\;dx\]

\expandafter\input{\file@loc Integrals/2311-Compute-Integral-0008.HELP.tex}

\[\dfrac{d}{dt}(g(t))=\answer{\sin\left(\sin\left(t + 7\right)\right)}\]
\end{problem}}%}

%%%%%%%%%%%%%%%%%%%%%%




\latexProblemContent{
\begin{problem}

Use the Fundamental Theorem of Calculus to find the derivative of the function.
\[g(t)=\int_{3}^{t} {e^{\left(e^{\left(x + 8\right)}\right)}}\;dx\]

\expandafter\input{\file@loc Integrals/2311-Compute-Integral-0008.HELP.tex}

\[\dfrac{d}{dt}(g(t))=\answer{e^{\left(e^{\left(t + 8\right)}\right)}}\]
\end{problem}}%}

%%%%%%%%%%%%%%%%%%%%%%




\latexProblemContent{
\begin{problem}

Use the Fundamental Theorem of Calculus to find the derivative of the function.
\[g(t)=\int_{4}^{t} {\cos\left(\cos\left(x + 8\right)\right)}\;dx\]

\expandafter\input{\file@loc Integrals/2311-Compute-Integral-0008.HELP.tex}

\[\dfrac{d}{dt}(g(t))=\answer{\cos\left(\cos\left(t + 8\right)\right)}\]
\end{problem}}%}

%%%%%%%%%%%%%%%%%%%%%%




\latexProblemContent{
\begin{problem}

Use the Fundamental Theorem of Calculus to find the derivative of the function.
\[g(t)=\int_{5}^{t} {{\left(x + 7\right)}^{\frac{1}{4}}}\;dx\]

\expandafter\input{\file@loc Integrals/2311-Compute-Integral-0008.HELP.tex}

\[\dfrac{d}{dt}(g(t))=\answer{{\left(t + 7\right)}^{\frac{1}{4}}}\]
\end{problem}}%}

%%%%%%%%%%%%%%%%%%%%%%




\latexProblemContent{
\begin{problem}

Use the Fundamental Theorem of Calculus to find the derivative of the function.
\[g(t)=\int_{4}^{t} {e^{\left(e^{\left(x - 1\right)}\right)}}\;dx\]

\expandafter\input{\file@loc Integrals/2311-Compute-Integral-0008.HELP.tex}

\[\dfrac{d}{dt}(g(t))=\answer{e^{\left(e^{\left(t - 1\right)}\right)}}\]
\end{problem}}%}

%%%%%%%%%%%%%%%%%%%%%%




\latexProblemContent{
\begin{problem}

Use the Fundamental Theorem of Calculus to find the derivative of the function.
\[g(t)=\int_{4}^{t} {{\left(x - 4\right)}^{9}}\;dx\]

\expandafter\input{\file@loc Integrals/2311-Compute-Integral-0008.HELP.tex}

\[\dfrac{d}{dt}(g(t))=\answer{{\left(t - 4\right)}^{9}}\]
\end{problem}}%}

%%%%%%%%%%%%%%%%%%%%%%




\latexProblemContent{
\begin{problem}

Use the Fundamental Theorem of Calculus to find the derivative of the function.
\[g(t)=\int_{2}^{t} {\sin\left(\sin\left(x + 7\right)\right)}\;dx\]

\expandafter\input{\file@loc Integrals/2311-Compute-Integral-0008.HELP.tex}

\[\dfrac{d}{dt}(g(t))=\answer{\sin\left(\sin\left(t + 7\right)\right)}\]
\end{problem}}%}

%%%%%%%%%%%%%%%%%%%%%%




\latexProblemContent{
\begin{problem}

Use the Fundamental Theorem of Calculus to find the derivative of the function.
\[g(t)=\int_{3}^{t} {{\left(x + 2\right)}^{\frac{1}{4}}}\;dx\]

\expandafter\input{\file@loc Integrals/2311-Compute-Integral-0008.HELP.tex}

\[\dfrac{d}{dt}(g(t))=\answer{{\left(t + 2\right)}^{\frac{1}{4}}}\]
\end{problem}}%}

%%%%%%%%%%%%%%%%%%%%%%




\latexProblemContent{
\begin{problem}

Use the Fundamental Theorem of Calculus to find the derivative of the function.
\[g(t)=\int_{4}^{t} {\cos\left(\cos\left(x + 9\right)\right)}\;dx\]

\expandafter\input{\file@loc Integrals/2311-Compute-Integral-0008.HELP.tex}

\[\dfrac{d}{dt}(g(t))=\answer{\cos\left(\cos\left(t + 9\right)\right)}\]
\end{problem}}%}

%%%%%%%%%%%%%%%%%%%%%%




\latexProblemContent{
\begin{problem}

Use the Fundamental Theorem of Calculus to find the derivative of the function.
\[g(t)=\int_{1}^{t} {{\left(x - 3\right)}^{9}}\;dx\]

\expandafter\input{\file@loc Integrals/2311-Compute-Integral-0008.HELP.tex}

\[\dfrac{d}{dt}(g(t))=\answer{{\left(t - 3\right)}^{9}}\]
\end{problem}}%}

%%%%%%%%%%%%%%%%%%%%%%




\latexProblemContent{
\begin{problem}

Use the Fundamental Theorem of Calculus to find the derivative of the function.
\[g(t)=\int_{2}^{t} {{\left(x - 5\right)}^{4}}\;dx\]

\expandafter\input{\file@loc Integrals/2311-Compute-Integral-0008.HELP.tex}

\[\dfrac{d}{dt}(g(t))=\answer{{\left(t - 5\right)}^{4}}\]
\end{problem}}%}

%%%%%%%%%%%%%%%%%%%%%%




\latexProblemContent{
\begin{problem}

Use the Fundamental Theorem of Calculus to find the derivative of the function.
\[g(t)=\int_{5}^{t} {{\left(x + 5\right)}^{\frac{1}{4}}}\;dx\]

\expandafter\input{\file@loc Integrals/2311-Compute-Integral-0008.HELP.tex}

\[\dfrac{d}{dt}(g(t))=\answer{{\left(t + 5\right)}^{\frac{1}{4}}}\]
\end{problem}}%}

%%%%%%%%%%%%%%%%%%%%%%




\latexProblemContent{
\begin{problem}

Use the Fundamental Theorem of Calculus to find the derivative of the function.
\[g(t)=\int_{4}^{t} {x + 2}\;dx\]

\expandafter\input{\file@loc Integrals/2311-Compute-Integral-0008.HELP.tex}

\[\dfrac{d}{dt}(g(t))=\answer{t + 2}\]
\end{problem}}%}

%%%%%%%%%%%%%%%%%%%%%%




\latexProblemContent{
\begin{problem}

Use the Fundamental Theorem of Calculus to find the derivative of the function.
\[g(t)=\int_{2}^{t} {{\left(x - 3\right)}^{4}}\;dx\]

\expandafter\input{\file@loc Integrals/2311-Compute-Integral-0008.HELP.tex}

\[\dfrac{d}{dt}(g(t))=\answer{{\left(t - 3\right)}^{4}}\]
\end{problem}}%}

%%%%%%%%%%%%%%%%%%%%%%




\latexProblemContent{
\begin{problem}

Use the Fundamental Theorem of Calculus to find the derivative of the function.
\[g(t)=\int_{4}^{t} {{\left(x - 4\right)}^{\frac{1}{4}}}\;dx\]

\expandafter\input{\file@loc Integrals/2311-Compute-Integral-0008.HELP.tex}

\[\dfrac{d}{dt}(g(t))=\answer{{\left(t - 4\right)}^{\frac{1}{4}}}\]
\end{problem}}%}

%%%%%%%%%%%%%%%%%%%%%%




\latexProblemContent{
\begin{problem}

Use the Fundamental Theorem of Calculus to find the derivative of the function.
\[g(t)=\int_{2}^{t} {x - 16}\;dx\]

\expandafter\input{\file@loc Integrals/2311-Compute-Integral-0008.HELP.tex}

\[\dfrac{d}{dt}(g(t))=\answer{t - 16}\]
\end{problem}}%}

%%%%%%%%%%%%%%%%%%%%%%




\latexProblemContent{
\begin{problem}

Use the Fundamental Theorem of Calculus to find the derivative of the function.
\[g(t)=\int_{3}^{t} {\log\left(\log\left(x - 4\right)\right)}\;dx\]

\expandafter\input{\file@loc Integrals/2311-Compute-Integral-0008.HELP.tex}

\[\dfrac{d}{dt}(g(t))=\answer{\log\left(\log\left(t - 4\right)\right)}\]
\end{problem}}%}

%%%%%%%%%%%%%%%%%%%%%%




\latexProblemContent{
\begin{problem}

Use the Fundamental Theorem of Calculus to find the derivative of the function.
\[g(t)=\int_{3}^{t} {{\left(x - 3\right)}^{\frac{1}{4}}}\;dx\]

\expandafter\input{\file@loc Integrals/2311-Compute-Integral-0008.HELP.tex}

\[\dfrac{d}{dt}(g(t))=\answer{{\left(t - 3\right)}^{\frac{1}{4}}}\]
\end{problem}}%}

%%%%%%%%%%%%%%%%%%%%%%




\latexProblemContent{
\begin{problem}

Use the Fundamental Theorem of Calculus to find the derivative of the function.
\[g(t)=\int_{2}^{t} {x + 10}\;dx\]

\expandafter\input{\file@loc Integrals/2311-Compute-Integral-0008.HELP.tex}

\[\dfrac{d}{dt}(g(t))=\answer{t + 10}\]
\end{problem}}%}

%%%%%%%%%%%%%%%%%%%%%%




\latexProblemContent{
\begin{problem}

Use the Fundamental Theorem of Calculus to find the derivative of the function.
\[g(t)=\int_{1}^{t} {\sin\left(\sin\left(x + 9\right)\right)}\;dx\]

\expandafter\input{\file@loc Integrals/2311-Compute-Integral-0008.HELP.tex}

\[\dfrac{d}{dt}(g(t))=\answer{\sin\left(\sin\left(t + 9\right)\right)}\]
\end{problem}}%}

%%%%%%%%%%%%%%%%%%%%%%




%%%%%%%%%%%%%%%%%%%%%%




\latexProblemContent{
\begin{problem}

Use the Fundamental Theorem of Calculus to find the derivative of the function.
\[g(t)=\int_{2}^{t} {{\left(x - 10\right)}^{9}}\;dx\]

\expandafter\input{\file@loc Integrals/2311-Compute-Integral-0008.HELP.tex}

\[\dfrac{d}{dt}(g(t))=\answer{{\left(t - 10\right)}^{9}}\]
\end{problem}}%}

%%%%%%%%%%%%%%%%%%%%%%




\latexProblemContent{
\begin{problem}

Use the Fundamental Theorem of Calculus to find the derivative of the function.
\[g(t)=\int_{5}^{t} {e^{\left(e^{\left(x - 9\right)}\right)}}\;dx\]

\expandafter\input{\file@loc Integrals/2311-Compute-Integral-0008.HELP.tex}

\[\dfrac{d}{dt}(g(t))=\answer{e^{\left(e^{\left(t - 9\right)}\right)}}\]
\end{problem}}%}

%%%%%%%%%%%%%%%%%%%%%%




%%%%%%%%%%%%%%%%%%%%%%




\latexProblemContent{
\begin{problem}

Use the Fundamental Theorem of Calculus to find the derivative of the function.
\[g(t)=\int_{2}^{t} {x - 2}\;dx\]

\expandafter\input{\file@loc Integrals/2311-Compute-Integral-0008.HELP.tex}

\[\dfrac{d}{dt}(g(t))=\answer{t - 2}\]
\end{problem}}%}

%%%%%%%%%%%%%%%%%%%%%%




\latexProblemContent{
\begin{problem}

Use the Fundamental Theorem of Calculus to find the derivative of the function.
\[g(t)=\int_{3}^{t} {x + 9}\;dx\]

\expandafter\input{\file@loc Integrals/2311-Compute-Integral-0008.HELP.tex}

\[\dfrac{d}{dt}(g(t))=\answer{t + 9}\]
\end{problem}}%}

%%%%%%%%%%%%%%%%%%%%%%




\latexProblemContent{
\begin{problem}

Use the Fundamental Theorem of Calculus to find the derivative of the function.
\[g(t)=\int_{5}^{t} {{\left(x - 5\right)}^{4}}\;dx\]

\expandafter\input{\file@loc Integrals/2311-Compute-Integral-0008.HELP.tex}

\[\dfrac{d}{dt}(g(t))=\answer{{\left(t - 5\right)}^{4}}\]
\end{problem}}%}

%%%%%%%%%%%%%%%%%%%%%%




\latexProblemContent{
\begin{problem}

Use the Fundamental Theorem of Calculus to find the derivative of the function.
\[g(t)=\int_{3}^{t} {{\left(x + 5\right)}^{16}}\;dx\]

\expandafter\input{\file@loc Integrals/2311-Compute-Integral-0008.HELP.tex}

\[\dfrac{d}{dt}(g(t))=\answer{{\left(t + 5\right)}^{16}}\]
\end{problem}}%}

%%%%%%%%%%%%%%%%%%%%%%




\latexProblemContent{
\begin{problem}

Use the Fundamental Theorem of Calculus to find the derivative of the function.
\[g(t)=\int_{3}^{t} {{\left(x - 4\right)}^{9}}\;dx\]

\expandafter\input{\file@loc Integrals/2311-Compute-Integral-0008.HELP.tex}

\[\dfrac{d}{dt}(g(t))=\answer{{\left(t - 4\right)}^{9}}\]
\end{problem}}%}

%%%%%%%%%%%%%%%%%%%%%%




\latexProblemContent{
\begin{problem}

Use the Fundamental Theorem of Calculus to find the derivative of the function.
\[g(t)=\int_{1}^{t} {x + 4}\;dx\]

\expandafter\input{\file@loc Integrals/2311-Compute-Integral-0008.HELP.tex}

\[\dfrac{d}{dt}(g(t))=\answer{t + 4}\]
\end{problem}}%}

%%%%%%%%%%%%%%%%%%%%%%




\latexProblemContent{
\begin{problem}

Use the Fundamental Theorem of Calculus to find the derivative of the function.
\[g(t)=\int_{5}^{t} {{\left(x - 1\right)}^{16}}\;dx\]

\expandafter\input{\file@loc Integrals/2311-Compute-Integral-0008.HELP.tex}

\[\dfrac{d}{dt}(g(t))=\answer{{\left(t - 1\right)}^{16}}\]
\end{problem}}%}

%%%%%%%%%%%%%%%%%%%%%%




\latexProblemContent{
\begin{problem}

Use the Fundamental Theorem of Calculus to find the derivative of the function.
\[g(t)=\int_{3}^{t} {{\left(x + 1\right)}^{4}}\;dx\]

\expandafter\input{\file@loc Integrals/2311-Compute-Integral-0008.HELP.tex}

\[\dfrac{d}{dt}(g(t))=\answer{{\left(t + 1\right)}^{4}}\]
\end{problem}}%}

%%%%%%%%%%%%%%%%%%%%%%




\latexProblemContent{
\begin{problem}

Use the Fundamental Theorem of Calculus to find the derivative of the function.
\[g(t)=\int_{4}^{t} {{\left(x + 10\right)}^{16}}\;dx\]

\expandafter\input{\file@loc Integrals/2311-Compute-Integral-0008.HELP.tex}

\[\dfrac{d}{dt}(g(t))=\answer{{\left(t + 10\right)}^{16}}\]
\end{problem}}%}

%%%%%%%%%%%%%%%%%%%%%%




\latexProblemContent{
\begin{problem}

Use the Fundamental Theorem of Calculus to find the derivative of the function.
\[g(t)=\int_{4}^{t} {{\left(x - 8\right)}^{9}}\;dx\]

\expandafter\input{\file@loc Integrals/2311-Compute-Integral-0008.HELP.tex}

\[\dfrac{d}{dt}(g(t))=\answer{{\left(t - 8\right)}^{9}}\]
\end{problem}}%}

%%%%%%%%%%%%%%%%%%%%%%




\latexProblemContent{
\begin{problem}

Use the Fundamental Theorem of Calculus to find the derivative of the function.
\[g(t)=\int_{4}^{t} {x + 20}\;dx\]

\expandafter\input{\file@loc Integrals/2311-Compute-Integral-0008.HELP.tex}

\[\dfrac{d}{dt}(g(t))=\answer{t + 20}\]
\end{problem}}%}

%%%%%%%%%%%%%%%%%%%%%%




\latexProblemContent{
\begin{problem}

Use the Fundamental Theorem of Calculus to find the derivative of the function.
\[g(t)=\int_{2}^{t} {{\left(x + 8\right)}^{9}}\;dx\]

\expandafter\input{\file@loc Integrals/2311-Compute-Integral-0008.HELP.tex}

\[\dfrac{d}{dt}(g(t))=\answer{{\left(t + 8\right)}^{9}}\]
\end{problem}}%}

%%%%%%%%%%%%%%%%%%%%%%




\latexProblemContent{
\begin{problem}

Use the Fundamental Theorem of Calculus to find the derivative of the function.
\[g(t)=\int_{1}^{t} {x + 6}\;dx\]

\expandafter\input{\file@loc Integrals/2311-Compute-Integral-0008.HELP.tex}

\[\dfrac{d}{dt}(g(t))=\answer{t + 6}\]
\end{problem}}%}

%%%%%%%%%%%%%%%%%%%%%%




\latexProblemContent{
\begin{problem}

Use the Fundamental Theorem of Calculus to find the derivative of the function.
\[g(t)=\int_{4}^{t} {e^{\left(e^{\left(x - 6\right)}\right)}}\;dx\]

\expandafter\input{\file@loc Integrals/2311-Compute-Integral-0008.HELP.tex}

\[\dfrac{d}{dt}(g(t))=\answer{e^{\left(e^{\left(t - 6\right)}\right)}}\]
\end{problem}}%}

%%%%%%%%%%%%%%%%%%%%%%




\latexProblemContent{
\begin{problem}

Use the Fundamental Theorem of Calculus to find the derivative of the function.
\[g(t)=\int_{5}^{t} {{\left(x + 9\right)}^{9}}\;dx\]

\expandafter\input{\file@loc Integrals/2311-Compute-Integral-0008.HELP.tex}

\[\dfrac{d}{dt}(g(t))=\answer{{\left(t + 9\right)}^{9}}\]
\end{problem}}%}

%%%%%%%%%%%%%%%%%%%%%%




\latexProblemContent{
\begin{problem}

Use the Fundamental Theorem of Calculus to find the derivative of the function.
\[g(t)=\int_{5}^{t} {x + 8}\;dx\]

\expandafter\input{\file@loc Integrals/2311-Compute-Integral-0008.HELP.tex}

\[\dfrac{d}{dt}(g(t))=\answer{t + 8}\]
\end{problem}}%}

%%%%%%%%%%%%%%%%%%%%%%




%%%%%%%%%%%%%%%%%%%%%%




\latexProblemContent{
\begin{problem}

Use the Fundamental Theorem of Calculus to find the derivative of the function.
\[g(t)=\int_{5}^{t} {\cos\left(\cos\left(x + 2\right)\right)}\;dx\]

\expandafter\input{\file@loc Integrals/2311-Compute-Integral-0008.HELP.tex}

\[\dfrac{d}{dt}(g(t))=\answer{\cos\left(\cos\left(t + 2\right)\right)}\]
\end{problem}}%}

%%%%%%%%%%%%%%%%%%%%%%




\latexProblemContent{
\begin{problem}

Use the Fundamental Theorem of Calculus to find the derivative of the function.
\[g(t)=\int_{3}^{t} {x - 5}\;dx\]

\expandafter\input{\file@loc Integrals/2311-Compute-Integral-0008.HELP.tex}

\[\dfrac{d}{dt}(g(t))=\answer{t - 5}\]
\end{problem}}%}

%%%%%%%%%%%%%%%%%%%%%%




\latexProblemContent{
\begin{problem}

Use the Fundamental Theorem of Calculus to find the derivative of the function.
\[g(t)=\int_{5}^{t} {x - 8}\;dx\]

\expandafter\input{\file@loc Integrals/2311-Compute-Integral-0008.HELP.tex}

\[\dfrac{d}{dt}(g(t))=\answer{t - 8}\]
\end{problem}}%}

%%%%%%%%%%%%%%%%%%%%%%




\latexProblemContent{
\begin{problem}

Use the Fundamental Theorem of Calculus to find the derivative of the function.
\[g(t)=\int_{1}^{t} {{\left(x + 8\right)}^{\frac{1}{4}}}\;dx\]

\expandafter\input{\file@loc Integrals/2311-Compute-Integral-0008.HELP.tex}

\[\dfrac{d}{dt}(g(t))=\answer{{\left(t + 8\right)}^{\frac{1}{4}}}\]
\end{problem}}%}

%%%%%%%%%%%%%%%%%%%%%%




\latexProblemContent{
\begin{problem}

Use the Fundamental Theorem of Calculus to find the derivative of the function.
\[g(t)=\int_{5}^{t} {{\left(x - 5\right)}^{9}}\;dx\]

\expandafter\input{\file@loc Integrals/2311-Compute-Integral-0008.HELP.tex}

\[\dfrac{d}{dt}(g(t))=\answer{{\left(t - 5\right)}^{9}}\]
\end{problem}}%}

%%%%%%%%%%%%%%%%%%%%%%




\latexProblemContent{
\begin{problem}

Use the Fundamental Theorem of Calculus to find the derivative of the function.
\[g(t)=\int_{5}^{t} {x - 14}\;dx\]

\expandafter\input{\file@loc Integrals/2311-Compute-Integral-0008.HELP.tex}

\[\dfrac{d}{dt}(g(t))=\answer{t - 14}\]
\end{problem}}%}

%%%%%%%%%%%%%%%%%%%%%%




\latexProblemContent{
\begin{problem}

Use the Fundamental Theorem of Calculus to find the derivative of the function.
\[g(t)=\int_{3}^{t} {x - 10}\;dx\]

\expandafter\input{\file@loc Integrals/2311-Compute-Integral-0008.HELP.tex}

\[\dfrac{d}{dt}(g(t))=\answer{t - 10}\]
\end{problem}}%}

%%%%%%%%%%%%%%%%%%%%%%




\latexProblemContent{
\begin{problem}

Use the Fundamental Theorem of Calculus to find the derivative of the function.
\[g(t)=\int_{5}^{t} {\sin\left(\sin\left(x + 1\right)\right)}\;dx\]

\expandafter\input{\file@loc Integrals/2311-Compute-Integral-0008.HELP.tex}

\[\dfrac{d}{dt}(g(t))=\answer{\sin\left(\sin\left(t + 1\right)\right)}\]
\end{problem}}%}

%%%%%%%%%%%%%%%%%%%%%%




\latexProblemContent{
\begin{problem}

Use the Fundamental Theorem of Calculus to find the derivative of the function.
\[g(t)=\int_{5}^{t} {x - 6}\;dx\]

\expandafter\input{\file@loc Integrals/2311-Compute-Integral-0008.HELP.tex}

\[\dfrac{d}{dt}(g(t))=\answer{t - 6}\]
\end{problem}}%}

%%%%%%%%%%%%%%%%%%%%%%




\latexProblemContent{
\begin{problem}

Use the Fundamental Theorem of Calculus to find the derivative of the function.
\[g(t)=\int_{2}^{t} {e^{\left(e^{\left(x + 5\right)}\right)}}\;dx\]

\expandafter\input{\file@loc Integrals/2311-Compute-Integral-0008.HELP.tex}

\[\dfrac{d}{dt}(g(t))=\answer{e^{\left(e^{\left(t + 5\right)}\right)}}\]
\end{problem}}%}

%%%%%%%%%%%%%%%%%%%%%%




\latexProblemContent{
\begin{problem}

Use the Fundamental Theorem of Calculus to find the derivative of the function.
\[g(t)=\int_{3}^{t} {{\left(x + 8\right)}^{\frac{1}{4}}}\;dx\]

\expandafter\input{\file@loc Integrals/2311-Compute-Integral-0008.HELP.tex}

\[\dfrac{d}{dt}(g(t))=\answer{{\left(t + 8\right)}^{\frac{1}{4}}}\]
\end{problem}}%}

%%%%%%%%%%%%%%%%%%%%%%




%%%%%%%%%%%%%%%%%%%%%%




%%%%%%%%%%%%%%%%%%%%%%




%%%%%%%%%%%%%%%%%%%%%%




\latexProblemContent{
\begin{problem}

Use the Fundamental Theorem of Calculus to find the derivative of the function.
\[g(t)=\int_{5}^{t} {\log\left(\log\left(x - 1\right)\right)}\;dx\]

\expandafter\input{\file@loc Integrals/2311-Compute-Integral-0008.HELP.tex}

\[\dfrac{d}{dt}(g(t))=\answer{\log\left(\log\left(t - 1\right)\right)}\]
\end{problem}}%}

%%%%%%%%%%%%%%%%%%%%%%




\latexProblemContent{
\begin{problem}

Use the Fundamental Theorem of Calculus to find the derivative of the function.
\[g(t)=\int_{2}^{t} {{\left(x - 9\right)}^{9}}\;dx\]

\expandafter\input{\file@loc Integrals/2311-Compute-Integral-0008.HELP.tex}

\[\dfrac{d}{dt}(g(t))=\answer{{\left(t - 9\right)}^{9}}\]
\end{problem}}%}

%%%%%%%%%%%%%%%%%%%%%%




\latexProblemContent{
\begin{problem}

Use the Fundamental Theorem of Calculus to find the derivative of the function.
\[g(t)=\int_{3}^{t} {x + 6}\;dx\]

\expandafter\input{\file@loc Integrals/2311-Compute-Integral-0008.HELP.tex}

\[\dfrac{d}{dt}(g(t))=\answer{t + 6}\]
\end{problem}}%}

%%%%%%%%%%%%%%%%%%%%%%




\latexProblemContent{
\begin{problem}

Use the Fundamental Theorem of Calculus to find the derivative of the function.
\[g(t)=\int_{1}^{t} {{\left(x + 10\right)}^{4}}\;dx\]

\expandafter\input{\file@loc Integrals/2311-Compute-Integral-0008.HELP.tex}

\[\dfrac{d}{dt}(g(t))=\answer{{\left(t + 10\right)}^{4}}\]
\end{problem}}%}

%%%%%%%%%%%%%%%%%%%%%%




\latexProblemContent{
\begin{problem}

Use the Fundamental Theorem of Calculus to find the derivative of the function.
\[g(t)=\int_{3}^{t} {\sin\left(\sin\left(x + 3\right)\right)}\;dx\]

\expandafter\input{\file@loc Integrals/2311-Compute-Integral-0008.HELP.tex}

\[\dfrac{d}{dt}(g(t))=\answer{\sin\left(\sin\left(t + 3\right)\right)}\]
\end{problem}}%}

%%%%%%%%%%%%%%%%%%%%%%




\latexProblemContent{
\begin{problem}

Use the Fundamental Theorem of Calculus to find the derivative of the function.
\[g(t)=\int_{2}^{t} {x - 6}\;dx\]

\expandafter\input{\file@loc Integrals/2311-Compute-Integral-0008.HELP.tex}

\[\dfrac{d}{dt}(g(t))=\answer{t - 6}\]
\end{problem}}%}

%%%%%%%%%%%%%%%%%%%%%%




\latexProblemContent{
\begin{problem}

Use the Fundamental Theorem of Calculus to find the derivative of the function.
\[g(t)=\int_{3}^{t} {{\left(x - 7\right)}^{9}}\;dx\]

\expandafter\input{\file@loc Integrals/2311-Compute-Integral-0008.HELP.tex}

\[\dfrac{d}{dt}(g(t))=\answer{{\left(t - 7\right)}^{9}}\]
\end{problem}}%}

%%%%%%%%%%%%%%%%%%%%%%




\latexProblemContent{
\begin{problem}

Use the Fundamental Theorem of Calculus to find the derivative of the function.
\[g(t)=\int_{1}^{t} {{\left(x - 10\right)}^{9}}\;dx\]

\expandafter\input{\file@loc Integrals/2311-Compute-Integral-0008.HELP.tex}

\[\dfrac{d}{dt}(g(t))=\answer{{\left(t - 10\right)}^{9}}\]
\end{problem}}%}

%%%%%%%%%%%%%%%%%%%%%%




\latexProblemContent{
\begin{problem}

Use the Fundamental Theorem of Calculus to find the derivative of the function.
\[g(t)=\int_{3}^{t} {{\left(x + 7\right)}^{\frac{1}{4}}}\;dx\]

\expandafter\input{\file@loc Integrals/2311-Compute-Integral-0008.HELP.tex}

\[\dfrac{d}{dt}(g(t))=\answer{{\left(t + 7\right)}^{\frac{1}{4}}}\]
\end{problem}}%}

%%%%%%%%%%%%%%%%%%%%%%




\latexProblemContent{
\begin{problem}

Use the Fundamental Theorem of Calculus to find the derivative of the function.
\[g(t)=\int_{3}^{t} {{\left(x - 7\right)}^{4}}\;dx\]

\expandafter\input{\file@loc Integrals/2311-Compute-Integral-0008.HELP.tex}

\[\dfrac{d}{dt}(g(t))=\answer{{\left(t - 7\right)}^{4}}\]
\end{problem}}%}

%%%%%%%%%%%%%%%%%%%%%%




\latexProblemContent{
\begin{problem}

Use the Fundamental Theorem of Calculus to find the derivative of the function.
\[g(t)=\int_{3}^{t} {{\left(x - 6\right)}^{4}}\;dx\]

\expandafter\input{\file@loc Integrals/2311-Compute-Integral-0008.HELP.tex}

\[\dfrac{d}{dt}(g(t))=\answer{{\left(t - 6\right)}^{4}}\]
\end{problem}}%}

%%%%%%%%%%%%%%%%%%%%%%




\latexProblemContent{
\begin{problem}

Use the Fundamental Theorem of Calculus to find the derivative of the function.
\[g(t)=\int_{2}^{t} {x + 2}\;dx\]

\expandafter\input{\file@loc Integrals/2311-Compute-Integral-0008.HELP.tex}

\[\dfrac{d}{dt}(g(t))=\answer{t + 2}\]
\end{problem}}%}

%%%%%%%%%%%%%%%%%%%%%%




\latexProblemContent{
\begin{problem}

Use the Fundamental Theorem of Calculus to find the derivative of the function.
\[g(t)=\int_{4}^{t} {\sin\left(\sin\left(x - 10\right)\right)}\;dx\]

\expandafter\input{\file@loc Integrals/2311-Compute-Integral-0008.HELP.tex}

\[\dfrac{d}{dt}(g(t))=\answer{\sin\left(\sin\left(t - 10\right)\right)}\]
\end{problem}}%}

%%%%%%%%%%%%%%%%%%%%%%




\latexProblemContent{
\begin{problem}

Use the Fundamental Theorem of Calculus to find the derivative of the function.
\[g(t)=\int_{5}^{t} {{\left(x + 10\right)}^{9}}\;dx\]

\expandafter\input{\file@loc Integrals/2311-Compute-Integral-0008.HELP.tex}

\[\dfrac{d}{dt}(g(t))=\answer{{\left(t + 10\right)}^{9}}\]
\end{problem}}%}

%%%%%%%%%%%%%%%%%%%%%%




\latexProblemContent{
\begin{problem}

Use the Fundamental Theorem of Calculus to find the derivative of the function.
\[g(t)=\int_{1}^{t} {e^{\left(e^{\left(x - 7\right)}\right)}}\;dx\]

\expandafter\input{\file@loc Integrals/2311-Compute-Integral-0008.HELP.tex}

\[\dfrac{d}{dt}(g(t))=\answer{e^{\left(e^{\left(t - 7\right)}\right)}}\]
\end{problem}}%}

%%%%%%%%%%%%%%%%%%%%%%




\latexProblemContent{
\begin{problem}

Use the Fundamental Theorem of Calculus to find the derivative of the function.
\[g(t)=\int_{5}^{t} {\cos\left(\cos\left(x + 7\right)\right)}\;dx\]

\expandafter\input{\file@loc Integrals/2311-Compute-Integral-0008.HELP.tex}

\[\dfrac{d}{dt}(g(t))=\answer{\cos\left(\cos\left(t + 7\right)\right)}\]
\end{problem}}%}

%%%%%%%%%%%%%%%%%%%%%%




\latexProblemContent{
\begin{problem}

Use the Fundamental Theorem of Calculus to find the derivative of the function.
\[g(t)=\int_{3}^{t} {\sin\left(\sin\left(x + 2\right)\right)}\;dx\]

\expandafter\input{\file@loc Integrals/2311-Compute-Integral-0008.HELP.tex}

\[\dfrac{d}{dt}(g(t))=\answer{\sin\left(\sin\left(t + 2\right)\right)}\]
\end{problem}}%}

%%%%%%%%%%%%%%%%%%%%%%




\latexProblemContent{
\begin{problem}

Use the Fundamental Theorem of Calculus to find the derivative of the function.
\[g(t)=\int_{3}^{t} {{\left(x + 1\right)}^{16}}\;dx\]

\expandafter\input{\file@loc Integrals/2311-Compute-Integral-0008.HELP.tex}

\[\dfrac{d}{dt}(g(t))=\answer{{\left(t + 1\right)}^{16}}\]
\end{problem}}%}

%%%%%%%%%%%%%%%%%%%%%%




%%%%%%%%%%%%%%%%%%%%%%




\latexProblemContent{
\begin{problem}

Use the Fundamental Theorem of Calculus to find the derivative of the function.
\[g(t)=\int_{3}^{t} {\cos\left(\cos\left(x + 2\right)\right)}\;dx\]

\expandafter\input{\file@loc Integrals/2311-Compute-Integral-0008.HELP.tex}

\[\dfrac{d}{dt}(g(t))=\answer{\cos\left(\cos\left(t + 2\right)\right)}\]
\end{problem}}%}

%%%%%%%%%%%%%%%%%%%%%%




\latexProblemContent{
\begin{problem}

Use the Fundamental Theorem of Calculus to find the derivative of the function.
\[g(t)=\int_{2}^{t} {{\left(x + 5\right)}^{\frac{1}{4}}}\;dx\]

\expandafter\input{\file@loc Integrals/2311-Compute-Integral-0008.HELP.tex}

\[\dfrac{d}{dt}(g(t))=\answer{{\left(t + 5\right)}^{\frac{1}{4}}}\]
\end{problem}}%}

%%%%%%%%%%%%%%%%%%%%%%




\latexProblemContent{
\begin{problem}

Use the Fundamental Theorem of Calculus to find the derivative of the function.
\[g(t)=\int_{1}^{t} {x - 7}\;dx\]

\expandafter\input{\file@loc Integrals/2311-Compute-Integral-0008.HELP.tex}

\[\dfrac{d}{dt}(g(t))=\answer{t - 7}\]
\end{problem}}%}

%%%%%%%%%%%%%%%%%%%%%%




\latexProblemContent{
\begin{problem}

Use the Fundamental Theorem of Calculus to find the derivative of the function.
\[g(t)=\int_{1}^{t} {\sin\left(\sin\left(x + 6\right)\right)}\;dx\]

\expandafter\input{\file@loc Integrals/2311-Compute-Integral-0008.HELP.tex}

\[\dfrac{d}{dt}(g(t))=\answer{\sin\left(\sin\left(t + 6\right)\right)}\]
\end{problem}}%}

%%%%%%%%%%%%%%%%%%%%%%




\latexProblemContent{
\begin{problem}

Use the Fundamental Theorem of Calculus to find the derivative of the function.
\[g(t)=\int_{1}^{t} {x - 14}\;dx\]

\expandafter\input{\file@loc Integrals/2311-Compute-Integral-0008.HELP.tex}

\[\dfrac{d}{dt}(g(t))=\answer{t - 14}\]
\end{problem}}%}

%%%%%%%%%%%%%%%%%%%%%%




\latexProblemContent{
\begin{problem}

Use the Fundamental Theorem of Calculus to find the derivative of the function.
\[g(t)=\int_{5}^{t} {{\left(x + 9\right)}^{\frac{1}{4}}}\;dx\]

\expandafter\input{\file@loc Integrals/2311-Compute-Integral-0008.HELP.tex}

\[\dfrac{d}{dt}(g(t))=\answer{{\left(t + 9\right)}^{\frac{1}{4}}}\]
\end{problem}}%}

%%%%%%%%%%%%%%%%%%%%%%




\latexProblemContent{
\begin{problem}

Use the Fundamental Theorem of Calculus to find the derivative of the function.
\[g(t)=\int_{4}^{t} {x + 4}\;dx\]

\expandafter\input{\file@loc Integrals/2311-Compute-Integral-0008.HELP.tex}

\[\dfrac{d}{dt}(g(t))=\answer{t + 4}\]
\end{problem}}%}

%%%%%%%%%%%%%%%%%%%%%%




\latexProblemContent{
\begin{problem}

Use the Fundamental Theorem of Calculus to find the derivative of the function.
\[g(t)=\int_{1}^{t} {{\left(x - 3\right)}^{4}}\;dx\]

\expandafter\input{\file@loc Integrals/2311-Compute-Integral-0008.HELP.tex}

\[\dfrac{d}{dt}(g(t))=\answer{{\left(t - 3\right)}^{4}}\]
\end{problem}}%}

%%%%%%%%%%%%%%%%%%%%%%




\latexProblemContent{
\begin{problem}

Use the Fundamental Theorem of Calculus to find the derivative of the function.
\[g(t)=\int_{5}^{t} {\log\left(\log\left(x - 4\right)\right)}\;dx\]

\expandafter\input{\file@loc Integrals/2311-Compute-Integral-0008.HELP.tex}

\[\dfrac{d}{dt}(g(t))=\answer{\log\left(\log\left(t - 4\right)\right)}\]
\end{problem}}%}

%%%%%%%%%%%%%%%%%%%%%%




\latexProblemContent{
\begin{problem}

Use the Fundamental Theorem of Calculus to find the derivative of the function.
\[g(t)=\int_{3}^{t} {{\left(x + 9\right)}^{16}}\;dx\]

\expandafter\input{\file@loc Integrals/2311-Compute-Integral-0008.HELP.tex}

\[\dfrac{d}{dt}(g(t))=\answer{{\left(t + 9\right)}^{16}}\]
\end{problem}}%}

%%%%%%%%%%%%%%%%%%%%%%




\latexProblemContent{
\begin{problem}

Use the Fundamental Theorem of Calculus to find the derivative of the function.
\[g(t)=\int_{4}^{t} {{\left(x - 1\right)}^{9}}\;dx\]

\expandafter\input{\file@loc Integrals/2311-Compute-Integral-0008.HELP.tex}

\[\dfrac{d}{dt}(g(t))=\answer{{\left(t - 1\right)}^{9}}\]
\end{problem}}%}

%%%%%%%%%%%%%%%%%%%%%%




%%%%%%%%%%%%%%%%%%%%%%




\latexProblemContent{
\begin{problem}

Use the Fundamental Theorem of Calculus to find the derivative of the function.
\[g(t)=\int_{3}^{t} {\sin\left(\sin\left(x - 4\right)\right)}\;dx\]

\expandafter\input{\file@loc Integrals/2311-Compute-Integral-0008.HELP.tex}

\[\dfrac{d}{dt}(g(t))=\answer{\sin\left(\sin\left(t - 4\right)\right)}\]
\end{problem}}%}

%%%%%%%%%%%%%%%%%%%%%%




\latexProblemContent{
\begin{problem}

Use the Fundamental Theorem of Calculus to find the derivative of the function.
\[g(t)=\int_{4}^{t} {\cos\left(\cos\left(x - 2\right)\right)}\;dx\]

\expandafter\input{\file@loc Integrals/2311-Compute-Integral-0008.HELP.tex}

\[\dfrac{d}{dt}(g(t))=\answer{\cos\left(\cos\left(t - 2\right)\right)}\]
\end{problem}}%}

%%%%%%%%%%%%%%%%%%%%%%




\latexProblemContent{
\begin{problem}

Use the Fundamental Theorem of Calculus to find the derivative of the function.
\[g(t)=\int_{2}^{t} {{\left(x + 3\right)}^{9}}\;dx\]

\expandafter\input{\file@loc Integrals/2311-Compute-Integral-0008.HELP.tex}

\[\dfrac{d}{dt}(g(t))=\answer{{\left(t + 3\right)}^{9}}\]
\end{problem}}%}

%%%%%%%%%%%%%%%%%%%%%%




\latexProblemContent{
\begin{problem}

Use the Fundamental Theorem of Calculus to find the derivative of the function.
\[g(t)=\int_{2}^{t} {x - 8}\;dx\]

\expandafter\input{\file@loc Integrals/2311-Compute-Integral-0008.HELP.tex}

\[\dfrac{d}{dt}(g(t))=\answer{t - 8}\]
\end{problem}}%}

%%%%%%%%%%%%%%%%%%%%%%




\latexProblemContent{
\begin{problem}

Use the Fundamental Theorem of Calculus to find the derivative of the function.
\[g(t)=\int_{2}^{t} {\cos\left(\cos\left(x + 3\right)\right)}\;dx\]

\expandafter\input{\file@loc Integrals/2311-Compute-Integral-0008.HELP.tex}

\[\dfrac{d}{dt}(g(t))=\answer{\cos\left(\cos\left(t + 3\right)\right)}\]
\end{problem}}%}

%%%%%%%%%%%%%%%%%%%%%%




\latexProblemContent{
\begin{problem}

Use the Fundamental Theorem of Calculus to find the derivative of the function.
\[g(t)=\int_{2}^{t} {{\left(x - 6\right)}^{16}}\;dx\]

\expandafter\input{\file@loc Integrals/2311-Compute-Integral-0008.HELP.tex}

\[\dfrac{d}{dt}(g(t))=\answer{{\left(t - 6\right)}^{16}}\]
\end{problem}}%}

%%%%%%%%%%%%%%%%%%%%%%




\latexProblemContent{
\begin{problem}

Use the Fundamental Theorem of Calculus to find the derivative of the function.
\[g(t)=\int_{3}^{t} {x + 2}\;dx\]

\expandafter\input{\file@loc Integrals/2311-Compute-Integral-0008.HELP.tex}

\[\dfrac{d}{dt}(g(t))=\answer{t + 2}\]
\end{problem}}%}

%%%%%%%%%%%%%%%%%%%%%%




\latexProblemContent{
\begin{problem}

Use the Fundamental Theorem of Calculus to find the derivative of the function.
\[g(t)=\int_{4}^{t} {\cos\left(\cos\left(x - 10\right)\right)}\;dx\]

\expandafter\input{\file@loc Integrals/2311-Compute-Integral-0008.HELP.tex}

\[\dfrac{d}{dt}(g(t))=\answer{\cos\left(\cos\left(t - 10\right)\right)}\]
\end{problem}}%}

%%%%%%%%%%%%%%%%%%%%%%




\latexProblemContent{
\begin{problem}

Use the Fundamental Theorem of Calculus to find the derivative of the function.
\[g(t)=\int_{2}^{t} {x - 14}\;dx\]

\expandafter\input{\file@loc Integrals/2311-Compute-Integral-0008.HELP.tex}

\[\dfrac{d}{dt}(g(t))=\answer{t - 14}\]
\end{problem}}%}

%%%%%%%%%%%%%%%%%%%%%%




\latexProblemContent{
\begin{problem}

Use the Fundamental Theorem of Calculus to find the derivative of the function.
\[g(t)=\int_{4}^{t} {\cos\left(\cos\left(x + 3\right)\right)}\;dx\]

\expandafter\input{\file@loc Integrals/2311-Compute-Integral-0008.HELP.tex}

\[\dfrac{d}{dt}(g(t))=\answer{\cos\left(\cos\left(t + 3\right)\right)}\]
\end{problem}}%}

%%%%%%%%%%%%%%%%%%%%%%




%%%%%%%%%%%%%%%%%%%%%%




\latexProblemContent{
\begin{problem}

Use the Fundamental Theorem of Calculus to find the derivative of the function.
\[g(t)=\int_{2}^{t} {{\left(x - 4\right)}^{\frac{1}{4}}}\;dx\]

\expandafter\input{\file@loc Integrals/2311-Compute-Integral-0008.HELP.tex}

\[\dfrac{d}{dt}(g(t))=\answer{{\left(t - 4\right)}^{\frac{1}{4}}}\]
\end{problem}}%}

%%%%%%%%%%%%%%%%%%%%%%




\latexProblemContent{
\begin{problem}

Use the Fundamental Theorem of Calculus to find the derivative of the function.
\[g(t)=\int_{2}^{t} {{\left(x + 5\right)}^{4}}\;dx\]

\expandafter\input{\file@loc Integrals/2311-Compute-Integral-0008.HELP.tex}

\[\dfrac{d}{dt}(g(t))=\answer{{\left(t + 5\right)}^{4}}\]
\end{problem}}%}

%%%%%%%%%%%%%%%%%%%%%%




\latexProblemContent{
\begin{problem}

Use the Fundamental Theorem of Calculus to find the derivative of the function.
\[g(t)=\int_{2}^{t} {\sin\left(\sin\left(x + 6\right)\right)}\;dx\]

\expandafter\input{\file@loc Integrals/2311-Compute-Integral-0008.HELP.tex}

\[\dfrac{d}{dt}(g(t))=\answer{\sin\left(\sin\left(t + 6\right)\right)}\]
\end{problem}}%}

%%%%%%%%%%%%%%%%%%%%%%




\latexProblemContent{
\begin{problem}

Use the Fundamental Theorem of Calculus to find the derivative of the function.
\[g(t)=\int_{1}^{t} {x + 3}\;dx\]

\expandafter\input{\file@loc Integrals/2311-Compute-Integral-0008.HELP.tex}

\[\dfrac{d}{dt}(g(t))=\answer{t + 3}\]
\end{problem}}%}

%%%%%%%%%%%%%%%%%%%%%%




\latexProblemContent{
\begin{problem}

Use the Fundamental Theorem of Calculus to find the derivative of the function.
\[g(t)=\int_{2}^{t} {{\left(x + 8\right)}^{4}}\;dx\]

\expandafter\input{\file@loc Integrals/2311-Compute-Integral-0008.HELP.tex}

\[\dfrac{d}{dt}(g(t))=\answer{{\left(t + 8\right)}^{4}}\]
\end{problem}}%}

%%%%%%%%%%%%%%%%%%%%%%




\latexProblemContent{
\begin{problem}

Use the Fundamental Theorem of Calculus to find the derivative of the function.
\[g(t)=\int_{1}^{t} {{\left(x + 9\right)}^{4}}\;dx\]

\expandafter\input{\file@loc Integrals/2311-Compute-Integral-0008.HELP.tex}

\[\dfrac{d}{dt}(g(t))=\answer{{\left(t + 9\right)}^{4}}\]
\end{problem}}%}

%%%%%%%%%%%%%%%%%%%%%%




\latexProblemContent{
\begin{problem}

Use the Fundamental Theorem of Calculus to find the derivative of the function.
\[g(t)=\int_{5}^{t} {x + 1}\;dx\]

\expandafter\input{\file@loc Integrals/2311-Compute-Integral-0008.HELP.tex}

\[\dfrac{d}{dt}(g(t))=\answer{t + 1}\]
\end{problem}}%}

%%%%%%%%%%%%%%%%%%%%%%




\latexProblemContent{
\begin{problem}

Use the Fundamental Theorem of Calculus to find the derivative of the function.
\[g(t)=\int_{1}^{t} {x - 18}\;dx\]

\expandafter\input{\file@loc Integrals/2311-Compute-Integral-0008.HELP.tex}

\[\dfrac{d}{dt}(g(t))=\answer{t - 18}\]
\end{problem}}%}

%%%%%%%%%%%%%%%%%%%%%%




\latexProblemContent{
\begin{problem}

Use the Fundamental Theorem of Calculus to find the derivative of the function.
\[g(t)=\int_{1}^{t} {\sin\left(\sin\left(x + 2\right)\right)}\;dx\]

\expandafter\input{\file@loc Integrals/2311-Compute-Integral-0008.HELP.tex}

\[\dfrac{d}{dt}(g(t))=\answer{\sin\left(\sin\left(t + 2\right)\right)}\]
\end{problem}}%}

%%%%%%%%%%%%%%%%%%%%%%




\latexProblemContent{
\begin{problem}

Use the Fundamental Theorem of Calculus to find the derivative of the function.
\[g(t)=\int_{1}^{t} {{\left(x + 10\right)}^{9}}\;dx\]

\expandafter\input{\file@loc Integrals/2311-Compute-Integral-0008.HELP.tex}

\[\dfrac{d}{dt}(g(t))=\answer{{\left(t + 10\right)}^{9}}\]
\end{problem}}%}

%%%%%%%%%%%%%%%%%%%%%%




\latexProblemContent{
\begin{problem}

Use the Fundamental Theorem of Calculus to find the derivative of the function.
\[g(t)=\int_{1}^{t} {{\left(x + 3\right)}^{16}}\;dx\]

\expandafter\input{\file@loc Integrals/2311-Compute-Integral-0008.HELP.tex}

\[\dfrac{d}{dt}(g(t))=\answer{{\left(t + 3\right)}^{16}}\]
\end{problem}}%}

%%%%%%%%%%%%%%%%%%%%%%




\latexProblemContent{
\begin{problem}

Use the Fundamental Theorem of Calculus to find the derivative of the function.
\[g(t)=\int_{1}^{t} {{\left(x + 2\right)}^{4}}\;dx\]

\expandafter\input{\file@loc Integrals/2311-Compute-Integral-0008.HELP.tex}

\[\dfrac{d}{dt}(g(t))=\answer{{\left(t + 2\right)}^{4}}\]
\end{problem}}%}

%%%%%%%%%%%%%%%%%%%%%%




\latexProblemContent{
\begin{problem}

Use the Fundamental Theorem of Calculus to find the derivative of the function.
\[g(t)=\int_{1}^{t} {\sin\left(\sin\left(x - 2\right)\right)}\;dx\]

\expandafter\input{\file@loc Integrals/2311-Compute-Integral-0008.HELP.tex}

\[\dfrac{d}{dt}(g(t))=\answer{\sin\left(\sin\left(t - 2\right)\right)}\]
\end{problem}}%}

%%%%%%%%%%%%%%%%%%%%%%




\latexProblemContent{
\begin{problem}

Use the Fundamental Theorem of Calculus to find the derivative of the function.
\[g(t)=\int_{5}^{t} {e^{\left(e^{\left(x + 4\right)}\right)}}\;dx\]

\expandafter\input{\file@loc Integrals/2311-Compute-Integral-0008.HELP.tex}

\[\dfrac{d}{dt}(g(t))=\answer{e^{\left(e^{\left(t + 4\right)}\right)}}\]
\end{problem}}%}

%%%%%%%%%%%%%%%%%%%%%%




\latexProblemContent{
\begin{problem}

Use the Fundamental Theorem of Calculus to find the derivative of the function.
\[g(t)=\int_{4}^{t} {{\left(x - 4\right)}^{4}}\;dx\]

\expandafter\input{\file@loc Integrals/2311-Compute-Integral-0008.HELP.tex}

\[\dfrac{d}{dt}(g(t))=\answer{{\left(t - 4\right)}^{4}}\]
\end{problem}}%}

%%%%%%%%%%%%%%%%%%%%%%




\latexProblemContent{
\begin{problem}

Use the Fundamental Theorem of Calculus to find the derivative of the function.
\[g(t)=\int_{5}^{t} {\sin\left(\sin\left(x + 10\right)\right)}\;dx\]

\expandafter\input{\file@loc Integrals/2311-Compute-Integral-0008.HELP.tex}

\[\dfrac{d}{dt}(g(t))=\answer{\sin\left(\sin\left(t + 10\right)\right)}\]
\end{problem}}%}

%%%%%%%%%%%%%%%%%%%%%%




\latexProblemContent{
\begin{problem}

Use the Fundamental Theorem of Calculus to find the derivative of the function.
\[g(t)=\int_{5}^{t} {{\left(x - 8\right)}^{9}}\;dx\]

\expandafter\input{\file@loc Integrals/2311-Compute-Integral-0008.HELP.tex}

\[\dfrac{d}{dt}(g(t))=\answer{{\left(t - 8\right)}^{9}}\]
\end{problem}}%}

%%%%%%%%%%%%%%%%%%%%%%




\latexProblemContent{
\begin{problem}

Use the Fundamental Theorem of Calculus to find the derivative of the function.
\[g(t)=\int_{1}^{t} {x - 6}\;dx\]

\expandafter\input{\file@loc Integrals/2311-Compute-Integral-0008.HELP.tex}

\[\dfrac{d}{dt}(g(t))=\answer{t - 6}\]
\end{problem}}%}

%%%%%%%%%%%%%%%%%%%%%%




\latexProblemContent{
\begin{problem}

Use the Fundamental Theorem of Calculus to find the derivative of the function.
\[g(t)=\int_{4}^{t} {{\left(x + 1\right)}^{9}}\;dx\]

\expandafter\input{\file@loc Integrals/2311-Compute-Integral-0008.HELP.tex}

\[\dfrac{d}{dt}(g(t))=\answer{{\left(t + 1\right)}^{9}}\]
\end{problem}}%}

%%%%%%%%%%%%%%%%%%%%%%




\latexProblemContent{
\begin{problem}

Use the Fundamental Theorem of Calculus to find the derivative of the function.
\[g(t)=\int_{4}^{t} {\cos\left(\cos\left(x + 7\right)\right)}\;dx\]

\expandafter\input{\file@loc Integrals/2311-Compute-Integral-0008.HELP.tex}

\[\dfrac{d}{dt}(g(t))=\answer{\cos\left(\cos\left(t + 7\right)\right)}\]
\end{problem}}%}

%%%%%%%%%%%%%%%%%%%%%%




\latexProblemContent{
\begin{problem}

Use the Fundamental Theorem of Calculus to find the derivative of the function.
\[g(t)=\int_{5}^{t} {e^{\left(e^{\left(x + 10\right)}\right)}}\;dx\]

\expandafter\input{\file@loc Integrals/2311-Compute-Integral-0008.HELP.tex}

\[\dfrac{d}{dt}(g(t))=\answer{e^{\left(e^{\left(t + 10\right)}\right)}}\]
\end{problem}}%}

%%%%%%%%%%%%%%%%%%%%%%




\latexProblemContent{
\begin{problem}

Use the Fundamental Theorem of Calculus to find the derivative of the function.
\[g(t)=\int_{3}^{t} {{\left(x - 1\right)}^{4}}\;dx\]

\expandafter\input{\file@loc Integrals/2311-Compute-Integral-0008.HELP.tex}

\[\dfrac{d}{dt}(g(t))=\answer{{\left(t - 1\right)}^{4}}\]
\end{problem}}%}

%%%%%%%%%%%%%%%%%%%%%%




\latexProblemContent{
\begin{problem}

Use the Fundamental Theorem of Calculus to find the derivative of the function.
\[g(t)=\int_{2}^{t} {x + 9}\;dx\]

\expandafter\input{\file@loc Integrals/2311-Compute-Integral-0008.HELP.tex}

\[\dfrac{d}{dt}(g(t))=\answer{t + 9}\]
\end{problem}}%}

%%%%%%%%%%%%%%%%%%%%%%




%%%%%%%%%%%%%%%%%%%%%%




\latexProblemContent{
\begin{problem}

Use the Fundamental Theorem of Calculus to find the derivative of the function.
\[g(t)=\int_{1}^{t} {{\left(x - 5\right)}^{4}}\;dx\]

\expandafter\input{\file@loc Integrals/2311-Compute-Integral-0008.HELP.tex}

\[\dfrac{d}{dt}(g(t))=\answer{{\left(t - 5\right)}^{4}}\]
\end{problem}}%}

%%%%%%%%%%%%%%%%%%%%%%




\latexProblemContent{
\begin{problem}

Use the Fundamental Theorem of Calculus to find the derivative of the function.
\[g(t)=\int_{5}^{t} {{\left(x + 6\right)}^{16}}\;dx\]

\expandafter\input{\file@loc Integrals/2311-Compute-Integral-0008.HELP.tex}

\[\dfrac{d}{dt}(g(t))=\answer{{\left(t + 6\right)}^{16}}\]
\end{problem}}%}

%%%%%%%%%%%%%%%%%%%%%%




\latexProblemContent{
\begin{problem}

Use the Fundamental Theorem of Calculus to find the derivative of the function.
\[g(t)=\int_{5}^{t} {{\left(x - 10\right)}^{4}}\;dx\]

\expandafter\input{\file@loc Integrals/2311-Compute-Integral-0008.HELP.tex}

\[\dfrac{d}{dt}(g(t))=\answer{{\left(t - 10\right)}^{4}}\]
\end{problem}}%}

%%%%%%%%%%%%%%%%%%%%%%




\latexProblemContent{
\begin{problem}

Use the Fundamental Theorem of Calculus to find the derivative of the function.
\[g(t)=\int_{2}^{t} {\log\left(\log\left(x - 1\right)\right)}\;dx\]

\expandafter\input{\file@loc Integrals/2311-Compute-Integral-0008.HELP.tex}

\[\dfrac{d}{dt}(g(t))=\answer{\log\left(\log\left(t - 1\right)\right)}\]
\end{problem}}%}

%%%%%%%%%%%%%%%%%%%%%%




%%%%%%%%%%%%%%%%%%%%%%




\latexProblemContent{
\begin{problem}

Use the Fundamental Theorem of Calculus to find the derivative of the function.
\[g(t)=\int_{1}^{t} {{\left(x - 4\right)}^{4}}\;dx\]

\expandafter\input{\file@loc Integrals/2311-Compute-Integral-0008.HELP.tex}

\[\dfrac{d}{dt}(g(t))=\answer{{\left(t - 4\right)}^{4}}\]
\end{problem}}%}

%%%%%%%%%%%%%%%%%%%%%%




\latexProblemContent{
\begin{problem}

Use the Fundamental Theorem of Calculus to find the derivative of the function.
\[g(t)=\int_{4}^{t} {\cos\left(\cos\left(x - 7\right)\right)}\;dx\]

\expandafter\input{\file@loc Integrals/2311-Compute-Integral-0008.HELP.tex}

\[\dfrac{d}{dt}(g(t))=\answer{\cos\left(\cos\left(t - 7\right)\right)}\]
\end{problem}}%}

%%%%%%%%%%%%%%%%%%%%%%




\latexProblemContent{
\begin{problem}

Use the Fundamental Theorem of Calculus to find the derivative of the function.
\[g(t)=\int_{2}^{t} {{\left(x - 7\right)}^{\frac{1}{4}}}\;dx\]

\expandafter\input{\file@loc Integrals/2311-Compute-Integral-0008.HELP.tex}

\[\dfrac{d}{dt}(g(t))=\answer{{\left(t - 7\right)}^{\frac{1}{4}}}\]
\end{problem}}%}

%%%%%%%%%%%%%%%%%%%%%%




\latexProblemContent{
\begin{problem}

Use the Fundamental Theorem of Calculus to find the derivative of the function.
\[g(t)=\int_{1}^{t} {x - 16}\;dx\]

\expandafter\input{\file@loc Integrals/2311-Compute-Integral-0008.HELP.tex}

\[\dfrac{d}{dt}(g(t))=\answer{t - 16}\]
\end{problem}}%}

%%%%%%%%%%%%%%%%%%%%%%




%%%%%%%%%%%%%%%%%%%%%%




\latexProblemContent{
\begin{problem}

Use the Fundamental Theorem of Calculus to find the derivative of the function.
\[g(t)=\int_{5}^{t} {{\left(x - 7\right)}^{\frac{1}{4}}}\;dx\]

\expandafter\input{\file@loc Integrals/2311-Compute-Integral-0008.HELP.tex}

\[\dfrac{d}{dt}(g(t))=\answer{{\left(t - 7\right)}^{\frac{1}{4}}}\]
\end{problem}}%}

%%%%%%%%%%%%%%%%%%%%%%




\latexProblemContent{
\begin{problem}

Use the Fundamental Theorem of Calculus to find the derivative of the function.
\[g(t)=\int_{1}^{t} {e^{\left(e^{\left(x + 6\right)}\right)}}\;dx\]

\expandafter\input{\file@loc Integrals/2311-Compute-Integral-0008.HELP.tex}

\[\dfrac{d}{dt}(g(t))=\answer{e^{\left(e^{\left(t + 6\right)}\right)}}\]
\end{problem}}%}

%%%%%%%%%%%%%%%%%%%%%%




\latexProblemContent{
\begin{problem}

Use the Fundamental Theorem of Calculus to find the derivative of the function.
\[g(t)=\int_{4}^{t} {\log\left(\log\left(x - 3\right)\right)}\;dx\]

\expandafter\input{\file@loc Integrals/2311-Compute-Integral-0008.HELP.tex}

\[\dfrac{d}{dt}(g(t))=\answer{\log\left(\log\left(t - 3\right)\right)}\]
\end{problem}}%}

%%%%%%%%%%%%%%%%%%%%%%




\latexProblemContent{
\begin{problem}

Use the Fundamental Theorem of Calculus to find the derivative of the function.
\[g(t)=\int_{3}^{t} {x - 4}\;dx\]

\expandafter\input{\file@loc Integrals/2311-Compute-Integral-0008.HELP.tex}

\[\dfrac{d}{dt}(g(t))=\answer{t - 4}\]
\end{problem}}%}

%%%%%%%%%%%%%%%%%%%%%%




\latexProblemContent{
\begin{problem}

Use the Fundamental Theorem of Calculus to find the derivative of the function.
\[g(t)=\int_{5}^{t} {x - 1}\;dx\]

\expandafter\input{\file@loc Integrals/2311-Compute-Integral-0008.HELP.tex}

\[\dfrac{d}{dt}(g(t))=\answer{t - 1}\]
\end{problem}}%}

%%%%%%%%%%%%%%%%%%%%%%




\latexProblemContent{
\begin{problem}

Use the Fundamental Theorem of Calculus to find the derivative of the function.
\[g(t)=\int_{3}^{t} {{\left(x + 5\right)}^{9}}\;dx\]

\expandafter\input{\file@loc Integrals/2311-Compute-Integral-0008.HELP.tex}

\[\dfrac{d}{dt}(g(t))=\answer{{\left(t + 5\right)}^{9}}\]
\end{problem}}%}

%%%%%%%%%%%%%%%%%%%%%%




\latexProblemContent{
\begin{problem}

Use the Fundamental Theorem of Calculus to find the derivative of the function.
\[g(t)=\int_{4}^{t} {{\left(x + 1\right)}^{\frac{1}{4}}}\;dx\]

\expandafter\input{\file@loc Integrals/2311-Compute-Integral-0008.HELP.tex}

\[\dfrac{d}{dt}(g(t))=\answer{{\left(t + 1\right)}^{\frac{1}{4}}}\]
\end{problem}}%}

%%%%%%%%%%%%%%%%%%%%%%




\latexProblemContent{
\begin{problem}

Use the Fundamental Theorem of Calculus to find the derivative of the function.
\[g(t)=\int_{4}^{t} {{\left(x - 2\right)}^{9}}\;dx\]

\expandafter\input{\file@loc Integrals/2311-Compute-Integral-0008.HELP.tex}

\[\dfrac{d}{dt}(g(t))=\answer{{\left(t - 2\right)}^{9}}\]
\end{problem}}%}

%%%%%%%%%%%%%%%%%%%%%%




\latexProblemContent{
\begin{problem}

Use the Fundamental Theorem of Calculus to find the derivative of the function.
\[g(t)=\int_{4}^{t} {x - 8}\;dx\]

\expandafter\input{\file@loc Integrals/2311-Compute-Integral-0008.HELP.tex}

\[\dfrac{d}{dt}(g(t))=\answer{t - 8}\]
\end{problem}}%}

%%%%%%%%%%%%%%%%%%%%%%




\latexProblemContent{
\begin{problem}

Use the Fundamental Theorem of Calculus to find the derivative of the function.
\[g(t)=\int_{1}^{t} {{\left(x - 9\right)}^{16}}\;dx\]

\expandafter\input{\file@loc Integrals/2311-Compute-Integral-0008.HELP.tex}

\[\dfrac{d}{dt}(g(t))=\answer{{\left(t - 9\right)}^{16}}\]
\end{problem}}%}

%%%%%%%%%%%%%%%%%%%%%%




\latexProblemContent{
\begin{problem}

Use the Fundamental Theorem of Calculus to find the derivative of the function.
\[g(t)=\int_{4}^{t} {\sin\left(\sin\left(x + 4\right)\right)}\;dx\]

\expandafter\input{\file@loc Integrals/2311-Compute-Integral-0008.HELP.tex}

\[\dfrac{d}{dt}(g(t))=\answer{\sin\left(\sin\left(t + 4\right)\right)}\]
\end{problem}}%}

%%%%%%%%%%%%%%%%%%%%%%




\latexProblemContent{
\begin{problem}

Use the Fundamental Theorem of Calculus to find the derivative of the function.
\[g(t)=\int_{1}^{t} {{\left(x + 6\right)}^{4}}\;dx\]

\expandafter\input{\file@loc Integrals/2311-Compute-Integral-0008.HELP.tex}

\[\dfrac{d}{dt}(g(t))=\answer{{\left(t + 6\right)}^{4}}\]
\end{problem}}%}

%%%%%%%%%%%%%%%%%%%%%%




\latexProblemContent{
\begin{problem}

Use the Fundamental Theorem of Calculus to find the derivative of the function.
\[g(t)=\int_{1}^{t} {\sin\left(\sin\left(x - 4\right)\right)}\;dx\]

\expandafter\input{\file@loc Integrals/2311-Compute-Integral-0008.HELP.tex}

\[\dfrac{d}{dt}(g(t))=\answer{\sin\left(\sin\left(t - 4\right)\right)}\]
\end{problem}}%}

%%%%%%%%%%%%%%%%%%%%%%




\latexProblemContent{
\begin{problem}

Use the Fundamental Theorem of Calculus to find the derivative of the function.
\[g(t)=\int_{3}^{t} {x + 4}\;dx\]

\expandafter\input{\file@loc Integrals/2311-Compute-Integral-0008.HELP.tex}

\[\dfrac{d}{dt}(g(t))=\answer{t + 4}\]
\end{problem}}%}

%%%%%%%%%%%%%%%%%%%%%%




\latexProblemContent{
\begin{problem}

Use the Fundamental Theorem of Calculus to find the derivative of the function.
\[g(t)=\int_{1}^{t} {x - 9}\;dx\]

\expandafter\input{\file@loc Integrals/2311-Compute-Integral-0008.HELP.tex}

\[\dfrac{d}{dt}(g(t))=\answer{t - 9}\]
\end{problem}}%}

%%%%%%%%%%%%%%%%%%%%%%




\latexProblemContent{
\begin{problem}

Use the Fundamental Theorem of Calculus to find the derivative of the function.
\[g(t)=\int_{5}^{t} {{\left(x + 5\right)}^{9}}\;dx\]

\expandafter\input{\file@loc Integrals/2311-Compute-Integral-0008.HELP.tex}

\[\dfrac{d}{dt}(g(t))=\answer{{\left(t + 5\right)}^{9}}\]
\end{problem}}%}

%%%%%%%%%%%%%%%%%%%%%%




\latexProblemContent{
\begin{problem}

Use the Fundamental Theorem of Calculus to find the derivative of the function.
\[g(t)=\int_{4}^{t} {\log\left(\log\left(x - 4\right)\right)}\;dx\]

\expandafter\input{\file@loc Integrals/2311-Compute-Integral-0008.HELP.tex}

\[\dfrac{d}{dt}(g(t))=\answer{\log\left(\log\left(t - 4\right)\right)}\]
\end{problem}}%}

%%%%%%%%%%%%%%%%%%%%%%




\latexProblemContent{
\begin{problem}

Use the Fundamental Theorem of Calculus to find the derivative of the function.
\[g(t)=\int_{1}^{t} {x + 1}\;dx\]

\expandafter\input{\file@loc Integrals/2311-Compute-Integral-0008.HELP.tex}

\[\dfrac{d}{dt}(g(t))=\answer{t + 1}\]
\end{problem}}%}

%%%%%%%%%%%%%%%%%%%%%%




\latexProblemContent{
\begin{problem}

Use the Fundamental Theorem of Calculus to find the derivative of the function.
\[g(t)=\int_{2}^{t} {{\left(x - 2\right)}^{4}}\;dx\]

\expandafter\input{\file@loc Integrals/2311-Compute-Integral-0008.HELP.tex}

\[\dfrac{d}{dt}(g(t))=\answer{{\left(t - 2\right)}^{4}}\]
\end{problem}}%}

%%%%%%%%%%%%%%%%%%%%%%




\latexProblemContent{
\begin{problem}

Use the Fundamental Theorem of Calculus to find the derivative of the function.
\[g(t)=\int_{4}^{t} {\cos\left(\cos\left(x - 6\right)\right)}\;dx\]

\expandafter\input{\file@loc Integrals/2311-Compute-Integral-0008.HELP.tex}

\[\dfrac{d}{dt}(g(t))=\answer{\cos\left(\cos\left(t - 6\right)\right)}\]
\end{problem}}%}

%%%%%%%%%%%%%%%%%%%%%%




\latexProblemContent{
\begin{problem}

Use the Fundamental Theorem of Calculus to find the derivative of the function.
\[g(t)=\int_{4}^{t} {{\left(x + 2\right)}^{9}}\;dx\]

\expandafter\input{\file@loc Integrals/2311-Compute-Integral-0008.HELP.tex}

\[\dfrac{d}{dt}(g(t))=\answer{{\left(t + 2\right)}^{9}}\]
\end{problem}}%}

%%%%%%%%%%%%%%%%%%%%%%




\latexProblemContent{
\begin{problem}

Use the Fundamental Theorem of Calculus to find the derivative of the function.
\[g(t)=\int_{3}^{t} {\sin\left(\sin\left(x + 9\right)\right)}\;dx\]

\expandafter\input{\file@loc Integrals/2311-Compute-Integral-0008.HELP.tex}

\[\dfrac{d}{dt}(g(t))=\answer{\sin\left(\sin\left(t + 9\right)\right)}\]
\end{problem}}%}

%%%%%%%%%%%%%%%%%%%%%%




\latexProblemContent{
\begin{problem}

Use the Fundamental Theorem of Calculus to find the derivative of the function.
\[g(t)=\int_{5}^{t} {{\left(x - 1\right)}^{4}}\;dx\]

\expandafter\input{\file@loc Integrals/2311-Compute-Integral-0008.HELP.tex}

\[\dfrac{d}{dt}(g(t))=\answer{{\left(t - 1\right)}^{4}}\]
\end{problem}}%}

%%%%%%%%%%%%%%%%%%%%%%




\latexProblemContent{
\begin{problem}

Use the Fundamental Theorem of Calculus to find the derivative of the function.
\[g(t)=\int_{3}^{t} {\log\left(\log\left(x - 2\right)\right)}\;dx\]

\expandafter\input{\file@loc Integrals/2311-Compute-Integral-0008.HELP.tex}

\[\dfrac{d}{dt}(g(t))=\answer{\log\left(\log\left(t - 2\right)\right)}\]
\end{problem}}%}

%%%%%%%%%%%%%%%%%%%%%%




\latexProblemContent{
\begin{problem}

Use the Fundamental Theorem of Calculus to find the derivative of the function.
\[g(t)=\int_{4}^{t} {{\left(x - 1\right)}^{4}}\;dx\]

\expandafter\input{\file@loc Integrals/2311-Compute-Integral-0008.HELP.tex}

\[\dfrac{d}{dt}(g(t))=\answer{{\left(t - 1\right)}^{4}}\]
\end{problem}}%}

%%%%%%%%%%%%%%%%%%%%%%




\latexProblemContent{
\begin{problem}

Use the Fundamental Theorem of Calculus to find the derivative of the function.
\[g(t)=\int_{3}^{t} {\cos\left(\cos\left(x - 2\right)\right)}\;dx\]

\expandafter\input{\file@loc Integrals/2311-Compute-Integral-0008.HELP.tex}

\[\dfrac{d}{dt}(g(t))=\answer{\cos\left(\cos\left(t - 2\right)\right)}\]
\end{problem}}%}

%%%%%%%%%%%%%%%%%%%%%%




\latexProblemContent{
\begin{problem}

Use the Fundamental Theorem of Calculus to find the derivative of the function.
\[g(t)=\int_{4}^{t} {{\left(x + 1\right)}^{16}}\;dx\]

\expandafter\input{\file@loc Integrals/2311-Compute-Integral-0008.HELP.tex}

\[\dfrac{d}{dt}(g(t))=\answer{{\left(t + 1\right)}^{16}}\]
\end{problem}}%}

%%%%%%%%%%%%%%%%%%%%%%




\latexProblemContent{
\begin{problem}

Use the Fundamental Theorem of Calculus to find the derivative of the function.
\[g(t)=\int_{3}^{t} {{\left(x - 9\right)}^{16}}\;dx\]

\expandafter\input{\file@loc Integrals/2311-Compute-Integral-0008.HELP.tex}

\[\dfrac{d}{dt}(g(t))=\answer{{\left(t - 9\right)}^{16}}\]
\end{problem}}%}

%%%%%%%%%%%%%%%%%%%%%%




\latexProblemContent{
\begin{problem}

Use the Fundamental Theorem of Calculus to find the derivative of the function.
\[g(t)=\int_{5}^{t} {{\left(x - 4\right)}^{9}}\;dx\]

\expandafter\input{\file@loc Integrals/2311-Compute-Integral-0008.HELP.tex}

\[\dfrac{d}{dt}(g(t))=\answer{{\left(t - 4\right)}^{9}}\]
\end{problem}}%}

%%%%%%%%%%%%%%%%%%%%%%




%%%%%%%%%%%%%%%%%%%%%%




\latexProblemContent{
\begin{problem}

Use the Fundamental Theorem of Calculus to find the derivative of the function.
\[g(t)=\int_{4}^{t} {\log\left(\log\left(x - 1\right)\right)}\;dx\]

\expandafter\input{\file@loc Integrals/2311-Compute-Integral-0008.HELP.tex}

\[\dfrac{d}{dt}(g(t))=\answer{\log\left(\log\left(t - 1\right)\right)}\]
\end{problem}}%}

%%%%%%%%%%%%%%%%%%%%%%




\latexProblemContent{
\begin{problem}

Use the Fundamental Theorem of Calculus to find the derivative of the function.
\[g(t)=\int_{2}^{t} {\sin\left(\sin\left(x - 1\right)\right)}\;dx\]

\expandafter\input{\file@loc Integrals/2311-Compute-Integral-0008.HELP.tex}

\[\dfrac{d}{dt}(g(t))=\answer{\sin\left(\sin\left(t - 1\right)\right)}\]
\end{problem}}%}

%%%%%%%%%%%%%%%%%%%%%%




\latexProblemContent{
\begin{problem}

Use the Fundamental Theorem of Calculus to find the derivative of the function.
\[g(t)=\int_{1}^{t} {e^{\left(e^{\left(x - 2\right)}\right)}}\;dx\]

\expandafter\input{\file@loc Integrals/2311-Compute-Integral-0008.HELP.tex}

\[\dfrac{d}{dt}(g(t))=\answer{e^{\left(e^{\left(t - 2\right)}\right)}}\]
\end{problem}}%}

%%%%%%%%%%%%%%%%%%%%%%




\latexProblemContent{
\begin{problem}

Use the Fundamental Theorem of Calculus to find the derivative of the function.
\[g(t)=\int_{5}^{t} {e^{\left(e^{\left(x + 6\right)}\right)}}\;dx\]

\expandafter\input{\file@loc Integrals/2311-Compute-Integral-0008.HELP.tex}

\[\dfrac{d}{dt}(g(t))=\answer{e^{\left(e^{\left(t + 6\right)}\right)}}\]
\end{problem}}%}

%%%%%%%%%%%%%%%%%%%%%%




%%%%%%%%%%%%%%%%%%%%%%




%%%%%%%%%%%%%%%%%%%%%%




\latexProblemContent{
\begin{problem}

Use the Fundamental Theorem of Calculus to find the derivative of the function.
\[g(t)=\int_{2}^{t} {{\left(x + 7\right)}^{4}}\;dx\]

\expandafter\input{\file@loc Integrals/2311-Compute-Integral-0008.HELP.tex}

\[\dfrac{d}{dt}(g(t))=\answer{{\left(t + 7\right)}^{4}}\]
\end{problem}}%}

%%%%%%%%%%%%%%%%%%%%%%




\latexProblemContent{
\begin{problem}

Use the Fundamental Theorem of Calculus to find the derivative of the function.
\[g(t)=\int_{4}^{t} {\cos\left(\cos\left(x + 4\right)\right)}\;dx\]

\expandafter\input{\file@loc Integrals/2311-Compute-Integral-0008.HELP.tex}

\[\dfrac{d}{dt}(g(t))=\answer{\cos\left(\cos\left(t + 4\right)\right)}\]
\end{problem}}%}

%%%%%%%%%%%%%%%%%%%%%%




%%%%%%%%%%%%%%%%%%%%%%




\latexProblemContent{
\begin{problem}

Use the Fundamental Theorem of Calculus to find the derivative of the function.
\[g(t)=\int_{3}^{t} {{\left(x + 4\right)}^{4}}\;dx\]

\expandafter\input{\file@loc Integrals/2311-Compute-Integral-0008.HELP.tex}

\[\dfrac{d}{dt}(g(t))=\answer{{\left(t + 4\right)}^{4}}\]
\end{problem}}%}

%%%%%%%%%%%%%%%%%%%%%%




\latexProblemContent{
\begin{problem}

Use the Fundamental Theorem of Calculus to find the derivative of the function.
\[g(t)=\int_{4}^{t} {x + 6}\;dx\]

\expandafter\input{\file@loc Integrals/2311-Compute-Integral-0008.HELP.tex}

\[\dfrac{d}{dt}(g(t))=\answer{t + 6}\]
\end{problem}}%}

%%%%%%%%%%%%%%%%%%%%%%




\latexProblemContent{
\begin{problem}

Use the Fundamental Theorem of Calculus to find the derivative of the function.
\[g(t)=\int_{4}^{t} {e^{\left(e^{\left(x + 2\right)}\right)}}\;dx\]

\expandafter\input{\file@loc Integrals/2311-Compute-Integral-0008.HELP.tex}

\[\dfrac{d}{dt}(g(t))=\answer{e^{\left(e^{\left(t + 2\right)}\right)}}\]
\end{problem}}%}

%%%%%%%%%%%%%%%%%%%%%%




%%%%%%%%%%%%%%%%%%%%%%




%%%%%%%%%%%%%%%%%%%%%%




\latexProblemContent{
\begin{problem}

Use the Fundamental Theorem of Calculus to find the derivative of the function.
\[g(t)=\int_{3}^{t} {{\left(x - 5\right)}^{16}}\;dx\]

\expandafter\input{\file@loc Integrals/2311-Compute-Integral-0008.HELP.tex}

\[\dfrac{d}{dt}(g(t))=\answer{{\left(t - 5\right)}^{16}}\]
\end{problem}}%}

%%%%%%%%%%%%%%%%%%%%%%




\latexProblemContent{
\begin{problem}

Use the Fundamental Theorem of Calculus to find the derivative of the function.
\[g(t)=\int_{5}^{t} {\sin\left(\sin\left(x - 8\right)\right)}\;dx\]

\expandafter\input{\file@loc Integrals/2311-Compute-Integral-0008.HELP.tex}

\[\dfrac{d}{dt}(g(t))=\answer{\sin\left(\sin\left(t - 8\right)\right)}\]
\end{problem}}%}

%%%%%%%%%%%%%%%%%%%%%%




\latexProblemContent{
\begin{problem}

Use the Fundamental Theorem of Calculus to find the derivative of the function.
\[g(t)=\int_{2}^{t} {x - 9}\;dx\]

\expandafter\input{\file@loc Integrals/2311-Compute-Integral-0008.HELP.tex}

\[\dfrac{d}{dt}(g(t))=\answer{t - 9}\]
\end{problem}}%}

%%%%%%%%%%%%%%%%%%%%%%




\latexProblemContent{
\begin{problem}

Use the Fundamental Theorem of Calculus to find the derivative of the function.
\[g(t)=\int_{5}^{t} {{\left(x - 8\right)}^{4}}\;dx\]

\expandafter\input{\file@loc Integrals/2311-Compute-Integral-0008.HELP.tex}

\[\dfrac{d}{dt}(g(t))=\answer{{\left(t - 8\right)}^{4}}\]
\end{problem}}%}

%%%%%%%%%%%%%%%%%%%%%%




\latexProblemContent{
\begin{problem}

Use the Fundamental Theorem of Calculus to find the derivative of the function.
\[g(t)=\int_{5}^{t} {{\left(x + 8\right)}^{9}}\;dx\]

\expandafter\input{\file@loc Integrals/2311-Compute-Integral-0008.HELP.tex}

\[\dfrac{d}{dt}(g(t))=\answer{{\left(t + 8\right)}^{9}}\]
\end{problem}}%}

%%%%%%%%%%%%%%%%%%%%%%




%%%%%%%%%%%%%%%%%%%%%%




%%%%%%%%%%%%%%%%%%%%%%




\latexProblemContent{
\begin{problem}

Use the Fundamental Theorem of Calculus to find the derivative of the function.
\[g(t)=\int_{2}^{t} {{\left(x - 4\right)}^{4}}\;dx\]

\expandafter\input{\file@loc Integrals/2311-Compute-Integral-0008.HELP.tex}

\[\dfrac{d}{dt}(g(t))=\answer{{\left(t - 4\right)}^{4}}\]
\end{problem}}%}

%%%%%%%%%%%%%%%%%%%%%%




\latexProblemContent{
\begin{problem}

Use the Fundamental Theorem of Calculus to find the derivative of the function.
\[g(t)=\int_{1}^{t} {{\left(x - 7\right)}^{16}}\;dx\]

\expandafter\input{\file@loc Integrals/2311-Compute-Integral-0008.HELP.tex}

\[\dfrac{d}{dt}(g(t))=\answer{{\left(t - 7\right)}^{16}}\]
\end{problem}}%}

%%%%%%%%%%%%%%%%%%%%%%




\latexProblemContent{
\begin{problem}

Use the Fundamental Theorem of Calculus to find the derivative of the function.
\[g(t)=\int_{4}^{t} {x + 18}\;dx\]

\expandafter\input{\file@loc Integrals/2311-Compute-Integral-0008.HELP.tex}

\[\dfrac{d}{dt}(g(t))=\answer{t + 18}\]
\end{problem}}%}

%%%%%%%%%%%%%%%%%%%%%%




%%%%%%%%%%%%%%%%%%%%%%




\latexProblemContent{
\begin{problem}

Use the Fundamental Theorem of Calculus to find the derivative of the function.
\[g(t)=\int_{2}^{t} {\sin\left(\sin\left(x + 4\right)\right)}\;dx\]

\expandafter\input{\file@loc Integrals/2311-Compute-Integral-0008.HELP.tex}

\[\dfrac{d}{dt}(g(t))=\answer{\sin\left(\sin\left(t + 4\right)\right)}\]
\end{problem}}%}

%%%%%%%%%%%%%%%%%%%%%%




%%%%%%%%%%%%%%%%%%%%%%




\latexProblemContent{
\begin{problem}

Use the Fundamental Theorem of Calculus to find the derivative of the function.
\[g(t)=\int_{2}^{t} {{\left(x - 3\right)}^{\frac{1}{4}}}\;dx\]

\expandafter\input{\file@loc Integrals/2311-Compute-Integral-0008.HELP.tex}

\[\dfrac{d}{dt}(g(t))=\answer{{\left(t - 3\right)}^{\frac{1}{4}}}\]
\end{problem}}%}

%%%%%%%%%%%%%%%%%%%%%%




\latexProblemContent{
\begin{problem}

Use the Fundamental Theorem of Calculus to find the derivative of the function.
\[g(t)=\int_{3}^{t} {\cos\left(\cos\left(x + 7\right)\right)}\;dx\]

\expandafter\input{\file@loc Integrals/2311-Compute-Integral-0008.HELP.tex}

\[\dfrac{d}{dt}(g(t))=\answer{\cos\left(\cos\left(t + 7\right)\right)}\]
\end{problem}}%}

%%%%%%%%%%%%%%%%%%%%%%




\latexProblemContent{
\begin{problem}

Use the Fundamental Theorem of Calculus to find the derivative of the function.
\[g(t)=\int_{3}^{t} {\cos\left(\cos\left(x + 6\right)\right)}\;dx\]

\expandafter\input{\file@loc Integrals/2311-Compute-Integral-0008.HELP.tex}

\[\dfrac{d}{dt}(g(t))=\answer{\cos\left(\cos\left(t + 6\right)\right)}\]
\end{problem}}%}

%%%%%%%%%%%%%%%%%%%%%%




%%%%%%%%%%%%%%%%%%%%%%




\latexProblemContent{
\begin{problem}

Use the Fundamental Theorem of Calculus to find the derivative of the function.
\[g(t)=\int_{2}^{t} {e^{\left(e^{\left(x + 7\right)}\right)}}\;dx\]

\expandafter\input{\file@loc Integrals/2311-Compute-Integral-0008.HELP.tex}

\[\dfrac{d}{dt}(g(t))=\answer{e^{\left(e^{\left(t + 7\right)}\right)}}\]
\end{problem}}%}

%%%%%%%%%%%%%%%%%%%%%%




%%%%%%%%%%%%%%%%%%%%%%




%%%%%%%%%%%%%%%%%%%%%%




\latexProblemContent{
\begin{problem}

Use the Fundamental Theorem of Calculus to find the derivative of the function.
\[g(t)=\int_{5}^{t} {\sin\left(\sin\left(x - 4\right)\right)}\;dx\]

\expandafter\input{\file@loc Integrals/2311-Compute-Integral-0008.HELP.tex}

\[\dfrac{d}{dt}(g(t))=\answer{\sin\left(\sin\left(t - 4\right)\right)}\]
\end{problem}}%}

%%%%%%%%%%%%%%%%%%%%%%




\latexProblemContent{
\begin{problem}

Use the Fundamental Theorem of Calculus to find the derivative of the function.
\[g(t)=\int_{4}^{t} {{\left(x + 4\right)}^{4}}\;dx\]

\expandafter\input{\file@loc Integrals/2311-Compute-Integral-0008.HELP.tex}

\[\dfrac{d}{dt}(g(t))=\answer{{\left(t + 4\right)}^{4}}\]
\end{problem}}%}

%%%%%%%%%%%%%%%%%%%%%%




\latexProblemContent{
\begin{problem}

Use the Fundamental Theorem of Calculus to find the derivative of the function.
\[g(t)=\int_{1}^{t} {{\left(x + 5\right)}^{4}}\;dx\]

\expandafter\input{\file@loc Integrals/2311-Compute-Integral-0008.HELP.tex}

\[\dfrac{d}{dt}(g(t))=\answer{{\left(t + 5\right)}^{4}}\]
\end{problem}}%}

%%%%%%%%%%%%%%%%%%%%%%




\latexProblemContent{
\begin{problem}

Use the Fundamental Theorem of Calculus to find the derivative of the function.
\[g(t)=\int_{3}^{t} {x + 18}\;dx\]

\expandafter\input{\file@loc Integrals/2311-Compute-Integral-0008.HELP.tex}

\[\dfrac{d}{dt}(g(t))=\answer{t + 18}\]
\end{problem}}%}

%%%%%%%%%%%%%%%%%%%%%%




%%%%%%%%%%%%%%%%%%%%%%




%%%%%%%%%%%%%%%%%%%%%%




\latexProblemContent{
\begin{problem}

Use the Fundamental Theorem of Calculus to find the derivative of the function.
\[g(t)=\int_{1}^{t} {\sin\left(\sin\left(x + 4\right)\right)}\;dx\]

\expandafter\input{\file@loc Integrals/2311-Compute-Integral-0008.HELP.tex}

\[\dfrac{d}{dt}(g(t))=\answer{\sin\left(\sin\left(t + 4\right)\right)}\]
\end{problem}}%}

%%%%%%%%%%%%%%%%%%%%%%




\latexProblemContent{
\begin{problem}

Use the Fundamental Theorem of Calculus to find the derivative of the function.
\[g(t)=\int_{4}^{t} {{\left(x + 10\right)}^{4}}\;dx\]

\expandafter\input{\file@loc Integrals/2311-Compute-Integral-0008.HELP.tex}

\[\dfrac{d}{dt}(g(t))=\answer{{\left(t + 10\right)}^{4}}\]
\end{problem}}%}

%%%%%%%%%%%%%%%%%%%%%%




\latexProblemContent{
\begin{problem}

Use the Fundamental Theorem of Calculus to find the derivative of the function.
\[g(t)=\int_{4}^{t} {x - 7}\;dx\]

\expandafter\input{\file@loc Integrals/2311-Compute-Integral-0008.HELP.tex}

\[\dfrac{d}{dt}(g(t))=\answer{t - 7}\]
\end{problem}}%}

%%%%%%%%%%%%%%%%%%%%%%




%%%%%%%%%%%%%%%%%%%%%%




%%%%%%%%%%%%%%%%%%%%%%




\latexProblemContent{
\begin{problem}

Use the Fundamental Theorem of Calculus to find the derivative of the function.
\[g(t)=\int_{4}^{t} {{\left(x - 9\right)}^{9}}\;dx\]

\expandafter\input{\file@loc Integrals/2311-Compute-Integral-0008.HELP.tex}

\[\dfrac{d}{dt}(g(t))=\answer{{\left(t - 9\right)}^{9}}\]
\end{problem}}%}

%%%%%%%%%%%%%%%%%%%%%%




\latexProblemContent{
\begin{problem}

Use the Fundamental Theorem of Calculus to find the derivative of the function.
\[g(t)=\int_{2}^{t} {x + 20}\;dx\]

\expandafter\input{\file@loc Integrals/2311-Compute-Integral-0008.HELP.tex}

\[\dfrac{d}{dt}(g(t))=\answer{t + 20}\]
\end{problem}}%}

%%%%%%%%%%%%%%%%%%%%%%




\latexProblemContent{
\begin{problem}

Use the Fundamental Theorem of Calculus to find the derivative of the function.
\[g(t)=\int_{3}^{t} {{\left(x + 4\right)}^{9}}\;dx\]

\expandafter\input{\file@loc Integrals/2311-Compute-Integral-0008.HELP.tex}

\[\dfrac{d}{dt}(g(t))=\answer{{\left(t + 4\right)}^{9}}\]
\end{problem}}%}

%%%%%%%%%%%%%%%%%%%%%%




\latexProblemContent{
\begin{problem}

Use the Fundamental Theorem of Calculus to find the derivative of the function.
\[g(t)=\int_{1}^{t} {{\left(x + 6\right)}^{9}}\;dx\]

\expandafter\input{\file@loc Integrals/2311-Compute-Integral-0008.HELP.tex}

\[\dfrac{d}{dt}(g(t))=\answer{{\left(t + 6\right)}^{9}}\]
\end{problem}}%}

%%%%%%%%%%%%%%%%%%%%%%




\latexProblemContent{
\begin{problem}

Use the Fundamental Theorem of Calculus to find the derivative of the function.
\[g(t)=\int_{1}^{t} {e^{\left(e^{\left(x + 2\right)}\right)}}\;dx\]

\expandafter\input{\file@loc Integrals/2311-Compute-Integral-0008.HELP.tex}

\[\dfrac{d}{dt}(g(t))=\answer{e^{\left(e^{\left(t + 2\right)}\right)}}\]
\end{problem}}%}

%%%%%%%%%%%%%%%%%%%%%%




%%%%%%%%%%%%%%%%%%%%%%




\latexProblemContent{
\begin{problem}

Use the Fundamental Theorem of Calculus to find the derivative of the function.
\[g(t)=\int_{2}^{t} {x + 12}\;dx\]

\expandafter\input{\file@loc Integrals/2311-Compute-Integral-0008.HELP.tex}

\[\dfrac{d}{dt}(g(t))=\answer{t + 12}\]
\end{problem}}%}

%%%%%%%%%%%%%%%%%%%%%%




\latexProblemContent{
\begin{problem}

Use the Fundamental Theorem of Calculus to find the derivative of the function.
\[g(t)=\int_{1}^{t} {x - 20}\;dx\]

\expandafter\input{\file@loc Integrals/2311-Compute-Integral-0008.HELP.tex}

\[\dfrac{d}{dt}(g(t))=\answer{t - 20}\]
\end{problem}}%}

%%%%%%%%%%%%%%%%%%%%%%




\latexProblemContent{
\begin{problem}

Use the Fundamental Theorem of Calculus to find the derivative of the function.
\[g(t)=\int_{3}^{t} {\sin\left(\sin\left(x - 7\right)\right)}\;dx\]

\expandafter\input{\file@loc Integrals/2311-Compute-Integral-0008.HELP.tex}

\[\dfrac{d}{dt}(g(t))=\answer{\sin\left(\sin\left(t - 7\right)\right)}\]
\end{problem}}%}

%%%%%%%%%%%%%%%%%%%%%%




\latexProblemContent{
\begin{problem}

Use the Fundamental Theorem of Calculus to find the derivative of the function.
\[g(t)=\int_{2}^{t} {{\left(x + 10\right)}^{9}}\;dx\]

\expandafter\input{\file@loc Integrals/2311-Compute-Integral-0008.HELP.tex}

\[\dfrac{d}{dt}(g(t))=\answer{{\left(t + 10\right)}^{9}}\]
\end{problem}}%}

%%%%%%%%%%%%%%%%%%%%%%




\latexProblemContent{
\begin{problem}

Use the Fundamental Theorem of Calculus to find the derivative of the function.
\[g(t)=\int_{3}^{t} {{\left(x - 8\right)}^{\frac{1}{4}}}\;dx\]

\expandafter\input{\file@loc Integrals/2311-Compute-Integral-0008.HELP.tex}

\[\dfrac{d}{dt}(g(t))=\answer{{\left(t - 8\right)}^{\frac{1}{4}}}\]
\end{problem}}%}

%%%%%%%%%%%%%%%%%%%%%%




\latexProblemContent{
\begin{problem}

Use the Fundamental Theorem of Calculus to find the derivative of the function.
\[g(t)=\int_{5}^{t} {x - 16}\;dx\]

\expandafter\input{\file@loc Integrals/2311-Compute-Integral-0008.HELP.tex}

\[\dfrac{d}{dt}(g(t))=\answer{t - 16}\]
\end{problem}}%}

%%%%%%%%%%%%%%%%%%%%%%




\latexProblemContent{
\begin{problem}

Use the Fundamental Theorem of Calculus to find the derivative of the function.
\[g(t)=\int_{5}^{t} {e^{\left(e^{\left(x - 5\right)}\right)}}\;dx\]

\expandafter\input{\file@loc Integrals/2311-Compute-Integral-0008.HELP.tex}

\[\dfrac{d}{dt}(g(t))=\answer{e^{\left(e^{\left(t - 5\right)}\right)}}\]
\end{problem}}%}

%%%%%%%%%%%%%%%%%%%%%%




\latexProblemContent{
\begin{problem}

Use the Fundamental Theorem of Calculus to find the derivative of the function.
\[g(t)=\int_{4}^{t} {\log\left(\log\left(x - 2\right)\right)}\;dx\]

\expandafter\input{\file@loc Integrals/2311-Compute-Integral-0008.HELP.tex}

\[\dfrac{d}{dt}(g(t))=\answer{\log\left(\log\left(t - 2\right)\right)}\]
\end{problem}}%}

%%%%%%%%%%%%%%%%%%%%%%




%%%%%%%%%%%%%%%%%%%%%%




\latexProblemContent{
\begin{problem}

Use the Fundamental Theorem of Calculus to find the derivative of the function.
\[g(t)=\int_{4}^{t} {e^{\left(e^{\left(x - 3\right)}\right)}}\;dx\]

\expandafter\input{\file@loc Integrals/2311-Compute-Integral-0008.HELP.tex}

\[\dfrac{d}{dt}(g(t))=\answer{e^{\left(e^{\left(t - 3\right)}\right)}}\]
\end{problem}}%}

%%%%%%%%%%%%%%%%%%%%%%




%%%%%%%%%%%%%%%%%%%%%%




\latexProblemContent{
\begin{problem}

Use the Fundamental Theorem of Calculus to find the derivative of the function.
\[g(t)=\int_{4}^{t} {{\left(x + 8\right)}^{\frac{1}{4}}}\;dx\]

\expandafter\input{\file@loc Integrals/2311-Compute-Integral-0008.HELP.tex}

\[\dfrac{d}{dt}(g(t))=\answer{{\left(t + 8\right)}^{\frac{1}{4}}}\]
\end{problem}}%}

%%%%%%%%%%%%%%%%%%%%%%




%%%%%%%%%%%%%%%%%%%%%%




\latexProblemContent{
\begin{problem}

Use the Fundamental Theorem of Calculus to find the derivative of the function.
\[g(t)=\int_{5}^{t} {{\left(x + 9\right)}^{16}}\;dx\]

\expandafter\input{\file@loc Integrals/2311-Compute-Integral-0008.HELP.tex}

\[\dfrac{d}{dt}(g(t))=\answer{{\left(t + 9\right)}^{16}}\]
\end{problem}}%}

%%%%%%%%%%%%%%%%%%%%%%




\latexProblemContent{
\begin{problem}

Use the Fundamental Theorem of Calculus to find the derivative of the function.
\[g(t)=\int_{5}^{t} {{\left(x + 10\right)}^{4}}\;dx\]

\expandafter\input{\file@loc Integrals/2311-Compute-Integral-0008.HELP.tex}

\[\dfrac{d}{dt}(g(t))=\answer{{\left(t + 10\right)}^{4}}\]
\end{problem}}%}

%%%%%%%%%%%%%%%%%%%%%%




\latexProblemContent{
\begin{problem}

Use the Fundamental Theorem of Calculus to find the derivative of the function.
\[g(t)=\int_{1}^{t} {{\left(x - 1\right)}^{9}}\;dx\]

\expandafter\input{\file@loc Integrals/2311-Compute-Integral-0008.HELP.tex}

\[\dfrac{d}{dt}(g(t))=\answer{{\left(t - 1\right)}^{9}}\]
\end{problem}}%}

%%%%%%%%%%%%%%%%%%%%%%




\latexProblemContent{
\begin{problem}

Use the Fundamental Theorem of Calculus to find the derivative of the function.
\[g(t)=\int_{4}^{t} {\sin\left(\sin\left(x + 8\right)\right)}\;dx\]

\expandafter\input{\file@loc Integrals/2311-Compute-Integral-0008.HELP.tex}

\[\dfrac{d}{dt}(g(t))=\answer{\sin\left(\sin\left(t + 8\right)\right)}\]
\end{problem}}%}

%%%%%%%%%%%%%%%%%%%%%%




\latexProblemContent{
\begin{problem}

Use the Fundamental Theorem of Calculus to find the derivative of the function.
\[g(t)=\int_{5}^{t} {x + 10}\;dx\]

\expandafter\input{\file@loc Integrals/2311-Compute-Integral-0008.HELP.tex}

\[\dfrac{d}{dt}(g(t))=\answer{t + 10}\]
\end{problem}}%}

%%%%%%%%%%%%%%%%%%%%%%




\latexProblemContent{
\begin{problem}

Use the Fundamental Theorem of Calculus to find the derivative of the function.
\[g(t)=\int_{1}^{t} {{\left(x - 8\right)}^{\frac{1}{4}}}\;dx\]

\expandafter\input{\file@loc Integrals/2311-Compute-Integral-0008.HELP.tex}

\[\dfrac{d}{dt}(g(t))=\answer{{\left(t - 8\right)}^{\frac{1}{4}}}\]
\end{problem}}%}

%%%%%%%%%%%%%%%%%%%%%%




\latexProblemContent{
\begin{problem}

Use the Fundamental Theorem of Calculus to find the derivative of the function.
\[g(t)=\int_{2}^{t} {{\left(x + 6\right)}^{\frac{1}{4}}}\;dx\]

\expandafter\input{\file@loc Integrals/2311-Compute-Integral-0008.HELP.tex}

\[\dfrac{d}{dt}(g(t))=\answer{{\left(t + 6\right)}^{\frac{1}{4}}}\]
\end{problem}}%}

%%%%%%%%%%%%%%%%%%%%%%




%%%%%%%%%%%%%%%%%%%%%%




\latexProblemContent{
\begin{problem}

Use the Fundamental Theorem of Calculus to find the derivative of the function.
\[g(t)=\int_{1}^{t} {{\left(x + 9\right)}^{9}}\;dx\]

\expandafter\input{\file@loc Integrals/2311-Compute-Integral-0008.HELP.tex}

\[\dfrac{d}{dt}(g(t))=\answer{{\left(t + 9\right)}^{9}}\]
\end{problem}}%}

%%%%%%%%%%%%%%%%%%%%%%




\latexProblemContent{
\begin{problem}

Use the Fundamental Theorem of Calculus to find the derivative of the function.
\[g(t)=\int_{2}^{t} {\sin\left(\sin\left(x - 2\right)\right)}\;dx\]

\expandafter\input{\file@loc Integrals/2311-Compute-Integral-0008.HELP.tex}

\[\dfrac{d}{dt}(g(t))=\answer{\sin\left(\sin\left(t - 2\right)\right)}\]
\end{problem}}%}

%%%%%%%%%%%%%%%%%%%%%%




%%%%%%%%%%%%%%%%%%%%%%




%%%%%%%%%%%%%%%%%%%%%%




%%%%%%%%%%%%%%%%%%%%%%




\latexProblemContent{
\begin{problem}

Use the Fundamental Theorem of Calculus to find the derivative of the function.
\[g(t)=\int_{3}^{t} {\sin\left(\sin\left(x - 1\right)\right)}\;dx\]

\expandafter\input{\file@loc Integrals/2311-Compute-Integral-0008.HELP.tex}

\[\dfrac{d}{dt}(g(t))=\answer{\sin\left(\sin\left(t - 1\right)\right)}\]
\end{problem}}%}

%%%%%%%%%%%%%%%%%%%%%%




\latexProblemContent{
\begin{problem}

Use the Fundamental Theorem of Calculus to find the derivative of the function.
\[g(t)=\int_{1}^{t} {e^{\left(e^{\left(x - 9\right)}\right)}}\;dx\]

\expandafter\input{\file@loc Integrals/2311-Compute-Integral-0008.HELP.tex}

\[\dfrac{d}{dt}(g(t))=\answer{e^{\left(e^{\left(t - 9\right)}\right)}}\]
\end{problem}}%}

%%%%%%%%%%%%%%%%%%%%%%




\latexProblemContent{
\begin{problem}

Use the Fundamental Theorem of Calculus to find the derivative of the function.
\[g(t)=\int_{3}^{t} {{\left(x + 1\right)}^{9}}\;dx\]

\expandafter\input{\file@loc Integrals/2311-Compute-Integral-0008.HELP.tex}

\[\dfrac{d}{dt}(g(t))=\answer{{\left(t + 1\right)}^{9}}\]
\end{problem}}%}

%%%%%%%%%%%%%%%%%%%%%%




\latexProblemContent{
\begin{problem}

Use the Fundamental Theorem of Calculus to find the derivative of the function.
\[g(t)=\int_{4}^{t} {{\left(x - 6\right)}^{\frac{1}{4}}}\;dx\]

\expandafter\input{\file@loc Integrals/2311-Compute-Integral-0008.HELP.tex}

\[\dfrac{d}{dt}(g(t))=\answer{{\left(t - 6\right)}^{\frac{1}{4}}}\]
\end{problem}}%}

%%%%%%%%%%%%%%%%%%%%%%




\latexProblemContent{
\begin{problem}

Use the Fundamental Theorem of Calculus to find the derivative of the function.
\[g(t)=\int_{4}^{t} {\sin\left(\sin\left(x + 5\right)\right)}\;dx\]

\expandafter\input{\file@loc Integrals/2311-Compute-Integral-0008.HELP.tex}

\[\dfrac{d}{dt}(g(t))=\answer{\sin\left(\sin\left(t + 5\right)\right)}\]
\end{problem}}%}

%%%%%%%%%%%%%%%%%%%%%%




\latexProblemContent{
\begin{problem}

Use the Fundamental Theorem of Calculus to find the derivative of the function.
\[g(t)=\int_{4}^{t} {{\left(x + 6\right)}^{16}}\;dx\]

\expandafter\input{\file@loc Integrals/2311-Compute-Integral-0008.HELP.tex}

\[\dfrac{d}{dt}(g(t))=\answer{{\left(t + 6\right)}^{16}}\]
\end{problem}}%}

%%%%%%%%%%%%%%%%%%%%%%




%%%%%%%%%%%%%%%%%%%%%%




\latexProblemContent{
\begin{problem}

Use the Fundamental Theorem of Calculus to find the derivative of the function.
\[g(t)=\int_{2}^{t} {\cos\left(\cos\left(x - 7\right)\right)}\;dx\]

\expandafter\input{\file@loc Integrals/2311-Compute-Integral-0008.HELP.tex}

\[\dfrac{d}{dt}(g(t))=\answer{\cos\left(\cos\left(t - 7\right)\right)}\]
\end{problem}}%}

%%%%%%%%%%%%%%%%%%%%%%




%%%%%%%%%%%%%%%%%%%%%%




\latexProblemContent{
\begin{problem}

Use the Fundamental Theorem of Calculus to find the derivative of the function.
\[g(t)=\int_{3}^{t} {{\left(x - 4\right)}^{4}}\;dx\]

\expandafter\input{\file@loc Integrals/2311-Compute-Integral-0008.HELP.tex}

\[\dfrac{d}{dt}(g(t))=\answer{{\left(t - 4\right)}^{4}}\]
\end{problem}}%}

%%%%%%%%%%%%%%%%%%%%%%




\latexProblemContent{
\begin{problem}

Use the Fundamental Theorem of Calculus to find the derivative of the function.
\[g(t)=\int_{4}^{t} {\sin\left(\sin\left(x - 2\right)\right)}\;dx\]

\expandafter\input{\file@loc Integrals/2311-Compute-Integral-0008.HELP.tex}

\[\dfrac{d}{dt}(g(t))=\answer{\sin\left(\sin\left(t - 2\right)\right)}\]
\end{problem}}%}

%%%%%%%%%%%%%%%%%%%%%%




\latexProblemContent{
\begin{problem}

Use the Fundamental Theorem of Calculus to find the derivative of the function.
\[g(t)=\int_{4}^{t} {\log\left(\log\left(x - 5\right)\right)}\;dx\]

\expandafter\input{\file@loc Integrals/2311-Compute-Integral-0008.HELP.tex}

\[\dfrac{d}{dt}(g(t))=\answer{\log\left(\log\left(t - 5\right)\right)}\]
\end{problem}}%}

%%%%%%%%%%%%%%%%%%%%%%




%%%%%%%%%%%%%%%%%%%%%%




\latexProblemContent{
\begin{problem}

Use the Fundamental Theorem of Calculus to find the derivative of the function.
\[g(t)=\int_{3}^{t} {{\left(x + 7\right)}^{16}}\;dx\]

\expandafter\input{\file@loc Integrals/2311-Compute-Integral-0008.HELP.tex}

\[\dfrac{d}{dt}(g(t))=\answer{{\left(t + 7\right)}^{16}}\]
\end{problem}}%}

%%%%%%%%%%%%%%%%%%%%%%




\latexProblemContent{
\begin{problem}

Use the Fundamental Theorem of Calculus to find the derivative of the function.
\[g(t)=\int_{4}^{t} {{\left(x + 4\right)}^{9}}\;dx\]

\expandafter\input{\file@loc Integrals/2311-Compute-Integral-0008.HELP.tex}

\[\dfrac{d}{dt}(g(t))=\answer{{\left(t + 4\right)}^{9}}\]
\end{problem}}%}

%%%%%%%%%%%%%%%%%%%%%%




\latexProblemContent{
\begin{problem}

Use the Fundamental Theorem of Calculus to find the derivative of the function.
\[g(t)=\int_{4}^{t} {{\left(x - 5\right)}^{9}}\;dx\]

\expandafter\input{\file@loc Integrals/2311-Compute-Integral-0008.HELP.tex}

\[\dfrac{d}{dt}(g(t))=\answer{{\left(t - 5\right)}^{9}}\]
\end{problem}}%}

%%%%%%%%%%%%%%%%%%%%%%




\latexProblemContent{
\begin{problem}

Use the Fundamental Theorem of Calculus to find the derivative of the function.
\[g(t)=\int_{5}^{t} {{\left(x + 4\right)}^{\frac{1}{4}}}\;dx\]

\expandafter\input{\file@loc Integrals/2311-Compute-Integral-0008.HELP.tex}

\[\dfrac{d}{dt}(g(t))=\answer{{\left(t + 4\right)}^{\frac{1}{4}}}\]
\end{problem}}%}

%%%%%%%%%%%%%%%%%%%%%%




\latexProblemContent{
\begin{problem}

Use the Fundamental Theorem of Calculus to find the derivative of the function.
\[g(t)=\int_{2}^{t} {{\left(x + 4\right)}^{9}}\;dx\]

\expandafter\input{\file@loc Integrals/2311-Compute-Integral-0008.HELP.tex}

\[\dfrac{d}{dt}(g(t))=\answer{{\left(t + 4\right)}^{9}}\]
\end{problem}}%}

%%%%%%%%%%%%%%%%%%%%%%




%%%%%%%%%%%%%%%%%%%%%%




%%%%%%%%%%%%%%%%%%%%%%




\latexProblemContent{
\begin{problem}

Use the Fundamental Theorem of Calculus to find the derivative of the function.
\[g(t)=\int_{1}^{t} {{\left(x - 6\right)}^{9}}\;dx\]

\expandafter\input{\file@loc Integrals/2311-Compute-Integral-0008.HELP.tex}

\[\dfrac{d}{dt}(g(t))=\answer{{\left(t - 6\right)}^{9}}\]
\end{problem}}%}

%%%%%%%%%%%%%%%%%%%%%%




\latexProblemContent{
\begin{problem}

Use the Fundamental Theorem of Calculus to find the derivative of the function.
\[g(t)=\int_{2}^{t} {{\left(x - 6\right)}^{9}}\;dx\]

\expandafter\input{\file@loc Integrals/2311-Compute-Integral-0008.HELP.tex}

\[\dfrac{d}{dt}(g(t))=\answer{{\left(t - 6\right)}^{9}}\]
\end{problem}}%}

%%%%%%%%%%%%%%%%%%%%%%




\latexProblemContent{
\begin{problem}

Use the Fundamental Theorem of Calculus to find the derivative of the function.
\[g(t)=\int_{2}^{t} {{\left(x - 3\right)}^{9}}\;dx\]

\expandafter\input{\file@loc Integrals/2311-Compute-Integral-0008.HELP.tex}

\[\dfrac{d}{dt}(g(t))=\answer{{\left(t - 3\right)}^{9}}\]
\end{problem}}%}

%%%%%%%%%%%%%%%%%%%%%%




%%%%%%%%%%%%%%%%%%%%%%




\latexProblemContent{
\begin{problem}

Use the Fundamental Theorem of Calculus to find the derivative of the function.
\[g(t)=\int_{3}^{t} {x + 1}\;dx\]

\expandafter\input{\file@loc Integrals/2311-Compute-Integral-0008.HELP.tex}

\[\dfrac{d}{dt}(g(t))=\answer{t + 1}\]
\end{problem}}%}

%%%%%%%%%%%%%%%%%%%%%%




\latexProblemContent{
\begin{problem}

Use the Fundamental Theorem of Calculus to find the derivative of the function.
\[g(t)=\int_{4}^{t} {{\left(x - 2\right)}^{16}}\;dx\]

\expandafter\input{\file@loc Integrals/2311-Compute-Integral-0008.HELP.tex}

\[\dfrac{d}{dt}(g(t))=\answer{{\left(t - 2\right)}^{16}}\]
\end{problem}}%}

%%%%%%%%%%%%%%%%%%%%%%




%%%%%%%%%%%%%%%%%%%%%%




\latexProblemContent{
\begin{problem}

Use the Fundamental Theorem of Calculus to find the derivative of the function.
\[g(t)=\int_{3}^{t} {{\left(x + 9\right)}^{9}}\;dx\]

\expandafter\input{\file@loc Integrals/2311-Compute-Integral-0008.HELP.tex}

\[\dfrac{d}{dt}(g(t))=\answer{{\left(t + 9\right)}^{9}}\]
\end{problem}}%}

%%%%%%%%%%%%%%%%%%%%%%




\latexProblemContent{
\begin{problem}

Use the Fundamental Theorem of Calculus to find the derivative of the function.
\[g(t)=\int_{3}^{t} {{\left(x + 3\right)}^{9}}\;dx\]

\expandafter\input{\file@loc Integrals/2311-Compute-Integral-0008.HELP.tex}

\[\dfrac{d}{dt}(g(t))=\answer{{\left(t + 3\right)}^{9}}\]
\end{problem}}%}

%%%%%%%%%%%%%%%%%%%%%%




%%%%%%%%%%%%%%%%%%%%%%




\latexProblemContent{
\begin{problem}

Use the Fundamental Theorem of Calculus to find the derivative of the function.
\[g(t)=\int_{5}^{t} {x + 9}\;dx\]

\expandafter\input{\file@loc Integrals/2311-Compute-Integral-0008.HELP.tex}

\[\dfrac{d}{dt}(g(t))=\answer{t + 9}\]
\end{problem}}%}

%%%%%%%%%%%%%%%%%%%%%%




\latexProblemContent{
\begin{problem}

Use the Fundamental Theorem of Calculus to find the derivative of the function.
\[g(t)=\int_{2}^{t} {\sin\left(\sin\left(x - 7\right)\right)}\;dx\]

\expandafter\input{\file@loc Integrals/2311-Compute-Integral-0008.HELP.tex}

\[\dfrac{d}{dt}(g(t))=\answer{\sin\left(\sin\left(t - 7\right)\right)}\]
\end{problem}}%}

%%%%%%%%%%%%%%%%%%%%%%




%%%%%%%%%%%%%%%%%%%%%%




%%%%%%%%%%%%%%%%%%%%%%




\latexProblemContent{
\begin{problem}

Use the Fundamental Theorem of Calculus to find the derivative of the function.
\[g(t)=\int_{3}^{t} {\log\left(\log\left(x - 1\right)\right)}\;dx\]

\expandafter\input{\file@loc Integrals/2311-Compute-Integral-0008.HELP.tex}

\[\dfrac{d}{dt}(g(t))=\answer{\log\left(\log\left(t - 1\right)\right)}\]
\end{problem}}%}

%%%%%%%%%%%%%%%%%%%%%%




\latexProblemContent{
\begin{problem}

Use the Fundamental Theorem of Calculus to find the derivative of the function.
\[g(t)=\int_{2}^{t} {\log\left(\log\left(x - 3\right)\right)}\;dx\]

\expandafter\input{\file@loc Integrals/2311-Compute-Integral-0008.HELP.tex}

\[\dfrac{d}{dt}(g(t))=\answer{\log\left(\log\left(t - 3\right)\right)}\]
\end{problem}}%}

%%%%%%%%%%%%%%%%%%%%%%




\latexProblemContent{
\begin{problem}

Use the Fundamental Theorem of Calculus to find the derivative of the function.
\[g(t)=\int_{2}^{t} {\sin\left(\sin\left(x + 5\right)\right)}\;dx\]

\expandafter\input{\file@loc Integrals/2311-Compute-Integral-0008.HELP.tex}

\[\dfrac{d}{dt}(g(t))=\answer{\sin\left(\sin\left(t + 5\right)\right)}\]
\end{problem}}%}

%%%%%%%%%%%%%%%%%%%%%%




\latexProblemContent{
\begin{problem}

Use the Fundamental Theorem of Calculus to find the derivative of the function.
\[g(t)=\int_{5}^{t} {{\left(x - 3\right)}^{16}}\;dx\]

\expandafter\input{\file@loc Integrals/2311-Compute-Integral-0008.HELP.tex}

\[\dfrac{d}{dt}(g(t))=\answer{{\left(t - 3\right)}^{16}}\]
\end{problem}}%}

%%%%%%%%%%%%%%%%%%%%%%




%%%%%%%%%%%%%%%%%%%%%%




\latexProblemContent{
\begin{problem}

Use the Fundamental Theorem of Calculus to find the derivative of the function.
\[g(t)=\int_{1}^{t} {\cos\left(\cos\left(x + 10\right)\right)}\;dx\]

\expandafter\input{\file@loc Integrals/2311-Compute-Integral-0008.HELP.tex}

\[\dfrac{d}{dt}(g(t))=\answer{\cos\left(\cos\left(t + 10\right)\right)}\]
\end{problem}}%}

%%%%%%%%%%%%%%%%%%%%%%




\latexProblemContent{
\begin{problem}

Use the Fundamental Theorem of Calculus to find the derivative of the function.
\[g(t)=\int_{2}^{t} {x + 1}\;dx\]

\expandafter\input{\file@loc Integrals/2311-Compute-Integral-0008.HELP.tex}

\[\dfrac{d}{dt}(g(t))=\answer{t + 1}\]
\end{problem}}%}

%%%%%%%%%%%%%%%%%%%%%%




%%%%%%%%%%%%%%%%%%%%%%




\latexProblemContent{
\begin{problem}

Use the Fundamental Theorem of Calculus to find the derivative of the function.
\[g(t)=\int_{3}^{t} {\log\left(\log\left(x - 3\right)\right)}\;dx\]

\expandafter\input{\file@loc Integrals/2311-Compute-Integral-0008.HELP.tex}

\[\dfrac{d}{dt}(g(t))=\answer{\log\left(\log\left(t - 3\right)\right)}\]
\end{problem}}%}

%%%%%%%%%%%%%%%%%%%%%%




%%%%%%%%%%%%%%%%%%%%%%




\latexProblemContent{
\begin{problem}

Use the Fundamental Theorem of Calculus to find the derivative of the function.
\[g(t)=\int_{3}^{t} {x - 8}\;dx\]

\expandafter\input{\file@loc Integrals/2311-Compute-Integral-0008.HELP.tex}

\[\dfrac{d}{dt}(g(t))=\answer{t - 8}\]
\end{problem}}%}

%%%%%%%%%%%%%%%%%%%%%%




%%%%%%%%%%%%%%%%%%%%%%




\latexProblemContent{
\begin{problem}

Use the Fundamental Theorem of Calculus to find the derivative of the function.
\[g(t)=\int_{2}^{t} {{\left(x + 10\right)}^{\frac{1}{4}}}\;dx\]

\expandafter\input{\file@loc Integrals/2311-Compute-Integral-0008.HELP.tex}

\[\dfrac{d}{dt}(g(t))=\answer{{\left(t + 10\right)}^{\frac{1}{4}}}\]
\end{problem}}%}

%%%%%%%%%%%%%%%%%%%%%%




\latexProblemContent{
\begin{problem}

Use the Fundamental Theorem of Calculus to find the derivative of the function.
\[g(t)=\int_{3}^{t} {x + 8}\;dx\]

\expandafter\input{\file@loc Integrals/2311-Compute-Integral-0008.HELP.tex}

\[\dfrac{d}{dt}(g(t))=\answer{t + 8}\]
\end{problem}}%}

%%%%%%%%%%%%%%%%%%%%%%




%%%%%%%%%%%%%%%%%%%%%%




%%%%%%%%%%%%%%%%%%%%%%




\latexProblemContent{
\begin{problem}

Use the Fundamental Theorem of Calculus to find the derivative of the function.
\[g(t)=\int_{3}^{t} {\sin\left(\sin\left(x - 5\right)\right)}\;dx\]

\expandafter\input{\file@loc Integrals/2311-Compute-Integral-0008.HELP.tex}

\[\dfrac{d}{dt}(g(t))=\answer{\sin\left(\sin\left(t - 5\right)\right)}\]
\end{problem}}%}

%%%%%%%%%%%%%%%%%%%%%%




\latexProblemContent{
\begin{problem}

Use the Fundamental Theorem of Calculus to find the derivative of the function.
\[g(t)=\int_{4}^{t} {{\left(x - 7\right)}^{16}}\;dx\]

\expandafter\input{\file@loc Integrals/2311-Compute-Integral-0008.HELP.tex}

\[\dfrac{d}{dt}(g(t))=\answer{{\left(t - 7\right)}^{16}}\]
\end{problem}}%}

%%%%%%%%%%%%%%%%%%%%%%




\latexProblemContent{
\begin{problem}

Use the Fundamental Theorem of Calculus to find the derivative of the function.
\[g(t)=\int_{4}^{t} {{\left(x - 6\right)}^{9}}\;dx\]

\expandafter\input{\file@loc Integrals/2311-Compute-Integral-0008.HELP.tex}

\[\dfrac{d}{dt}(g(t))=\answer{{\left(t - 6\right)}^{9}}\]
\end{problem}}%}

%%%%%%%%%%%%%%%%%%%%%%




\latexProblemContent{
\begin{problem}

Use the Fundamental Theorem of Calculus to find the derivative of the function.
\[g(t)=\int_{1}^{t} {x - 1}\;dx\]

\expandafter\input{\file@loc Integrals/2311-Compute-Integral-0008.HELP.tex}

\[\dfrac{d}{dt}(g(t))=\answer{t - 1}\]
\end{problem}}%}

%%%%%%%%%%%%%%%%%%%%%%




\latexProblemContent{
\begin{problem}

Use the Fundamental Theorem of Calculus to find the derivative of the function.
\[g(t)=\int_{3}^{t} {{\left(x - 9\right)}^{4}}\;dx\]

\expandafter\input{\file@loc Integrals/2311-Compute-Integral-0008.HELP.tex}

\[\dfrac{d}{dt}(g(t))=\answer{{\left(t - 9\right)}^{4}}\]
\end{problem}}%}

%%%%%%%%%%%%%%%%%%%%%%




\latexProblemContent{
\begin{problem}

Use the Fundamental Theorem of Calculus to find the derivative of the function.
\[g(t)=\int_{1}^{t} {x - 2}\;dx\]

\expandafter\input{\file@loc Integrals/2311-Compute-Integral-0008.HELP.tex}

\[\dfrac{d}{dt}(g(t))=\answer{t - 2}\]
\end{problem}}%}

%%%%%%%%%%%%%%%%%%%%%%




\latexProblemContent{
\begin{problem}

Use the Fundamental Theorem of Calculus to find the derivative of the function.
\[g(t)=\int_{5}^{t} {{\left(x - 10\right)}^{9}}\;dx\]

\expandafter\input{\file@loc Integrals/2311-Compute-Integral-0008.HELP.tex}

\[\dfrac{d}{dt}(g(t))=\answer{{\left(t - 10\right)}^{9}}\]
\end{problem}}%}

%%%%%%%%%%%%%%%%%%%%%%




\latexProblemContent{
\begin{problem}

Use the Fundamental Theorem of Calculus to find the derivative of the function.
\[g(t)=\int_{3}^{t} {e^{\left(e^{\left(x + 7\right)}\right)}}\;dx\]

\expandafter\input{\file@loc Integrals/2311-Compute-Integral-0008.HELP.tex}

\[\dfrac{d}{dt}(g(t))=\answer{e^{\left(e^{\left(t + 7\right)}\right)}}\]
\end{problem}}%}

%%%%%%%%%%%%%%%%%%%%%%




\latexProblemContent{
\begin{problem}

Use the Fundamental Theorem of Calculus to find the derivative of the function.
\[g(t)=\int_{2}^{t} {{\left(x - 9\right)}^{16}}\;dx\]

\expandafter\input{\file@loc Integrals/2311-Compute-Integral-0008.HELP.tex}

\[\dfrac{d}{dt}(g(t))=\answer{{\left(t - 9\right)}^{16}}\]
\end{problem}}%}

%%%%%%%%%%%%%%%%%%%%%%




\latexProblemContent{
\begin{problem}

Use the Fundamental Theorem of Calculus to find the derivative of the function.
\[g(t)=\int_{5}^{t} {{\left(x - 2\right)}^{\frac{1}{4}}}\;dx\]

\expandafter\input{\file@loc Integrals/2311-Compute-Integral-0008.HELP.tex}

\[\dfrac{d}{dt}(g(t))=\answer{{\left(t - 2\right)}^{\frac{1}{4}}}\]
\end{problem}}%}

%%%%%%%%%%%%%%%%%%%%%%




\latexProblemContent{
\begin{problem}

Use the Fundamental Theorem of Calculus to find the derivative of the function.
\[g(t)=\int_{5}^{t} {e^{\left(e^{\left(x + 3\right)}\right)}}\;dx\]

\expandafter\input{\file@loc Integrals/2311-Compute-Integral-0008.HELP.tex}

\[\dfrac{d}{dt}(g(t))=\answer{e^{\left(e^{\left(t + 3\right)}\right)}}\]
\end{problem}}%}

%%%%%%%%%%%%%%%%%%%%%%




\latexProblemContent{
\begin{problem}

Use the Fundamental Theorem of Calculus to find the derivative of the function.
\[g(t)=\int_{4}^{t} {{\left(x + 7\right)}^{4}}\;dx\]

\expandafter\input{\file@loc Integrals/2311-Compute-Integral-0008.HELP.tex}

\[\dfrac{d}{dt}(g(t))=\answer{{\left(t + 7\right)}^{4}}\]
\end{problem}}%}

%%%%%%%%%%%%%%%%%%%%%%




%%%%%%%%%%%%%%%%%%%%%%




%%%%%%%%%%%%%%%%%%%%%%




\latexProblemContent{
\begin{problem}

Use the Fundamental Theorem of Calculus to find the derivative of the function.
\[g(t)=\int_{5}^{t} {{\left(x - 8\right)}^{\frac{1}{4}}}\;dx\]

\expandafter\input{\file@loc Integrals/2311-Compute-Integral-0008.HELP.tex}

\[\dfrac{d}{dt}(g(t))=\answer{{\left(t - 8\right)}^{\frac{1}{4}}}\]
\end{problem}}%}

%%%%%%%%%%%%%%%%%%%%%%




\latexProblemContent{
\begin{problem}

Use the Fundamental Theorem of Calculus to find the derivative of the function.
\[g(t)=\int_{3}^{t} {{\left(x + 10\right)}^{\frac{1}{4}}}\;dx\]

\expandafter\input{\file@loc Integrals/2311-Compute-Integral-0008.HELP.tex}

\[\dfrac{d}{dt}(g(t))=\answer{{\left(t + 10\right)}^{\frac{1}{4}}}\]
\end{problem}}%}

%%%%%%%%%%%%%%%%%%%%%%




%%%%%%%%%%%%%%%%%%%%%%




\latexProblemContent{
\begin{problem}

Use the Fundamental Theorem of Calculus to find the derivative of the function.
\[g(t)=\int_{5}^{t} {x - 7}\;dx\]

\expandafter\input{\file@loc Integrals/2311-Compute-Integral-0008.HELP.tex}

\[\dfrac{d}{dt}(g(t))=\answer{t - 7}\]
\end{problem}}%}

%%%%%%%%%%%%%%%%%%%%%%




%%%%%%%%%%%%%%%%%%%%%%




%%%%%%%%%%%%%%%%%%%%%%




\latexProblemContent{
\begin{problem}

Use the Fundamental Theorem of Calculus to find the derivative of the function.
\[g(t)=\int_{1}^{t} {{\left(x - 10\right)}^{\frac{1}{4}}}\;dx\]

\expandafter\input{\file@loc Integrals/2311-Compute-Integral-0008.HELP.tex}

\[\dfrac{d}{dt}(g(t))=\answer{{\left(t - 10\right)}^{\frac{1}{4}}}\]
\end{problem}}%}

%%%%%%%%%%%%%%%%%%%%%%




\latexProblemContent{
\begin{problem}

Use the Fundamental Theorem of Calculus to find the derivative of the function.
\[g(t)=\int_{2}^{t} {{\left(x + 1\right)}^{9}}\;dx\]

\expandafter\input{\file@loc Integrals/2311-Compute-Integral-0008.HELP.tex}

\[\dfrac{d}{dt}(g(t))=\answer{{\left(t + 1\right)}^{9}}\]
\end{problem}}%}

%%%%%%%%%%%%%%%%%%%%%%




%%%%%%%%%%%%%%%%%%%%%%




\latexProblemContent{
\begin{problem}

Use the Fundamental Theorem of Calculus to find the derivative of the function.
\[g(t)=\int_{2}^{t} {x - 7}\;dx\]

\expandafter\input{\file@loc Integrals/2311-Compute-Integral-0008.HELP.tex}

\[\dfrac{d}{dt}(g(t))=\answer{t - 7}\]
\end{problem}}%}

%%%%%%%%%%%%%%%%%%%%%%




\latexProblemContent{
\begin{problem}

Use the Fundamental Theorem of Calculus to find the derivative of the function.
\[g(t)=\int_{3}^{t} {{\left(x + 6\right)}^{16}}\;dx\]

\expandafter\input{\file@loc Integrals/2311-Compute-Integral-0008.HELP.tex}

\[\dfrac{d}{dt}(g(t))=\answer{{\left(t + 6\right)}^{16}}\]
\end{problem}}%}

%%%%%%%%%%%%%%%%%%%%%%




\latexProblemContent{
\begin{problem}

Use the Fundamental Theorem of Calculus to find the derivative of the function.
\[g(t)=\int_{1}^{t} {\log\left(\log\left(x - 1\right)\right)}\;dx\]

\expandafter\input{\file@loc Integrals/2311-Compute-Integral-0008.HELP.tex}

\[\dfrac{d}{dt}(g(t))=\answer{\log\left(\log\left(t - 1\right)\right)}\]
\end{problem}}%}

%%%%%%%%%%%%%%%%%%%%%%




%%%%%%%%%%%%%%%%%%%%%%




\latexProblemContent{
\begin{problem}

Use the Fundamental Theorem of Calculus to find the derivative of the function.
\[g(t)=\int_{5}^{t} {{\left(x - 2\right)}^{9}}\;dx\]

\expandafter\input{\file@loc Integrals/2311-Compute-Integral-0008.HELP.tex}

\[\dfrac{d}{dt}(g(t))=\answer{{\left(t - 2\right)}^{9}}\]
\end{problem}}%}

%%%%%%%%%%%%%%%%%%%%%%




%%%%%%%%%%%%%%%%%%%%%%




\latexProblemContent{
\begin{problem}

Use the Fundamental Theorem of Calculus to find the derivative of the function.
\[g(t)=\int_{5}^{t} {\cos\left(\cos\left(x + 8\right)\right)}\;dx\]

\expandafter\input{\file@loc Integrals/2311-Compute-Integral-0008.HELP.tex}

\[\dfrac{d}{dt}(g(t))=\answer{\cos\left(\cos\left(t + 8\right)\right)}\]
\end{problem}}%}

%%%%%%%%%%%%%%%%%%%%%%




\latexProblemContent{
\begin{problem}

Use the Fundamental Theorem of Calculus to find the derivative of the function.
\[g(t)=\int_{2}^{t} {\sin\left(\sin\left(x + 1\right)\right)}\;dx\]

\expandafter\input{\file@loc Integrals/2311-Compute-Integral-0008.HELP.tex}

\[\dfrac{d}{dt}(g(t))=\answer{\sin\left(\sin\left(t + 1\right)\right)}\]
\end{problem}}%}

%%%%%%%%%%%%%%%%%%%%%%




%%%%%%%%%%%%%%%%%%%%%%




\latexProblemContent{
\begin{problem}

Use the Fundamental Theorem of Calculus to find the derivative of the function.
\[g(t)=\int_{3}^{t} {{\left(x - 2\right)}^{4}}\;dx\]

\expandafter\input{\file@loc Integrals/2311-Compute-Integral-0008.HELP.tex}

\[\dfrac{d}{dt}(g(t))=\answer{{\left(t - 2\right)}^{4}}\]
\end{problem}}%}

%%%%%%%%%%%%%%%%%%%%%%




%%%%%%%%%%%%%%%%%%%%%%




\latexProblemContent{
\begin{problem}

Use the Fundamental Theorem of Calculus to find the derivative of the function.
\[g(t)=\int_{2}^{t} {\log\left(\log\left(x - 5\right)\right)}\;dx\]

\expandafter\input{\file@loc Integrals/2311-Compute-Integral-0008.HELP.tex}

\[\dfrac{d}{dt}(g(t))=\answer{\log\left(\log\left(t - 5\right)\right)}\]
\end{problem}}%}

%%%%%%%%%%%%%%%%%%%%%%




%%%%%%%%%%%%%%%%%%%%%%




\latexProblemContent{
\begin{problem}

Use the Fundamental Theorem of Calculus to find the derivative of the function.
\[g(t)=\int_{3}^{t} {x - 2}\;dx\]

\expandafter\input{\file@loc Integrals/2311-Compute-Integral-0008.HELP.tex}

\[\dfrac{d}{dt}(g(t))=\answer{t - 2}\]
\end{problem}}%}

%%%%%%%%%%%%%%%%%%%%%%




\latexProblemContent{
\begin{problem}

Use the Fundamental Theorem of Calculus to find the derivative of the function.
\[g(t)=\int_{1}^{t} {\cos\left(\cos\left(x + 3\right)\right)}\;dx\]

\expandafter\input{\file@loc Integrals/2311-Compute-Integral-0008.HELP.tex}

\[\dfrac{d}{dt}(g(t))=\answer{\cos\left(\cos\left(t + 3\right)\right)}\]
\end{problem}}%}

%%%%%%%%%%%%%%%%%%%%%%




\latexProblemContent{
\begin{problem}

Use the Fundamental Theorem of Calculus to find the derivative of the function.
\[g(t)=\int_{2}^{t} {x - 12}\;dx\]

\expandafter\input{\file@loc Integrals/2311-Compute-Integral-0008.HELP.tex}

\[\dfrac{d}{dt}(g(t))=\answer{t - 12}\]
\end{problem}}%}

%%%%%%%%%%%%%%%%%%%%%%




\latexProblemContent{
\begin{problem}

Use the Fundamental Theorem of Calculus to find the derivative of the function.
\[g(t)=\int_{3}^{t} {{\left(x - 8\right)}^{4}}\;dx\]

\expandafter\input{\file@loc Integrals/2311-Compute-Integral-0008.HELP.tex}

\[\dfrac{d}{dt}(g(t))=\answer{{\left(t - 8\right)}^{4}}\]
\end{problem}}%}

%%%%%%%%%%%%%%%%%%%%%%




\latexProblemContent{
\begin{problem}

Use the Fundamental Theorem of Calculus to find the derivative of the function.
\[g(t)=\int_{2}^{t} {{\left(x + 9\right)}^{9}}\;dx\]

\expandafter\input{\file@loc Integrals/2311-Compute-Integral-0008.HELP.tex}

\[\dfrac{d}{dt}(g(t))=\answer{{\left(t + 9\right)}^{9}}\]
\end{problem}}%}

%%%%%%%%%%%%%%%%%%%%%%




\latexProblemContent{
\begin{problem}

Use the Fundamental Theorem of Calculus to find the derivative of the function.
\[g(t)=\int_{2}^{t} {e^{\left(e^{\left(x - 7\right)}\right)}}\;dx\]

\expandafter\input{\file@loc Integrals/2311-Compute-Integral-0008.HELP.tex}

\[\dfrac{d}{dt}(g(t))=\answer{e^{\left(e^{\left(t - 7\right)}\right)}}\]
\end{problem}}%}

%%%%%%%%%%%%%%%%%%%%%%




\latexProblemContent{
\begin{problem}

Use the Fundamental Theorem of Calculus to find the derivative of the function.
\[g(t)=\int_{1}^{t} {\sin\left(\sin\left(x - 8\right)\right)}\;dx\]

\expandafter\input{\file@loc Integrals/2311-Compute-Integral-0008.HELP.tex}

\[\dfrac{d}{dt}(g(t))=\answer{\sin\left(\sin\left(t - 8\right)\right)}\]
\end{problem}}%}

%%%%%%%%%%%%%%%%%%%%%%




%%%%%%%%%%%%%%%%%%%%%%




\latexProblemContent{
\begin{problem}

Use the Fundamental Theorem of Calculus to find the derivative of the function.
\[g(t)=\int_{1}^{t} {e^{\left(e^{\left(x + 7\right)}\right)}}\;dx\]

\expandafter\input{\file@loc Integrals/2311-Compute-Integral-0008.HELP.tex}

\[\dfrac{d}{dt}(g(t))=\answer{e^{\left(e^{\left(t + 7\right)}\right)}}\]
\end{problem}}%}

%%%%%%%%%%%%%%%%%%%%%%




%%%%%%%%%%%%%%%%%%%%%%




%%%%%%%%%%%%%%%%%%%%%%




\latexProblemContent{
\begin{problem}

Use the Fundamental Theorem of Calculus to find the derivative of the function.
\[g(t)=\int_{3}^{t} {{\left(x + 4\right)}^{16}}\;dx\]

\expandafter\input{\file@loc Integrals/2311-Compute-Integral-0008.HELP.tex}

\[\dfrac{d}{dt}(g(t))=\answer{{\left(t + 4\right)}^{16}}\]
\end{problem}}%}

%%%%%%%%%%%%%%%%%%%%%%




%%%%%%%%%%%%%%%%%%%%%%




\latexProblemContent{
\begin{problem}

Use the Fundamental Theorem of Calculus to find the derivative of the function.
\[g(t)=\int_{4}^{t} {e^{\left(e^{\left(x - 9\right)}\right)}}\;dx\]

\expandafter\input{\file@loc Integrals/2311-Compute-Integral-0008.HELP.tex}

\[\dfrac{d}{dt}(g(t))=\answer{e^{\left(e^{\left(t - 9\right)}\right)}}\]
\end{problem}}%}

%%%%%%%%%%%%%%%%%%%%%%




\latexProblemContent{
\begin{problem}

Use the Fundamental Theorem of Calculus to find the derivative of the function.
\[g(t)=\int_{2}^{t} {e^{\left(e^{\left(x + 4\right)}\right)}}\;dx\]

\expandafter\input{\file@loc Integrals/2311-Compute-Integral-0008.HELP.tex}

\[\dfrac{d}{dt}(g(t))=\answer{e^{\left(e^{\left(t + 4\right)}\right)}}\]
\end{problem}}%}

%%%%%%%%%%%%%%%%%%%%%%




\latexProblemContent{
\begin{problem}

Use the Fundamental Theorem of Calculus to find the derivative of the function.
\[g(t)=\int_{1}^{t} {{\left(x + 9\right)}^{\frac{1}{4}}}\;dx\]

\expandafter\input{\file@loc Integrals/2311-Compute-Integral-0008.HELP.tex}

\[\dfrac{d}{dt}(g(t))=\answer{{\left(t + 9\right)}^{\frac{1}{4}}}\]
\end{problem}}%}

%%%%%%%%%%%%%%%%%%%%%%




%%%%%%%%%%%%%%%%%%%%%%




%%%%%%%%%%%%%%%%%%%%%%




%%%%%%%%%%%%%%%%%%%%%%




\latexProblemContent{
\begin{problem}

Use the Fundamental Theorem of Calculus to find the derivative of the function.
\[g(t)=\int_{2}^{t} {\sin\left(\sin\left(x + 10\right)\right)}\;dx\]

\expandafter\input{\file@loc Integrals/2311-Compute-Integral-0008.HELP.tex}

\[\dfrac{d}{dt}(g(t))=\answer{\sin\left(\sin\left(t + 10\right)\right)}\]
\end{problem}}%}

%%%%%%%%%%%%%%%%%%%%%%




%%%%%%%%%%%%%%%%%%%%%%




%%%%%%%%%%%%%%%%%%%%%%




\latexProblemContent{
\begin{problem}

Use the Fundamental Theorem of Calculus to find the derivative of the function.
\[g(t)=\int_{5}^{t} {x - 9}\;dx\]

\expandafter\input{\file@loc Integrals/2311-Compute-Integral-0008.HELP.tex}

\[\dfrac{d}{dt}(g(t))=\answer{t - 9}\]
\end{problem}}%}

%%%%%%%%%%%%%%%%%%%%%%




%%%%%%%%%%%%%%%%%%%%%%




\latexProblemContent{
\begin{problem}

Use the Fundamental Theorem of Calculus to find the derivative of the function.
\[g(t)=\int_{1}^{t} {{\left(x + 1\right)}^{4}}\;dx\]

\expandafter\input{\file@loc Integrals/2311-Compute-Integral-0008.HELP.tex}

\[\dfrac{d}{dt}(g(t))=\answer{{\left(t + 1\right)}^{4}}\]
\end{problem}}%}

%%%%%%%%%%%%%%%%%%%%%%




\latexProblemContent{
\begin{problem}

Use the Fundamental Theorem of Calculus to find the derivative of the function.
\[g(t)=\int_{4}^{t} {{\left(x + 1\right)}^{4}}\;dx\]

\expandafter\input{\file@loc Integrals/2311-Compute-Integral-0008.HELP.tex}

\[\dfrac{d}{dt}(g(t))=\answer{{\left(t + 1\right)}^{4}}\]
\end{problem}}%}

%%%%%%%%%%%%%%%%%%%%%%




%%%%%%%%%%%%%%%%%%%%%%




%%%%%%%%%%%%%%%%%%%%%%




%%%%%%%%%%%%%%%%%%%%%%




%%%%%%%%%%%%%%%%%%%%%%


