%%%%%%%%%%%%%%%%%%%%%%%
%%\tagged{Cat@One, Cat@Two, Cat@Three, Cat@Four, Cat@Five, Ans@ShortAns, Type@Compute, Topic@Integral, Sub@Definite, Sub@Theorems_FTC, Sub@Poly}{

\latexProblemContent{
\begin{problem}

Use the Fundamental Theorem of Calculus to evaluate the integral.

\expandafter\input{\file@loc Integrals/2311-Compute-Integral-0010.HELP.tex}

\[\int_{2}^{3} {x^{3} - 3 \, x + 2}\;dx=\answer{\frac{43}{4}}\]
\end{problem}}%}

%%%%%%%%%%%%%%%%%%%%%%


\latexProblemContent{
\begin{problem}

Use the Fundamental Theorem of Calculus to evaluate the integral.

\expandafter\input{\file@loc Integrals/2311-Compute-Integral-0010.HELP.tex}

\[\int_{4}^{10} {x + 1}\;dx=\answer{48}\]
\end{problem}}%}

%%%%%%%%%%%%%%%%%%%%%%


\latexProblemContent{
\begin{problem}

Use the Fundamental Theorem of Calculus to evaluate the integral.

\expandafter\input{\file@loc Integrals/2311-Compute-Integral-0010.HELP.tex}

\[\int_{-6}^{-2} {x^{3} + 15 \, x^{2} + 75 \, x + 125}\;dx=\answer{20}\]
\end{problem}}%}

%%%%%%%%%%%%%%%%%%%%%%


\latexProblemContent{
\begin{problem}

Use the Fundamental Theorem of Calculus to evaluate the integral.

\expandafter\input{\file@loc Integrals/2311-Compute-Integral-0010.HELP.tex}

\[\int_{-2}^{4} {x^{2} - 4 \, x - 12}\;dx=\answer{-72}\]
\end{problem}}%}

%%%%%%%%%%%%%%%%%%%%%%


\latexProblemContent{
\begin{problem}

Use the Fundamental Theorem of Calculus to evaluate the integral.

\expandafter\input{\file@loc Integrals/2311-Compute-Integral-0010.HELP.tex}

\[\int_{-5}^{1} {x - 2}\;dx=\answer{-24}\]
\end{problem}}%}

%%%%%%%%%%%%%%%%%%%%%%


\latexProblemContent{
\begin{problem}

Use the Fundamental Theorem of Calculus to evaluate the integral.

\expandafter\input{\file@loc Integrals/2311-Compute-Integral-0010.HELP.tex}

\[\int_{1}^{2} {x^{2} - 3 \, x - 4}\;dx=\answer{-\frac{37}{6}}\]
\end{problem}}%}

%%%%%%%%%%%%%%%%%%%%%%


\latexProblemContent{
\begin{problem}

Use the Fundamental Theorem of Calculus to evaluate the integral.

\expandafter\input{\file@loc Integrals/2311-Compute-Integral-0010.HELP.tex}

\[\int_{5}^{7} {x^{2} + 10 \, x + 25}\;dx=\answer{\frac{728}{3}}\]
\end{problem}}%}

%%%%%%%%%%%%%%%%%%%%%%


\latexProblemContent{
\begin{problem}

Use the Fundamental Theorem of Calculus to evaluate the integral.

\expandafter\input{\file@loc Integrals/2311-Compute-Integral-0010.HELP.tex}

\[\int_{-7}^{9} {x^{2} + 10 \, x + 16}\;dx=\answer{\frac{2320}{3}}\]
\end{problem}}%}

%%%%%%%%%%%%%%%%%%%%%%


\latexProblemContent{
\begin{problem}

Use the Fundamental Theorem of Calculus to evaluate the integral.

\expandafter\input{\file@loc Integrals/2311-Compute-Integral-0010.HELP.tex}

\[\int_{-7}^{-6} {x^{3} + 8 \, x^{2} + 20 \, x + 16}\;dx=\answer{-\frac{619}{12}}\]
\end{problem}}%}

%%%%%%%%%%%%%%%%%%%%%%


\latexProblemContent{
\begin{problem}

Use the Fundamental Theorem of Calculus to evaluate the integral.

\expandafter\input{\file@loc Integrals/2311-Compute-Integral-0010.HELP.tex}

\[\int_{-6}^{8} {x^{3} - 4 \, x^{2} - 35 \, x + 150}\;dx=\answer{\frac{4018}{3}}\]
\end{problem}}%}

%%%%%%%%%%%%%%%%%%%%%%


\latexProblemContent{
\begin{problem}

Use the Fundamental Theorem of Calculus to evaluate the integral.

\expandafter\input{\file@loc Integrals/2311-Compute-Integral-0010.HELP.tex}

\[\int_{-4}^{-3} {x^{2} - 8 \, x + 16}\;dx=\answer{\frac{169}{3}}\]
\end{problem}}%}

%%%%%%%%%%%%%%%%%%%%%%


\latexProblemContent{
\begin{problem}

Use the Fundamental Theorem of Calculus to evaluate the integral.

\expandafter\input{\file@loc Integrals/2311-Compute-Integral-0010.HELP.tex}

\[\int_{-6}^{7} {x^{2} - 7 \, x + 10}\;dx=\answer{\frac{1625}{6}}\]
\end{problem}}%}

%%%%%%%%%%%%%%%%%%%%%%


\latexProblemContent{
\begin{problem}

Use the Fundamental Theorem of Calculus to evaluate the integral.

\expandafter\input{\file@loc Integrals/2311-Compute-Integral-0010.HELP.tex}

\[\int_{-5}^{-1} {x^{2} - 8 \, x + 16}\;dx=\answer{\frac{604}{3}}\]
\end{problem}}%}

%%%%%%%%%%%%%%%%%%%%%%


\latexProblemContent{
\begin{problem}

Use the Fundamental Theorem of Calculus to evaluate the integral.

\expandafter\input{\file@loc Integrals/2311-Compute-Integral-0010.HELP.tex}

\[\int_{4}^{12} {x^{3} - 6 \, x^{2} + 12 \, x - 8}\;dx=\answer{2496}\]
\end{problem}}%}

%%%%%%%%%%%%%%%%%%%%%%


\latexProblemContent{
\begin{problem}

Use the Fundamental Theorem of Calculus to evaluate the integral.

\expandafter\input{\file@loc Integrals/2311-Compute-Integral-0010.HELP.tex}

\[\int_{-10}^{-5} {x^{2} - 3 \, x - 18}\;dx=\answer{\frac{1885}{6}}\]
\end{problem}}%}

%%%%%%%%%%%%%%%%%%%%%%


\latexProblemContent{
\begin{problem}

Use the Fundamental Theorem of Calculus to evaluate the integral.

\expandafter\input{\file@loc Integrals/2311-Compute-Integral-0010.HELP.tex}

\[\int_{-6}^{-3} {x^{2} - 4 \, x - 5}\;dx=\answer{102}\]
\end{problem}}%}

%%%%%%%%%%%%%%%%%%%%%%


\latexProblemContent{
\begin{problem}

Use the Fundamental Theorem of Calculus to evaluate the integral.

\expandafter\input{\file@loc Integrals/2311-Compute-Integral-0010.HELP.tex}

\[\int_{0}^{11} {x^{2} + 8 \, x + 15}\;dx=\answer{\frac{3278}{3}}\]
\end{problem}}%}

%%%%%%%%%%%%%%%%%%%%%%


\latexProblemContent{
\begin{problem}

Use the Fundamental Theorem of Calculus to evaluate the integral.

\expandafter\input{\file@loc Integrals/2311-Compute-Integral-0010.HELP.tex}

\[\int_{5}^{9} {x^{3} - 12 \, x^{2} + 48 \, x - 64}\;dx=\answer{156}\]
\end{problem}}%}

%%%%%%%%%%%%%%%%%%%%%%


\latexProblemContent{
\begin{problem}

Use the Fundamental Theorem of Calculus to evaluate the integral.

\expandafter\input{\file@loc Integrals/2311-Compute-Integral-0010.HELP.tex}

\[\int_{-3}^{2} {x^{3} - 15 \, x^{2} + 75 \, x - 125}\;dx=\answer{-\frac{4015}{4}}\]
\end{problem}}%}

%%%%%%%%%%%%%%%%%%%%%%


\latexProblemContent{
\begin{problem}

Use the Fundamental Theorem of Calculus to evaluate the integral.

\expandafter\input{\file@loc Integrals/2311-Compute-Integral-0010.HELP.tex}

\[\int_{2}^{8} {x^{3} - 3 \, x^{2} - 9 \, x + 27}\;dx=\answer{408}\]
\end{problem}}%}

%%%%%%%%%%%%%%%%%%%%%%


\latexProblemContent{
\begin{problem}

Use the Fundamental Theorem of Calculus to evaluate the integral.

\expandafter\input{\file@loc Integrals/2311-Compute-Integral-0010.HELP.tex}

\[\int_{-1}^{5} {x^{2} + 11 \, x + 24}\;dx=\answer{318}\]
\end{problem}}%}

%%%%%%%%%%%%%%%%%%%%%%


\latexProblemContent{
\begin{problem}

Use the Fundamental Theorem of Calculus to evaluate the integral.

\expandafter\input{\file@loc Integrals/2311-Compute-Integral-0010.HELP.tex}

\[\int_{-1}^{9} {x^{2} + 8 \, x + 16}\;dx=\answer{\frac{2170}{3}}\]
\end{problem}}%}

%%%%%%%%%%%%%%%%%%%%%%


\latexProblemContent{
\begin{problem}

Use the Fundamental Theorem of Calculus to evaluate the integral.

\expandafter\input{\file@loc Integrals/2311-Compute-Integral-0010.HELP.tex}

\[\int_{-2}^{7} {x - 3}\;dx=\answer{-\frac{9}{2}}\]
\end{problem}}%}

%%%%%%%%%%%%%%%%%%%%%%


\latexProblemContent{
\begin{problem}

Use the Fundamental Theorem of Calculus to evaluate the integral.

\expandafter\input{\file@loc Integrals/2311-Compute-Integral-0010.HELP.tex}

\[\int_{1}^{4} {x^{2} + 6 \, x + 9}\;dx=\answer{93}\]
\end{problem}}%}

%%%%%%%%%%%%%%%%%%%%%%


\latexProblemContent{
\begin{problem}

Use the Fundamental Theorem of Calculus to evaluate the integral.

\expandafter\input{\file@loc Integrals/2311-Compute-Integral-0010.HELP.tex}

\[\int_{-8}^{-5} {x^{2} + 4 \, x - 32}\;dx=\answer{-45}\]
\end{problem}}%}

%%%%%%%%%%%%%%%%%%%%%%


\latexProblemContent{
\begin{problem}

Use the Fundamental Theorem of Calculus to evaluate the integral.

\expandafter\input{\file@loc Integrals/2311-Compute-Integral-0010.HELP.tex}

\[\int_{5}^{11} {x^{2} + 7 \, x + 12}\;dx=\answer{810}\]
\end{problem}}%}

%%%%%%%%%%%%%%%%%%%%%%


\latexProblemContent{
\begin{problem}

Use the Fundamental Theorem of Calculus to evaluate the integral.

\expandafter\input{\file@loc Integrals/2311-Compute-Integral-0010.HELP.tex}

\[\int_{-2}^{10} {x^{2} + 5 \, x + 4}\;dx=\answer{624}\]
\end{problem}}%}

%%%%%%%%%%%%%%%%%%%%%%


\latexProblemContent{
\begin{problem}

Use the Fundamental Theorem of Calculus to evaluate the integral.

\expandafter\input{\file@loc Integrals/2311-Compute-Integral-0010.HELP.tex}

\[\int_{3}^{8} {x^{3} + 6 \, x^{2} + 12 \, x + 8}\;dx=\answer{\frac{9375}{4}}\]
\end{problem}}%}

%%%%%%%%%%%%%%%%%%%%%%


\latexProblemContent{
\begin{problem}

Use the Fundamental Theorem of Calculus to evaluate the integral.

\expandafter\input{\file@loc Integrals/2311-Compute-Integral-0010.HELP.tex}

\[\int_{-5}^{-3} {x^{3} - 15 \, x^{2} + 75 \, x - 125}\;dx=\answer{-1476}\]
\end{problem}}%}

%%%%%%%%%%%%%%%%%%%%%%


\latexProblemContent{
\begin{problem}

Use the Fundamental Theorem of Calculus to evaluate the integral.

\expandafter\input{\file@loc Integrals/2311-Compute-Integral-0010.HELP.tex}

\[\int_{-5}^{7} {x^{2} - 2 \, x + 1}\;dx=\answer{144}\]
\end{problem}}%}

%%%%%%%%%%%%%%%%%%%%%%


\latexProblemContent{
\begin{problem}

Use the Fundamental Theorem of Calculus to evaluate the integral.

\expandafter\input{\file@loc Integrals/2311-Compute-Integral-0010.HELP.tex}

\[\int_{-1}^{7} {x^{2} - 2 \, x + 1}\;dx=\answer{\frac{224}{3}}\]
\end{problem}}%}

%%%%%%%%%%%%%%%%%%%%%%


\latexProblemContent{
\begin{problem}

Use the Fundamental Theorem of Calculus to evaluate the integral.

\expandafter\input{\file@loc Integrals/2311-Compute-Integral-0010.HELP.tex}

\[\int_{-8}^{8} {x^{2} + 8 \, x + 16}\;dx=\answer{\frac{1792}{3}}\]
\end{problem}}%}

%%%%%%%%%%%%%%%%%%%%%%


\latexProblemContent{
\begin{problem}

Use the Fundamental Theorem of Calculus to evaluate the integral.

\expandafter\input{\file@loc Integrals/2311-Compute-Integral-0010.HELP.tex}

\[\int_{-3}^{5} {x^{2} + 8 \, x + 16}\;dx=\answer{\frac{728}{3}}\]
\end{problem}}%}

%%%%%%%%%%%%%%%%%%%%%%


\latexProblemContent{
\begin{problem}

Use the Fundamental Theorem of Calculus to evaluate the integral.

\expandafter\input{\file@loc Integrals/2311-Compute-Integral-0010.HELP.tex}

\[\int_{-7}^{4} {x^{3} - 12 \, x^{2} + 48 \, x - 64}\;dx=\answer{-\frac{14641}{4}}\]
\end{problem}}%}

%%%%%%%%%%%%%%%%%%%%%%


\latexProblemContent{
\begin{problem}

Use the Fundamental Theorem of Calculus to evaluate the integral.

\expandafter\input{\file@loc Integrals/2311-Compute-Integral-0010.HELP.tex}

\[\int_{0}^{1} {x^{3} + 15 \, x^{2} + 75 \, x + 125}\;dx=\answer{\frac{671}{4}}\]
\end{problem}}%}

%%%%%%%%%%%%%%%%%%%%%%


\latexProblemContent{
\begin{problem}

Use the Fundamental Theorem of Calculus to evaluate the integral.

\expandafter\input{\file@loc Integrals/2311-Compute-Integral-0010.HELP.tex}

\[\int_{-2}^{10} {x^{2} - 2 \, x + 1}\;dx=\answer{252}\]
\end{problem}}%}

%%%%%%%%%%%%%%%%%%%%%%


\latexProblemContent{
\begin{problem}

Use the Fundamental Theorem of Calculus to evaluate the integral.

\expandafter\input{\file@loc Integrals/2311-Compute-Integral-0010.HELP.tex}

\[\int_{4}^{6} {x^{2} + 8 \, x + 12}\;dx=\answer{\frac{464}{3}}\]
\end{problem}}%}

%%%%%%%%%%%%%%%%%%%%%%


\latexProblemContent{
\begin{problem}

Use the Fundamental Theorem of Calculus to evaluate the integral.

\expandafter\input{\file@loc Integrals/2311-Compute-Integral-0010.HELP.tex}

\[\int_{-1}^{11} {x - 2}\;dx=\answer{36}\]
\end{problem}}%}

%%%%%%%%%%%%%%%%%%%%%%


\latexProblemContent{
\begin{problem}

Use the Fundamental Theorem of Calculus to evaluate the integral.

\expandafter\input{\file@loc Integrals/2311-Compute-Integral-0010.HELP.tex}

\[\int_{0}^{3} {x - 3}\;dx=\answer{-\frac{9}{2}}\]
\end{problem}}%}

%%%%%%%%%%%%%%%%%%%%%%


\latexProblemContent{
\begin{problem}

Use the Fundamental Theorem of Calculus to evaluate the integral.

\expandafter\input{\file@loc Integrals/2311-Compute-Integral-0010.HELP.tex}

\[\int_{3}^{4} {x + 2}\;dx=\answer{\frac{11}{2}}\]
\end{problem}}%}

%%%%%%%%%%%%%%%%%%%%%%


\latexProblemContent{
\begin{problem}

Use the Fundamental Theorem of Calculus to evaluate the integral.

\expandafter\input{\file@loc Integrals/2311-Compute-Integral-0010.HELP.tex}

\[\int_{-9}^{-2} {x^{2} + 2 \, x + 1}\;dx=\answer{\frac{511}{3}}\]
\end{problem}}%}

%%%%%%%%%%%%%%%%%%%%%%


\latexProblemContent{
\begin{problem}

Use the Fundamental Theorem of Calculus to evaluate the integral.

\expandafter\input{\file@loc Integrals/2311-Compute-Integral-0010.HELP.tex}

\[\int_{2}^{7} {x^{2} - x - 30}\;dx=\answer{-\frac{365}{6}}\]
\end{problem}}%}

%%%%%%%%%%%%%%%%%%%%%%


\latexProblemContent{
\begin{problem}

Use the Fundamental Theorem of Calculus to evaluate the integral.

\expandafter\input{\file@loc Integrals/2311-Compute-Integral-0010.HELP.tex}

\[\int_{0}^{1} {x^{2} + 10 \, x + 25}\;dx=\answer{\frac{91}{3}}\]
\end{problem}}%}

%%%%%%%%%%%%%%%%%%%%%%


\latexProblemContent{
\begin{problem}

Use the Fundamental Theorem of Calculus to evaluate the integral.

\expandafter\input{\file@loc Integrals/2311-Compute-Integral-0010.HELP.tex}

\[\int_{4}^{10} {x^{3} - 3 \, x^{2} + 3 \, x - 1}\;dx=\answer{1620}\]
\end{problem}}%}

%%%%%%%%%%%%%%%%%%%%%%


\latexProblemContent{
\begin{problem}

Use the Fundamental Theorem of Calculus to evaluate the integral.

\expandafter\input{\file@loc Integrals/2311-Compute-Integral-0010.HELP.tex}

\[\int_{-8}^{9} {x^{3} + 6 \, x^{2} + 12 \, x + 8}\;dx=\answer{\frac{13345}{4}}\]
\end{problem}}%}

%%%%%%%%%%%%%%%%%%%%%%


\latexProblemContent{
\begin{problem}

Use the Fundamental Theorem of Calculus to evaluate the integral.

\expandafter\input{\file@loc Integrals/2311-Compute-Integral-0010.HELP.tex}

\[\int_{1}^{10} {x^{3} + 6 \, x^{2} + 12 \, x + 8}\;dx=\answer{\frac{20655}{4}}\]
\end{problem}}%}

%%%%%%%%%%%%%%%%%%%%%%


\latexProblemContent{
\begin{problem}

Use the Fundamental Theorem of Calculus to evaluate the integral.

\expandafter\input{\file@loc Integrals/2311-Compute-Integral-0010.HELP.tex}

\[\int_{-8}^{1} {x^{2} + 9 \, x + 14}\;dx=\answer{\frac{27}{2}}\]
\end{problem}}%}

%%%%%%%%%%%%%%%%%%%%%%


\latexProblemContent{
\begin{problem}

Use the Fundamental Theorem of Calculus to evaluate the integral.

\expandafter\input{\file@loc Integrals/2311-Compute-Integral-0010.HELP.tex}

\[\int_{-2}^{3} {x^{3} + 15 \, x^{2} + 75 \, x + 125}\;dx=\answer{\frac{4015}{4}}\]
\end{problem}}%}

%%%%%%%%%%%%%%%%%%%%%%


\latexProblemContent{
\begin{problem}

Use the Fundamental Theorem of Calculus to evaluate the integral.

\expandafter\input{\file@loc Integrals/2311-Compute-Integral-0010.HELP.tex}

\[\int_{0}^{6} {x^{2} - 2 \, x - 24}\;dx=\answer{-108}\]
\end{problem}}%}

%%%%%%%%%%%%%%%%%%%%%%


\latexProblemContent{
\begin{problem}

Use the Fundamental Theorem of Calculus to evaluate the integral.

\expandafter\input{\file@loc Integrals/2311-Compute-Integral-0010.HELP.tex}

\[\int_{4}^{11} {x^{3} + 9 \, x^{2} + 27 \, x + 27}\;dx=\answer{\frac{36015}{4}}\]
\end{problem}}%}

%%%%%%%%%%%%%%%%%%%%%%


\latexProblemContent{
\begin{problem}

Use the Fundamental Theorem of Calculus to evaluate the integral.

\expandafter\input{\file@loc Integrals/2311-Compute-Integral-0010.HELP.tex}

\[\int_{-1}^{9} {x^{3} - 14 \, x^{2} + 64 \, x - 96}\;dx=\answer{-\frac{500}{3}}\]
\end{problem}}%}

%%%%%%%%%%%%%%%%%%%%%%


\latexProblemContent{
\begin{problem}

Use the Fundamental Theorem of Calculus to evaluate the integral.

\expandafter\input{\file@loc Integrals/2311-Compute-Integral-0010.HELP.tex}

\[\int_{-9}^{-2} {x^{3} + 3 \, x^{2} - 24 \, x - 80}\;dx=\answer{-\frac{2205}{4}}\]
\end{problem}}%}

%%%%%%%%%%%%%%%%%%%%%%


\latexProblemContent{
\begin{problem}

Use the Fundamental Theorem of Calculus to evaluate the integral.

\expandafter\input{\file@loc Integrals/2311-Compute-Integral-0010.HELP.tex}

\[\int_{-4}^{3} {x^{3} + 11 \, x^{2} + 32 \, x + 28}\;dx=\answer{\frac{4487}{12}}\]
\end{problem}}%}

%%%%%%%%%%%%%%%%%%%%%%


\latexProblemContent{
\begin{problem}

Use the Fundamental Theorem of Calculus to evaluate the integral.

\expandafter\input{\file@loc Integrals/2311-Compute-Integral-0010.HELP.tex}

\[\int_{1}^{3} {x^{3} + 9 \, x^{2} + 15 \, x + 7}\;dx=\answer{172}\]
\end{problem}}%}

%%%%%%%%%%%%%%%%%%%%%%


\latexProblemContent{
\begin{problem}

Use the Fundamental Theorem of Calculus to evaluate the integral.

\expandafter\input{\file@loc Integrals/2311-Compute-Integral-0010.HELP.tex}

\[\int_{-3}^{11} {x + 5}\;dx=\answer{126}\]
\end{problem}}%}

%%%%%%%%%%%%%%%%%%%%%%


\latexProblemContent{
\begin{problem}

Use the Fundamental Theorem of Calculus to evaluate the integral.

\expandafter\input{\file@loc Integrals/2311-Compute-Integral-0010.HELP.tex}

\[\int_{-1}^{-1} {x^{2} + 4 \, x - 5}\;dx=\answer{0}\]
\end{problem}}%}

%%%%%%%%%%%%%%%%%%%%%%


\latexProblemContent{
\begin{problem}

Use the Fundamental Theorem of Calculus to evaluate the integral.

\expandafter\input{\file@loc Integrals/2311-Compute-Integral-0010.HELP.tex}

\[\int_{3}^{8} {x^{2} - 10 \, x + 25}\;dx=\answer{\frac{35}{3}}\]
\end{problem}}%}

%%%%%%%%%%%%%%%%%%%%%%


\latexProblemContent{
\begin{problem}

Use the Fundamental Theorem of Calculus to evaluate the integral.

\expandafter\input{\file@loc Integrals/2311-Compute-Integral-0010.HELP.tex}

\[\int_{-9}^{-8} {x^{2} + 10 \, x + 25}\;dx=\answer{\frac{37}{3}}\]
\end{problem}}%}

%%%%%%%%%%%%%%%%%%%%%%


\latexProblemContent{
\begin{problem}

Use the Fundamental Theorem of Calculus to evaluate the integral.

\expandafter\input{\file@loc Integrals/2311-Compute-Integral-0010.HELP.tex}

\[\int_{-3}^{5} {x^{3} + 12 \, x^{2} + 48 \, x + 64}\;dx=\answer{1640}\]
\end{problem}}%}

%%%%%%%%%%%%%%%%%%%%%%


\latexProblemContent{
\begin{problem}

Use the Fundamental Theorem of Calculus to evaluate the integral.

\expandafter\input{\file@loc Integrals/2311-Compute-Integral-0010.HELP.tex}

\[\int_{-7}^{-6} {x^{3} + 4 \, x^{2} + 5 \, x + 2}\;dx=\answer{-\frac{1649}{12}}\]
\end{problem}}%}

%%%%%%%%%%%%%%%%%%%%%%


\latexProblemContent{
\begin{problem}

Use the Fundamental Theorem of Calculus to evaluate the integral.

\expandafter\input{\file@loc Integrals/2311-Compute-Integral-0010.HELP.tex}

\[\int_{-6}^{10} {x^{2} - 10 \, x + 25}\;dx=\answer{\frac{1456}{3}}\]
\end{problem}}%}

%%%%%%%%%%%%%%%%%%%%%%


\latexProblemContent{
\begin{problem}

Use the Fundamental Theorem of Calculus to evaluate the integral.

\expandafter\input{\file@loc Integrals/2311-Compute-Integral-0010.HELP.tex}

\[\int_{-8}^{6} {x - 2}\;dx=\answer{-42}\]
\end{problem}}%}

%%%%%%%%%%%%%%%%%%%%%%


\latexProblemContent{
\begin{problem}

Use the Fundamental Theorem of Calculus to evaluate the integral.

\expandafter\input{\file@loc Integrals/2311-Compute-Integral-0010.HELP.tex}

\[\int_{-5}^{8} {x^{2} - 2 \, x + 1}\;dx=\answer{\frac{559}{3}}\]
\end{problem}}%}

%%%%%%%%%%%%%%%%%%%%%%


\latexProblemContent{
\begin{problem}

Use the Fundamental Theorem of Calculus to evaluate the integral.

\expandafter\input{\file@loc Integrals/2311-Compute-Integral-0010.HELP.tex}

\[\int_{-7}^{-4} {x^{2} - 8 \, x + 16}\;dx=\answer{273}\]
\end{problem}}%}

%%%%%%%%%%%%%%%%%%%%%%


\latexProblemContent{
\begin{problem}

Use the Fundamental Theorem of Calculus to evaluate the integral.

\expandafter\input{\file@loc Integrals/2311-Compute-Integral-0010.HELP.tex}

\[\int_{-1}^{12} {x^{3} - 2 \, x^{2} - 32 \, x + 96}\;dx=\answer{\frac{35893}{12}}\]
\end{problem}}%}

%%%%%%%%%%%%%%%%%%%%%%


\latexProblemContent{
\begin{problem}

Use the Fundamental Theorem of Calculus to evaluate the integral.

\expandafter\input{\file@loc Integrals/2311-Compute-Integral-0010.HELP.tex}

\[\int_{3}^{10} {x^{3} + 2 \, x^{2} - 20 \, x + 24}\;dx=\answer{\frac{28637}{12}}\]
\end{problem}}%}

%%%%%%%%%%%%%%%%%%%%%%


\latexProblemContent{
\begin{problem}

Use the Fundamental Theorem of Calculus to evaluate the integral.

\expandafter\input{\file@loc Integrals/2311-Compute-Integral-0010.HELP.tex}

\[\int_{4}^{7} {x^{2} + 6 \, x + 9}\;dx=\answer{219}\]
\end{problem}}%}

%%%%%%%%%%%%%%%%%%%%%%


\latexProblemContent{
\begin{problem}

Use the Fundamental Theorem of Calculus to evaluate the integral.

\expandafter\input{\file@loc Integrals/2311-Compute-Integral-0010.HELP.tex}

\[\int_{3}^{3} {x^{3} + 9 \, x^{2} + 27 \, x + 27}\;dx=\answer{0}\]
\end{problem}}%}

%%%%%%%%%%%%%%%%%%%%%%


\latexProblemContent{
\begin{problem}

Use the Fundamental Theorem of Calculus to evaluate the integral.

\expandafter\input{\file@loc Integrals/2311-Compute-Integral-0010.HELP.tex}

\[\int_{-1}^{11} {x^{2} + 6 \, x + 9}\;dx=\answer{912}\]
\end{problem}}%}

%%%%%%%%%%%%%%%%%%%%%%


\latexProblemContent{
\begin{problem}

Use the Fundamental Theorem of Calculus to evaluate the integral.

\expandafter\input{\file@loc Integrals/2311-Compute-Integral-0010.HELP.tex}

\[\int_{5}^{8} {x - 1}\;dx=\answer{\frac{33}{2}}\]
\end{problem}}%}

%%%%%%%%%%%%%%%%%%%%%%


\latexProblemContent{
\begin{problem}

Use the Fundamental Theorem of Calculus to evaluate the integral.

\expandafter\input{\file@loc Integrals/2311-Compute-Integral-0010.HELP.tex}

\[\int_{1}^{10} {x^{2} + 2 \, x + 1}\;dx=\answer{441}\]
\end{problem}}%}

%%%%%%%%%%%%%%%%%%%%%%


\latexProblemContent{
\begin{problem}

Use the Fundamental Theorem of Calculus to evaluate the integral.

\expandafter\input{\file@loc Integrals/2311-Compute-Integral-0010.HELP.tex}

\[\int_{-7}^{2} {x^{2} - 6 \, x + 9}\;dx=\answer{333}\]
\end{problem}}%}

%%%%%%%%%%%%%%%%%%%%%%


\latexProblemContent{
\begin{problem}

Use the Fundamental Theorem of Calculus to evaluate the integral.

\expandafter\input{\file@loc Integrals/2311-Compute-Integral-0010.HELP.tex}

\[\int_{1}^{11} {x^{3} + 12 \, x^{2} + 48 \, x + 64}\;dx=\answer{12500}\]
\end{problem}}%}

%%%%%%%%%%%%%%%%%%%%%%


\latexProblemContent{
\begin{problem}

Use the Fundamental Theorem of Calculus to evaluate the integral.

\expandafter\input{\file@loc Integrals/2311-Compute-Integral-0010.HELP.tex}

\[\int_{-9}^{12} {x^{3} + 4 \, x^{2} - 35 \, x - 150}\;dx=\answer{\frac{10269}{4}}\]
\end{problem}}%}

%%%%%%%%%%%%%%%%%%%%%%


\latexProblemContent{
\begin{problem}

Use the Fundamental Theorem of Calculus to evaluate the integral.

\expandafter\input{\file@loc Integrals/2311-Compute-Integral-0010.HELP.tex}

\[\int_{1}^{9} {x^{2} + 10 \, x + 25}\;dx=\answer{\frac{2528}{3}}\]
\end{problem}}%}

%%%%%%%%%%%%%%%%%%%%%%


\latexProblemContent{
\begin{problem}

Use the Fundamental Theorem of Calculus to evaluate the integral.

\expandafter\input{\file@loc Integrals/2311-Compute-Integral-0010.HELP.tex}

\[\int_{-4}^{1} {x + 2}\;dx=\answer{\frac{5}{2}}\]
\end{problem}}%}

%%%%%%%%%%%%%%%%%%%%%%


\latexProblemContent{
\begin{problem}

Use the Fundamental Theorem of Calculus to evaluate the integral.

\expandafter\input{\file@loc Integrals/2311-Compute-Integral-0010.HELP.tex}

\[\int_{-1}^{6} {x^{3} + 9 \, x^{2} + 27 \, x + 27}\;dx=\answer{\frac{6545}{4}}\]
\end{problem}}%}

%%%%%%%%%%%%%%%%%%%%%%


\latexProblemContent{
\begin{problem}

Use the Fundamental Theorem of Calculus to evaluate the integral.

\expandafter\input{\file@loc Integrals/2311-Compute-Integral-0010.HELP.tex}

\[\int_{1}^{8} {x^{2} - 9 \, x + 20}\;dx=\answer{\frac{161}{6}}\]
\end{problem}}%}

%%%%%%%%%%%%%%%%%%%%%%


\latexProblemContent{
\begin{problem}

Use the Fundamental Theorem of Calculus to evaluate the integral.

\expandafter\input{\file@loc Integrals/2311-Compute-Integral-0010.HELP.tex}

\[\int_{-4}^{-1} {x + 1}\;dx=\answer{-\frac{9}{2}}\]
\end{problem}}%}

%%%%%%%%%%%%%%%%%%%%%%


\latexProblemContent{
\begin{problem}

Use the Fundamental Theorem of Calculus to evaluate the integral.

\expandafter\input{\file@loc Integrals/2311-Compute-Integral-0010.HELP.tex}

\[\int_{2}^{11} {x^{3} - 3 \, x^{2} - 9 \, x + 27}\;dx=\answer{\frac{8199}{4}}\]
\end{problem}}%}

%%%%%%%%%%%%%%%%%%%%%%


\latexProblemContent{
\begin{problem}

Use the Fundamental Theorem of Calculus to evaluate the integral.

\expandafter\input{\file@loc Integrals/2311-Compute-Integral-0010.HELP.tex}

\[\int_{-8}^{6} {x^{2} + 5 \, x - 24}\;dx=\answer{-\frac{490}{3}}\]
\end{problem}}%}

%%%%%%%%%%%%%%%%%%%%%%


\latexProblemContent{
\begin{problem}

Use the Fundamental Theorem of Calculus to evaluate the integral.

\expandafter\input{\file@loc Integrals/2311-Compute-Integral-0010.HELP.tex}

\[\int_{4}^{12} {x^{2} - 2 \, x - 15}\;dx=\answer{\frac{920}{3}}\]
\end{problem}}%}

%%%%%%%%%%%%%%%%%%%%%%


\latexProblemContent{
\begin{problem}

Use the Fundamental Theorem of Calculus to evaluate the integral.

\expandafter\input{\file@loc Integrals/2311-Compute-Integral-0010.HELP.tex}

\[\int_{0}^{11} {x^{2} + 8 \, x + 16}\;dx=\answer{\frac{3311}{3}}\]
\end{problem}}%}

%%%%%%%%%%%%%%%%%%%%%%


\latexProblemContent{
\begin{problem}

Use the Fundamental Theorem of Calculus to evaluate the integral.

\expandafter\input{\file@loc Integrals/2311-Compute-Integral-0010.HELP.tex}

\[\int_{-8}^{-8} {x - 1}\;dx=\answer{0}\]
\end{problem}}%}

%%%%%%%%%%%%%%%%%%%%%%


\latexProblemContent{
\begin{problem}

Use the Fundamental Theorem of Calculus to evaluate the integral.

\expandafter\input{\file@loc Integrals/2311-Compute-Integral-0010.HELP.tex}

\[\int_{-9}^{7} {x^{2} + 6 \, x + 9}\;dx=\answer{\frac{1216}{3}}\]
\end{problem}}%}

%%%%%%%%%%%%%%%%%%%%%%


\latexProblemContent{
\begin{problem}

Use the Fundamental Theorem of Calculus to evaluate the integral.

\expandafter\input{\file@loc Integrals/2311-Compute-Integral-0010.HELP.tex}

\[\int_{-4}^{8} {x^{3} - 15 \, x^{2} + 75 \, x - 125}\;dx=\answer{-1620}\]
\end{problem}}%}

%%%%%%%%%%%%%%%%%%%%%%


\latexProblemContent{
\begin{problem}

Use the Fundamental Theorem of Calculus to evaluate the integral.

\expandafter\input{\file@loc Integrals/2311-Compute-Integral-0010.HELP.tex}

\[\int_{-1}^{10} {x^{2} - 7 \, x + 10}\;dx=\answer{\frac{583}{6}}\]
\end{problem}}%}

%%%%%%%%%%%%%%%%%%%%%%


\latexProblemContent{
\begin{problem}

Use the Fundamental Theorem of Calculus to evaluate the integral.

\expandafter\input{\file@loc Integrals/2311-Compute-Integral-0010.HELP.tex}

\[\int_{-5}^{0} {x + 3}\;dx=\answer{\frac{5}{2}}\]
\end{problem}}%}

%%%%%%%%%%%%%%%%%%%%%%


\latexProblemContent{
\begin{problem}

Use the Fundamental Theorem of Calculus to evaluate the integral.

\expandafter\input{\file@loc Integrals/2311-Compute-Integral-0010.HELP.tex}

\[\int_{-6}^{10} {x^{3} + x^{2} - 21 \, x - 45}\;dx=\answer{\frac{3568}{3}}\]
\end{problem}}%}

%%%%%%%%%%%%%%%%%%%%%%


\latexProblemContent{
\begin{problem}

Use the Fundamental Theorem of Calculus to evaluate the integral.

\expandafter\input{\file@loc Integrals/2311-Compute-Integral-0010.HELP.tex}

\[\int_{-3}^{2} {x^{2} - 9 \, x + 20}\;dx=\answer{\frac{805}{6}}\]
\end{problem}}%}

%%%%%%%%%%%%%%%%%%%%%%


\latexProblemContent{
\begin{problem}

Use the Fundamental Theorem of Calculus to evaluate the integral.

\expandafter\input{\file@loc Integrals/2311-Compute-Integral-0010.HELP.tex}

\[\int_{-1}^{3} {x + 5}\;dx=\answer{24}\]
\end{problem}}%}

%%%%%%%%%%%%%%%%%%%%%%


\latexProblemContent{
\begin{problem}

Use the Fundamental Theorem of Calculus to evaluate the integral.

\expandafter\input{\file@loc Integrals/2311-Compute-Integral-0010.HELP.tex}

\[\int_{3}^{7} {x - 3}\;dx=\answer{8}\]
\end{problem}}%}

%%%%%%%%%%%%%%%%%%%%%%


\latexProblemContent{
\begin{problem}

Use the Fundamental Theorem of Calculus to evaluate the integral.

\expandafter\input{\file@loc Integrals/2311-Compute-Integral-0010.HELP.tex}

\[\int_{-6}^{-5} {x^{3} + x^{2} - 16 \, x + 20}\;dx=\answer{-\frac{353}{12}}\]
\end{problem}}%}

%%%%%%%%%%%%%%%%%%%%%%


\latexProblemContent{
\begin{problem}

Use the Fundamental Theorem of Calculus to evaluate the integral.

\expandafter\input{\file@loc Integrals/2311-Compute-Integral-0010.HELP.tex}

\[\int_{2}^{6} {x + 4}\;dx=\answer{32}\]
\end{problem}}%}

%%%%%%%%%%%%%%%%%%%%%%


\latexProblemContent{
\begin{problem}

Use the Fundamental Theorem of Calculus to evaluate the integral.

\expandafter\input{\file@loc Integrals/2311-Compute-Integral-0010.HELP.tex}

\[\int_{3}^{9} {x^{3} + 7 \, x^{2} - 5 \, x - 75}\;dx=\answer{2628}\]
\end{problem}}%}

%%%%%%%%%%%%%%%%%%%%%%


\latexProblemContent{
\begin{problem}

Use the Fundamental Theorem of Calculus to evaluate the integral.

\expandafter\input{\file@loc Integrals/2311-Compute-Integral-0010.HELP.tex}

\[\int_{1}^{4} {x^{3} + 8 \, x^{2} + 16 \, x}\;dx=\answer{\frac{1407}{4}}\]
\end{problem}}%}

%%%%%%%%%%%%%%%%%%%%%%


\latexProblemContent{
\begin{problem}

Use the Fundamental Theorem of Calculus to evaluate the integral.

\expandafter\input{\file@loc Integrals/2311-Compute-Integral-0010.HELP.tex}

\[\int_{4}^{5} {x^{3} - x^{2} - 16 \, x - 20}\;dx=\answer{-\frac{241}{12}}\]
\end{problem}}%}

%%%%%%%%%%%%%%%%%%%%%%


\latexProblemContent{
\begin{problem}

Use the Fundamental Theorem of Calculus to evaluate the integral.

\expandafter\input{\file@loc Integrals/2311-Compute-Integral-0010.HELP.tex}

\[\int_{4}^{12} {x^{3} + 2 \, x^{2} - 15 \, x - 36}\;dx=\answer{\frac{14944}{3}}\]
\end{problem}}%}

%%%%%%%%%%%%%%%%%%%%%%


\latexProblemContent{
\begin{problem}

Use the Fundamental Theorem of Calculus to evaluate the integral.

\expandafter\input{\file@loc Integrals/2311-Compute-Integral-0010.HELP.tex}

\[\int_{0}^{1} {x + 4}\;dx=\answer{\frac{9}{2}}\]
\end{problem}}%}

%%%%%%%%%%%%%%%%%%%%%%


\latexProblemContent{
\begin{problem}

Use the Fundamental Theorem of Calculus to evaluate the integral.

\expandafter\input{\file@loc Integrals/2311-Compute-Integral-0010.HELP.tex}

\[\int_{-3}^{8} {x^{2} + 2 \, x - 15}\;dx=\answer{\frac{209}{3}}\]
\end{problem}}%}

%%%%%%%%%%%%%%%%%%%%%%


\latexProblemContent{
\begin{problem}

Use the Fundamental Theorem of Calculus to evaluate the integral.

\expandafter\input{\file@loc Integrals/2311-Compute-Integral-0010.HELP.tex}

\[\int_{4}^{11} {x + 5}\;dx=\answer{\frac{175}{2}}\]
\end{problem}}%}

%%%%%%%%%%%%%%%%%%%%%%


\latexProblemContent{
\begin{problem}

Use the Fundamental Theorem of Calculus to evaluate the integral.

\expandafter\input{\file@loc Integrals/2311-Compute-Integral-0010.HELP.tex}

\[\int_{-7}^{3} {x^{2} + 6 \, x + 9}\;dx=\answer{\frac{280}{3}}\]
\end{problem}}%}

%%%%%%%%%%%%%%%%%%%%%%


\latexProblemContent{
\begin{problem}

Use the Fundamental Theorem of Calculus to evaluate the integral.

\expandafter\input{\file@loc Integrals/2311-Compute-Integral-0010.HELP.tex}

\[\int_{2}^{6} {x^{2} - 9 \, x + 8}\;dx=\answer{-\frac{128}{3}}\]
\end{problem}}%}

%%%%%%%%%%%%%%%%%%%%%%


\latexProblemContent{
\begin{problem}

Use the Fundamental Theorem of Calculus to evaluate the integral.

\expandafter\input{\file@loc Integrals/2311-Compute-Integral-0010.HELP.tex}

\[\int_{-9}^{10} {x^{3} + 12 \, x^{2} + 48 \, x + 64}\;dx=\answer{\frac{37791}{4}}\]
\end{problem}}%}

%%%%%%%%%%%%%%%%%%%%%%


\latexProblemContent{
\begin{problem}

Use the Fundamental Theorem of Calculus to evaluate the integral.

\expandafter\input{\file@loc Integrals/2311-Compute-Integral-0010.HELP.tex}

\[\int_{1}^{12} {x^{3} - 6 \, x^{2} + 12 \, x - 8}\;dx=\answer{\frac{9999}{4}}\]
\end{problem}}%}

%%%%%%%%%%%%%%%%%%%%%%


\latexProblemContent{
\begin{problem}

Use the Fundamental Theorem of Calculus to evaluate the integral.

\expandafter\input{\file@loc Integrals/2311-Compute-Integral-0010.HELP.tex}

\[\int_{-1}^{6} {x^{3} + 3 \, x^{2} + 3 \, x + 1}\;dx=\answer{\frac{2401}{4}}\]
\end{problem}}%}

%%%%%%%%%%%%%%%%%%%%%%


\latexProblemContent{
\begin{problem}

Use the Fundamental Theorem of Calculus to evaluate the integral.

\expandafter\input{\file@loc Integrals/2311-Compute-Integral-0010.HELP.tex}

\[\int_{5}^{5} {x - 2}\;dx=\answer{0}\]
\end{problem}}%}

%%%%%%%%%%%%%%%%%%%%%%


\latexProblemContent{
\begin{problem}

Use the Fundamental Theorem of Calculus to evaluate the integral.

\expandafter\input{\file@loc Integrals/2311-Compute-Integral-0010.HELP.tex}

\[\int_{-10}^{3} {x^{2} - 5 \, x - 14}\;dx=\answer{\frac{2327}{6}}\]
\end{problem}}%}

%%%%%%%%%%%%%%%%%%%%%%


\latexProblemContent{
\begin{problem}

Use the Fundamental Theorem of Calculus to evaluate the integral.

\expandafter\input{\file@loc Integrals/2311-Compute-Integral-0010.HELP.tex}

\[\int_{-9}^{-6} {x^{2} + 10 \, x + 25}\;dx=\answer{21}\]
\end{problem}}%}

%%%%%%%%%%%%%%%%%%%%%%


\latexProblemContent{
\begin{problem}

Use the Fundamental Theorem of Calculus to evaluate the integral.

\expandafter\input{\file@loc Integrals/2311-Compute-Integral-0010.HELP.tex}

\[\int_{3}^{9} {x + 4}\;dx=\answer{60}\]
\end{problem}}%}

%%%%%%%%%%%%%%%%%%%%%%


\latexProblemContent{
\begin{problem}

Use the Fundamental Theorem of Calculus to evaluate the integral.

\expandafter\input{\file@loc Integrals/2311-Compute-Integral-0010.HELP.tex}

\[\int_{-7}^{6} {x^{2} + 8 \, x + 16}\;dx=\answer{\frac{1027}{3}}\]
\end{problem}}%}

%%%%%%%%%%%%%%%%%%%%%%


\latexProblemContent{
\begin{problem}

Use the Fundamental Theorem of Calculus to evaluate the integral.

\expandafter\input{\file@loc Integrals/2311-Compute-Integral-0010.HELP.tex}

\[\int_{-10}^{12} {x^{2} + 8 \, x + 16}\;dx=\answer{\frac{4312}{3}}\]
\end{problem}}%}

%%%%%%%%%%%%%%%%%%%%%%


\latexProblemContent{
\begin{problem}

Use the Fundamental Theorem of Calculus to evaluate the integral.

\expandafter\input{\file@loc Integrals/2311-Compute-Integral-0010.HELP.tex}

\[\int_{3}^{10} {x^{3} - 15 \, x^{2} + 75 \, x - 125}\;dx=\answer{\frac{609}{4}}\]
\end{problem}}%}

%%%%%%%%%%%%%%%%%%%%%%


\latexProblemContent{
\begin{problem}

Use the Fundamental Theorem of Calculus to evaluate the integral.

\expandafter\input{\file@loc Integrals/2311-Compute-Integral-0010.HELP.tex}

\[\int_{-10}^{8} {x^{3} + 3 \, x^{2} + 3 \, x + 1}\;dx=\answer{0}\]
\end{problem}}%}

%%%%%%%%%%%%%%%%%%%%%%


\latexProblemContent{
\begin{problem}

Use the Fundamental Theorem of Calculus to evaluate the integral.

\expandafter\input{\file@loc Integrals/2311-Compute-Integral-0010.HELP.tex}

\[\int_{3}^{12} {x + 2}\;dx=\answer{\frac{171}{2}}\]
\end{problem}}%}

%%%%%%%%%%%%%%%%%%%%%%


\latexProblemContent{
\begin{problem}

Use the Fundamental Theorem of Calculus to evaluate the integral.

\expandafter\input{\file@loc Integrals/2311-Compute-Integral-0010.HELP.tex}

\[\int_{1}^{10} {x^{2} + 2 \, x - 3}\;dx=\answer{405}\]
\end{problem}}%}

%%%%%%%%%%%%%%%%%%%%%%


\latexProblemContent{
\begin{problem}

Use the Fundamental Theorem of Calculus to evaluate the integral.

\expandafter\input{\file@loc Integrals/2311-Compute-Integral-0010.HELP.tex}

\[\int_{-3}^{-2} {x^{3} + 3 \, x^{2} - 45 \, x - 175}\;dx=\answer{-\frac{239}{4}}\]
\end{problem}}%}

%%%%%%%%%%%%%%%%%%%%%%


\latexProblemContent{
\begin{problem}

Use the Fundamental Theorem of Calculus to evaluate the integral.

\expandafter\input{\file@loc Integrals/2311-Compute-Integral-0010.HELP.tex}

\[\int_{-5}^{4} {x^{2} - 9 \, x + 18}\;dx=\answer{\frac{531}{2}}\]
\end{problem}}%}

%%%%%%%%%%%%%%%%%%%%%%


\latexProblemContent{
\begin{problem}

Use the Fundamental Theorem of Calculus to evaluate the integral.

\expandafter\input{\file@loc Integrals/2311-Compute-Integral-0010.HELP.tex}

\[\int_{-8}^{8} {x^{2} - 2 \, x + 1}\;dx=\answer{\frac{1072}{3}}\]
\end{problem}}%}

%%%%%%%%%%%%%%%%%%%%%%


\latexProblemContent{
\begin{problem}

Use the Fundamental Theorem of Calculus to evaluate the integral.

\expandafter\input{\file@loc Integrals/2311-Compute-Integral-0010.HELP.tex}

\[\int_{-3}^{-3} {x + 3}\;dx=\answer{0}\]
\end{problem}}%}

%%%%%%%%%%%%%%%%%%%%%%


\latexProblemContent{
\begin{problem}

Use the Fundamental Theorem of Calculus to evaluate the integral.

\expandafter\input{\file@loc Integrals/2311-Compute-Integral-0010.HELP.tex}

\[\int_{-8}^{4} {x^{2} - 7 \, x - 8}\;dx=\answer{264}\]
\end{problem}}%}

%%%%%%%%%%%%%%%%%%%%%%


\latexProblemContent{
\begin{problem}

Use the Fundamental Theorem of Calculus to evaluate the integral.

\expandafter\input{\file@loc Integrals/2311-Compute-Integral-0010.HELP.tex}

\[\int_{-7}^{6} {x^{2} - 9 \, x + 18}\;dx=\answer{\frac{2873}{6}}\]
\end{problem}}%}

%%%%%%%%%%%%%%%%%%%%%%


\latexProblemContent{
\begin{problem}

Use the Fundamental Theorem of Calculus to evaluate the integral.

\expandafter\input{\file@loc Integrals/2311-Compute-Integral-0010.HELP.tex}

\[\int_{-9}^{4} {x^{3} - 10 \, x^{2} + 32 \, x - 32}\;dx=\answer{-\frac{68107}{12}}\]
\end{problem}}%}

%%%%%%%%%%%%%%%%%%%%%%


\latexProblemContent{
\begin{problem}

Use the Fundamental Theorem of Calculus to evaluate the integral.

\expandafter\input{\file@loc Integrals/2311-Compute-Integral-0010.HELP.tex}

\[\int_{-6}^{-4} {x^{2} + 10 \, x + 25}\;dx=\answer{\frac{2}{3}}\]
\end{problem}}%}

%%%%%%%%%%%%%%%%%%%%%%


\latexProblemContent{
\begin{problem}

Use the Fundamental Theorem of Calculus to evaluate the integral.

\expandafter\input{\file@loc Integrals/2311-Compute-Integral-0010.HELP.tex}

\[\int_{-3}^{0} {x^{3} + 2 \, x^{2} + x}\;dx=\answer{-\frac{27}{4}}\]
\end{problem}}%}

%%%%%%%%%%%%%%%%%%%%%%


\latexProblemContent{
\begin{problem}

Use the Fundamental Theorem of Calculus to evaluate the integral.

\expandafter\input{\file@loc Integrals/2311-Compute-Integral-0010.HELP.tex}

\[\int_{-1}^{0} {x - 4}\;dx=\answer{-\frac{9}{2}}\]
\end{problem}}%}

%%%%%%%%%%%%%%%%%%%%%%


\latexProblemContent{
\begin{problem}

Use the Fundamental Theorem of Calculus to evaluate the integral.

\expandafter\input{\file@loc Integrals/2311-Compute-Integral-0010.HELP.tex}

\[\int_{-9}^{4} {x + 4}\;dx=\answer{\frac{39}{2}}\]
\end{problem}}%}

%%%%%%%%%%%%%%%%%%%%%%


\latexProblemContent{
\begin{problem}

Use the Fundamental Theorem of Calculus to evaluate the integral.

\expandafter\input{\file@loc Integrals/2311-Compute-Integral-0010.HELP.tex}

\[\int_{4}^{12} {x^{2} + 8 \, x + 16}\;dx=\answer{\frac{3584}{3}}\]
\end{problem}}%}

%%%%%%%%%%%%%%%%%%%%%%


\latexProblemContent{
\begin{problem}

Use the Fundamental Theorem of Calculus to evaluate the integral.

\expandafter\input{\file@loc Integrals/2311-Compute-Integral-0010.HELP.tex}

\[\int_{3}^{7} {x^{2} - 2 \, x + 1}\;dx=\answer{\frac{208}{3}}\]
\end{problem}}%}

%%%%%%%%%%%%%%%%%%%%%%


\latexProblemContent{
\begin{problem}

Use the Fundamental Theorem of Calculus to evaluate the integral.

\expandafter\input{\file@loc Integrals/2311-Compute-Integral-0010.HELP.tex}

\[\int_{-10}^{8} {x^{2} - 6 \, x - 16}\;dx=\answer{324}\]
\end{problem}}%}

%%%%%%%%%%%%%%%%%%%%%%


\latexProblemContent{
\begin{problem}

Use the Fundamental Theorem of Calculus to evaluate the integral.

\expandafter\input{\file@loc Integrals/2311-Compute-Integral-0010.HELP.tex}

\[\int_{-10}^{9} {x^{2} + 13 \, x + 40}\;dx=\answer{\frac{7277}{6}}\]
\end{problem}}%}

%%%%%%%%%%%%%%%%%%%%%%


\latexProblemContent{
\begin{problem}

Use the Fundamental Theorem of Calculus to evaluate the integral.

\expandafter\input{\file@loc Integrals/2311-Compute-Integral-0010.HELP.tex}

\[\int_{-9}^{1} {x^{3} + 15 \, x^{2} + 75 \, x + 125}\;dx=\answer{260}\]
\end{problem}}%}

%%%%%%%%%%%%%%%%%%%%%%


\latexProblemContent{
\begin{problem}

Use the Fundamental Theorem of Calculus to evaluate the integral.

\expandafter\input{\file@loc Integrals/2311-Compute-Integral-0010.HELP.tex}

\[\int_{1}^{12} {x^{3} + 9 \, x^{2} + 15 \, x + 7}\;dx=\answer{\frac{46057}{4}}\]
\end{problem}}%}

%%%%%%%%%%%%%%%%%%%%%%


\latexProblemContent{
\begin{problem}

Use the Fundamental Theorem of Calculus to evaluate the integral.

\expandafter\input{\file@loc Integrals/2311-Compute-Integral-0010.HELP.tex}

\[\int_{-5}^{-5} {x + 1}\;dx=\answer{0}\]
\end{problem}}%}

%%%%%%%%%%%%%%%%%%%%%%


\latexProblemContent{
\begin{problem}

Use the Fundamental Theorem of Calculus to evaluate the integral.

\expandafter\input{\file@loc Integrals/2311-Compute-Integral-0010.HELP.tex}

\[\int_{0}^{6} {x^{3} - 6 \, x^{2} + 9 \, x - 4}\;dx=\answer{30}\]
\end{problem}}%}

%%%%%%%%%%%%%%%%%%%%%%


\latexProblemContent{
\begin{problem}

Use the Fundamental Theorem of Calculus to evaluate the integral.

\expandafter\input{\file@loc Integrals/2311-Compute-Integral-0010.HELP.tex}

\[\int_{4}^{12} {x + 1}\;dx=\answer{72}\]
\end{problem}}%}

%%%%%%%%%%%%%%%%%%%%%%


\latexProblemContent{
\begin{problem}

Use the Fundamental Theorem of Calculus to evaluate the integral.

\expandafter\input{\file@loc Integrals/2311-Compute-Integral-0010.HELP.tex}

\[\int_{-4}^{-4} {x^{3} - 3 \, x^{2} + 3 \, x - 1}\;dx=\answer{0}\]
\end{problem}}%}

%%%%%%%%%%%%%%%%%%%%%%


\latexProblemContent{
\begin{problem}

Use the Fundamental Theorem of Calculus to evaluate the integral.

\expandafter\input{\file@loc Integrals/2311-Compute-Integral-0010.HELP.tex}

\[\int_{-6}^{6} {x^{2} + 6 \, x + 9}\;dx=\answer{252}\]
\end{problem}}%}

%%%%%%%%%%%%%%%%%%%%%%


\latexProblemContent{
\begin{problem}

Use the Fundamental Theorem of Calculus to evaluate the integral.

\expandafter\input{\file@loc Integrals/2311-Compute-Integral-0010.HELP.tex}

\[\int_{-8}^{-6} {x^{2} - 2 \, x + 1}\;dx=\answer{\frac{386}{3}}\]
\end{problem}}%}

%%%%%%%%%%%%%%%%%%%%%%


\latexProblemContent{
\begin{problem}

Use the Fundamental Theorem of Calculus to evaluate the integral.

\expandafter\input{\file@loc Integrals/2311-Compute-Integral-0010.HELP.tex}

\[\int_{4}^{8} {x - 3}\;dx=\answer{12}\]
\end{problem}}%}

%%%%%%%%%%%%%%%%%%%%%%


\latexProblemContent{
\begin{problem}

Use the Fundamental Theorem of Calculus to evaluate the integral.

\expandafter\input{\file@loc Integrals/2311-Compute-Integral-0010.HELP.tex}

\[\int_{-9}^{-4} {x^{2} + 6 \, x + 9}\;dx=\answer{\frac{215}{3}}\]
\end{problem}}%}

%%%%%%%%%%%%%%%%%%%%%%


\latexProblemContent{
\begin{problem}

Use the Fundamental Theorem of Calculus to evaluate the integral.

\expandafter\input{\file@loc Integrals/2311-Compute-Integral-0010.HELP.tex}

\[\int_{2}^{11} {x^{2} - 6 \, x + 9}\;dx=\answer{171}\]
\end{problem}}%}

%%%%%%%%%%%%%%%%%%%%%%


\latexProblemContent{
\begin{problem}

Use the Fundamental Theorem of Calculus to evaluate the integral.

\expandafter\input{\file@loc Integrals/2311-Compute-Integral-0010.HELP.tex}

\[\int_{-1}^{7} {x^{3} + 4 \, x^{2} + 5 \, x + 2}\;dx=\answer{\frac{3584}{3}}\]
\end{problem}}%}

%%%%%%%%%%%%%%%%%%%%%%


\latexProblemContent{
\begin{problem}

Use the Fundamental Theorem of Calculus to evaluate the integral.

\expandafter\input{\file@loc Integrals/2311-Compute-Integral-0010.HELP.tex}

\[\int_{-2}^{8} {x^{3} - x^{2} - 5 \, x - 3}\;dx=\answer{\frac{2000}{3}}\]
\end{problem}}%}

%%%%%%%%%%%%%%%%%%%%%%


\latexProblemContent{
\begin{problem}

Use the Fundamental Theorem of Calculus to evaluate the integral.

\expandafter\input{\file@loc Integrals/2311-Compute-Integral-0010.HELP.tex}

\[\int_{-8}^{-1} {x^{2} + 12 \, x + 32}\;dx=\answer{\frac{49}{3}}\]
\end{problem}}%}

%%%%%%%%%%%%%%%%%%%%%%


\latexProblemContent{
\begin{problem}

Use the Fundamental Theorem of Calculus to evaluate the integral.

\expandafter\input{\file@loc Integrals/2311-Compute-Integral-0010.HELP.tex}

\[\int_{-9}^{10} {x^{3} + 15 \, x^{2} + 75 \, x + 125}\;dx=\answer{\frac{50369}{4}}\]
\end{problem}}%}

%%%%%%%%%%%%%%%%%%%%%%


\latexProblemContent{
\begin{problem}

Use the Fundamental Theorem of Calculus to evaluate the integral.

\expandafter\input{\file@loc Integrals/2311-Compute-Integral-0010.HELP.tex}

\[\int_{-4}^{3} {x - 4}\;dx=\answer{-\frac{63}{2}}\]
\end{problem}}%}

%%%%%%%%%%%%%%%%%%%%%%


\latexProblemContent{
\begin{problem}

Use the Fundamental Theorem of Calculus to evaluate the integral.

\expandafter\input{\file@loc Integrals/2311-Compute-Integral-0010.HELP.tex}

\[\int_{-1}^{8} {x^{3} + 14 \, x^{2} + 65 \, x + 100}\;dx=\answer{\frac{25461}{4}}\]
\end{problem}}%}

%%%%%%%%%%%%%%%%%%%%%%


\latexProblemContent{
\begin{problem}

Use the Fundamental Theorem of Calculus to evaluate the integral.

\expandafter\input{\file@loc Integrals/2311-Compute-Integral-0010.HELP.tex}

\[\int_{4}^{7} {x^{3} + 9 \, x^{2} + 27 \, x + 27}\;dx=\answer{\frac{7599}{4}}\]
\end{problem}}%}

%%%%%%%%%%%%%%%%%%%%%%


\latexProblemContent{
\begin{problem}

Use the Fundamental Theorem of Calculus to evaluate the integral.

\expandafter\input{\file@loc Integrals/2311-Compute-Integral-0010.HELP.tex}

\[\int_{-9}^{-7} {x^{2} + 8 \, x + 16}\;dx=\answer{\frac{98}{3}}\]
\end{problem}}%}

%%%%%%%%%%%%%%%%%%%%%%


\latexProblemContent{
\begin{problem}

Use the Fundamental Theorem of Calculus to evaluate the integral.

\expandafter\input{\file@loc Integrals/2311-Compute-Integral-0010.HELP.tex}

\[\int_{-9}^{2} {x^{2} + 10 \, x + 24}\;dx=\answer{\frac{374}{3}}\]
\end{problem}}%}

%%%%%%%%%%%%%%%%%%%%%%


\latexProblemContent{
\begin{problem}

Use the Fundamental Theorem of Calculus to evaluate the integral.

\expandafter\input{\file@loc Integrals/2311-Compute-Integral-0010.HELP.tex}

\[\int_{3}^{10} {x^{3} - 6 \, x^{2} + 12 \, x - 8}\;dx=\answer{\frac{4095}{4}}\]
\end{problem}}%}

%%%%%%%%%%%%%%%%%%%%%%


\latexProblemContent{
\begin{problem}

Use the Fundamental Theorem of Calculus to evaluate the integral.

\expandafter\input{\file@loc Integrals/2311-Compute-Integral-0010.HELP.tex}

\[\int_{-3}^{4} {x - 3}\;dx=\answer{-\frac{35}{2}}\]
\end{problem}}%}

%%%%%%%%%%%%%%%%%%%%%%


\latexProblemContent{
\begin{problem}

Use the Fundamental Theorem of Calculus to evaluate the integral.

\expandafter\input{\file@loc Integrals/2311-Compute-Integral-0010.HELP.tex}

\[\int_{-6}^{7} {x^{3} - 6 \, x^{2} + 12 \, x - 8}\;dx=\answer{-\frac{3471}{4}}\]
\end{problem}}%}

%%%%%%%%%%%%%%%%%%%%%%


\latexProblemContent{
\begin{problem}

Use the Fundamental Theorem of Calculus to evaluate the integral.

\expandafter\input{\file@loc Integrals/2311-Compute-Integral-0010.HELP.tex}

\[\int_{2}^{8} {x^{3} + 12 \, x^{2} + 48 \, x + 64}\;dx=\answer{4860}\]
\end{problem}}%}

%%%%%%%%%%%%%%%%%%%%%%


\latexProblemContent{
\begin{problem}

Use the Fundamental Theorem of Calculus to evaluate the integral.

\expandafter\input{\file@loc Integrals/2311-Compute-Integral-0010.HELP.tex}

\[\int_{1}^{2} {x^{2} - 4 \, x + 4}\;dx=\answer{\frac{1}{3}}\]
\end{problem}}%}

%%%%%%%%%%%%%%%%%%%%%%


\latexProblemContent{
\begin{problem}

Use the Fundamental Theorem of Calculus to evaluate the integral.

\expandafter\input{\file@loc Integrals/2311-Compute-Integral-0010.HELP.tex}

\[\int_{-7}^{-3} {x^{2} + 10 \, x + 25}\;dx=\answer{\frac{16}{3}}\]
\end{problem}}%}

%%%%%%%%%%%%%%%%%%%%%%


\latexProblemContent{
\begin{problem}

Use the Fundamental Theorem of Calculus to evaluate the integral.

\expandafter\input{\file@loc Integrals/2311-Compute-Integral-0010.HELP.tex}

\[\int_{5}^{11} {x + 4}\;dx=\answer{72}\]
\end{problem}}%}

%%%%%%%%%%%%%%%%%%%%%%


\latexProblemContent{
\begin{problem}

Use the Fundamental Theorem of Calculus to evaluate the integral.

\expandafter\input{\file@loc Integrals/2311-Compute-Integral-0010.HELP.tex}

\[\int_{-2}^{4} {x^{2} + 4 \, x + 4}\;dx=\answer{72}\]
\end{problem}}%}

%%%%%%%%%%%%%%%%%%%%%%


\latexProblemContent{
\begin{problem}

Use the Fundamental Theorem of Calculus to evaluate the integral.

\expandafter\input{\file@loc Integrals/2311-Compute-Integral-0010.HELP.tex}

\[\int_{-8}^{-4} {x - 2}\;dx=\answer{-32}\]
\end{problem}}%}

%%%%%%%%%%%%%%%%%%%%%%


\latexProblemContent{
\begin{problem}

Use the Fundamental Theorem of Calculus to evaluate the integral.

\expandafter\input{\file@loc Integrals/2311-Compute-Integral-0010.HELP.tex}

\[\int_{2}^{10} {x^{2} + 4 \, x + 4}\;dx=\answer{\frac{1664}{3}}\]
\end{problem}}%}

%%%%%%%%%%%%%%%%%%%%%%


\latexProblemContent{
\begin{problem}

Use the Fundamental Theorem of Calculus to evaluate the integral.

\expandafter\input{\file@loc Integrals/2311-Compute-Integral-0010.HELP.tex}

\[\int_{0}^{6} {x^{3} + 14 \, x^{2} + 65 \, x + 100}\;dx=\answer{3102}\]
\end{problem}}%}

%%%%%%%%%%%%%%%%%%%%%%


\latexProblemContent{
\begin{problem}

Use the Fundamental Theorem of Calculus to evaluate the integral.

\expandafter\input{\file@loc Integrals/2311-Compute-Integral-0010.HELP.tex}

\[\int_{3}^{4} {x^{3} - 9 \, x^{2} + 27 \, x - 27}\;dx=\answer{\frac{1}{4}}\]
\end{problem}}%}

%%%%%%%%%%%%%%%%%%%%%%


\latexProblemContent{
\begin{problem}

Use the Fundamental Theorem of Calculus to evaluate the integral.

\expandafter\input{\file@loc Integrals/2311-Compute-Integral-0010.HELP.tex}

\[\int_{2}^{11} {x + 1}\;dx=\answer{\frac{135}{2}}\]
\end{problem}}%}

%%%%%%%%%%%%%%%%%%%%%%


\latexProblemContent{
\begin{problem}

Use the Fundamental Theorem of Calculus to evaluate the integral.

\expandafter\input{\file@loc Integrals/2311-Compute-Integral-0010.HELP.tex}

\[\int_{3}^{12} {x - 5}\;dx=\answer{\frac{45}{2}}\]
\end{problem}}%}

%%%%%%%%%%%%%%%%%%%%%%


\latexProblemContent{
\begin{problem}

Use the Fundamental Theorem of Calculus to evaluate the integral.

\expandafter\input{\file@loc Integrals/2311-Compute-Integral-0010.HELP.tex}

\[\int_{-5}^{2} {x^{3} - 6 \, x^{2} + 32}\;dx=\answer{-\frac{777}{4}}\]
\end{problem}}%}

%%%%%%%%%%%%%%%%%%%%%%


\latexProblemContent{
\begin{problem}

Use the Fundamental Theorem of Calculus to evaluate the integral.

\expandafter\input{\file@loc Integrals/2311-Compute-Integral-0010.HELP.tex}

\[\int_{2}^{5} {x^{3} + 12 \, x^{2} + 48 \, x + 64}\;dx=\answer{\frac{5265}{4}}\]
\end{problem}}%}

%%%%%%%%%%%%%%%%%%%%%%


\latexProblemContent{
\begin{problem}

Use the Fundamental Theorem of Calculus to evaluate the integral.

\expandafter\input{\file@loc Integrals/2311-Compute-Integral-0010.HELP.tex}

\[\int_{-6}^{2} {x - 3}\;dx=\answer{-40}\]
\end{problem}}%}

%%%%%%%%%%%%%%%%%%%%%%


\latexProblemContent{
\begin{problem}

Use the Fundamental Theorem of Calculus to evaluate the integral.

\expandafter\input{\file@loc Integrals/2311-Compute-Integral-0010.HELP.tex}

\[\int_{-7}^{-5} {x^{2} - 4 \, x + 4}\;dx=\answer{\frac{386}{3}}\]
\end{problem}}%}

%%%%%%%%%%%%%%%%%%%%%%


\latexProblemContent{
\begin{problem}

Use the Fundamental Theorem of Calculus to evaluate the integral.

\expandafter\input{\file@loc Integrals/2311-Compute-Integral-0010.HELP.tex}

\[\int_{-2}^{0} {x^{2} + 6 \, x + 9}\;dx=\answer{\frac{26}{3}}\]
\end{problem}}%}

%%%%%%%%%%%%%%%%%%%%%%


\latexProblemContent{
\begin{problem}

Use the Fundamental Theorem of Calculus to evaluate the integral.

\expandafter\input{\file@loc Integrals/2311-Compute-Integral-0010.HELP.tex}

\[\int_{3}^{8} {x^{3} - 3 \, x - 2}\;dx=\answer{\frac{3645}{4}}\]
\end{problem}}%}

%%%%%%%%%%%%%%%%%%%%%%


\latexProblemContent{
\begin{problem}

Use the Fundamental Theorem of Calculus to evaluate the integral.

\expandafter\input{\file@loc Integrals/2311-Compute-Integral-0010.HELP.tex}

\[\int_{-3}^{-1} {x^{2} - 25}\;dx=\answer{-\frac{124}{3}}\]
\end{problem}}%}

%%%%%%%%%%%%%%%%%%%%%%


\latexProblemContent{
\begin{problem}

Use the Fundamental Theorem of Calculus to evaluate the integral.

\expandafter\input{\file@loc Integrals/2311-Compute-Integral-0010.HELP.tex}

\[\int_{2}^{12} {x - 5}\;dx=\answer{20}\]
\end{problem}}%}

%%%%%%%%%%%%%%%%%%%%%%


\latexProblemContent{
\begin{problem}

Use the Fundamental Theorem of Calculus to evaluate the integral.

\expandafter\input{\file@loc Integrals/2311-Compute-Integral-0010.HELP.tex}

\[\int_{-10}^{5} {x^{3} + 2 \, x^{2} - 55 \, x - 200}\;dx=\answer{-\frac{10125}{4}}\]
\end{problem}}%}

%%%%%%%%%%%%%%%%%%%%%%


\latexProblemContent{
\begin{problem}

Use the Fundamental Theorem of Calculus to evaluate the integral.

\expandafter\input{\file@loc Integrals/2311-Compute-Integral-0010.HELP.tex}

\[\int_{3}^{4} {x^{2} + 2 \, x + 1}\;dx=\answer{\frac{61}{3}}\]
\end{problem}}%}

%%%%%%%%%%%%%%%%%%%%%%


\latexProblemContent{
\begin{problem}

Use the Fundamental Theorem of Calculus to evaluate the integral.

\expandafter\input{\file@loc Integrals/2311-Compute-Integral-0010.HELP.tex}

\[\int_{-10}^{-4} {x^{2} - 4}\;dx=\answer{288}\]
\end{problem}}%}

%%%%%%%%%%%%%%%%%%%%%%


\latexProblemContent{
\begin{problem}

Use the Fundamental Theorem of Calculus to evaluate the integral.

\expandafter\input{\file@loc Integrals/2311-Compute-Integral-0010.HELP.tex}

\[\int_{-10}^{-2} {x^{2} - 4 \, x + 4}\;dx=\answer{\frac{1664}{3}}\]
\end{problem}}%}

%%%%%%%%%%%%%%%%%%%%%%


\latexProblemContent{
\begin{problem}

Use the Fundamental Theorem of Calculus to evaluate the integral.

\expandafter\input{\file@loc Integrals/2311-Compute-Integral-0010.HELP.tex}

\[\int_{-3}^{-3} {x - 1}\;dx=\answer{0}\]
\end{problem}}%}

%%%%%%%%%%%%%%%%%%%%%%


\latexProblemContent{
\begin{problem}

Use the Fundamental Theorem of Calculus to evaluate the integral.

\expandafter\input{\file@loc Integrals/2311-Compute-Integral-0010.HELP.tex}

\[\int_{1}^{8} {x^{3} + 15 \, x^{2} + 75 \, x + 125}\;dx=\answer{\frac{27265}{4}}\]
\end{problem}}%}

%%%%%%%%%%%%%%%%%%%%%%


\latexProblemContent{
\begin{problem}

Use the Fundamental Theorem of Calculus to evaluate the integral.

\expandafter\input{\file@loc Integrals/2311-Compute-Integral-0010.HELP.tex}

\[\int_{-7}^{4} {x^{2} - 4 \, x + 4}\;dx=\answer{\frac{737}{3}}\]
\end{problem}}%}

%%%%%%%%%%%%%%%%%%%%%%


\latexProblemContent{
\begin{problem}

Use the Fundamental Theorem of Calculus to evaluate the integral.

\expandafter\input{\file@loc Integrals/2311-Compute-Integral-0010.HELP.tex}

\[\int_{-10}^{2} {x^{3} - 15 \, x^{2} + 75 \, x - 125}\;dx=\answer{-12636}\]
\end{problem}}%}

%%%%%%%%%%%%%%%%%%%%%%


\latexProblemContent{
\begin{problem}

Use the Fundamental Theorem of Calculus to evaluate the integral.

\expandafter\input{\file@loc Integrals/2311-Compute-Integral-0010.HELP.tex}

\[\int_{-7}^{-1} {x^{3} + 8 \, x^{2} + 16 \, x}\;dx=\answer{-72}\]
\end{problem}}%}

%%%%%%%%%%%%%%%%%%%%%%


\latexProblemContent{
\begin{problem}

Use the Fundamental Theorem of Calculus to evaluate the integral.

\expandafter\input{\file@loc Integrals/2311-Compute-Integral-0010.HELP.tex}

\[\int_{3}^{5} {x^{2} - 10 \, x + 25}\;dx=\answer{\frac{8}{3}}\]
\end{problem}}%}

%%%%%%%%%%%%%%%%%%%%%%


\latexProblemContent{
\begin{problem}

Use the Fundamental Theorem of Calculus to evaluate the integral.

\expandafter\input{\file@loc Integrals/2311-Compute-Integral-0010.HELP.tex}

\[\int_{-5}^{2} {x^{2} - 6 \, x + 9}\;dx=\answer{\frac{511}{3}}\]
\end{problem}}%}

%%%%%%%%%%%%%%%%%%%%%%


\latexProblemContent{
\begin{problem}

Use the Fundamental Theorem of Calculus to evaluate the integral.

\expandafter\input{\file@loc Integrals/2311-Compute-Integral-0010.HELP.tex}

\[\int_{-1}^{10} {x^{2} - 6 \, x + 9}\;dx=\answer{\frac{407}{3}}\]
\end{problem}}%}

%%%%%%%%%%%%%%%%%%%%%%


\latexProblemContent{
\begin{problem}

Use the Fundamental Theorem of Calculus to evaluate the integral.

\expandafter\input{\file@loc Integrals/2311-Compute-Integral-0010.HELP.tex}

\[\int_{4}^{10} {x^{3} + 12 \, x^{2} + 48 \, x + 64}\;dx=\answer{8580}\]
\end{problem}}%}

%%%%%%%%%%%%%%%%%%%%%%


\latexProblemContent{
\begin{problem}

Use the Fundamental Theorem of Calculus to evaluate the integral.

\expandafter\input{\file@loc Integrals/2311-Compute-Integral-0010.HELP.tex}

\[\int_{-7}^{-1} {x^{3} - 6 \, x^{2} + 12 \, x - 8}\;dx=\answer{-1620}\]
\end{problem}}%}

%%%%%%%%%%%%%%%%%%%%%%


\latexProblemContent{
\begin{problem}

Use the Fundamental Theorem of Calculus to evaluate the integral.

\expandafter\input{\file@loc Integrals/2311-Compute-Integral-0010.HELP.tex}

\[\int_{-3}^{8} {x^{3} - 9 \, x^{2} + 24 \, x - 16}\;dx=\answer{-\frac{517}{4}}\]
\end{problem}}%}

%%%%%%%%%%%%%%%%%%%%%%


\latexProblemContent{
\begin{problem}

Use the Fundamental Theorem of Calculus to evaluate the integral.

\expandafter\input{\file@loc Integrals/2311-Compute-Integral-0010.HELP.tex}

\[\int_{-2}^{-1} {x^{2} - 2 \, x - 8}\;dx=\answer{-\frac{8}{3}}\]
\end{problem}}%}

%%%%%%%%%%%%%%%%%%%%%%


\latexProblemContent{
\begin{problem}

Use the Fundamental Theorem of Calculus to evaluate the integral.

\expandafter\input{\file@loc Integrals/2311-Compute-Integral-0010.HELP.tex}

\[\int_{-7}^{2} {x^{2} + 8 \, x + 16}\;dx=\answer{81}\]
\end{problem}}%}

%%%%%%%%%%%%%%%%%%%%%%


\latexProblemContent{
\begin{problem}

Use the Fundamental Theorem of Calculus to evaluate the integral.

\expandafter\input{\file@loc Integrals/2311-Compute-Integral-0010.HELP.tex}

\[\int_{-2}^{0} {x + 5}\;dx=\answer{8}\]
\end{problem}}%}

%%%%%%%%%%%%%%%%%%%%%%


%%%%%%%%%%%%%%%%%%%%%%


\latexProblemContent{
\begin{problem}

Use the Fundamental Theorem of Calculus to evaluate the integral.

\expandafter\input{\file@loc Integrals/2311-Compute-Integral-0010.HELP.tex}

\[\int_{-6}^{9} {x + 5}\;dx=\answer{\frac{195}{2}}\]
\end{problem}}%}

%%%%%%%%%%%%%%%%%%%%%%


\latexProblemContent{
\begin{problem}

Use the Fundamental Theorem of Calculus to evaluate the integral.

\expandafter\input{\file@loc Integrals/2311-Compute-Integral-0010.HELP.tex}

\[\int_{-4}^{10} {x^{3} - 6 \, x^{2} + 12 \, x - 8}\;dx=\answer{700}\]
\end{problem}}%}

%%%%%%%%%%%%%%%%%%%%%%


\latexProblemContent{
\begin{problem}

Use the Fundamental Theorem of Calculus to evaluate the integral.

\expandafter\input{\file@loc Integrals/2311-Compute-Integral-0010.HELP.tex}

\[\int_{-6}^{7} {x - 2}\;dx=\answer{-\frac{39}{2}}\]
\end{problem}}%}

%%%%%%%%%%%%%%%%%%%%%%


\latexProblemContent{
\begin{problem}

Use the Fundamental Theorem of Calculus to evaluate the integral.

\expandafter\input{\file@loc Integrals/2311-Compute-Integral-0010.HELP.tex}

\[\int_{1}^{8} {x^{3} + 2 \, x^{2} - 20 \, x + 24}\;dx=\answer{\frac{10829}{12}}\]
\end{problem}}%}

%%%%%%%%%%%%%%%%%%%%%%


\latexProblemContent{
\begin{problem}

Use the Fundamental Theorem of Calculus to evaluate the integral.

\expandafter\input{\file@loc Integrals/2311-Compute-Integral-0010.HELP.tex}

\[\int_{5}^{11} {x - 5}\;dx=\answer{18}\]
\end{problem}}%}

%%%%%%%%%%%%%%%%%%%%%%


\latexProblemContent{
\begin{problem}

Use the Fundamental Theorem of Calculus to evaluate the integral.

\expandafter\input{\file@loc Integrals/2311-Compute-Integral-0010.HELP.tex}

\[\int_{-5}^{1} {x^{2} - 6 \, x - 7}\;dx=\answer{72}\]
\end{problem}}%}

%%%%%%%%%%%%%%%%%%%%%%


\latexProblemContent{
\begin{problem}

Use the Fundamental Theorem of Calculus to evaluate the integral.

\expandafter\input{\file@loc Integrals/2311-Compute-Integral-0010.HELP.tex}

\[\int_{-6}^{1} {x^{3} - x^{2} - 5 \, x - 3}\;dx=\answer{-\frac{3955}{12}}\]
\end{problem}}%}

%%%%%%%%%%%%%%%%%%%%%%


\latexProblemContent{
\begin{problem}

Use the Fundamental Theorem of Calculus to evaluate the integral.

\expandafter\input{\file@loc Integrals/2311-Compute-Integral-0010.HELP.tex}

\[\int_{-9}^{11} {x^{2} + 7 \, x + 6}\;dx=\answer{\frac{2840}{3}}\]
\end{problem}}%}

%%%%%%%%%%%%%%%%%%%%%%


\latexProblemContent{
\begin{problem}

Use the Fundamental Theorem of Calculus to evaluate the integral.

\expandafter\input{\file@loc Integrals/2311-Compute-Integral-0010.HELP.tex}

\[\int_{-1}^{4} {x^{2} + 7 \, x + 10}\;dx=\answer{\frac{745}{6}}\]
\end{problem}}%}

%%%%%%%%%%%%%%%%%%%%%%


\latexProblemContent{
\begin{problem}

Use the Fundamental Theorem of Calculus to evaluate the integral.

\expandafter\input{\file@loc Integrals/2311-Compute-Integral-0010.HELP.tex}

\[\int_{-6}^{10} {x^{3} + 15 \, x^{2} + 75 \, x + 125}\;dx=\answer{12656}\]
\end{problem}}%}

%%%%%%%%%%%%%%%%%%%%%%


\latexProblemContent{
\begin{problem}

Use the Fundamental Theorem of Calculus to evaluate the integral.

\expandafter\input{\file@loc Integrals/2311-Compute-Integral-0010.HELP.tex}

\[\int_{-3}^{0} {x^{2} - 6 \, x + 9}\;dx=\answer{63}\]
\end{problem}}%}

%%%%%%%%%%%%%%%%%%%%%%


\latexProblemContent{
\begin{problem}

Use the Fundamental Theorem of Calculus to evaluate the integral.

\expandafter\input{\file@loc Integrals/2311-Compute-Integral-0010.HELP.tex}

\[\int_{-10}^{-1} {x^{2} + 6 \, x + 9}\;dx=\answer{117}\]
\end{problem}}%}

%%%%%%%%%%%%%%%%%%%%%%


\latexProblemContent{
\begin{problem}

Use the Fundamental Theorem of Calculus to evaluate the integral.

\expandafter\input{\file@loc Integrals/2311-Compute-Integral-0010.HELP.tex}

\[\int_{3}^{6} {x - 2}\;dx=\answer{\frac{15}{2}}\]
\end{problem}}%}

%%%%%%%%%%%%%%%%%%%%%%


\latexProblemContent{
\begin{problem}

Use the Fundamental Theorem of Calculus to evaluate the integral.

\expandafter\input{\file@loc Integrals/2311-Compute-Integral-0010.HELP.tex}

\[\int_{-10}^{-9} {x^{2} - 6 \, x + 8}\;dx=\answer{\frac{466}{3}}\]
\end{problem}}%}

%%%%%%%%%%%%%%%%%%%%%%


\latexProblemContent{
\begin{problem}

Use the Fundamental Theorem of Calculus to evaluate the integral.

\expandafter\input{\file@loc Integrals/2311-Compute-Integral-0010.HELP.tex}

\[\int_{-9}^{9} {x^{3} - 3 \, x^{2} + 3 \, x - 1}\;dx=\answer{-1476}\]
\end{problem}}%}

%%%%%%%%%%%%%%%%%%%%%%


\latexProblemContent{
\begin{problem}

Use the Fundamental Theorem of Calculus to evaluate the integral.

\expandafter\input{\file@loc Integrals/2311-Compute-Integral-0010.HELP.tex}

\[\int_{-5}^{10} {x^{2} + 4 \, x + 4}\;dx=\answer{585}\]
\end{problem}}%}

%%%%%%%%%%%%%%%%%%%%%%


\latexProblemContent{
\begin{problem}

Use the Fundamental Theorem of Calculus to evaluate the integral.

\expandafter\input{\file@loc Integrals/2311-Compute-Integral-0010.HELP.tex}

\[\int_{-9}^{12} {x^{3} + 12 \, x^{2} + 48 \, x + 64}\;dx=\answer{\frac{64911}{4}}\]
\end{problem}}%}

%%%%%%%%%%%%%%%%%%%%%%


\latexProblemContent{
\begin{problem}

Use the Fundamental Theorem of Calculus to evaluate the integral.

\expandafter\input{\file@loc Integrals/2311-Compute-Integral-0010.HELP.tex}

\[\int_{0}^{2} {x^{3} - 12 \, x^{2} + 48 \, x - 64}\;dx=\answer{-60}\]
\end{problem}}%}

%%%%%%%%%%%%%%%%%%%%%%


\latexProblemContent{
\begin{problem}

Use the Fundamental Theorem of Calculus to evaluate the integral.

\expandafter\input{\file@loc Integrals/2311-Compute-Integral-0010.HELP.tex}

\[\int_{-1}^{11} {x - 5}\;dx=\answer{0}\]
\end{problem}}%}

%%%%%%%%%%%%%%%%%%%%%%


\latexProblemContent{
\begin{problem}

Use the Fundamental Theorem of Calculus to evaluate the integral.

\expandafter\input{\file@loc Integrals/2311-Compute-Integral-0010.HELP.tex}

\[\int_{-10}^{-1} {x^{2} + 8 \, x + 16}\;dx=\answer{81}\]
\end{problem}}%}

%%%%%%%%%%%%%%%%%%%%%%


\latexProblemContent{
\begin{problem}

Use the Fundamental Theorem of Calculus to evaluate the integral.

\expandafter\input{\file@loc Integrals/2311-Compute-Integral-0010.HELP.tex}

\[\int_{3}^{6} {x - 3}\;dx=\answer{\frac{9}{2}}\]
\end{problem}}%}

%%%%%%%%%%%%%%%%%%%%%%


\latexProblemContent{
\begin{problem}

Use the Fundamental Theorem of Calculus to evaluate the integral.

\expandafter\input{\file@loc Integrals/2311-Compute-Integral-0010.HELP.tex}

\[\int_{-3}^{-1} {x^{2} - 6 \, x + 9}\;dx=\answer{\frac{152}{3}}\]
\end{problem}}%}

%%%%%%%%%%%%%%%%%%%%%%


\latexProblemContent{
\begin{problem}

Use the Fundamental Theorem of Calculus to evaluate the integral.

\expandafter\input{\file@loc Integrals/2311-Compute-Integral-0010.HELP.tex}

\[\int_{-9}^{-9} {x^{2} - 6 \, x + 9}\;dx=\answer{0}\]
\end{problem}}%}

%%%%%%%%%%%%%%%%%%%%%%


\latexProblemContent{
\begin{problem}

Use the Fundamental Theorem of Calculus to evaluate the integral.

\expandafter\input{\file@loc Integrals/2311-Compute-Integral-0010.HELP.tex}

\[\int_{1}^{7} {x^{2} - 12 \, x + 35}\;dx=\answer{36}\]
\end{problem}}%}

%%%%%%%%%%%%%%%%%%%%%%


\latexProblemContent{
\begin{problem}

Use the Fundamental Theorem of Calculus to evaluate the integral.

\expandafter\input{\file@loc Integrals/2311-Compute-Integral-0010.HELP.tex}

\[\int_{1}^{3} {x^{2} + 4 \, x + 4}\;dx=\answer{\frac{98}{3}}\]
\end{problem}}%}

%%%%%%%%%%%%%%%%%%%%%%


\latexProblemContent{
\begin{problem}

Use the Fundamental Theorem of Calculus to evaluate the integral.

\expandafter\input{\file@loc Integrals/2311-Compute-Integral-0010.HELP.tex}

\[\int_{2}^{11} {x^{3} - 6 \, x^{2} + 12 \, x - 8}\;dx=\answer{\frac{6561}{4}}\]
\end{problem}}%}

%%%%%%%%%%%%%%%%%%%%%%


\latexProblemContent{
\begin{problem}

Use the Fundamental Theorem of Calculus to evaluate the integral.

\expandafter\input{\file@loc Integrals/2311-Compute-Integral-0010.HELP.tex}

\[\int_{5}^{11} {x^{2} - 7 \, x + 10}\;dx=\answer{126}\]
\end{problem}}%}

%%%%%%%%%%%%%%%%%%%%%%


\latexProblemContent{
\begin{problem}

Use the Fundamental Theorem of Calculus to evaluate the integral.

\expandafter\input{\file@loc Integrals/2311-Compute-Integral-0010.HELP.tex}

\[\int_{4}^{10} {x - 1}\;dx=\answer{36}\]
\end{problem}}%}

%%%%%%%%%%%%%%%%%%%%%%


\latexProblemContent{
\begin{problem}

Use the Fundamental Theorem of Calculus to evaluate the integral.

\expandafter\input{\file@loc Integrals/2311-Compute-Integral-0010.HELP.tex}

\[\int_{-8}^{-1} {x^{3} - 27 \, x - 54}\;dx=\answer{-\frac{2205}{4}}\]
\end{problem}}%}

%%%%%%%%%%%%%%%%%%%%%%


\latexProblemContent{
\begin{problem}

Use the Fundamental Theorem of Calculus to evaluate the integral.

\expandafter\input{\file@loc Integrals/2311-Compute-Integral-0010.HELP.tex}

\[\int_{-9}^{12} {x^{2} - 9 \, x + 8}\;dx=\answer{\frac{1407}{2}}\]
\end{problem}}%}

%%%%%%%%%%%%%%%%%%%%%%


\latexProblemContent{
\begin{problem}

Use the Fundamental Theorem of Calculus to evaluate the integral.

\expandafter\input{\file@loc Integrals/2311-Compute-Integral-0010.HELP.tex}

\[\int_{-4}^{2} {x^{3} - 9 \, x^{2} + 27 \, x - 27}\;dx=\answer{-600}\]
\end{problem}}%}

%%%%%%%%%%%%%%%%%%%%%%


\latexProblemContent{
\begin{problem}

Use the Fundamental Theorem of Calculus to evaluate the integral.

\expandafter\input{\file@loc Integrals/2311-Compute-Integral-0010.HELP.tex}

\[\int_{-10}^{8} {x^{3} - 12 \, x^{2} + 36 \, x - 32}\;dx=\answer{-8748}\]
\end{problem}}%}

%%%%%%%%%%%%%%%%%%%%%%


\latexProblemContent{
\begin{problem}

Use the Fundamental Theorem of Calculus to evaluate the integral.

\expandafter\input{\file@loc Integrals/2311-Compute-Integral-0010.HELP.tex}

\[\int_{0}^{11} {x^{2} + 10 \, x + 25}\;dx=\answer{\frac{3971}{3}}\]
\end{problem}}%}

%%%%%%%%%%%%%%%%%%%%%%


\latexProblemContent{
\begin{problem}

Use the Fundamental Theorem of Calculus to evaluate the integral.

\expandafter\input{\file@loc Integrals/2311-Compute-Integral-0010.HELP.tex}

\[\int_{2}^{6} {x^{3} + 11 \, x^{2} + 35 \, x + 25}\;dx=\answer{\frac{5228}{3}}\]
\end{problem}}%}

%%%%%%%%%%%%%%%%%%%%%%


\latexProblemContent{
\begin{problem}

Use the Fundamental Theorem of Calculus to evaluate the integral.

\expandafter\input{\file@loc Integrals/2311-Compute-Integral-0010.HELP.tex}

\[\int_{-7}^{8} {x^{3} - 15 \, x^{2} + 75 \, x - 125}\;dx=\answer{-\frac{20655}{4}}\]
\end{problem}}%}

%%%%%%%%%%%%%%%%%%%%%%


\latexProblemContent{
\begin{problem}

Use the Fundamental Theorem of Calculus to evaluate the integral.

\expandafter\input{\file@loc Integrals/2311-Compute-Integral-0010.HELP.tex}

\[\int_{-3}^{4} {x^{3} - 27 \, x - 54}\;dx=\answer{-\frac{1715}{4}}\]
\end{problem}}%}

%%%%%%%%%%%%%%%%%%%%%%


\latexProblemContent{
\begin{problem}

Use the Fundamental Theorem of Calculus to evaluate the integral.

\expandafter\input{\file@loc Integrals/2311-Compute-Integral-0010.HELP.tex}

\[\int_{5}^{12} {x^{3} + 3 \, x^{2} + 3 \, x + 1}\;dx=\answer{\frac{27265}{4}}\]
\end{problem}}%}

%%%%%%%%%%%%%%%%%%%%%%


\latexProblemContent{
\begin{problem}

Use the Fundamental Theorem of Calculus to evaluate the integral.

\expandafter\input{\file@loc Integrals/2311-Compute-Integral-0010.HELP.tex}

\[\int_{-1}^{2} {x - 3}\;dx=\answer{-\frac{15}{2}}\]
\end{problem}}%}

%%%%%%%%%%%%%%%%%%%%%%


\latexProblemContent{
\begin{problem}

Use the Fundamental Theorem of Calculus to evaluate the integral.

\expandafter\input{\file@loc Integrals/2311-Compute-Integral-0010.HELP.tex}

\[\int_{-10}^{6} {x^{3} + 15 \, x^{2} + 75 \, x + 125}\;dx=\answer{3504}\]
\end{problem}}%}

%%%%%%%%%%%%%%%%%%%%%%


\latexProblemContent{
\begin{problem}

Use the Fundamental Theorem of Calculus to evaluate the integral.

\expandafter\input{\file@loc Integrals/2311-Compute-Integral-0010.HELP.tex}

\[\int_{1}^{12} {x^{3} - 3 \, x^{2} + 3 \, x - 1}\;dx=\answer{\frac{14641}{4}}\]
\end{problem}}%}

%%%%%%%%%%%%%%%%%%%%%%


\latexProblemContent{
\begin{problem}

Use the Fundamental Theorem of Calculus to evaluate the integral.

\expandafter\input{\file@loc Integrals/2311-Compute-Integral-0010.HELP.tex}

\[\int_{-1}^{-1} {x^{3} + 9 \, x^{2} + 27 \, x + 27}\;dx=\answer{0}\]
\end{problem}}%}

%%%%%%%%%%%%%%%%%%%%%%


\latexProblemContent{
\begin{problem}

Use the Fundamental Theorem of Calculus to evaluate the integral.

\expandafter\input{\file@loc Integrals/2311-Compute-Integral-0010.HELP.tex}

\[\int_{0}^{5} {x^{2} + 2 \, x + 1}\;dx=\answer{\frac{215}{3}}\]
\end{problem}}%}

%%%%%%%%%%%%%%%%%%%%%%


\latexProblemContent{
\begin{problem}

Use the Fundamental Theorem of Calculus to evaluate the integral.

\expandafter\input{\file@loc Integrals/2311-Compute-Integral-0010.HELP.tex}

\[\int_{-7}^{4} {x^{2} + 8 \, x + 16}\;dx=\answer{\frac{539}{3}}\]
\end{problem}}%}

%%%%%%%%%%%%%%%%%%%%%%


\latexProblemContent{
\begin{problem}

Use the Fundamental Theorem of Calculus to evaluate the integral.

\expandafter\input{\file@loc Integrals/2311-Compute-Integral-0010.HELP.tex}

\[\int_{5}^{8} {x^{3} + 3 \, x^{2} - 24 \, x + 28}\;dx=\answer{\frac{3483}{4}}\]
\end{problem}}%}

%%%%%%%%%%%%%%%%%%%%%%


\latexProblemContent{
\begin{problem}

Use the Fundamental Theorem of Calculus to evaluate the integral.

\expandafter\input{\file@loc Integrals/2311-Compute-Integral-0010.HELP.tex}

\[\int_{-2}^{10} {x^{2} - x - 20}\;dx=\answer{48}\]
\end{problem}}%}

%%%%%%%%%%%%%%%%%%%%%%


\latexProblemContent{
\begin{problem}

Use the Fundamental Theorem of Calculus to evaluate the integral.

\expandafter\input{\file@loc Integrals/2311-Compute-Integral-0010.HELP.tex}

\[\int_{-4}^{8} {x^{2} + 2 \, x + 1}\;dx=\answer{252}\]
\end{problem}}%}

%%%%%%%%%%%%%%%%%%%%%%


\latexProblemContent{
\begin{problem}

Use the Fundamental Theorem of Calculus to evaluate the integral.

\expandafter\input{\file@loc Integrals/2311-Compute-Integral-0010.HELP.tex}

\[\int_{3}^{7} {x^{3} - 15 \, x^{2} + 75 \, x - 125}\;dx=\answer{0}\]
\end{problem}}%}

%%%%%%%%%%%%%%%%%%%%%%


\latexProblemContent{
\begin{problem}

Use the Fundamental Theorem of Calculus to evaluate the integral.

\expandafter\input{\file@loc Integrals/2311-Compute-Integral-0010.HELP.tex}

\[\int_{0}^{7} {x^{3} - 15 \, x^{2} + 75 \, x - 125}\;dx=\answer{-\frac{609}{4}}\]
\end{problem}}%}

%%%%%%%%%%%%%%%%%%%%%%


\latexProblemContent{
\begin{problem}

Use the Fundamental Theorem of Calculus to evaluate the integral.

\expandafter\input{\file@loc Integrals/2311-Compute-Integral-0010.HELP.tex}

\[\int_{1}^{12} {x^{2} - 8 \, x + 16}\;dx=\answer{\frac{539}{3}}\]
\end{problem}}%}

%%%%%%%%%%%%%%%%%%%%%%


\latexProblemContent{
\begin{problem}

Use the Fundamental Theorem of Calculus to evaluate the integral.

\expandafter\input{\file@loc Integrals/2311-Compute-Integral-0010.HELP.tex}

\[\int_{-5}^{1} {x^{3} - 10 \, x^{2} + 32 \, x - 32}\;dx=\answer{-1152}\]
\end{problem}}%}

%%%%%%%%%%%%%%%%%%%%%%


\latexProblemContent{
\begin{problem}

Use the Fundamental Theorem of Calculus to evaluate the integral.

\expandafter\input{\file@loc Integrals/2311-Compute-Integral-0010.HELP.tex}

\[\int_{-3}^{-1} {x - 1}\;dx=\answer{-6}\]
\end{problem}}%}

%%%%%%%%%%%%%%%%%%%%%%


\latexProblemContent{
\begin{problem}

Use the Fundamental Theorem of Calculus to evaluate the integral.

\expandafter\input{\file@loc Integrals/2311-Compute-Integral-0010.HELP.tex}

\[\int_{2}^{5} {x^{2} + 8 \, x + 16}\;dx=\answer{171}\]
\end{problem}}%}

%%%%%%%%%%%%%%%%%%%%%%


\latexProblemContent{
\begin{problem}

Use the Fundamental Theorem of Calculus to evaluate the integral.

\expandafter\input{\file@loc Integrals/2311-Compute-Integral-0010.HELP.tex}

\[\int_{-1}^{-1} {x^{2} - 6 \, x + 9}\;dx=\answer{0}\]
\end{problem}}%}

%%%%%%%%%%%%%%%%%%%%%%


\latexProblemContent{
\begin{problem}

Use the Fundamental Theorem of Calculus to evaluate the integral.

\expandafter\input{\file@loc Integrals/2311-Compute-Integral-0010.HELP.tex}

\[\int_{-1}^{11} {x^{3} - 3 \, x^{2} + 3 \, x - 1}\;dx=\answer{2496}\]
\end{problem}}%}

%%%%%%%%%%%%%%%%%%%%%%


\latexProblemContent{
\begin{problem}

Use the Fundamental Theorem of Calculus to evaluate the integral.

\expandafter\input{\file@loc Integrals/2311-Compute-Integral-0010.HELP.tex}

\[\int_{-4}^{7} {x + 1}\;dx=\answer{\frac{55}{2}}\]
\end{problem}}%}

%%%%%%%%%%%%%%%%%%%%%%


\latexProblemContent{
\begin{problem}

Use the Fundamental Theorem of Calculus to evaluate the integral.

\expandafter\input{\file@loc Integrals/2311-Compute-Integral-0010.HELP.tex}

\[\int_{-1}^{5} {x^{3} - 2 \, x^{2} - 32 \, x + 96}\;dx=\answer{264}\]
\end{problem}}%}

%%%%%%%%%%%%%%%%%%%%%%


\latexProblemContent{
\begin{problem}

Use the Fundamental Theorem of Calculus to evaluate the integral.

\expandafter\input{\file@loc Integrals/2311-Compute-Integral-0010.HELP.tex}

\[\int_{3}^{10} {x^{3} - 14 \, x^{2} + 57 \, x - 72}\;dx=\answer{\frac{343}{12}}\]
\end{problem}}%}

%%%%%%%%%%%%%%%%%%%%%%


\latexProblemContent{
\begin{problem}

Use the Fundamental Theorem of Calculus to evaluate the integral.

\expandafter\input{\file@loc Integrals/2311-Compute-Integral-0010.HELP.tex}

\[\int_{-5}^{-1} {x^{2} - 10 \, x + 25}\;dx=\answer{\frac{784}{3}}\]
\end{problem}}%}

%%%%%%%%%%%%%%%%%%%%%%


\latexProblemContent{
\begin{problem}

Use the Fundamental Theorem of Calculus to evaluate the integral.

\expandafter\input{\file@loc Integrals/2311-Compute-Integral-0010.HELP.tex}

\[\int_{3}^{5} {x^{3} + 9 \, x^{2} + 15 \, x + 7}\;dx=\answer{564}\]
\end{problem}}%}

%%%%%%%%%%%%%%%%%%%%%%


\latexProblemContent{
\begin{problem}

Use the Fundamental Theorem of Calculus to evaluate the integral.

\expandafter\input{\file@loc Integrals/2311-Compute-Integral-0010.HELP.tex}

\[\int_{3}^{4} {x^{2} + 7 \, x + 6}\;dx=\answer{\frac{257}{6}}\]
\end{problem}}%}

%%%%%%%%%%%%%%%%%%%%%%


\latexProblemContent{
\begin{problem}

Use the Fundamental Theorem of Calculus to evaluate the integral.

\expandafter\input{\file@loc Integrals/2311-Compute-Integral-0010.HELP.tex}

\[\int_{-4}^{1} {x^{3} + 9 \, x^{2} + 27 \, x + 27}\;dx=\answer{\frac{255}{4}}\]
\end{problem}}%}

%%%%%%%%%%%%%%%%%%%%%%


\latexProblemContent{
\begin{problem}

Use the Fundamental Theorem of Calculus to evaluate the integral.

\expandafter\input{\file@loc Integrals/2311-Compute-Integral-0010.HELP.tex}

\[\int_{4}^{8} {x^{2} + 2 \, x + 1}\;dx=\answer{\frac{604}{3}}\]
\end{problem}}%}

%%%%%%%%%%%%%%%%%%%%%%


\latexProblemContent{
\begin{problem}

Use the Fundamental Theorem of Calculus to evaluate the integral.

\expandafter\input{\file@loc Integrals/2311-Compute-Integral-0010.HELP.tex}

\[\int_{-6}^{-2} {x^{3} - 3 \, x^{2} + 3 \, x - 1}\;dx=\answer{-580}\]
\end{problem}}%}

%%%%%%%%%%%%%%%%%%%%%%


\latexProblemContent{
\begin{problem}

Use the Fundamental Theorem of Calculus to evaluate the integral.

\expandafter\input{\file@loc Integrals/2311-Compute-Integral-0010.HELP.tex}

\[\int_{-3}^{3} {x - 2}\;dx=\answer{-12}\]
\end{problem}}%}

%%%%%%%%%%%%%%%%%%%%%%


\latexProblemContent{
\begin{problem}

Use the Fundamental Theorem of Calculus to evaluate the integral.

\expandafter\input{\file@loc Integrals/2311-Compute-Integral-0010.HELP.tex}

\[\int_{-2}^{3} {x^{3} - 2 \, x^{2} - 39 \, x - 72}\;dx=\answer{-\frac{5575}{12}}\]
\end{problem}}%}

%%%%%%%%%%%%%%%%%%%%%%


\latexProblemContent{
\begin{problem}

Use the Fundamental Theorem of Calculus to evaluate the integral.

\expandafter\input{\file@loc Integrals/2311-Compute-Integral-0010.HELP.tex}

\[\int_{-7}^{10} {x^{2} + 2 \, x + 1}\;dx=\answer{\frac{1547}{3}}\]
\end{problem}}%}

%%%%%%%%%%%%%%%%%%%%%%


\latexProblemContent{
\begin{problem}

Use the Fundamental Theorem of Calculus to evaluate the integral.

\expandafter\input{\file@loc Integrals/2311-Compute-Integral-0010.HELP.tex}

\[\int_{-8}^{-7} {x^{2} - 10 \, x + 25}\;dx=\answer{\frac{469}{3}}\]
\end{problem}}%}

%%%%%%%%%%%%%%%%%%%%%%


\latexProblemContent{
\begin{problem}

Use the Fundamental Theorem of Calculus to evaluate the integral.

\expandafter\input{\file@loc Integrals/2311-Compute-Integral-0010.HELP.tex}

\[\int_{-1}^{2} {x^{2} - 13 \, x + 40}\;dx=\answer{\frac{207}{2}}\]
\end{problem}}%}

%%%%%%%%%%%%%%%%%%%%%%


\latexProblemContent{
\begin{problem}

Use the Fundamental Theorem of Calculus to evaluate the integral.

\expandafter\input{\file@loc Integrals/2311-Compute-Integral-0010.HELP.tex}

\[\int_{2}^{5} {x^{3} + 3 \, x^{2} + 3 \, x + 1}\;dx=\answer{\frac{1215}{4}}\]
\end{problem}}%}

%%%%%%%%%%%%%%%%%%%%%%


\latexProblemContent{
\begin{problem}

Use the Fundamental Theorem of Calculus to evaluate the integral.

\expandafter\input{\file@loc Integrals/2311-Compute-Integral-0010.HELP.tex}

\[\int_{1}^{5} {x^{3} - 3 \, x^{2} + 3 \, x - 1}\;dx=\answer{64}\]
\end{problem}}%}

%%%%%%%%%%%%%%%%%%%%%%


\latexProblemContent{
\begin{problem}

Use the Fundamental Theorem of Calculus to evaluate the integral.

\expandafter\input{\file@loc Integrals/2311-Compute-Integral-0010.HELP.tex}

\[\int_{1}^{8} {x^{2} + 6 \, x - 16}\;dx=\answer{\frac{742}{3}}\]
\end{problem}}%}

%%%%%%%%%%%%%%%%%%%%%%


\latexProblemContent{
\begin{problem}

Use the Fundamental Theorem of Calculus to evaluate the integral.

\expandafter\input{\file@loc Integrals/2311-Compute-Integral-0010.HELP.tex}

\[\int_{-3}^{9} {x^{3} + 16 \, x^{2} + 80 \, x + 128}\;dx=\answer{10068}\]
\end{problem}}%}

%%%%%%%%%%%%%%%%%%%%%%


\latexProblemContent{
\begin{problem}

Use the Fundamental Theorem of Calculus to evaluate the integral.

\expandafter\input{\file@loc Integrals/2311-Compute-Integral-0010.HELP.tex}

\[\int_{-10}^{-10} {x - 3}\;dx=\answer{0}\]
\end{problem}}%}

%%%%%%%%%%%%%%%%%%%%%%


\latexProblemContent{
\begin{problem}

Use the Fundamental Theorem of Calculus to evaluate the integral.

\expandafter\input{\file@loc Integrals/2311-Compute-Integral-0010.HELP.tex}

\[\int_{-3}^{12} {x^{3} - 3 \, x^{2} + 3 \, x - 1}\;dx=\answer{\frac{14385}{4}}\]
\end{problem}}%}

%%%%%%%%%%%%%%%%%%%%%%


\latexProblemContent{
\begin{problem}

Use the Fundamental Theorem of Calculus to evaluate the integral.

\expandafter\input{\file@loc Integrals/2311-Compute-Integral-0010.HELP.tex}

\[\int_{-3}^{4} {x + 4}\;dx=\answer{\frac{63}{2}}\]
\end{problem}}%}

%%%%%%%%%%%%%%%%%%%%%%


\latexProblemContent{
\begin{problem}

Use the Fundamental Theorem of Calculus to evaluate the integral.

\expandafter\input{\file@loc Integrals/2311-Compute-Integral-0010.HELP.tex}

\[\int_{-1}^{8} {x^{3} - 3 \, x^{2} + 3 \, x - 1}\;dx=\answer{\frac{2385}{4}}\]
\end{problem}}%}

%%%%%%%%%%%%%%%%%%%%%%


\latexProblemContent{
\begin{problem}

Use the Fundamental Theorem of Calculus to evaluate the integral.

\expandafter\input{\file@loc Integrals/2311-Compute-Integral-0010.HELP.tex}

\[\int_{-9}^{9} {x^{3} + 9 \, x^{2} + 27 \, x + 27}\;dx=\answer{4860}\]
\end{problem}}%}

%%%%%%%%%%%%%%%%%%%%%%


\latexProblemContent{
\begin{problem}

Use the Fundamental Theorem of Calculus to evaluate the integral.

\expandafter\input{\file@loc Integrals/2311-Compute-Integral-0010.HELP.tex}

\[\int_{-5}^{6} {x + 3}\;dx=\answer{\frac{77}{2}}\]
\end{problem}}%}

%%%%%%%%%%%%%%%%%%%%%%


\latexProblemContent{
\begin{problem}

Use the Fundamental Theorem of Calculus to evaluate the integral.

\expandafter\input{\file@loc Integrals/2311-Compute-Integral-0010.HELP.tex}

\[\int_{-5}^{-5} {x^{3} - 27 \, x - 54}\;dx=\answer{0}\]
\end{problem}}%}

%%%%%%%%%%%%%%%%%%%%%%


\latexProblemContent{
\begin{problem}

Use the Fundamental Theorem of Calculus to evaluate the integral.

\expandafter\input{\file@loc Integrals/2311-Compute-Integral-0010.HELP.tex}

\[\int_{2}^{8} {x^{2} - 2 \, x - 35}\;dx=\answer{-102}\]
\end{problem}}%}

%%%%%%%%%%%%%%%%%%%%%%


\latexProblemContent{
\begin{problem}

Use the Fundamental Theorem of Calculus to evaluate the integral.

\expandafter\input{\file@loc Integrals/2311-Compute-Integral-0010.HELP.tex}

\[\int_{-4}^{8} {x^{3} + 12 \, x^{2} + 48 \, x + 64}\;dx=\answer{5184}\]
\end{problem}}%}

%%%%%%%%%%%%%%%%%%%%%%


\latexProblemContent{
\begin{problem}

Use the Fundamental Theorem of Calculus to evaluate the integral.

\expandafter\input{\file@loc Integrals/2311-Compute-Integral-0010.HELP.tex}

\[\int_{-8}^{3} {x^{3} + 6 \, x^{2} + 12 \, x + 8}\;dx=\answer{-\frac{671}{4}}\]
\end{problem}}%}

%%%%%%%%%%%%%%%%%%%%%%


\latexProblemContent{
\begin{problem}

Use the Fundamental Theorem of Calculus to evaluate the integral.

\expandafter\input{\file@loc Integrals/2311-Compute-Integral-0010.HELP.tex}

\[\int_{-2}^{7} {x - 5}\;dx=\answer{-\frac{45}{2}}\]
\end{problem}}%}

%%%%%%%%%%%%%%%%%%%%%%


\latexProblemContent{
\begin{problem}

Use the Fundamental Theorem of Calculus to evaluate the integral.

\expandafter\input{\file@loc Integrals/2311-Compute-Integral-0010.HELP.tex}

\[\int_{5}^{5} {x^{2} + 2 \, x + 1}\;dx=\answer{0}\]
\end{problem}}%}

%%%%%%%%%%%%%%%%%%%%%%


\latexProblemContent{
\begin{problem}

Use the Fundamental Theorem of Calculus to evaluate the integral.

\expandafter\input{\file@loc Integrals/2311-Compute-Integral-0010.HELP.tex}

\[\int_{5}^{7} {x^{2} - 4 \, x - 32}\;dx=\answer{-\frac{118}{3}}\]
\end{problem}}%}

%%%%%%%%%%%%%%%%%%%%%%


\latexProblemContent{
\begin{problem}

Use the Fundamental Theorem of Calculus to evaluate the integral.

\expandafter\input{\file@loc Integrals/2311-Compute-Integral-0010.HELP.tex}

\[\int_{4}^{5} {x^{2} - 2 \, x + 1}\;dx=\answer{\frac{37}{3}}\]
\end{problem}}%}

%%%%%%%%%%%%%%%%%%%%%%


%%%%%%%%%%%%%%%%%%%%%%


\latexProblemContent{
\begin{problem}

Use the Fundamental Theorem of Calculus to evaluate the integral.

\expandafter\input{\file@loc Integrals/2311-Compute-Integral-0010.HELP.tex}

\[\int_{-6}^{-4} {x^{2} - 3 \, x - 4}\;dx=\answer{\frac{218}{3}}\]
\end{problem}}%}

%%%%%%%%%%%%%%%%%%%%%%


\latexProblemContent{
\begin{problem}

Use the Fundamental Theorem of Calculus to evaluate the integral.

\expandafter\input{\file@loc Integrals/2311-Compute-Integral-0010.HELP.tex}

\[\int_{-5}^{8} {x^{3} - x^{2} - 21 \, x + 45}\;dx=\answer{\frac{9971}{12}}\]
\end{problem}}%}

%%%%%%%%%%%%%%%%%%%%%%


\latexProblemContent{
\begin{problem}

Use the Fundamental Theorem of Calculus to evaluate the integral.

\expandafter\input{\file@loc Integrals/2311-Compute-Integral-0010.HELP.tex}

\[\int_{1}^{10} {x^{3} + 5 \, x^{2} - 13 \, x + 7}\;dx=\answer{\frac{14337}{4}}\]
\end{problem}}%}

%%%%%%%%%%%%%%%%%%%%%%


\latexProblemContent{
\begin{problem}

Use the Fundamental Theorem of Calculus to evaluate the integral.

\expandafter\input{\file@loc Integrals/2311-Compute-Integral-0010.HELP.tex}

\[\int_{0}^{0} {x^{2} + 5 \, x + 6}\;dx=\answer{0}\]
\end{problem}}%}

%%%%%%%%%%%%%%%%%%%%%%


\latexProblemContent{
\begin{problem}

Use the Fundamental Theorem of Calculus to evaluate the integral.

\expandafter\input{\file@loc Integrals/2311-Compute-Integral-0010.HELP.tex}

\[\int_{-9}^{9} {x^{2} + 2 \, x - 24}\;dx=\answer{54}\]
\end{problem}}%}

%%%%%%%%%%%%%%%%%%%%%%


\latexProblemContent{
\begin{problem}

Use the Fundamental Theorem of Calculus to evaluate the integral.

\expandafter\input{\file@loc Integrals/2311-Compute-Integral-0010.HELP.tex}

\[\int_{-1}^{5} {x^{3} + 12 \, x^{2} + 48 \, x + 64}\;dx=\answer{1620}\]
\end{problem}}%}

%%%%%%%%%%%%%%%%%%%%%%


\latexProblemContent{
\begin{problem}

Use the Fundamental Theorem of Calculus to evaluate the integral.

\expandafter\input{\file@loc Integrals/2311-Compute-Integral-0010.HELP.tex}

\[\int_{5}^{6} {x^{2} - 8 \, x + 16}\;dx=\answer{\frac{7}{3}}\]
\end{problem}}%}

%%%%%%%%%%%%%%%%%%%%%%


\latexProblemContent{
\begin{problem}

Use the Fundamental Theorem of Calculus to evaluate the integral.

\expandafter\input{\file@loc Integrals/2311-Compute-Integral-0010.HELP.tex}

\[\int_{5}^{6} {x^{3} - 3 \, x^{2} + 3 \, x - 1}\;dx=\answer{\frac{369}{4}}\]
\end{problem}}%}

%%%%%%%%%%%%%%%%%%%%%%


\latexProblemContent{
\begin{problem}

Use the Fundamental Theorem of Calculus to evaluate the integral.

\expandafter\input{\file@loc Integrals/2311-Compute-Integral-0010.HELP.tex}

\[\int_{0}^{10} {x^{2} - 6 \, x + 9}\;dx=\answer{\frac{370}{3}}\]
\end{problem}}%}

%%%%%%%%%%%%%%%%%%%%%%


\latexProblemContent{
\begin{problem}

Use the Fundamental Theorem of Calculus to evaluate the integral.

\expandafter\input{\file@loc Integrals/2311-Compute-Integral-0010.HELP.tex}

\[\int_{2}^{4} {x^{2} - 6 \, x + 9}\;dx=\answer{\frac{2}{3}}\]
\end{problem}}%}

%%%%%%%%%%%%%%%%%%%%%%


%%%%%%%%%%%%%%%%%%%%%%


\latexProblemContent{
\begin{problem}

Use the Fundamental Theorem of Calculus to evaluate the integral.

\expandafter\input{\file@loc Integrals/2311-Compute-Integral-0010.HELP.tex}

\[\int_{-4}^{0} {x^{2} + 2 \, x - 15}\;dx=\answer{-\frac{164}{3}}\]
\end{problem}}%}

%%%%%%%%%%%%%%%%%%%%%%


\latexProblemContent{
\begin{problem}

Use the Fundamental Theorem of Calculus to evaluate the integral.

\expandafter\input{\file@loc Integrals/2311-Compute-Integral-0010.HELP.tex}

\[\int_{-8}^{10} {x^{2} - 4 \, x - 21}\;dx=\answer{54}\]
\end{problem}}%}

%%%%%%%%%%%%%%%%%%%%%%


\latexProblemContent{
\begin{problem}

Use the Fundamental Theorem of Calculus to evaluate the integral.

\expandafter\input{\file@loc Integrals/2311-Compute-Integral-0010.HELP.tex}

\[\int_{-6}^{3} {x + 1}\;dx=\answer{-\frac{9}{2}}\]
\end{problem}}%}

%%%%%%%%%%%%%%%%%%%%%%


\latexProblemContent{
\begin{problem}

Use the Fundamental Theorem of Calculus to evaluate the integral.

\expandafter\input{\file@loc Integrals/2311-Compute-Integral-0010.HELP.tex}

\[\int_{-4}^{0} {x^{2} + 7 \, x + 12}\;dx=\answer{\frac{40}{3}}\]
\end{problem}}%}

%%%%%%%%%%%%%%%%%%%%%%


\latexProblemContent{
\begin{problem}

Use the Fundamental Theorem of Calculus to evaluate the integral.

\expandafter\input{\file@loc Integrals/2311-Compute-Integral-0010.HELP.tex}

\[\int_{4}^{11} {x^{2} - 5 \, x - 6}\;dx=\answer{\frac{707}{6}}\]
\end{problem}}%}

%%%%%%%%%%%%%%%%%%%%%%


\latexProblemContent{
\begin{problem}

Use the Fundamental Theorem of Calculus to evaluate the integral.

\expandafter\input{\file@loc Integrals/2311-Compute-Integral-0010.HELP.tex}

\[\int_{-9}^{5} {x + 2}\;dx=\answer{0}\]
\end{problem}}%}

%%%%%%%%%%%%%%%%%%%%%%


\latexProblemContent{
\begin{problem}

Use the Fundamental Theorem of Calculus to evaluate the integral.

\expandafter\input{\file@loc Integrals/2311-Compute-Integral-0010.HELP.tex}

\[\int_{-7}^{-3} {x^{2} + 8 \, x + 16}\;dx=\answer{\frac{28}{3}}\]
\end{problem}}%}

%%%%%%%%%%%%%%%%%%%%%%


\latexProblemContent{
\begin{problem}

Use the Fundamental Theorem of Calculus to evaluate the integral.

\expandafter\input{\file@loc Integrals/2311-Compute-Integral-0010.HELP.tex}

\[\int_{-2}^{-2} {x^{3} + 9 \, x^{2} + 27 \, x + 27}\;dx=\answer{0}\]
\end{problem}}%}

%%%%%%%%%%%%%%%%%%%%%%


\latexProblemContent{
\begin{problem}

Use the Fundamental Theorem of Calculus to evaluate the integral.

\expandafter\input{\file@loc Integrals/2311-Compute-Integral-0010.HELP.tex}

\[\int_{-5}^{7} {x^{2} + 10 \, x + 25}\;dx=\answer{576}\]
\end{problem}}%}

%%%%%%%%%%%%%%%%%%%%%%


\latexProblemContent{
\begin{problem}

Use the Fundamental Theorem of Calculus to evaluate the integral.

\expandafter\input{\file@loc Integrals/2311-Compute-Integral-0010.HELP.tex}

\[\int_{-6}^{5} {x^{2} - 9}\;dx=\answer{\frac{44}{3}}\]
\end{problem}}%}

%%%%%%%%%%%%%%%%%%%%%%


\latexProblemContent{
\begin{problem}

Use the Fundamental Theorem of Calculus to evaluate the integral.

\expandafter\input{\file@loc Integrals/2311-Compute-Integral-0010.HELP.tex}

\[\int_{-1}^{4} {x^{2} + 7 \, x + 6}\;dx=\answer{\frac{625}{6}}\]
\end{problem}}%}

%%%%%%%%%%%%%%%%%%%%%%


\latexProblemContent{
\begin{problem}

Use the Fundamental Theorem of Calculus to evaluate the integral.

\expandafter\input{\file@loc Integrals/2311-Compute-Integral-0010.HELP.tex}

\[\int_{-9}^{3} {x - 2}\;dx=\answer{-60}\]
\end{problem}}%}

%%%%%%%%%%%%%%%%%%%%%%


\latexProblemContent{
\begin{problem}

Use the Fundamental Theorem of Calculus to evaluate the integral.

\expandafter\input{\file@loc Integrals/2311-Compute-Integral-0010.HELP.tex}

\[\int_{-2}^{2} {x^{3} + 7 \, x^{2} + 16 \, x + 12}\;dx=\answer{\frac{256}{3}}\]
\end{problem}}%}

%%%%%%%%%%%%%%%%%%%%%%


\latexProblemContent{
\begin{problem}

Use the Fundamental Theorem of Calculus to evaluate the integral.

\expandafter\input{\file@loc Integrals/2311-Compute-Integral-0010.HELP.tex}

\[\int_{-1}^{1} {x^{3} - 15 \, x^{2} + 75 \, x - 125}\;dx=\answer{-260}\]
\end{problem}}%}

%%%%%%%%%%%%%%%%%%%%%%


\latexProblemContent{
\begin{problem}

Use the Fundamental Theorem of Calculus to evaluate the integral.

\expandafter\input{\file@loc Integrals/2311-Compute-Integral-0010.HELP.tex}

\[\int_{-4}^{-2} {x^{2} + 8 \, x + 16}\;dx=\answer{\frac{8}{3}}\]
\end{problem}}%}

%%%%%%%%%%%%%%%%%%%%%%


\latexProblemContent{
\begin{problem}

Use the Fundamental Theorem of Calculus to evaluate the integral.

\expandafter\input{\file@loc Integrals/2311-Compute-Integral-0010.HELP.tex}

\[\int_{4}^{5} {x^{3} + 12 \, x^{2} + 36 \, x + 32}\;dx=\answer{\frac{2121}{4}}\]
\end{problem}}%}

%%%%%%%%%%%%%%%%%%%%%%


\latexProblemContent{
\begin{problem}

Use the Fundamental Theorem of Calculus to evaluate the integral.

\expandafter\input{\file@loc Integrals/2311-Compute-Integral-0010.HELP.tex}

\[\int_{-10}^{-8} {x^{2} + 4 \, x + 4}\;dx=\answer{\frac{296}{3}}\]
\end{problem}}%}

%%%%%%%%%%%%%%%%%%%%%%


\latexProblemContent{
\begin{problem}

Use the Fundamental Theorem of Calculus to evaluate the integral.

\expandafter\input{\file@loc Integrals/2311-Compute-Integral-0010.HELP.tex}

\[\int_{-1}^{-1} {x^{3} + 12 \, x^{2} + 48 \, x + 64}\;dx=\answer{0}\]
\end{problem}}%}

%%%%%%%%%%%%%%%%%%%%%%


\latexProblemContent{
\begin{problem}

Use the Fundamental Theorem of Calculus to evaluate the integral.

\expandafter\input{\file@loc Integrals/2311-Compute-Integral-0010.HELP.tex}

\[\int_{-4}^{8} {x^{2} + 11 \, x + 28}\;dx=\answer{792}\]
\end{problem}}%}

%%%%%%%%%%%%%%%%%%%%%%


\latexProblemContent{
\begin{problem}

Use the Fundamental Theorem of Calculus to evaluate the integral.

\expandafter\input{\file@loc Integrals/2311-Compute-Integral-0010.HELP.tex}

\[\int_{-2}^{12} {x^{3} + x^{2} - x - 1}\;dx=\answer{\frac{17024}{3}}\]
\end{problem}}%}

%%%%%%%%%%%%%%%%%%%%%%


\latexProblemContent{
\begin{problem}

Use the Fundamental Theorem of Calculus to evaluate the integral.

\expandafter\input{\file@loc Integrals/2311-Compute-Integral-0010.HELP.tex}

\[\int_{-8}^{2} {x^{2} + 5 \, x + 4}\;dx=\answer{\frac{190}{3}}\]
\end{problem}}%}

%%%%%%%%%%%%%%%%%%%%%%


\latexProblemContent{
\begin{problem}

Use the Fundamental Theorem of Calculus to evaluate the integral.

\expandafter\input{\file@loc Integrals/2311-Compute-Integral-0010.HELP.tex}

\[\int_{-4}^{11} {x + 5}\;dx=\answer{\frac{255}{2}}\]
\end{problem}}%}

%%%%%%%%%%%%%%%%%%%%%%


\latexProblemContent{
\begin{problem}

Use the Fundamental Theorem of Calculus to evaluate the integral.

\expandafter\input{\file@loc Integrals/2311-Compute-Integral-0010.HELP.tex}

\[\int_{3}^{11} {x^{2} - 10 \, x + 25}\;dx=\answer{\frac{224}{3}}\]
\end{problem}}%}

%%%%%%%%%%%%%%%%%%%%%%


\latexProblemContent{
\begin{problem}

Use the Fundamental Theorem of Calculus to evaluate the integral.

\expandafter\input{\file@loc Integrals/2311-Compute-Integral-0010.HELP.tex}

\[\int_{-1}^{5} {x^{3} - 9 \, x^{2} + 27 \, x - 27}\;dx=\answer{-60}\]
\end{problem}}%}

%%%%%%%%%%%%%%%%%%%%%%


\latexProblemContent{
\begin{problem}

Use the Fundamental Theorem of Calculus to evaluate the integral.

\expandafter\input{\file@loc Integrals/2311-Compute-Integral-0010.HELP.tex}

\[\int_{-8}^{2} {x^{2} + 4 \, x + 4}\;dx=\answer{\frac{280}{3}}\]
\end{problem}}%}

%%%%%%%%%%%%%%%%%%%%%%


\latexProblemContent{
\begin{problem}

Use the Fundamental Theorem of Calculus to evaluate the integral.

\expandafter\input{\file@loc Integrals/2311-Compute-Integral-0010.HELP.tex}

\[\int_{-6}^{11} {x^{2} - 2 \, x + 1}\;dx=\answer{\frac{1343}{3}}\]
\end{problem}}%}

%%%%%%%%%%%%%%%%%%%%%%


\latexProblemContent{
\begin{problem}

Use the Fundamental Theorem of Calculus to evaluate the integral.

\expandafter\input{\file@loc Integrals/2311-Compute-Integral-0010.HELP.tex}

\[\int_{-10}^{-10} {x^{2} - 2 \, x}\;dx=\answer{0}\]
\end{problem}}%}

%%%%%%%%%%%%%%%%%%%%%%


\latexProblemContent{
\begin{problem}

Use the Fundamental Theorem of Calculus to evaluate the integral.

\expandafter\input{\file@loc Integrals/2311-Compute-Integral-0010.HELP.tex}

\[\int_{-2}^{12} {x^{2} - 8 \, x + 16}\;dx=\answer{\frac{728}{3}}\]
\end{problem}}%}

%%%%%%%%%%%%%%%%%%%%%%


\latexProblemContent{
\begin{problem}

Use the Fundamental Theorem of Calculus to evaluate the integral.

\expandafter\input{\file@loc Integrals/2311-Compute-Integral-0010.HELP.tex}

\[\int_{-9}^{11} {x^{3} - 12 \, x^{2} + 48 \, x - 64}\;dx=\answer{-6540}\]
\end{problem}}%}

%%%%%%%%%%%%%%%%%%%%%%


\latexProblemContent{
\begin{problem}

Use the Fundamental Theorem of Calculus to evaluate the integral.

\expandafter\input{\file@loc Integrals/2311-Compute-Integral-0010.HELP.tex}

\[\int_{-2}^{4} {x^{3} - 12 \, x^{2} + 45 \, x - 50}\;dx=\answer{-258}\]
\end{problem}}%}

%%%%%%%%%%%%%%%%%%%%%%


\latexProblemContent{
\begin{problem}

Use the Fundamental Theorem of Calculus to evaluate the integral.

\expandafter\input{\file@loc Integrals/2311-Compute-Integral-0010.HELP.tex}

\[\int_{4}^{7} {x^{2} - 4 \, x + 4}\;dx=\answer{39}\]
\end{problem}}%}

%%%%%%%%%%%%%%%%%%%%%%


\latexProblemContent{
\begin{problem}

Use the Fundamental Theorem of Calculus to evaluate the integral.

\expandafter\input{\file@loc Integrals/2311-Compute-Integral-0010.HELP.tex}

\[\int_{3}^{4} {x^{2} - 4 \, x + 4}\;dx=\answer{\frac{7}{3}}\]
\end{problem}}%}

%%%%%%%%%%%%%%%%%%%%%%


\latexProblemContent{
\begin{problem}

Use the Fundamental Theorem of Calculus to evaluate the integral.

\expandafter\input{\file@loc Integrals/2311-Compute-Integral-0010.HELP.tex}

\[\int_{-8}^{2} {x - 2}\;dx=\answer{-50}\]
\end{problem}}%}

%%%%%%%%%%%%%%%%%%%%%%


\latexProblemContent{
\begin{problem}

Use the Fundamental Theorem of Calculus to evaluate the integral.

\expandafter\input{\file@loc Integrals/2311-Compute-Integral-0010.HELP.tex}

\[\int_{-7}^{-4} {x^{2} + 4 \, x - 32}\;dx=\answer{-69}\]
\end{problem}}%}

%%%%%%%%%%%%%%%%%%%%%%


\latexProblemContent{
\begin{problem}

Use the Fundamental Theorem of Calculus to evaluate the integral.

\expandafter\input{\file@loc Integrals/2311-Compute-Integral-0010.HELP.tex}

\[\int_{-7}^{-6} {x^{2} - x - 2}\;dx=\answer{\frac{281}{6}}\]
\end{problem}}%}

%%%%%%%%%%%%%%%%%%%%%%


\latexProblemContent{
\begin{problem}

Use the Fundamental Theorem of Calculus to evaluate the integral.

\expandafter\input{\file@loc Integrals/2311-Compute-Integral-0010.HELP.tex}

\[\int_{-8}^{-1} {x^{3} + 3 \, x^{2} + 3 \, x + 1}\;dx=\answer{-\frac{2401}{4}}\]
\end{problem}}%}

%%%%%%%%%%%%%%%%%%%%%%


\latexProblemContent{
\begin{problem}

Use the Fundamental Theorem of Calculus to evaluate the integral.

\expandafter\input{\file@loc Integrals/2311-Compute-Integral-0010.HELP.tex}

\[\int_{-2}^{12} {x^{3} + x^{2} - 5 \, x + 3}\;dx=\answer{\frac{16352}{3}}\]
\end{problem}}%}

%%%%%%%%%%%%%%%%%%%%%%


\latexProblemContent{
\begin{problem}

Use the Fundamental Theorem of Calculus to evaluate the integral.

\expandafter\input{\file@loc Integrals/2311-Compute-Integral-0010.HELP.tex}

\[\int_{-2}^{7} {x^{2} - 8 \, x + 16}\;dx=\answer{81}\]
\end{problem}}%}

%%%%%%%%%%%%%%%%%%%%%%


\latexProblemContent{
\begin{problem}

Use the Fundamental Theorem of Calculus to evaluate the integral.

\expandafter\input{\file@loc Integrals/2311-Compute-Integral-0010.HELP.tex}

\[\int_{1}^{4} {x^{3} + 5 \, x^{2} - 8 \, x - 48}\;dx=\answer{-\frac{141}{4}}\]
\end{problem}}%}

%%%%%%%%%%%%%%%%%%%%%%


\latexProblemContent{
\begin{problem}

Use the Fundamental Theorem of Calculus to evaluate the integral.

\expandafter\input{\file@loc Integrals/2311-Compute-Integral-0010.HELP.tex}

\[\int_{4}^{12} {x^{2} - x - 30}\;dx=\answer{\frac{752}{3}}\]
\end{problem}}%}

%%%%%%%%%%%%%%%%%%%%%%


\latexProblemContent{
\begin{problem}

Use the Fundamental Theorem of Calculus to evaluate the integral.

\expandafter\input{\file@loc Integrals/2311-Compute-Integral-0010.HELP.tex}

\[\int_{1}^{4} {x^{3} - 9 \, x^{2} + 27 \, x - 27}\;dx=\answer{-\frac{15}{4}}\]
\end{problem}}%}

%%%%%%%%%%%%%%%%%%%%%%


\latexProblemContent{
\begin{problem}

Use the Fundamental Theorem of Calculus to evaluate the integral.

\expandafter\input{\file@loc Integrals/2311-Compute-Integral-0010.HELP.tex}

\[\int_{-3}^{4} {x + 3}\;dx=\answer{\frac{49}{2}}\]
\end{problem}}%}

%%%%%%%%%%%%%%%%%%%%%%


\latexProblemContent{
\begin{problem}

Use the Fundamental Theorem of Calculus to evaluate the integral.

\expandafter\input{\file@loc Integrals/2311-Compute-Integral-0010.HELP.tex}

\[\int_{-6}^{-5} {x^{3} - 6 \, x^{2} + 12 \, x - 8}\;dx=\answer{-\frac{1695}{4}}\]
\end{problem}}%}

%%%%%%%%%%%%%%%%%%%%%%


\latexProblemContent{
\begin{problem}

Use the Fundamental Theorem of Calculus to evaluate the integral.

\expandafter\input{\file@loc Integrals/2311-Compute-Integral-0010.HELP.tex}

\[\int_{-4}^{3} {x^{2} + 8 \, x + 16}\;dx=\answer{\frac{343}{3}}\]
\end{problem}}%}

%%%%%%%%%%%%%%%%%%%%%%


\latexProblemContent{
\begin{problem}

Use the Fundamental Theorem of Calculus to evaluate the integral.

\expandafter\input{\file@loc Integrals/2311-Compute-Integral-0010.HELP.tex}

\[\int_{0}^{6} {x^{3} - 9 \, x^{2} + 27 \, x - 27}\;dx=\answer{0}\]
\end{problem}}%}

%%%%%%%%%%%%%%%%%%%%%%


\latexProblemContent{
\begin{problem}

Use the Fundamental Theorem of Calculus to evaluate the integral.

\expandafter\input{\file@loc Integrals/2311-Compute-Integral-0010.HELP.tex}

\[\int_{-1}^{7} {x^{3} + 7 \, x^{2} - 5 \, x - 75}\;dx=\answer{\frac{2048}{3}}\]
\end{problem}}%}

%%%%%%%%%%%%%%%%%%%%%%


\latexProblemContent{
\begin{problem}

Use the Fundamental Theorem of Calculus to evaluate the integral.

\expandafter\input{\file@loc Integrals/2311-Compute-Integral-0010.HELP.tex}

\[\int_{-3}^{-1} {x^{2} + 6 \, x + 9}\;dx=\answer{\frac{8}{3}}\]
\end{problem}}%}

%%%%%%%%%%%%%%%%%%%%%%


\latexProblemContent{
\begin{problem}

Use the Fundamental Theorem of Calculus to evaluate the integral.

\expandafter\input{\file@loc Integrals/2311-Compute-Integral-0010.HELP.tex}

\[\int_{-7}^{-7} {x^{2} - 4 \, x - 32}\;dx=\answer{0}\]
\end{problem}}%}

%%%%%%%%%%%%%%%%%%%%%%


\latexProblemContent{
\begin{problem}

Use the Fundamental Theorem of Calculus to evaluate the integral.

\expandafter\input{\file@loc Integrals/2311-Compute-Integral-0010.HELP.tex}

\[\int_{-5}^{11} {x^{3} - 3 \, x^{2} + 3 \, x - 1}\;dx=\answer{2176}\]
\end{problem}}%}

%%%%%%%%%%%%%%%%%%%%%%


\latexProblemContent{
\begin{problem}

Use the Fundamental Theorem of Calculus to evaluate the integral.

\expandafter\input{\file@loc Integrals/2311-Compute-Integral-0010.HELP.tex}

\[\int_{-7}^{11} {x^{3} - 15 \, x^{2} + 75 \, x - 125}\;dx=\answer{-4860}\]
\end{problem}}%}

%%%%%%%%%%%%%%%%%%%%%%


\latexProblemContent{
\begin{problem}

Use the Fundamental Theorem of Calculus to evaluate the integral.

\expandafter\input{\file@loc Integrals/2311-Compute-Integral-0010.HELP.tex}

\[\int_{-10}^{0} {x + 1}\;dx=\answer{-40}\]
\end{problem}}%}

%%%%%%%%%%%%%%%%%%%%%%


\latexProblemContent{
\begin{problem}

Use the Fundamental Theorem of Calculus to evaluate the integral.

\expandafter\input{\file@loc Integrals/2311-Compute-Integral-0010.HELP.tex}

\[\int_{4}^{5} {x^{3} + 4 \, x^{2} - 35 \, x - 150}\;dx=\answer{-\frac{1607}{12}}\]
\end{problem}}%}

%%%%%%%%%%%%%%%%%%%%%%


\latexProblemContent{
\begin{problem}

Use the Fundamental Theorem of Calculus to evaluate the integral.

\expandafter\input{\file@loc Integrals/2311-Compute-Integral-0010.HELP.tex}

\[\int_{-2}^{5} {x^{3} - 11 \, x^{2} + 32 \, x - 28}\;dx=\answer{-\frac{2345}{12}}\]
\end{problem}}%}

%%%%%%%%%%%%%%%%%%%%%%


\latexProblemContent{
\begin{problem}

Use the Fundamental Theorem of Calculus to evaluate the integral.

\expandafter\input{\file@loc Integrals/2311-Compute-Integral-0010.HELP.tex}

\[\int_{2}^{4} {x^{2} + 10 \, x + 25}\;dx=\answer{\frac{386}{3}}\]
\end{problem}}%}

%%%%%%%%%%%%%%%%%%%%%%


\latexProblemContent{
\begin{problem}

Use the Fundamental Theorem of Calculus to evaluate the integral.

\expandafter\input{\file@loc Integrals/2311-Compute-Integral-0010.HELP.tex}

\[\int_{2}^{5} {x^{2} + 7 \, x + 6}\;dx=\answer{\frac{261}{2}}\]
\end{problem}}%}

%%%%%%%%%%%%%%%%%%%%%%


\latexProblemContent{
\begin{problem}

Use the Fundamental Theorem of Calculus to evaluate the integral.

\expandafter\input{\file@loc Integrals/2311-Compute-Integral-0010.HELP.tex}

\[\int_{-2}^{5} {x^{2} - 6 \, x - 7}\;dx=\answer{-\frac{203}{3}}\]
\end{problem}}%}

%%%%%%%%%%%%%%%%%%%%%%


\latexProblemContent{
\begin{problem}

Use the Fundamental Theorem of Calculus to evaluate the integral.

\expandafter\input{\file@loc Integrals/2311-Compute-Integral-0010.HELP.tex}

\[\int_{2}^{2} {x^{3} + 7 \, x^{2} - 5 \, x - 75}\;dx=\answer{0}\]
\end{problem}}%}

%%%%%%%%%%%%%%%%%%%%%%


\latexProblemContent{
\begin{problem}

Use the Fundamental Theorem of Calculus to evaluate the integral.

\expandafter\input{\file@loc Integrals/2311-Compute-Integral-0010.HELP.tex}

\[\int_{-6}^{3} {x^{2} - 16}\;dx=\answer{-63}\]
\end{problem}}%}

%%%%%%%%%%%%%%%%%%%%%%


\latexProblemContent{
\begin{problem}

Use the Fundamental Theorem of Calculus to evaluate the integral.

\expandafter\input{\file@loc Integrals/2311-Compute-Integral-0010.HELP.tex}

\[\int_{-8}^{8} {x^{2} - 5 \, x + 6}\;dx=\answer{\frac{1312}{3}}\]
\end{problem}}%}

%%%%%%%%%%%%%%%%%%%%%%


\latexProblemContent{
\begin{problem}

Use the Fundamental Theorem of Calculus to evaluate the integral.

\expandafter\input{\file@loc Integrals/2311-Compute-Integral-0010.HELP.tex}

\[\int_{3}^{6} {x^{3} + 15 \, x^{2} + 75 \, x + 125}\;dx=\answer{\frac{10545}{4}}\]
\end{problem}}%}

%%%%%%%%%%%%%%%%%%%%%%


\latexProblemContent{
\begin{problem}

Use the Fundamental Theorem of Calculus to evaluate the integral.

\expandafter\input{\file@loc Integrals/2311-Compute-Integral-0010.HELP.tex}

\[\int_{-8}^{-4} {x^{3} - 9 \, x^{2} + 27 \, x - 27}\;dx=\answer{-3060}\]
\end{problem}}%}

%%%%%%%%%%%%%%%%%%%%%%


\latexProblemContent{
\begin{problem}

Use the Fundamental Theorem of Calculus to evaluate the integral.

\expandafter\input{\file@loc Integrals/2311-Compute-Integral-0010.HELP.tex}

\[\int_{-1}^{11} {x^{3} - 6 \, x^{2} - 15 \, x + 100}\;dx=\answer{1296}\]
\end{problem}}%}

%%%%%%%%%%%%%%%%%%%%%%


\latexProblemContent{
\begin{problem}

Use the Fundamental Theorem of Calculus to evaluate the integral.

\expandafter\input{\file@loc Integrals/2311-Compute-Integral-0010.HELP.tex}

\[\int_{-2}^{8} {x^{3} - 4 \, x^{2} - 11 \, x - 6}\;dx=\answer{-\frac{190}{3}}\]
\end{problem}}%}

%%%%%%%%%%%%%%%%%%%%%%


\latexProblemContent{
\begin{problem}

Use the Fundamental Theorem of Calculus to evaluate the integral.

\expandafter\input{\file@loc Integrals/2311-Compute-Integral-0010.HELP.tex}

\[\int_{-9}^{2} {x^{3} - 15 \, x^{2} + 75 \, x - 125}\;dx=\answer{-\frac{38335}{4}}\]
\end{problem}}%}

%%%%%%%%%%%%%%%%%%%%%%


\latexProblemContent{
\begin{problem}

Use the Fundamental Theorem of Calculus to evaluate the integral.

\expandafter\input{\file@loc Integrals/2311-Compute-Integral-0010.HELP.tex}

\[\int_{-9}^{9} {x^{2} + 2 \, x - 15}\;dx=\answer{216}\]
\end{problem}}%}

%%%%%%%%%%%%%%%%%%%%%%


\latexProblemContent{
\begin{problem}

Use the Fundamental Theorem of Calculus to evaluate the integral.

\expandafter\input{\file@loc Integrals/2311-Compute-Integral-0010.HELP.tex}

\[\int_{-5}^{-5} {x - 5}\;dx=\answer{0}\]
\end{problem}}%}

%%%%%%%%%%%%%%%%%%%%%%


\latexProblemContent{
\begin{problem}

Use the Fundamental Theorem of Calculus to evaluate the integral.

\expandafter\input{\file@loc Integrals/2311-Compute-Integral-0010.HELP.tex}

\[\int_{-6}^{8} {x^{2} + 2 \, x}\;dx=\answer{\frac{812}{3}}\]
\end{problem}}%}

%%%%%%%%%%%%%%%%%%%%%%


\latexProblemContent{
\begin{problem}

Use the Fundamental Theorem of Calculus to evaluate the integral.

\expandafter\input{\file@loc Integrals/2311-Compute-Integral-0010.HELP.tex}

\[\int_{-5}^{9} {x + 4}\;dx=\answer{84}\]
\end{problem}}%}

%%%%%%%%%%%%%%%%%%%%%%


\latexProblemContent{
\begin{problem}

Use the Fundamental Theorem of Calculus to evaluate the integral.

\expandafter\input{\file@loc Integrals/2311-Compute-Integral-0010.HELP.tex}

\[\int_{4}^{10} {x^{3} - 2 \, x^{2} - 32 \, x + 96}\;dx=\answer{1044}\]
\end{problem}}%}

%%%%%%%%%%%%%%%%%%%%%%


\latexProblemContent{
\begin{problem}

Use the Fundamental Theorem of Calculus to evaluate the integral.

\expandafter\input{\file@loc Integrals/2311-Compute-Integral-0010.HELP.tex}

\[\int_{-9}^{11} {x^{3} - 48 \, x + 128}\;dx=\answer{3620}\]
\end{problem}}%}

%%%%%%%%%%%%%%%%%%%%%%


\latexProblemContent{
\begin{problem}

Use the Fundamental Theorem of Calculus to evaluate the integral.

\expandafter\input{\file@loc Integrals/2311-Compute-Integral-0010.HELP.tex}

\[\int_{3}^{6} {x^{2} - 10 \, x + 25}\;dx=\answer{3}\]
\end{problem}}%}

%%%%%%%%%%%%%%%%%%%%%%


\latexProblemContent{
\begin{problem}

Use the Fundamental Theorem of Calculus to evaluate the integral.

\expandafter\input{\file@loc Integrals/2311-Compute-Integral-0010.HELP.tex}

\[\int_{-6}^{-1} {x^{2} - 4 \, x + 4}\;dx=\answer{\frac{485}{3}}\]
\end{problem}}%}

%%%%%%%%%%%%%%%%%%%%%%


\latexProblemContent{
\begin{problem}

Use the Fundamental Theorem of Calculus to evaluate the integral.

\expandafter\input{\file@loc Integrals/2311-Compute-Integral-0010.HELP.tex}

\[\int_{-8}^{-1} {x^{2} + 4 \, x + 4}\;dx=\answer{\frac{217}{3}}\]
\end{problem}}%}

%%%%%%%%%%%%%%%%%%%%%%


\latexProblemContent{
\begin{problem}

Use the Fundamental Theorem of Calculus to evaluate the integral.

\expandafter\input{\file@loc Integrals/2311-Compute-Integral-0010.HELP.tex}

\[\int_{-2}^{1} {x^{2} - 8 \, x + 16}\;dx=\answer{63}\]
\end{problem}}%}

%%%%%%%%%%%%%%%%%%%%%%


\latexProblemContent{
\begin{problem}

Use the Fundamental Theorem of Calculus to evaluate the integral.

\expandafter\input{\file@loc Integrals/2311-Compute-Integral-0010.HELP.tex}

\[\int_{-2}^{6} {x + 4}\;dx=\answer{48}\]
\end{problem}}%}

%%%%%%%%%%%%%%%%%%%%%%


\latexProblemContent{
\begin{problem}

Use the Fundamental Theorem of Calculus to evaluate the integral.

\expandafter\input{\file@loc Integrals/2311-Compute-Integral-0010.HELP.tex}

\[\int_{-9}^{5} {x^{3} + x^{2} - 21 \, x - 45}\;dx=\answer{-\frac{3724}{3}}\]
\end{problem}}%}

%%%%%%%%%%%%%%%%%%%%%%


\latexProblemContent{
\begin{problem}

Use the Fundamental Theorem of Calculus to evaluate the integral.

\expandafter\input{\file@loc Integrals/2311-Compute-Integral-0010.HELP.tex}

\[\int_{5}^{6} {x^{2} - 7 \, x + 6}\;dx=\answer{-\frac{13}{6}}\]
\end{problem}}%}

%%%%%%%%%%%%%%%%%%%%%%


\latexProblemContent{
\begin{problem}

Use the Fundamental Theorem of Calculus to evaluate the integral.

\expandafter\input{\file@loc Integrals/2311-Compute-Integral-0010.HELP.tex}

\[\int_{-1}^{0} {x^{2} + 4 \, x - 12}\;dx=\answer{-\frac{41}{3}}\]
\end{problem}}%}

%%%%%%%%%%%%%%%%%%%%%%


\latexProblemContent{
\begin{problem}

Use the Fundamental Theorem of Calculus to evaluate the integral.

\expandafter\input{\file@loc Integrals/2311-Compute-Integral-0010.HELP.tex}

\[\int_{-7}^{0} {x + 4}\;dx=\answer{\frac{7}{2}}\]
\end{problem}}%}

%%%%%%%%%%%%%%%%%%%%%%


\latexProblemContent{
\begin{problem}

Use the Fundamental Theorem of Calculus to evaluate the integral.

\expandafter\input{\file@loc Integrals/2311-Compute-Integral-0010.HELP.tex}

\[\int_{-8}^{1} {x^{2} - 16}\;dx=\answer{27}\]
\end{problem}}%}

%%%%%%%%%%%%%%%%%%%%%%


\latexProblemContent{
\begin{problem}

Use the Fundamental Theorem of Calculus to evaluate the integral.

\expandafter\input{\file@loc Integrals/2311-Compute-Integral-0010.HELP.tex}

\[\int_{-3}^{10} {x^{2} + x - 12}\;dx=\answer{\frac{1391}{6}}\]
\end{problem}}%}

%%%%%%%%%%%%%%%%%%%%%%


\latexProblemContent{
\begin{problem}

Use the Fundamental Theorem of Calculus to evaluate the integral.

\expandafter\input{\file@loc Integrals/2311-Compute-Integral-0010.HELP.tex}

\[\int_{-6}^{0} {x^{3} - 9 \, x^{2} + 27 \, x - 27}\;dx=\answer{-1620}\]
\end{problem}}%}

%%%%%%%%%%%%%%%%%%%%%%


\latexProblemContent{
\begin{problem}

Use the Fundamental Theorem of Calculus to evaluate the integral.

\expandafter\input{\file@loc Integrals/2311-Compute-Integral-0010.HELP.tex}

\[\int_{1}^{12} {x^{3} + x^{2} - 8 \, x - 12}\;dx=\answer{\frac{60665}{12}}\]
\end{problem}}%}

%%%%%%%%%%%%%%%%%%%%%%


\latexProblemContent{
\begin{problem}

Use the Fundamental Theorem of Calculus to evaluate the integral.

\expandafter\input{\file@loc Integrals/2311-Compute-Integral-0010.HELP.tex}

\[\int_{-9}^{-1} {x - 5}\;dx=\answer{-80}\]
\end{problem}}%}

%%%%%%%%%%%%%%%%%%%%%%


\latexProblemContent{
\begin{problem}

Use the Fundamental Theorem of Calculus to evaluate the integral.

\expandafter\input{\file@loc Integrals/2311-Compute-Integral-0010.HELP.tex}

\[\int_{-1}^{1} {x^{2} - 4 \, x + 4}\;dx=\answer{\frac{26}{3}}\]
\end{problem}}%}

%%%%%%%%%%%%%%%%%%%%%%


\latexProblemContent{
\begin{problem}

Use the Fundamental Theorem of Calculus to evaluate the integral.

\expandafter\input{\file@loc Integrals/2311-Compute-Integral-0010.HELP.tex}

\[\int_{0}^{8} {x^{2} - 2 \, x - 35}\;dx=\answer{-\frac{520}{3}}\]
\end{problem}}%}

%%%%%%%%%%%%%%%%%%%%%%


\latexProblemContent{
\begin{problem}

Use the Fundamental Theorem of Calculus to evaluate the integral.

\expandafter\input{\file@loc Integrals/2311-Compute-Integral-0010.HELP.tex}

\[\int_{-6}^{6} {x^{3} + 3 \, x^{2} - 9 \, x - 27}\;dx=\answer{108}\]
\end{problem}}%}

%%%%%%%%%%%%%%%%%%%%%%


\latexProblemContent{
\begin{problem}

Use the Fundamental Theorem of Calculus to evaluate the integral.

\expandafter\input{\file@loc Integrals/2311-Compute-Integral-0010.HELP.tex}

\[\int_{5}^{8} {x^{3} + 6 \, x^{2} + 9 \, x}\;dx=\answer{\frac{7269}{4}}\]
\end{problem}}%}

%%%%%%%%%%%%%%%%%%%%%%


\latexProblemContent{
\begin{problem}

Use the Fundamental Theorem of Calculus to evaluate the integral.

\expandafter\input{\file@loc Integrals/2311-Compute-Integral-0010.HELP.tex}

\[\int_{-4}^{-1} {x^{2} - 2 \, x - 3}\;dx=\answer{27}\]
\end{problem}}%}

%%%%%%%%%%%%%%%%%%%%%%


\latexProblemContent{
\begin{problem}

Use the Fundamental Theorem of Calculus to evaluate the integral.

\expandafter\input{\file@loc Integrals/2311-Compute-Integral-0010.HELP.tex}

\[\int_{-7}^{7} {x^{2} - 8 \, x + 16}\;dx=\answer{\frac{1358}{3}}\]
\end{problem}}%}

%%%%%%%%%%%%%%%%%%%%%%


\latexProblemContent{
\begin{problem}

Use the Fundamental Theorem of Calculus to evaluate the integral.

\expandafter\input{\file@loc Integrals/2311-Compute-Integral-0010.HELP.tex}

\[\int_{-8}^{-6} {x - 1}\;dx=\answer{-16}\]
\end{problem}}%}

%%%%%%%%%%%%%%%%%%%%%%


\latexProblemContent{
\begin{problem}

Use the Fundamental Theorem of Calculus to evaluate the integral.

\expandafter\input{\file@loc Integrals/2311-Compute-Integral-0010.HELP.tex}

\[\int_{4}^{6} {x - 3}\;dx=\answer{4}\]
\end{problem}}%}

%%%%%%%%%%%%%%%%%%%%%%


\latexProblemContent{
\begin{problem}

Use the Fundamental Theorem of Calculus to evaluate the integral.

\expandafter\input{\file@loc Integrals/2311-Compute-Integral-0010.HELP.tex}

\[\int_{-8}^{-6} {x^{2} + 8 \, x + 16}\;dx=\answer{\frac{56}{3}}\]
\end{problem}}%}

%%%%%%%%%%%%%%%%%%%%%%


\latexProblemContent{
\begin{problem}

Use the Fundamental Theorem of Calculus to evaluate the integral.

\expandafter\input{\file@loc Integrals/2311-Compute-Integral-0010.HELP.tex}

\[\int_{-9}^{-8} {x^{2} - 6 \, x + 9}\;dx=\answer{\frac{397}{3}}\]
\end{problem}}%}

%%%%%%%%%%%%%%%%%%%%%%


\latexProblemContent{
\begin{problem}

Use the Fundamental Theorem of Calculus to evaluate the integral.

\expandafter\input{\file@loc Integrals/2311-Compute-Integral-0010.HELP.tex}

\[\int_{2}^{8} {x^{2} - 2 \, x + 1}\;dx=\answer{114}\]
\end{problem}}%}

%%%%%%%%%%%%%%%%%%%%%%


\latexProblemContent{
\begin{problem}

Use the Fundamental Theorem of Calculus to evaluate the integral.

\expandafter\input{\file@loc Integrals/2311-Compute-Integral-0010.HELP.tex}

\[\int_{-7}^{-4} {x + 1}\;dx=\answer{-\frac{27}{2}}\]
\end{problem}}%}

%%%%%%%%%%%%%%%%%%%%%%


\latexProblemContent{
\begin{problem}

Use the Fundamental Theorem of Calculus to evaluate the integral.

\expandafter\input{\file@loc Integrals/2311-Compute-Integral-0010.HELP.tex}

\[\int_{2}^{9} {x^{3} + 3 \, x^{2} + 3 \, x + 1}\;dx=\answer{\frac{9919}{4}}\]
\end{problem}}%}

%%%%%%%%%%%%%%%%%%%%%%


\latexProblemContent{
\begin{problem}

Use the Fundamental Theorem of Calculus to evaluate the integral.

\expandafter\input{\file@loc Integrals/2311-Compute-Integral-0010.HELP.tex}

\[\int_{3}^{11} {x^{3} + x^{2} - 21 \, x - 45}\;dx=\answer{\frac{7616}{3}}\]
\end{problem}}%}

%%%%%%%%%%%%%%%%%%%%%%


\latexProblemContent{
\begin{problem}

Use the Fundamental Theorem of Calculus to evaluate the integral.

\expandafter\input{\file@loc Integrals/2311-Compute-Integral-0010.HELP.tex}

\[\int_{-1}^{12} {x - 2}\;dx=\answer{\frac{91}{2}}\]
\end{problem}}%}

%%%%%%%%%%%%%%%%%%%%%%


\latexProblemContent{
\begin{problem}

Use the Fundamental Theorem of Calculus to evaluate the integral.

\expandafter\input{\file@loc Integrals/2311-Compute-Integral-0010.HELP.tex}

\[\int_{-9}^{-9} {x^{3} + 5 \, x^{2} - 13 \, x + 7}\;dx=\answer{0}\]
\end{problem}}%}

%%%%%%%%%%%%%%%%%%%%%%


\latexProblemContent{
\begin{problem}

Use the Fundamental Theorem of Calculus to evaluate the integral.

\expandafter\input{\file@loc Integrals/2311-Compute-Integral-0010.HELP.tex}

\[\int_{-9}^{-3} {x^{2} - 6 \, x + 9}\;dx=\answer{504}\]
\end{problem}}%}

%%%%%%%%%%%%%%%%%%%%%%


\latexProblemContent{
\begin{problem}

Use the Fundamental Theorem of Calculus to evaluate the integral.

\expandafter\input{\file@loc Integrals/2311-Compute-Integral-0010.HELP.tex}

\[\int_{-5}^{12} {x^{2} - 6 \, x - 7}\;dx=\answer{\frac{425}{3}}\]
\end{problem}}%}

%%%%%%%%%%%%%%%%%%%%%%


\latexProblemContent{
\begin{problem}

Use the Fundamental Theorem of Calculus to evaluate the integral.

\expandafter\input{\file@loc Integrals/2311-Compute-Integral-0010.HELP.tex}

\[\int_{-2}^{7} {x^{3} - 3 \, x^{2} + 3 \, x - 1}\;dx=\answer{\frac{1215}{4}}\]
\end{problem}}%}

%%%%%%%%%%%%%%%%%%%%%%


\latexProblemContent{
\begin{problem}

Use the Fundamental Theorem of Calculus to evaluate the integral.

\expandafter\input{\file@loc Integrals/2311-Compute-Integral-0010.HELP.tex}

\[\int_{-7}^{4} {x^{3} - 4 \, x^{2} - 11 \, x - 6}\;dx=\answer{-\frac{11561}{12}}\]
\end{problem}}%}

%%%%%%%%%%%%%%%%%%%%%%


\latexProblemContent{
\begin{problem}

Use the Fundamental Theorem of Calculus to evaluate the integral.

\expandafter\input{\file@loc Integrals/2311-Compute-Integral-0010.HELP.tex}

\[\int_{-9}^{-9} {x - 4}\;dx=\answer{0}\]
\end{problem}}%}

%%%%%%%%%%%%%%%%%%%%%%


\latexProblemContent{
\begin{problem}

Use the Fundamental Theorem of Calculus to evaluate the integral.

\expandafter\input{\file@loc Integrals/2311-Compute-Integral-0010.HELP.tex}

\[\int_{2}^{12} {x^{2} - 6 \, x - 16}\;dx=\answer{-\frac{20}{3}}\]
\end{problem}}%}

%%%%%%%%%%%%%%%%%%%%%%


\latexProblemContent{
\begin{problem}

Use the Fundamental Theorem of Calculus to evaluate the integral.

\expandafter\input{\file@loc Integrals/2311-Compute-Integral-0010.HELP.tex}

\[\int_{-5}^{6} {x^{3} - 12 \, x^{2} + 48 \, x - 64}\;dx=\answer{-\frac{6545}{4}}\]
\end{problem}}%}

%%%%%%%%%%%%%%%%%%%%%%


\latexProblemContent{
\begin{problem}

Use the Fundamental Theorem of Calculus to evaluate the integral.

\expandafter\input{\file@loc Integrals/2311-Compute-Integral-0010.HELP.tex}

\[\int_{0}^{12} {x^{2} + x - 20}\;dx=\answer{408}\]
\end{problem}}%}

%%%%%%%%%%%%%%%%%%%%%%


\latexProblemContent{
\begin{problem}

Use the Fundamental Theorem of Calculus to evaluate the integral.

\expandafter\input{\file@loc Integrals/2311-Compute-Integral-0010.HELP.tex}

\[\int_{3}^{3} {x^{3} - 12 \, x + 16}\;dx=\answer{0}\]
\end{problem}}%}

%%%%%%%%%%%%%%%%%%%%%%


\latexProblemContent{
\begin{problem}

Use the Fundamental Theorem of Calculus to evaluate the integral.

\expandafter\input{\file@loc Integrals/2311-Compute-Integral-0010.HELP.tex}

\[\int_{-3}^{9} {x^{2} - 6 \, x + 5}\;dx=\answer{96}\]
\end{problem}}%}

%%%%%%%%%%%%%%%%%%%%%%


\latexProblemContent{
\begin{problem}

Use the Fundamental Theorem of Calculus to evaluate the integral.

\expandafter\input{\file@loc Integrals/2311-Compute-Integral-0010.HELP.tex}

\[\int_{-9}^{3} {x^{3} + 8 \, x^{2} + 16 \, x}\;dx=\answer{-180}\]
\end{problem}}%}

%%%%%%%%%%%%%%%%%%%%%%


\latexProblemContent{
\begin{problem}

Use the Fundamental Theorem of Calculus to evaluate the integral.

\expandafter\input{\file@loc Integrals/2311-Compute-Integral-0010.HELP.tex}

\[\int_{-10}^{11} {x + 1}\;dx=\answer{\frac{63}{2}}\]
\end{problem}}%}

%%%%%%%%%%%%%%%%%%%%%%


\latexProblemContent{
\begin{problem}

Use the Fundamental Theorem of Calculus to evaluate the integral.

\expandafter\input{\file@loc Integrals/2311-Compute-Integral-0010.HELP.tex}

\[\int_{2}^{9} {x^{3} + 7 \, x^{2} - 5 \, x - 75}\;dx=\answer{\frac{31213}{12}}\]
\end{problem}}%}

%%%%%%%%%%%%%%%%%%%%%%


\latexProblemContent{
\begin{problem}

Use the Fundamental Theorem of Calculus to evaluate the integral.

\expandafter\input{\file@loc Integrals/2311-Compute-Integral-0010.HELP.tex}

\[\int_{2}^{12} {x^{3} - 6 \, x^{2} + 12 \, x - 8}\;dx=\answer{2500}\]
\end{problem}}%}

%%%%%%%%%%%%%%%%%%%%%%


\latexProblemContent{
\begin{problem}

Use the Fundamental Theorem of Calculus to evaluate the integral.

\expandafter\input{\file@loc Integrals/2311-Compute-Integral-0010.HELP.tex}

\[\int_{-5}^{6} {x^{2} - 4 \, x + 4}\;dx=\answer{\frac{407}{3}}\]
\end{problem}}%}

%%%%%%%%%%%%%%%%%%%%%%


\latexProblemContent{
\begin{problem}

Use the Fundamental Theorem of Calculus to evaluate the integral.

\expandafter\input{\file@loc Integrals/2311-Compute-Integral-0010.HELP.tex}

\[\int_{-1}^{-1} {x^{3} - 15 \, x^{2} + 75 \, x - 125}\;dx=\answer{0}\]
\end{problem}}%}

%%%%%%%%%%%%%%%%%%%%%%


\latexProblemContent{
\begin{problem}

Use the Fundamental Theorem of Calculus to evaluate the integral.

\expandafter\input{\file@loc Integrals/2311-Compute-Integral-0010.HELP.tex}

\[\int_{-9}^{8} {x^{2} + 8 \, x + 16}\;dx=\answer{\frac{1853}{3}}\]
\end{problem}}%}

%%%%%%%%%%%%%%%%%%%%%%


\latexProblemContent{
\begin{problem}

Use the Fundamental Theorem of Calculus to evaluate the integral.

\expandafter\input{\file@loc Integrals/2311-Compute-Integral-0010.HELP.tex}

\[\int_{-5}^{-3} {x^{3} - 3 \, x^{2} - 9 \, x + 27}\;dx=\answer{-108}\]
\end{problem}}%}

%%%%%%%%%%%%%%%%%%%%%%


%%%%%%%%%%%%%%%%%%%%%%


\latexProblemContent{
\begin{problem}

Use the Fundamental Theorem of Calculus to evaluate the integral.

\expandafter\input{\file@loc Integrals/2311-Compute-Integral-0010.HELP.tex}

\[\int_{-8}^{9} {x^{2} + 8 \, x + 16}\;dx=\answer{\frac{2261}{3}}\]
\end{problem}}%}

%%%%%%%%%%%%%%%%%%%%%%


\latexProblemContent{
\begin{problem}

Use the Fundamental Theorem of Calculus to evaluate the integral.

\expandafter\input{\file@loc Integrals/2311-Compute-Integral-0010.HELP.tex}

\[\int_{-2}^{11} {x + 5}\;dx=\answer{\frac{247}{2}}\]
\end{problem}}%}

%%%%%%%%%%%%%%%%%%%%%%


\latexProblemContent{
\begin{problem}

Use the Fundamental Theorem of Calculus to evaluate the integral.

\expandafter\input{\file@loc Integrals/2311-Compute-Integral-0010.HELP.tex}

\[\int_{-7}^{5} {x^{2} - 8 \, x + 16}\;dx=\answer{444}\]
\end{problem}}%}

%%%%%%%%%%%%%%%%%%%%%%


\latexProblemContent{
\begin{problem}

Use the Fundamental Theorem of Calculus to evaluate the integral.

\expandafter\input{\file@loc Integrals/2311-Compute-Integral-0010.HELP.tex}

\[\int_{1}^{11} {x - 5}\;dx=\answer{10}\]
\end{problem}}%}

%%%%%%%%%%%%%%%%%%%%%%


\latexProblemContent{
\begin{problem}

Use the Fundamental Theorem of Calculus to evaluate the integral.

\expandafter\input{\file@loc Integrals/2311-Compute-Integral-0010.HELP.tex}

\[\int_{1}^{7} {x - 3}\;dx=\answer{6}\]
\end{problem}}%}

%%%%%%%%%%%%%%%%%%%%%%


\latexProblemContent{
\begin{problem}

Use the Fundamental Theorem of Calculus to evaluate the integral.

\expandafter\input{\file@loc Integrals/2311-Compute-Integral-0010.HELP.tex}

\[\int_{-10}^{-4} {x^{2} + 2 \, x + 1}\;dx=\answer{234}\]
\end{problem}}%}

%%%%%%%%%%%%%%%%%%%%%%


\latexProblemContent{
\begin{problem}

Use the Fundamental Theorem of Calculus to evaluate the integral.

\expandafter\input{\file@loc Integrals/2311-Compute-Integral-0010.HELP.tex}

\[\int_{2}^{6} {x^{3} + 6 \, x^{2} - 32}\;dx=\answer{608}\]
\end{problem}}%}

%%%%%%%%%%%%%%%%%%%%%%


\latexProblemContent{
\begin{problem}

Use the Fundamental Theorem of Calculus to evaluate the integral.

\expandafter\input{\file@loc Integrals/2311-Compute-Integral-0010.HELP.tex}

\[\int_{1}^{7} {x^{2} + x - 12}\;dx=\answer{66}\]
\end{problem}}%}

%%%%%%%%%%%%%%%%%%%%%%


\latexProblemContent{
\begin{problem}

Use the Fundamental Theorem of Calculus to evaluate the integral.

\expandafter\input{\file@loc Integrals/2311-Compute-Integral-0010.HELP.tex}

\[\int_{0}^{8} {x - 1}\;dx=\answer{24}\]
\end{problem}}%}

%%%%%%%%%%%%%%%%%%%%%%


\latexProblemContent{
\begin{problem}

Use the Fundamental Theorem of Calculus to evaluate the integral.

\expandafter\input{\file@loc Integrals/2311-Compute-Integral-0010.HELP.tex}

\[\int_{2}^{11} {x^{2} - 4 \, x + 4}\;dx=\answer{243}\]
\end{problem}}%}

%%%%%%%%%%%%%%%%%%%%%%


\latexProblemContent{
\begin{problem}

Use the Fundamental Theorem of Calculus to evaluate the integral.

\expandafter\input{\file@loc Integrals/2311-Compute-Integral-0010.HELP.tex}

\[\int_{-1}^{10} {x^{2} + 10 \, x + 16}\;dx=\answer{\frac{3014}{3}}\]
\end{problem}}%}

%%%%%%%%%%%%%%%%%%%%%%


\latexProblemContent{
\begin{problem}

Use the Fundamental Theorem of Calculus to evaluate the integral.

\expandafter\input{\file@loc Integrals/2311-Compute-Integral-0010.HELP.tex}

\[\int_{0}^{10} {x^{3} + 18 \, x^{2} + 105 \, x + 200}\;dx=\answer{15750}\]
\end{problem}}%}

%%%%%%%%%%%%%%%%%%%%%%


\latexProblemContent{
\begin{problem}

Use the Fundamental Theorem of Calculus to evaluate the integral.

\expandafter\input{\file@loc Integrals/2311-Compute-Integral-0010.HELP.tex}

\[\int_{-10}^{2} {x^{2} - 10 \, x + 25}\;dx=\answer{1116}\]
\end{problem}}%}

%%%%%%%%%%%%%%%%%%%%%%


\latexProblemContent{
\begin{problem}

Use the Fundamental Theorem of Calculus to evaluate the integral.

\expandafter\input{\file@loc Integrals/2311-Compute-Integral-0010.HELP.tex}

\[\int_{3}^{6} {x^{2} - 4 \, x}\;dx=\answer{9}\]
\end{problem}}%}

%%%%%%%%%%%%%%%%%%%%%%


\latexProblemContent{
\begin{problem}

Use the Fundamental Theorem of Calculus to evaluate the integral.

\expandafter\input{\file@loc Integrals/2311-Compute-Integral-0010.HELP.tex}

\[\int_{3}^{3} {x^{2} - 4 \, x - 5}\;dx=\answer{0}\]
\end{problem}}%}

%%%%%%%%%%%%%%%%%%%%%%


\latexProblemContent{
\begin{problem}

Use the Fundamental Theorem of Calculus to evaluate the integral.

\expandafter\input{\file@loc Integrals/2311-Compute-Integral-0010.HELP.tex}

\[\int_{5}^{10} {x^{2} - 6 \, x + 9}\;dx=\answer{\frac{335}{3}}\]
\end{problem}}%}

%%%%%%%%%%%%%%%%%%%%%%


\latexProblemContent{
\begin{problem}

Use the Fundamental Theorem of Calculus to evaluate the integral.

\expandafter\input{\file@loc Integrals/2311-Compute-Integral-0010.HELP.tex}

\[\int_{-4}^{-4} {x^{2} - 4 \, x + 4}\;dx=\answer{0}\]
\end{problem}}%}

%%%%%%%%%%%%%%%%%%%%%%


\latexProblemContent{
\begin{problem}

Use the Fundamental Theorem of Calculus to evaluate the integral.

\expandafter\input{\file@loc Integrals/2311-Compute-Integral-0010.HELP.tex}

\[\int_{-10}^{12} {x^{3} - 15 \, x^{2} + 75 \, x - 125}\;dx=\answer{-12056}\]
\end{problem}}%}

%%%%%%%%%%%%%%%%%%%%%%


\latexProblemContent{
\begin{problem}

Use the Fundamental Theorem of Calculus to evaluate the integral.

\expandafter\input{\file@loc Integrals/2311-Compute-Integral-0010.HELP.tex}

\[\int_{4}^{10} {x^{3} - 6 \, x^{2} + 12 \, x - 8}\;dx=\answer{1020}\]
\end{problem}}%}

%%%%%%%%%%%%%%%%%%%%%%


\latexProblemContent{
\begin{problem}

Use the Fundamental Theorem of Calculus to evaluate the integral.

\expandafter\input{\file@loc Integrals/2311-Compute-Integral-0010.HELP.tex}

\[\int_{-9}^{2} {x^{2} - 2 \, x}\;dx=\answer{\frac{968}{3}}\]
\end{problem}}%}

%%%%%%%%%%%%%%%%%%%%%%


\latexProblemContent{
\begin{problem}

Use the Fundamental Theorem of Calculus to evaluate the integral.

\expandafter\input{\file@loc Integrals/2311-Compute-Integral-0010.HELP.tex}

\[\int_{-4}^{8} {x^{3} + 15 \, x^{2} + 75 \, x + 125}\;dx=\answer{7140}\]
\end{problem}}%}

%%%%%%%%%%%%%%%%%%%%%%


\latexProblemContent{
\begin{problem}

Use the Fundamental Theorem of Calculus to evaluate the integral.

\expandafter\input{\file@loc Integrals/2311-Compute-Integral-0010.HELP.tex}

\[\int_{-6}^{6} {x^{2} - 9 \, x + 20}\;dx=\answer{384}\]
\end{problem}}%}

%%%%%%%%%%%%%%%%%%%%%%


\latexProblemContent{
\begin{problem}

Use the Fundamental Theorem of Calculus to evaluate the integral.

\expandafter\input{\file@loc Integrals/2311-Compute-Integral-0010.HELP.tex}

\[\int_{1}^{9} {x^{3} - 15 \, x^{2} + 75 \, x - 125}\;dx=\answer{0}\]
\end{problem}}%}

%%%%%%%%%%%%%%%%%%%%%%


\latexProblemContent{
\begin{problem}

Use the Fundamental Theorem of Calculus to evaluate the integral.

\expandafter\input{\file@loc Integrals/2311-Compute-Integral-0010.HELP.tex}

\[\int_{-2}^{8} {x^{2} - 2 \, x + 1}\;dx=\answer{\frac{370}{3}}\]
\end{problem}}%}

%%%%%%%%%%%%%%%%%%%%%%


\latexProblemContent{
\begin{problem}

Use the Fundamental Theorem of Calculus to evaluate the integral.

\expandafter\input{\file@loc Integrals/2311-Compute-Integral-0010.HELP.tex}

\[\int_{-2}^{0} {x^{3} - 8 \, x^{2} + 20 \, x - 16}\;dx=\answer{-\frac{292}{3}}\]
\end{problem}}%}

%%%%%%%%%%%%%%%%%%%%%%


\latexProblemContent{
\begin{problem}

Use the Fundamental Theorem of Calculus to evaluate the integral.

\expandafter\input{\file@loc Integrals/2311-Compute-Integral-0010.HELP.tex}

\[\int_{-7}^{12} {x^{3} + 12 \, x^{2} + 48 \, x + 64}\;dx=\answer{\frac{65455}{4}}\]
\end{problem}}%}

%%%%%%%%%%%%%%%%%%%%%%


\latexProblemContent{
\begin{problem}

Use the Fundamental Theorem of Calculus to evaluate the integral.

\expandafter\input{\file@loc Integrals/2311-Compute-Integral-0010.HELP.tex}

\[\int_{-2}^{1} {x^{3} + 6 \, x^{2} + 12 \, x + 8}\;dx=\answer{\frac{81}{4}}\]
\end{problem}}%}

%%%%%%%%%%%%%%%%%%%%%%


\latexProblemContent{
\begin{problem}

Use the Fundamental Theorem of Calculus to evaluate the integral.

\expandafter\input{\file@loc Integrals/2311-Compute-Integral-0010.HELP.tex}

\[\int_{-4}^{-2} {x^{2} + 6 \, x + 9}\;dx=\answer{\frac{2}{3}}\]
\end{problem}}%}

%%%%%%%%%%%%%%%%%%%%%%


\latexProblemContent{
\begin{problem}

Use the Fundamental Theorem of Calculus to evaluate the integral.

\expandafter\input{\file@loc Integrals/2311-Compute-Integral-0010.HELP.tex}

\[\int_{-8}^{-5} {x^{2} + 6 \, x + 9}\;dx=\answer{39}\]
\end{problem}}%}

%%%%%%%%%%%%%%%%%%%%%%


\latexProblemContent{
\begin{problem}

Use the Fundamental Theorem of Calculus to evaluate the integral.

\expandafter\input{\file@loc Integrals/2311-Compute-Integral-0010.HELP.tex}

\[\int_{-7}^{-1} {x + 1}\;dx=\answer{-18}\]
\end{problem}}%}

%%%%%%%%%%%%%%%%%%%%%%


\latexProblemContent{
\begin{problem}

Use the Fundamental Theorem of Calculus to evaluate the integral.

\expandafter\input{\file@loc Integrals/2311-Compute-Integral-0010.HELP.tex}

\[\int_{-7}^{-3} {x^{3} - 13 \, x^{2} + 55 \, x - 75}\;dx=\answer{-\frac{10048}{3}}\]
\end{problem}}%}

%%%%%%%%%%%%%%%%%%%%%%


\latexProblemContent{
\begin{problem}

Use the Fundamental Theorem of Calculus to evaluate the integral.

\expandafter\input{\file@loc Integrals/2311-Compute-Integral-0010.HELP.tex}

\[\int_{-10}^{-3} {x^{3} + 10 \, x^{2} + 17 \, x + 8}\;dx=\answer{\frac{553}{12}}\]
\end{problem}}%}

%%%%%%%%%%%%%%%%%%%%%%


\latexProblemContent{
\begin{problem}

Use the Fundamental Theorem of Calculus to evaluate the integral.

\expandafter\input{\file@loc Integrals/2311-Compute-Integral-0010.HELP.tex}

\[\int_{-9}^{8} {x^{3} + 8 \, x^{2} + 13 \, x + 6}\;dx=\answer{\frac{32215}{12}}\]
\end{problem}}%}

%%%%%%%%%%%%%%%%%%%%%%


\latexProblemContent{
\begin{problem}

Use the Fundamental Theorem of Calculus to evaluate the integral.

\expandafter\input{\file@loc Integrals/2311-Compute-Integral-0010.HELP.tex}

\[\int_{-4}^{4} {x^{3} + 2 \, x^{2} + x}\;dx=\answer{\frac{256}{3}}\]
\end{problem}}%}

%%%%%%%%%%%%%%%%%%%%%%


\latexProblemContent{
\begin{problem}

Use the Fundamental Theorem of Calculus to evaluate the integral.

\expandafter\input{\file@loc Integrals/2311-Compute-Integral-0010.HELP.tex}

\[\int_{-5}^{9} {x^{3} - 15 \, x^{2} + 75 \, x - 125}\;dx=\answer{-2436}\]
\end{problem}}%}

%%%%%%%%%%%%%%%%%%%%%%


\latexProblemContent{
\begin{problem}

Use the Fundamental Theorem of Calculus to evaluate the integral.

\expandafter\input{\file@loc Integrals/2311-Compute-Integral-0010.HELP.tex}

\[\int_{5}^{10} {x^{2} + 8 \, x + 16}\;dx=\answer{\frac{2015}{3}}\]
\end{problem}}%}

%%%%%%%%%%%%%%%%%%%%%%


\latexProblemContent{
\begin{problem}

Use the Fundamental Theorem of Calculus to evaluate the integral.

\expandafter\input{\file@loc Integrals/2311-Compute-Integral-0010.HELP.tex}

\[\int_{-7}^{9} {x^{3} - 3 \, x^{2} + 3 \, x - 1}\;dx=\answer{0}\]
\end{problem}}%}

%%%%%%%%%%%%%%%%%%%%%%


\latexProblemContent{
\begin{problem}

Use the Fundamental Theorem of Calculus to evaluate the integral.

\expandafter\input{\file@loc Integrals/2311-Compute-Integral-0010.HELP.tex}

\[\int_{-3}^{2} {x^{3} + 15 \, x^{2} + 75 \, x + 125}\;dx=\answer{\frac{2385}{4}}\]
\end{problem}}%}

%%%%%%%%%%%%%%%%%%%%%%


\latexProblemContent{
\begin{problem}

Use the Fundamental Theorem of Calculus to evaluate the integral.

\expandafter\input{\file@loc Integrals/2311-Compute-Integral-0010.HELP.tex}

\[\int_{-5}^{6} {x^{3} - 4 \, x^{2} - 28 \, x - 32}\;dx=\answer{-\frac{9515}{12}}\]
\end{problem}}%}

%%%%%%%%%%%%%%%%%%%%%%


\latexProblemContent{
\begin{problem}

Use the Fundamental Theorem of Calculus to evaluate the integral.

\expandafter\input{\file@loc Integrals/2311-Compute-Integral-0010.HELP.tex}

\[\int_{5}^{8} {x^{2} + 8 \, x + 16}\;dx=\answer{333}\]
\end{problem}}%}

%%%%%%%%%%%%%%%%%%%%%%


\latexProblemContent{
\begin{problem}

Use the Fundamental Theorem of Calculus to evaluate the integral.

\expandafter\input{\file@loc Integrals/2311-Compute-Integral-0010.HELP.tex}

\[\int_{3}^{5} {x^{3} + 12 \, x^{2} + 45 \, x + 54}\;dx=\answer{996}\]
\end{problem}}%}

%%%%%%%%%%%%%%%%%%%%%%


\latexProblemContent{
\begin{problem}

Use the Fundamental Theorem of Calculus to evaluate the integral.

\expandafter\input{\file@loc Integrals/2311-Compute-Integral-0010.HELP.tex}

\[\int_{2}^{3} {x^{2} - 6 \, x + 9}\;dx=\answer{\frac{1}{3}}\]
\end{problem}}%}

%%%%%%%%%%%%%%%%%%%%%%


\latexProblemContent{
\begin{problem}

Use the Fundamental Theorem of Calculus to evaluate the integral.

\expandafter\input{\file@loc Integrals/2311-Compute-Integral-0010.HELP.tex}

\[\int_{1}^{6} {x - 4}\;dx=\answer{-\frac{5}{2}}\]
\end{problem}}%}

%%%%%%%%%%%%%%%%%%%%%%


\latexProblemContent{
\begin{problem}

Use the Fundamental Theorem of Calculus to evaluate the integral.

\expandafter\input{\file@loc Integrals/2311-Compute-Integral-0010.HELP.tex}

\[\int_{5}^{7} {x^{2} + 3 \, x + 2}\;dx=\answer{\frac{338}{3}}\]
\end{problem}}%}

%%%%%%%%%%%%%%%%%%%%%%


\latexProblemContent{
\begin{problem}

Use the Fundamental Theorem of Calculus to evaluate the integral.

\expandafter\input{\file@loc Integrals/2311-Compute-Integral-0010.HELP.tex}

\[\int_{2}^{6} {x^{2} - 4 \, x + 4}\;dx=\answer{\frac{64}{3}}\]
\end{problem}}%}

%%%%%%%%%%%%%%%%%%%%%%


\latexProblemContent{
\begin{problem}

Use the Fundamental Theorem of Calculus to evaluate the integral.

\expandafter\input{\file@loc Integrals/2311-Compute-Integral-0010.HELP.tex}

\[\int_{-9}^{-4} {x + 3}\;dx=\answer{-\frac{35}{2}}\]
\end{problem}}%}

%%%%%%%%%%%%%%%%%%%%%%


\latexProblemContent{
\begin{problem}

Use the Fundamental Theorem of Calculus to evaluate the integral.

\expandafter\input{\file@loc Integrals/2311-Compute-Integral-0010.HELP.tex}

\[\int_{-10}^{12} {x^{3} + 7 \, x^{2} + 11 \, x + 5}\;dx=\answer{\frac{28204}{3}}\]
\end{problem}}%}

%%%%%%%%%%%%%%%%%%%%%%


\latexProblemContent{
\begin{problem}

Use the Fundamental Theorem of Calculus to evaluate the integral.

\expandafter\input{\file@loc Integrals/2311-Compute-Integral-0010.HELP.tex}

\[\int_{-3}^{6} {x - 4}\;dx=\answer{-\frac{45}{2}}\]
\end{problem}}%}

%%%%%%%%%%%%%%%%%%%%%%


\latexProblemContent{
\begin{problem}

Use the Fundamental Theorem of Calculus to evaluate the integral.

\expandafter\input{\file@loc Integrals/2311-Compute-Integral-0010.HELP.tex}

\[\int_{-1}^{1} {x^{3} - 12 \, x^{2} + 48 \, x - 64}\;dx=\answer{-136}\]
\end{problem}}%}

%%%%%%%%%%%%%%%%%%%%%%


\latexProblemContent{
\begin{problem}

Use the Fundamental Theorem of Calculus to evaluate the integral.

\expandafter\input{\file@loc Integrals/2311-Compute-Integral-0010.HELP.tex}

\[\int_{0}^{0} {x - 2}\;dx=\answer{0}\]
\end{problem}}%}

%%%%%%%%%%%%%%%%%%%%%%


\latexProblemContent{
\begin{problem}

Use the Fundamental Theorem of Calculus to evaluate the integral.

\expandafter\input{\file@loc Integrals/2311-Compute-Integral-0010.HELP.tex}

\[\int_{-2}^{11} {x^{3} + 9 \, x^{2} + 27 \, x + 27}\;dx=\answer{\frac{38415}{4}}\]
\end{problem}}%}

%%%%%%%%%%%%%%%%%%%%%%


\latexProblemContent{
\begin{problem}

Use the Fundamental Theorem of Calculus to evaluate the integral.

\expandafter\input{\file@loc Integrals/2311-Compute-Integral-0010.HELP.tex}

\[\int_{-3}^{0} {x - 5}\;dx=\answer{-\frac{39}{2}}\]
\end{problem}}%}

%%%%%%%%%%%%%%%%%%%%%%


\latexProblemContent{
\begin{problem}

Use the Fundamental Theorem of Calculus to evaluate the integral.

\expandafter\input{\file@loc Integrals/2311-Compute-Integral-0010.HELP.tex}

\[\int_{1}^{10} {x + 1}\;dx=\answer{\frac{117}{2}}\]
\end{problem}}%}

%%%%%%%%%%%%%%%%%%%%%%


\latexProblemContent{
\begin{problem}

Use the Fundamental Theorem of Calculus to evaluate the integral.

\expandafter\input{\file@loc Integrals/2311-Compute-Integral-0010.HELP.tex}

\[\int_{3}^{11} {x^{2} - 5 \, x - 14}\;dx=\answer{\frac{128}{3}}\]
\end{problem}}%}

%%%%%%%%%%%%%%%%%%%%%%


\latexProblemContent{
\begin{problem}

Use the Fundamental Theorem of Calculus to evaluate the integral.

\expandafter\input{\file@loc Integrals/2311-Compute-Integral-0010.HELP.tex}

\[\int_{0}^{8} {x + 1}\;dx=\answer{40}\]
\end{problem}}%}

%%%%%%%%%%%%%%%%%%%%%%


\latexProblemContent{
\begin{problem}

Use the Fundamental Theorem of Calculus to evaluate the integral.

\expandafter\input{\file@loc Integrals/2311-Compute-Integral-0010.HELP.tex}

\[\int_{-8}^{-8} {x^{3} + 12 \, x^{2} + 48 \, x + 64}\;dx=\answer{0}\]
\end{problem}}%}

%%%%%%%%%%%%%%%%%%%%%%


\latexProblemContent{
\begin{problem}

Use the Fundamental Theorem of Calculus to evaluate the integral.

\expandafter\input{\file@loc Integrals/2311-Compute-Integral-0010.HELP.tex}

\[\int_{-7}^{5} {x + 2}\;dx=\answer{12}\]
\end{problem}}%}

%%%%%%%%%%%%%%%%%%%%%%


\latexProblemContent{
\begin{problem}

Use the Fundamental Theorem of Calculus to evaluate the integral.

\expandafter\input{\file@loc Integrals/2311-Compute-Integral-0010.HELP.tex}

\[\int_{-2}^{8} {x + 5}\;dx=\answer{80}\]
\end{problem}}%}

%%%%%%%%%%%%%%%%%%%%%%


\latexProblemContent{
\begin{problem}

Use the Fundamental Theorem of Calculus to evaluate the integral.

\expandafter\input{\file@loc Integrals/2311-Compute-Integral-0010.HELP.tex}

\[\int_{-7}^{6} {x^{3} - 9 \, x^{2} + 15 \, x + 25}\;dx=\answer{-\frac{6903}{4}}\]
\end{problem}}%}

%%%%%%%%%%%%%%%%%%%%%%


\latexProblemContent{
\begin{problem}

Use the Fundamental Theorem of Calculus to evaluate the integral.

\expandafter\input{\file@loc Integrals/2311-Compute-Integral-0010.HELP.tex}

\[\int_{2}^{9} {x^{2} + 4 \, x + 4}\;dx=\answer{\frac{1267}{3}}\]
\end{problem}}%}

%%%%%%%%%%%%%%%%%%%%%%


\latexProblemContent{
\begin{problem}

Use the Fundamental Theorem of Calculus to evaluate the integral.

\expandafter\input{\file@loc Integrals/2311-Compute-Integral-0010.HELP.tex}

\[\int_{-7}^{-2} {x^{2} + 10 \, x + 25}\;dx=\answer{\frac{35}{3}}\]
\end{problem}}%}

%%%%%%%%%%%%%%%%%%%%%%


\latexProblemContent{
\begin{problem}

Use the Fundamental Theorem of Calculus to evaluate the integral.

\expandafter\input{\file@loc Integrals/2311-Compute-Integral-0010.HELP.tex}

\[\int_{1}^{5} {x^{2} + 5 \, x + 4}\;dx=\answer{\frac{352}{3}}\]
\end{problem}}%}

%%%%%%%%%%%%%%%%%%%%%%


\latexProblemContent{
\begin{problem}

Use the Fundamental Theorem of Calculus to evaluate the integral.

\expandafter\input{\file@loc Integrals/2311-Compute-Integral-0010.HELP.tex}

\[\int_{4}^{6} {x^{2} - 2 \, x + 1}\;dx=\answer{\frac{98}{3}}\]
\end{problem}}%}

%%%%%%%%%%%%%%%%%%%%%%


\latexProblemContent{
\begin{problem}

Use the Fundamental Theorem of Calculus to evaluate the integral.

\expandafter\input{\file@loc Integrals/2311-Compute-Integral-0010.HELP.tex}

\[\int_{-3}^{2} {x^{2} + 6 \, x + 9}\;dx=\answer{\frac{125}{3}}\]
\end{problem}}%}

%%%%%%%%%%%%%%%%%%%%%%


\latexProblemContent{
\begin{problem}

Use the Fundamental Theorem of Calculus to evaluate the integral.

\expandafter\input{\file@loc Integrals/2311-Compute-Integral-0010.HELP.tex}

\[\int_{-3}^{-2} {x^{3} - 3 \, x^{2} + 3 \, x - 1}\;dx=\answer{-\frac{175}{4}}\]
\end{problem}}%}

%%%%%%%%%%%%%%%%%%%%%%


\latexProblemContent{
\begin{problem}

Use the Fundamental Theorem of Calculus to evaluate the integral.

\expandafter\input{\file@loc Integrals/2311-Compute-Integral-0010.HELP.tex}

\[\int_{-3}^{8} {x^{2} + 4 \, x - 32}\;dx=\answer{-\frac{187}{3}}\]
\end{problem}}%}

%%%%%%%%%%%%%%%%%%%%%%


\latexProblemContent{
\begin{problem}

Use the Fundamental Theorem of Calculus to evaluate the integral.

\expandafter\input{\file@loc Integrals/2311-Compute-Integral-0010.HELP.tex}

\[\int_{-2}^{-2} {x^{2} - 2 \, x + 1}\;dx=\answer{0}\]
\end{problem}}%}

%%%%%%%%%%%%%%%%%%%%%%


\latexProblemContent{
\begin{problem}

Use the Fundamental Theorem of Calculus to evaluate the integral.

\expandafter\input{\file@loc Integrals/2311-Compute-Integral-0010.HELP.tex}

\[\int_{-6}^{-4} {x^{3} + 9 \, x^{2} + 27 \, x + 27}\;dx=\answer{-20}\]
\end{problem}}%}

%%%%%%%%%%%%%%%%%%%%%%


\latexProblemContent{
\begin{problem}

Use the Fundamental Theorem of Calculus to evaluate the integral.

\expandafter\input{\file@loc Integrals/2311-Compute-Integral-0010.HELP.tex}

\[\int_{4}^{4} {x - 1}\;dx=\answer{0}\]
\end{problem}}%}

%%%%%%%%%%%%%%%%%%%%%%


\latexProblemContent{
\begin{problem}

Use the Fundamental Theorem of Calculus to evaluate the integral.

\expandafter\input{\file@loc Integrals/2311-Compute-Integral-0010.HELP.tex}

\[\int_{-5}^{8} {x^{2} - 7 \, x + 12}\;dx=\answer{\frac{1391}{6}}\]
\end{problem}}%}

%%%%%%%%%%%%%%%%%%%%%%


\latexProblemContent{
\begin{problem}

Use the Fundamental Theorem of Calculus to evaluate the integral.

\expandafter\input{\file@loc Integrals/2311-Compute-Integral-0010.HELP.tex}

\[\int_{5}^{6} {x^{2} - 4 \, x + 4}\;dx=\answer{\frac{37}{3}}\]
\end{problem}}%}

%%%%%%%%%%%%%%%%%%%%%%


\latexProblemContent{
\begin{problem}

Use the Fundamental Theorem of Calculus to evaluate the integral.

\expandafter\input{\file@loc Integrals/2311-Compute-Integral-0010.HELP.tex}

\[\int_{-6}^{7} {x - 5}\;dx=\answer{-\frac{117}{2}}\]
\end{problem}}%}

%%%%%%%%%%%%%%%%%%%%%%


\latexProblemContent{
\begin{problem}

Use the Fundamental Theorem of Calculus to evaluate the integral.

\expandafter\input{\file@loc Integrals/2311-Compute-Integral-0010.HELP.tex}

\[\int_{-9}^{6} {x^{2} - 3 \, x - 4}\;dx=\answer{\frac{645}{2}}\]
\end{problem}}%}

%%%%%%%%%%%%%%%%%%%%%%


\latexProblemContent{
\begin{problem}

Use the Fundamental Theorem of Calculus to evaluate the integral.

\expandafter\input{\file@loc Integrals/2311-Compute-Integral-0010.HELP.tex}

\[\int_{1}^{12} {x^{2} - 2 \, x + 1}\;dx=\answer{\frac{1331}{3}}\]
\end{problem}}%}

%%%%%%%%%%%%%%%%%%%%%%


\latexProblemContent{
\begin{problem}

Use the Fundamental Theorem of Calculus to evaluate the integral.

\expandafter\input{\file@loc Integrals/2311-Compute-Integral-0010.HELP.tex}

\[\int_{5}^{10} {x + 3}\;dx=\answer{\frac{105}{2}}\]
\end{problem}}%}

%%%%%%%%%%%%%%%%%%%%%%


\latexProblemContent{
\begin{problem}

Use the Fundamental Theorem of Calculus to evaluate the integral.

\expandafter\input{\file@loc Integrals/2311-Compute-Integral-0010.HELP.tex}

\[\int_{-10}^{9} {x^{3} - 9 \, x^{2} + 27 \, x - 27}\;dx=\answer{-\frac{27265}{4}}\]
\end{problem}}%}

%%%%%%%%%%%%%%%%%%%%%%


\latexProblemContent{
\begin{problem}

Use the Fundamental Theorem of Calculus to evaluate the integral.

\expandafter\input{\file@loc Integrals/2311-Compute-Integral-0010.HELP.tex}

\[\int_{-4}^{0} {x^{3} + 11 \, x^{2} + 35 \, x + 25}\;dx=\answer{-\frac{28}{3}}\]
\end{problem}}%}

%%%%%%%%%%%%%%%%%%%%%%


\latexProblemContent{
\begin{problem}

Use the Fundamental Theorem of Calculus to evaluate the integral.

\expandafter\input{\file@loc Integrals/2311-Compute-Integral-0010.HELP.tex}

\[\int_{-8}^{3} {x^{2} + 8 \, x + 7}\;dx=\answer{\frac{110}{3}}\]
\end{problem}}%}

%%%%%%%%%%%%%%%%%%%%%%


\latexProblemContent{
\begin{problem}

Use the Fundamental Theorem of Calculus to evaluate the integral.

\expandafter\input{\file@loc Integrals/2311-Compute-Integral-0010.HELP.tex}

\[\int_{4}^{5} {x^{2} - 6 \, x + 9}\;dx=\answer{\frac{7}{3}}\]
\end{problem}}%}

%%%%%%%%%%%%%%%%%%%%%%


\latexProblemContent{
\begin{problem}

Use the Fundamental Theorem of Calculus to evaluate the integral.

\expandafter\input{\file@loc Integrals/2311-Compute-Integral-0010.HELP.tex}

\[\int_{-10}^{1} {x^{2} + 4 \, x + 4}\;dx=\answer{\frac{539}{3}}\]
\end{problem}}%}

%%%%%%%%%%%%%%%%%%%%%%


\latexProblemContent{
\begin{problem}

Use the Fundamental Theorem of Calculus to evaluate the integral.

\expandafter\input{\file@loc Integrals/2311-Compute-Integral-0010.HELP.tex}

\[\int_{2}^{6} {x^{3} + 9 \, x^{2} + 24 \, x + 20}\;dx=\answer{1408}\]
\end{problem}}%}

%%%%%%%%%%%%%%%%%%%%%%


\latexProblemContent{
\begin{problem}

Use the Fundamental Theorem of Calculus to evaluate the integral.

\expandafter\input{\file@loc Integrals/2311-Compute-Integral-0010.HELP.tex}

\[\int_{-9}^{-7} {x^{3} - 48 \, x - 128}\;dx=\answer{-528}\]
\end{problem}}%}

%%%%%%%%%%%%%%%%%%%%%%


\latexProblemContent{
\begin{problem}

Use the Fundamental Theorem of Calculus to evaluate the integral.

\expandafter\input{\file@loc Integrals/2311-Compute-Integral-0010.HELP.tex}

\[\int_{-5}^{0} {x - 2}\;dx=\answer{-\frac{45}{2}}\]
\end{problem}}%}

%%%%%%%%%%%%%%%%%%%%%%


\latexProblemContent{
\begin{problem}

Use the Fundamental Theorem of Calculus to evaluate the integral.

\expandafter\input{\file@loc Integrals/2311-Compute-Integral-0010.HELP.tex}

\[\int_{-7}^{2} {x^{3} + 15 \, x^{2} + 75 \, x + 125}\;dx=\answer{\frac{2385}{4}}\]
\end{problem}}%}

%%%%%%%%%%%%%%%%%%%%%%


\latexProblemContent{
\begin{problem}

Use the Fundamental Theorem of Calculus to evaluate the integral.

\expandafter\input{\file@loc Integrals/2311-Compute-Integral-0010.HELP.tex}

\[\int_{-10}^{7} {x^{3} + 3 \, x^{2} + 3 \, x + 1}\;dx=\answer{-\frac{2465}{4}}\]
\end{problem}}%}

%%%%%%%%%%%%%%%%%%%%%%


\latexProblemContent{
\begin{problem}

Use the Fundamental Theorem of Calculus to evaluate the integral.

\expandafter\input{\file@loc Integrals/2311-Compute-Integral-0010.HELP.tex}

\[\int_{-2}^{6} {x^{3} - 6 \, x^{2} + 12 \, x - 8}\;dx=\answer{0}\]
\end{problem}}%}

%%%%%%%%%%%%%%%%%%%%%%


\latexProblemContent{
\begin{problem}

Use the Fundamental Theorem of Calculus to evaluate the integral.

\expandafter\input{\file@loc Integrals/2311-Compute-Integral-0010.HELP.tex}

\[\int_{-10}^{-7} {x - 2}\;dx=\answer{-\frac{63}{2}}\]
\end{problem}}%}

%%%%%%%%%%%%%%%%%%%%%%


\latexProblemContent{
\begin{problem}

Use the Fundamental Theorem of Calculus to evaluate the integral.

\expandafter\input{\file@loc Integrals/2311-Compute-Integral-0010.HELP.tex}

\[\int_{-4}^{12} {x^{3} + 15 \, x^{2} + 75 \, x + 125}\;dx=\answer{20880}\]
\end{problem}}%}

%%%%%%%%%%%%%%%%%%%%%%


\latexProblemContent{
\begin{problem}

Use the Fundamental Theorem of Calculus to evaluate the integral.

\expandafter\input{\file@loc Integrals/2311-Compute-Integral-0010.HELP.tex}

\[\int_{-10}^{10} {x^{2} - 10 \, x + 25}\;dx=\answer{\frac{3500}{3}}\]
\end{problem}}%}

%%%%%%%%%%%%%%%%%%%%%%


\latexProblemContent{
\begin{problem}

Use the Fundamental Theorem of Calculus to evaluate the integral.

\expandafter\input{\file@loc Integrals/2311-Compute-Integral-0010.HELP.tex}

\[\int_{5}^{7} {x - 2}\;dx=\answer{8}\]
\end{problem}}%}

%%%%%%%%%%%%%%%%%%%%%%


\latexProblemContent{
\begin{problem}

Use the Fundamental Theorem of Calculus to evaluate the integral.

\expandafter\input{\file@loc Integrals/2311-Compute-Integral-0010.HELP.tex}

\[\int_{-1}^{1} {x - 2}\;dx=\answer{-4}\]
\end{problem}}%}

%%%%%%%%%%%%%%%%%%%%%%


