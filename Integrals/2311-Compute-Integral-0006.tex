%%%%%%%%%%%%%%%%%%%%%%%
%%\tagged{Cat@One, Cat@Two, Cat@Three, Cat@Four, Cat@Five, Ans@ShortAns, Type@Compute, Topic@Integral, Sub@Definite, Sub@Theorems_FTC}{

\latexProblemContent{
\begin{problem}

Use the Fundamental Theorem of Calculus to find the derivative of the function.
\[g(t)=\int_{5}^{t} {\sin\left(x - 4\right)}\;dx\]

\expandafter\input{\file@loc Integrals/2311-Compute-Integral-0006.HELP.tex}

\[\dfrac{d}{dt}(g(t))=\answer{\sin\left(t - 4\right)}\]
\end{problem}}%}

%%%%%%%%%%%%%%%%%%%%%%



\latexProblemContent{
\begin{problem}

Use the Fundamental Theorem of Calculus to find the derivative of the function.
\[g(t)=\int_{2}^{t} {\frac{1}{{\left(x + 2\right)}^{3}}}\;dx\]

\expandafter\input{\file@loc Integrals/2311-Compute-Integral-0006.HELP.tex}

\[\dfrac{d}{dt}(g(t))=\answer{\frac{1}{{\left(t + 2\right)}^{3}}}\]
\end{problem}}%}

%%%%%%%%%%%%%%%%%%%%%%



\latexProblemContent{
\begin{problem}

Use the Fundamental Theorem of Calculus to find the derivative of the function.
\[g(t)=\int_{3}^{t} {\sin\left(x + 1\right)}\;dx\]

\expandafter\input{\file@loc Integrals/2311-Compute-Integral-0006.HELP.tex}

\[\dfrac{d}{dt}(g(t))=\answer{\sin\left(t + 1\right)}\]
\end{problem}}%}

%%%%%%%%%%%%%%%%%%%%%%



\latexProblemContent{
\begin{problem}

Use the Fundamental Theorem of Calculus to find the derivative of the function.
\[g(t)=\int_{5}^{t} {x - 1}\;dx\]

\expandafter\input{\file@loc Integrals/2311-Compute-Integral-0006.HELP.tex}

\[\dfrac{d}{dt}(g(t))=\answer{t - 1}\]
\end{problem}}%}

%%%%%%%%%%%%%%%%%%%%%%



\latexProblemContent{
\begin{problem}

Use the Fundamental Theorem of Calculus to find the derivative of the function.
\[g(t)=\int_{2}^{t} {\sin\left(x + 1\right)}\;dx\]

\expandafter\input{\file@loc Integrals/2311-Compute-Integral-0006.HELP.tex}

\[\dfrac{d}{dt}(g(t))=\answer{\sin\left(t + 1\right)}\]
\end{problem}}%}

%%%%%%%%%%%%%%%%%%%%%%



\latexProblemContent{
\begin{problem}

Use the Fundamental Theorem of Calculus to find the derivative of the function.
\[g(t)=\int_{1}^{t} {\frac{1}{{\left(x - 6\right)}^{2}}}\;dx\]

\expandafter\input{\file@loc Integrals/2311-Compute-Integral-0006.HELP.tex}

\[\dfrac{d}{dt}(g(t))=\answer{\frac{1}{{\left(t - 6\right)}^{2}}}\]
\end{problem}}%}

%%%%%%%%%%%%%%%%%%%%%%



\latexProblemContent{
\begin{problem}

Use the Fundamental Theorem of Calculus to find the derivative of the function.
\[g(t)=\int_{3}^{t} {\frac{1}{{\left(x + 4\right)}^{3}}}\;dx\]

\expandafter\input{\file@loc Integrals/2311-Compute-Integral-0006.HELP.tex}

\[\dfrac{d}{dt}(g(t))=\answer{\frac{1}{{\left(t + 4\right)}^{3}}}\]
\end{problem}}%}

%%%%%%%%%%%%%%%%%%%%%%



\latexProblemContent{
\begin{problem}

Use the Fundamental Theorem of Calculus to find the derivative of the function.
\[g(t)=\int_{5}^{t} {{\left(x + 8\right)}^{2}}\;dx\]

\expandafter\input{\file@loc Integrals/2311-Compute-Integral-0006.HELP.tex}

\[\dfrac{d}{dt}(g(t))=\answer{{\left(t + 8\right)}^{2}}\]
\end{problem}}%}

%%%%%%%%%%%%%%%%%%%%%%



\latexProblemContent{
\begin{problem}

Use the Fundamental Theorem of Calculus to find the derivative of the function.
\[g(t)=\int_{3}^{t} {{\left(x + 6\right)}^{4}}\;dx\]

\expandafter\input{\file@loc Integrals/2311-Compute-Integral-0006.HELP.tex}

\[\dfrac{d}{dt}(g(t))=\answer{{\left(t + 6\right)}^{4}}\]
\end{problem}}%}

%%%%%%%%%%%%%%%%%%%%%%



\latexProblemContent{
\begin{problem}

Use the Fundamental Theorem of Calculus to find the derivative of the function.
\[g(t)=\int_{3}^{t} {{\left(x - 3\right)}^{4}}\;dx\]

\expandafter\input{\file@loc Integrals/2311-Compute-Integral-0006.HELP.tex}

\[\dfrac{d}{dt}(g(t))=\answer{{\left(t - 3\right)}^{4}}\]
\end{problem}}%}

%%%%%%%%%%%%%%%%%%%%%%



\latexProblemContent{
\begin{problem}

Use the Fundamental Theorem of Calculus to find the derivative of the function.
\[g(t)=\int_{2}^{t} {\log\left(x - 4\right)}\;dx\]

\expandafter\input{\file@loc Integrals/2311-Compute-Integral-0006.HELP.tex}

\[\dfrac{d}{dt}(g(t))=\answer{\log\left(t - 4\right)}\]
\end{problem}}%}

%%%%%%%%%%%%%%%%%%%%%%



\latexProblemContent{
\begin{problem}

Use the Fundamental Theorem of Calculus to find the derivative of the function.
\[g(t)=\int_{5}^{t} {\cos\left(x - 6\right)}\;dx\]

\expandafter\input{\file@loc Integrals/2311-Compute-Integral-0006.HELP.tex}

\[\dfrac{d}{dt}(g(t))=\answer{\cos\left(t - 6\right)}\]
\end{problem}}%}

%%%%%%%%%%%%%%%%%%%%%%



\latexProblemContent{
\begin{problem}

Use the Fundamental Theorem of Calculus to find the derivative of the function.
\[g(t)=\int_{3}^{t} {{\left(x - 4\right)}^{4}}\;dx\]

\expandafter\input{\file@loc Integrals/2311-Compute-Integral-0006.HELP.tex}

\[\dfrac{d}{dt}(g(t))=\answer{{\left(t - 4\right)}^{4}}\]
\end{problem}}%}

%%%%%%%%%%%%%%%%%%%%%%



\latexProblemContent{
\begin{problem}

Use the Fundamental Theorem of Calculus to find the derivative of the function.
\[g(t)=\int_{4}^{t} {\sin\left(x - 7\right)}\;dx\]

\expandafter\input{\file@loc Integrals/2311-Compute-Integral-0006.HELP.tex}

\[\dfrac{d}{dt}(g(t))=\answer{\sin\left(t - 7\right)}\]
\end{problem}}%}

%%%%%%%%%%%%%%%%%%%%%%



\latexProblemContent{
\begin{problem}

Use the Fundamental Theorem of Calculus to find the derivative of the function.
\[g(t)=\int_{5}^{t} {\log\left(x - 4\right)}\;dx\]

\expandafter\input{\file@loc Integrals/2311-Compute-Integral-0006.HELP.tex}

\[\dfrac{d}{dt}(g(t))=\answer{\log\left(t - 4\right)}\]
\end{problem}}%}

%%%%%%%%%%%%%%%%%%%%%%



\latexProblemContent{
\begin{problem}

Use the Fundamental Theorem of Calculus to find the derivative of the function.
\[g(t)=\int_{2}^{t} {\cos\left(x - 4\right)}\;dx\]

\expandafter\input{\file@loc Integrals/2311-Compute-Integral-0006.HELP.tex}

\[\dfrac{d}{dt}(g(t))=\answer{\cos\left(t - 4\right)}\]
\end{problem}}%}

%%%%%%%%%%%%%%%%%%%%%%



\latexProblemContent{
\begin{problem}

Use the Fundamental Theorem of Calculus to find the derivative of the function.
\[g(t)=\int_{1}^{t} {\sin\left(x + 4\right)}\;dx\]

\expandafter\input{\file@loc Integrals/2311-Compute-Integral-0006.HELP.tex}

\[\dfrac{d}{dt}(g(t))=\answer{\sin\left(t + 4\right)}\]
\end{problem}}%}

%%%%%%%%%%%%%%%%%%%%%%



\latexProblemContent{
\begin{problem}

Use the Fundamental Theorem of Calculus to find the derivative of the function.
\[g(t)=\int_{3}^{t} {\frac{1}{x + 4}}\;dx\]

\expandafter\input{\file@loc Integrals/2311-Compute-Integral-0006.HELP.tex}

\[\dfrac{d}{dt}(g(t))=\answer{\frac{1}{t + 4}}\]
\end{problem}}%}

%%%%%%%%%%%%%%%%%%%%%%



\latexProblemContent{
\begin{problem}

Use the Fundamental Theorem of Calculus to find the derivative of the function.
\[g(t)=\int_{2}^{t} {\sqrt{x + 7}}\;dx\]

\expandafter\input{\file@loc Integrals/2311-Compute-Integral-0006.HELP.tex}

\[\dfrac{d}{dt}(g(t))=\answer{\sqrt{t + 7}}\]
\end{problem}}%}

%%%%%%%%%%%%%%%%%%%%%%



\latexProblemContent{
\begin{problem}

Use the Fundamental Theorem of Calculus to find the derivative of the function.
\[g(t)=\int_{4}^{t} {\sqrt{x - 3}}\;dx\]

\expandafter\input{\file@loc Integrals/2311-Compute-Integral-0006.HELP.tex}

\[\dfrac{d}{dt}(g(t))=\answer{\sqrt{t - 3}}\]
\end{problem}}%}

%%%%%%%%%%%%%%%%%%%%%%



\latexProblemContent{
\begin{problem}

Use the Fundamental Theorem of Calculus to find the derivative of the function.
\[g(t)=\int_{4}^{t} {{\left(x - 6\right)}^{3}}\;dx\]

\expandafter\input{\file@loc Integrals/2311-Compute-Integral-0006.HELP.tex}

\[\dfrac{d}{dt}(g(t))=\answer{{\left(t - 6\right)}^{3}}\]
\end{problem}}%}

%%%%%%%%%%%%%%%%%%%%%%



\latexProblemContent{
\begin{problem}

Use the Fundamental Theorem of Calculus to find the derivative of the function.
\[g(t)=\int_{3}^{t} {x + 1}\;dx\]

\expandafter\input{\file@loc Integrals/2311-Compute-Integral-0006.HELP.tex}

\[\dfrac{d}{dt}(g(t))=\answer{t + 1}\]
\end{problem}}%}

%%%%%%%%%%%%%%%%%%%%%%



\latexProblemContent{
\begin{problem}

Use the Fundamental Theorem of Calculus to find the derivative of the function.
\[g(t)=\int_{3}^{t} {\frac{1}{{\left(x + 3\right)}^{3}}}\;dx\]

\expandafter\input{\file@loc Integrals/2311-Compute-Integral-0006.HELP.tex}

\[\dfrac{d}{dt}(g(t))=\answer{\frac{1}{{\left(t + 3\right)}^{3}}}\]
\end{problem}}%}

%%%%%%%%%%%%%%%%%%%%%%



\latexProblemContent{
\begin{problem}

Use the Fundamental Theorem of Calculus to find the derivative of the function.
\[g(t)=\int_{1}^{t} {\frac{1}{x - 5}}\;dx\]

\expandafter\input{\file@loc Integrals/2311-Compute-Integral-0006.HELP.tex}

\[\dfrac{d}{dt}(g(t))=\answer{\frac{1}{t - 5}}\]
\end{problem}}%}

%%%%%%%%%%%%%%%%%%%%%%



\latexProblemContent{
\begin{problem}

Use the Fundamental Theorem of Calculus to find the derivative of the function.
\[g(t)=\int_{4}^{t} {x - 7}\;dx\]

\expandafter\input{\file@loc Integrals/2311-Compute-Integral-0006.HELP.tex}

\[\dfrac{d}{dt}(g(t))=\answer{t - 7}\]
\end{problem}}%}

%%%%%%%%%%%%%%%%%%%%%%



\latexProblemContent{
\begin{problem}

Use the Fundamental Theorem of Calculus to find the derivative of the function.
\[g(t)=\int_{1}^{t} {\sqrt{x - 3}}\;dx\]

\expandafter\input{\file@loc Integrals/2311-Compute-Integral-0006.HELP.tex}

\[\dfrac{d}{dt}(g(t))=\answer{\sqrt{t - 3}}\]
\end{problem}}%}

%%%%%%%%%%%%%%%%%%%%%%



\latexProblemContent{
\begin{problem}

Use the Fundamental Theorem of Calculus to find the derivative of the function.
\[g(t)=\int_{1}^{t} {\frac{1}{{\left(x - 5\right)}^{2}}}\;dx\]

\expandafter\input{\file@loc Integrals/2311-Compute-Integral-0006.HELP.tex}

\[\dfrac{d}{dt}(g(t))=\answer{\frac{1}{{\left(t - 5\right)}^{2}}}\]
\end{problem}}%}

%%%%%%%%%%%%%%%%%%%%%%



\latexProblemContent{
\begin{problem}

Use the Fundamental Theorem of Calculus to find the derivative of the function.
\[g(t)=\int_{1}^{t} {e^{\left(x - 7\right)}}\;dx\]

\expandafter\input{\file@loc Integrals/2311-Compute-Integral-0006.HELP.tex}

\[\dfrac{d}{dt}(g(t))=\answer{e^{\left(t - 7\right)}}\]
\end{problem}}%}

%%%%%%%%%%%%%%%%%%%%%%



\latexProblemContent{
\begin{problem}

Use the Fundamental Theorem of Calculus to find the derivative of the function.
\[g(t)=\int_{5}^{t} {e^{\left(x - 9\right)}}\;dx\]

\expandafter\input{\file@loc Integrals/2311-Compute-Integral-0006.HELP.tex}

\[\dfrac{d}{dt}(g(t))=\answer{e^{\left(t - 9\right)}}\]
\end{problem}}%}

%%%%%%%%%%%%%%%%%%%%%%



\latexProblemContent{
\begin{problem}

Use the Fundamental Theorem of Calculus to find the derivative of the function.
\[g(t)=\int_{4}^{t} {\frac{1}{x - 4}}\;dx\]

\expandafter\input{\file@loc Integrals/2311-Compute-Integral-0006.HELP.tex}

\[\dfrac{d}{dt}(g(t))=\answer{\frac{1}{t - 4}}\]
\end{problem}}%}

%%%%%%%%%%%%%%%%%%%%%%



\latexProblemContent{
\begin{problem}

Use the Fundamental Theorem of Calculus to find the derivative of the function.
\[g(t)=\int_{5}^{t} {{\left(x + 2\right)}^{2}}\;dx\]

\expandafter\input{\file@loc Integrals/2311-Compute-Integral-0006.HELP.tex}

\[\dfrac{d}{dt}(g(t))=\answer{{\left(t + 2\right)}^{2}}\]
\end{problem}}%}

%%%%%%%%%%%%%%%%%%%%%%



\latexProblemContent{
\begin{problem}

Use the Fundamental Theorem of Calculus to find the derivative of the function.
\[g(t)=\int_{5}^{t} {{\left(x - 2\right)}^{4}}\;dx\]

\expandafter\input{\file@loc Integrals/2311-Compute-Integral-0006.HELP.tex}

\[\dfrac{d}{dt}(g(t))=\answer{{\left(t - 2\right)}^{4}}\]
\end{problem}}%}

%%%%%%%%%%%%%%%%%%%%%%



\latexProblemContent{
\begin{problem}

Use the Fundamental Theorem of Calculus to find the derivative of the function.
\[g(t)=\int_{5}^{t} {\sin\left(x - 9\right)}\;dx\]

\expandafter\input{\file@loc Integrals/2311-Compute-Integral-0006.HELP.tex}

\[\dfrac{d}{dt}(g(t))=\answer{\sin\left(t - 9\right)}\]
\end{problem}}%}

%%%%%%%%%%%%%%%%%%%%%%



\latexProblemContent{
\begin{problem}

Use the Fundamental Theorem of Calculus to find the derivative of the function.
\[g(t)=\int_{4}^{t} {{\left(x - 3\right)}^{4}}\;dx\]

\expandafter\input{\file@loc Integrals/2311-Compute-Integral-0006.HELP.tex}

\[\dfrac{d}{dt}(g(t))=\answer{{\left(t - 3\right)}^{4}}\]
\end{problem}}%}

%%%%%%%%%%%%%%%%%%%%%%



\latexProblemContent{
\begin{problem}

Use the Fundamental Theorem of Calculus to find the derivative of the function.
\[g(t)=\int_{2}^{t} {{\left(x + 10\right)}^{3}}\;dx\]

\expandafter\input{\file@loc Integrals/2311-Compute-Integral-0006.HELP.tex}

\[\dfrac{d}{dt}(g(t))=\answer{{\left(t + 10\right)}^{3}}\]
\end{problem}}%}

%%%%%%%%%%%%%%%%%%%%%%



\latexProblemContent{
\begin{problem}

Use the Fundamental Theorem of Calculus to find the derivative of the function.
\[g(t)=\int_{3}^{t} {{\left(x + 1\right)}^{2}}\;dx\]

\expandafter\input{\file@loc Integrals/2311-Compute-Integral-0006.HELP.tex}

\[\dfrac{d}{dt}(g(t))=\answer{{\left(t + 1\right)}^{2}}\]
\end{problem}}%}

%%%%%%%%%%%%%%%%%%%%%%



\latexProblemContent{
\begin{problem}

Use the Fundamental Theorem of Calculus to find the derivative of the function.
\[g(t)=\int_{5}^{t} {{\left(x - 8\right)}^{2}}\;dx\]

\expandafter\input{\file@loc Integrals/2311-Compute-Integral-0006.HELP.tex}

\[\dfrac{d}{dt}(g(t))=\answer{{\left(t - 8\right)}^{2}}\]
\end{problem}}%}

%%%%%%%%%%%%%%%%%%%%%%



\latexProblemContent{
\begin{problem}

Use the Fundamental Theorem of Calculus to find the derivative of the function.
\[g(t)=\int_{4}^{t} {\frac{1}{{\left(x + 7\right)}^{2}}}\;dx\]

\expandafter\input{\file@loc Integrals/2311-Compute-Integral-0006.HELP.tex}

\[\dfrac{d}{dt}(g(t))=\answer{\frac{1}{{\left(t + 7\right)}^{2}}}\]
\end{problem}}%}

%%%%%%%%%%%%%%%%%%%%%%



\latexProblemContent{
\begin{problem}

Use the Fundamental Theorem of Calculus to find the derivative of the function.
\[g(t)=\int_{5}^{t} {{\left(x - 1\right)}^{4}}\;dx\]

\expandafter\input{\file@loc Integrals/2311-Compute-Integral-0006.HELP.tex}

\[\dfrac{d}{dt}(g(t))=\answer{{\left(t - 1\right)}^{4}}\]
\end{problem}}%}

%%%%%%%%%%%%%%%%%%%%%%



\latexProblemContent{
\begin{problem}

Use the Fundamental Theorem of Calculus to find the derivative of the function.
\[g(t)=\int_{5}^{t} {\cos\left(x + 1\right)}\;dx\]

\expandafter\input{\file@loc Integrals/2311-Compute-Integral-0006.HELP.tex}

\[\dfrac{d}{dt}(g(t))=\answer{\cos\left(t + 1\right)}\]
\end{problem}}%}

%%%%%%%%%%%%%%%%%%%%%%



\latexProblemContent{
\begin{problem}

Use the Fundamental Theorem of Calculus to find the derivative of the function.
\[g(t)=\int_{3}^{t} {e^{\left(x - 8\right)}}\;dx\]

\expandafter\input{\file@loc Integrals/2311-Compute-Integral-0006.HELP.tex}

\[\dfrac{d}{dt}(g(t))=\answer{e^{\left(t - 8\right)}}\]
\end{problem}}%}

%%%%%%%%%%%%%%%%%%%%%%



\latexProblemContent{
\begin{problem}

Use the Fundamental Theorem of Calculus to find the derivative of the function.
\[g(t)=\int_{4}^{t} {e^{\left(x + 6\right)}}\;dx\]

\expandafter\input{\file@loc Integrals/2311-Compute-Integral-0006.HELP.tex}

\[\dfrac{d}{dt}(g(t))=\answer{e^{\left(t + 6\right)}}\]
\end{problem}}%}

%%%%%%%%%%%%%%%%%%%%%%



\latexProblemContent{
\begin{problem}

Use the Fundamental Theorem of Calculus to find the derivative of the function.
\[g(t)=\int_{3}^{t} {\sin\left(x - 9\right)}\;dx\]

\expandafter\input{\file@loc Integrals/2311-Compute-Integral-0006.HELP.tex}

\[\dfrac{d}{dt}(g(t))=\answer{\sin\left(t - 9\right)}\]
\end{problem}}%}

%%%%%%%%%%%%%%%%%%%%%%



\latexProblemContent{
\begin{problem}

Use the Fundamental Theorem of Calculus to find the derivative of the function.
\[g(t)=\int_{5}^{t} {\frac{1}{{\left(x - 8\right)}^{3}}}\;dx\]

\expandafter\input{\file@loc Integrals/2311-Compute-Integral-0006.HELP.tex}

\[\dfrac{d}{dt}(g(t))=\answer{\frac{1}{{\left(t - 8\right)}^{3}}}\]
\end{problem}}%}

%%%%%%%%%%%%%%%%%%%%%%



\latexProblemContent{
\begin{problem}

Use the Fundamental Theorem of Calculus to find the derivative of the function.
\[g(t)=\int_{4}^{t} {\frac{1}{x - 8}}\;dx\]

\expandafter\input{\file@loc Integrals/2311-Compute-Integral-0006.HELP.tex}

\[\dfrac{d}{dt}(g(t))=\answer{\frac{1}{t - 8}}\]
\end{problem}}%}

%%%%%%%%%%%%%%%%%%%%%%



\latexProblemContent{
\begin{problem}

Use the Fundamental Theorem of Calculus to find the derivative of the function.
\[g(t)=\int_{1}^{t} {\log\left(x - 1\right)}\;dx\]

\expandafter\input{\file@loc Integrals/2311-Compute-Integral-0006.HELP.tex}

\[\dfrac{d}{dt}(g(t))=\answer{\log\left(t - 1\right)}\]
\end{problem}}%}

%%%%%%%%%%%%%%%%%%%%%%



\latexProblemContent{
\begin{problem}

Use the Fundamental Theorem of Calculus to find the derivative of the function.
\[g(t)=\int_{2}^{t} {\cos\left(x - 6\right)}\;dx\]

\expandafter\input{\file@loc Integrals/2311-Compute-Integral-0006.HELP.tex}

\[\dfrac{d}{dt}(g(t))=\answer{\cos\left(t - 6\right)}\]
\end{problem}}%}

%%%%%%%%%%%%%%%%%%%%%%



\latexProblemContent{
\begin{problem}

Use the Fundamental Theorem of Calculus to find the derivative of the function.
\[g(t)=\int_{2}^{t} {\cos\left(x + 1\right)}\;dx\]

\expandafter\input{\file@loc Integrals/2311-Compute-Integral-0006.HELP.tex}

\[\dfrac{d}{dt}(g(t))=\answer{\cos\left(t + 1\right)}\]
\end{problem}}%}

%%%%%%%%%%%%%%%%%%%%%%



%%%%%%%%%%%%%%%%%%%%%%



\latexProblemContent{
\begin{problem}

Use the Fundamental Theorem of Calculus to find the derivative of the function.
\[g(t)=\int_{3}^{t} {\cos\left(x + 4\right)}\;dx\]

\expandafter\input{\file@loc Integrals/2311-Compute-Integral-0006.HELP.tex}

\[\dfrac{d}{dt}(g(t))=\answer{\cos\left(t + 4\right)}\]
\end{problem}}%}

%%%%%%%%%%%%%%%%%%%%%%



\latexProblemContent{
\begin{problem}

Use the Fundamental Theorem of Calculus to find the derivative of the function.
\[g(t)=\int_{1}^{t} {\cos\left(x - 9\right)}\;dx\]

\expandafter\input{\file@loc Integrals/2311-Compute-Integral-0006.HELP.tex}

\[\dfrac{d}{dt}(g(t))=\answer{\cos\left(t - 9\right)}\]
\end{problem}}%}

%%%%%%%%%%%%%%%%%%%%%%



\latexProblemContent{
\begin{problem}

Use the Fundamental Theorem of Calculus to find the derivative of the function.
\[g(t)=\int_{3}^{t} {\sqrt{x + 3}}\;dx\]

\expandafter\input{\file@loc Integrals/2311-Compute-Integral-0006.HELP.tex}

\[\dfrac{d}{dt}(g(t))=\answer{\sqrt{t + 3}}\]
\end{problem}}%}

%%%%%%%%%%%%%%%%%%%%%%



\latexProblemContent{
\begin{problem}

Use the Fundamental Theorem of Calculus to find the derivative of the function.
\[g(t)=\int_{1}^{t} {e^{\left(x - 3\right)}}\;dx\]

\expandafter\input{\file@loc Integrals/2311-Compute-Integral-0006.HELP.tex}

\[\dfrac{d}{dt}(g(t))=\answer{e^{\left(t - 3\right)}}\]
\end{problem}}%}

%%%%%%%%%%%%%%%%%%%%%%



\latexProblemContent{
\begin{problem}

Use the Fundamental Theorem of Calculus to find the derivative of the function.
\[g(t)=\int_{3}^{t} {{\left(x + 10\right)}^{3}}\;dx\]

\expandafter\input{\file@loc Integrals/2311-Compute-Integral-0006.HELP.tex}

\[\dfrac{d}{dt}(g(t))=\answer{{\left(t + 10\right)}^{3}}\]
\end{problem}}%}

%%%%%%%%%%%%%%%%%%%%%%



\latexProblemContent{
\begin{problem}

Use the Fundamental Theorem of Calculus to find the derivative of the function.
\[g(t)=\int_{1}^{t} {\log\left(x - 2\right)}\;dx\]

\expandafter\input{\file@loc Integrals/2311-Compute-Integral-0006.HELP.tex}

\[\dfrac{d}{dt}(g(t))=\answer{\log\left(t - 2\right)}\]
\end{problem}}%}

%%%%%%%%%%%%%%%%%%%%%%



\latexProblemContent{
\begin{problem}

Use the Fundamental Theorem of Calculus to find the derivative of the function.
\[g(t)=\int_{1}^{t} {x + 8}\;dx\]

\expandafter\input{\file@loc Integrals/2311-Compute-Integral-0006.HELP.tex}

\[\dfrac{d}{dt}(g(t))=\answer{t + 8}\]
\end{problem}}%}

%%%%%%%%%%%%%%%%%%%%%%



\latexProblemContent{
\begin{problem}

Use the Fundamental Theorem of Calculus to find the derivative of the function.
\[g(t)=\int_{4}^{t} {\log\left(x - 5\right)}\;dx\]

\expandafter\input{\file@loc Integrals/2311-Compute-Integral-0006.HELP.tex}

\[\dfrac{d}{dt}(g(t))=\answer{\log\left(t - 5\right)}\]
\end{problem}}%}

%%%%%%%%%%%%%%%%%%%%%%



\latexProblemContent{
\begin{problem}

Use the Fundamental Theorem of Calculus to find the derivative of the function.
\[g(t)=\int_{5}^{t} {e^{\left(x - 4\right)}}\;dx\]

\expandafter\input{\file@loc Integrals/2311-Compute-Integral-0006.HELP.tex}

\[\dfrac{d}{dt}(g(t))=\answer{e^{\left(t - 4\right)}}\]
\end{problem}}%}

%%%%%%%%%%%%%%%%%%%%%%



\latexProblemContent{
\begin{problem}

Use the Fundamental Theorem of Calculus to find the derivative of the function.
\[g(t)=\int_{5}^{t} {{\left(x + 2\right)}^{4}}\;dx\]

\expandafter\input{\file@loc Integrals/2311-Compute-Integral-0006.HELP.tex}

\[\dfrac{d}{dt}(g(t))=\answer{{\left(t + 2\right)}^{4}}\]
\end{problem}}%}

%%%%%%%%%%%%%%%%%%%%%%



\latexProblemContent{
\begin{problem}

Use the Fundamental Theorem of Calculus to find the derivative of the function.
\[g(t)=\int_{5}^{t} {\cos\left(x + 6\right)}\;dx\]

\expandafter\input{\file@loc Integrals/2311-Compute-Integral-0006.HELP.tex}

\[\dfrac{d}{dt}(g(t))=\answer{\cos\left(t + 6\right)}\]
\end{problem}}%}

%%%%%%%%%%%%%%%%%%%%%%



\latexProblemContent{
\begin{problem}

Use the Fundamental Theorem of Calculus to find the derivative of the function.
\[g(t)=\int_{3}^{t} {{\left(x + 2\right)}^{2}}\;dx\]

\expandafter\input{\file@loc Integrals/2311-Compute-Integral-0006.HELP.tex}

\[\dfrac{d}{dt}(g(t))=\answer{{\left(t + 2\right)}^{2}}\]
\end{problem}}%}

%%%%%%%%%%%%%%%%%%%%%%



\latexProblemContent{
\begin{problem}

Use the Fundamental Theorem of Calculus to find the derivative of the function.
\[g(t)=\int_{5}^{t} {{\left(x - 9\right)}^{4}}\;dx\]

\expandafter\input{\file@loc Integrals/2311-Compute-Integral-0006.HELP.tex}

\[\dfrac{d}{dt}(g(t))=\answer{{\left(t - 9\right)}^{4}}\]
\end{problem}}%}

%%%%%%%%%%%%%%%%%%%%%%



\latexProblemContent{
\begin{problem}

Use the Fundamental Theorem of Calculus to find the derivative of the function.
\[g(t)=\int_{1}^{t} {{\left(x + 6\right)}^{4}}\;dx\]

\expandafter\input{\file@loc Integrals/2311-Compute-Integral-0006.HELP.tex}

\[\dfrac{d}{dt}(g(t))=\answer{{\left(t + 6\right)}^{4}}\]
\end{problem}}%}

%%%%%%%%%%%%%%%%%%%%%%



\latexProblemContent{
\begin{problem}

Use the Fundamental Theorem of Calculus to find the derivative of the function.
\[g(t)=\int_{5}^{t} {{\left(x - 7\right)}^{4}}\;dx\]

\expandafter\input{\file@loc Integrals/2311-Compute-Integral-0006.HELP.tex}

\[\dfrac{d}{dt}(g(t))=\answer{{\left(t - 7\right)}^{4}}\]
\end{problem}}%}

%%%%%%%%%%%%%%%%%%%%%%



\latexProblemContent{
\begin{problem}

Use the Fundamental Theorem of Calculus to find the derivative of the function.
\[g(t)=\int_{1}^{t} {x - 5}\;dx\]

\expandafter\input{\file@loc Integrals/2311-Compute-Integral-0006.HELP.tex}

\[\dfrac{d}{dt}(g(t))=\answer{t - 5}\]
\end{problem}}%}

%%%%%%%%%%%%%%%%%%%%%%



\latexProblemContent{
\begin{problem}

Use the Fundamental Theorem of Calculus to find the derivative of the function.
\[g(t)=\int_{2}^{t} {{\left(x + 6\right)}^{3}}\;dx\]

\expandafter\input{\file@loc Integrals/2311-Compute-Integral-0006.HELP.tex}

\[\dfrac{d}{dt}(g(t))=\answer{{\left(t + 6\right)}^{3}}\]
\end{problem}}%}

%%%%%%%%%%%%%%%%%%%%%%



\latexProblemContent{
\begin{problem}

Use the Fundamental Theorem of Calculus to find the derivative of the function.
\[g(t)=\int_{3}^{t} {\frac{1}{{\left(x + 8\right)}^{3}}}\;dx\]

\expandafter\input{\file@loc Integrals/2311-Compute-Integral-0006.HELP.tex}

\[\dfrac{d}{dt}(g(t))=\answer{\frac{1}{{\left(t + 8\right)}^{3}}}\]
\end{problem}}%}

%%%%%%%%%%%%%%%%%%%%%%



\latexProblemContent{
\begin{problem}

Use the Fundamental Theorem of Calculus to find the derivative of the function.
\[g(t)=\int_{2}^{t} {\sqrt{x + 10}}\;dx\]

\expandafter\input{\file@loc Integrals/2311-Compute-Integral-0006.HELP.tex}

\[\dfrac{d}{dt}(g(t))=\answer{\sqrt{t + 10}}\]
\end{problem}}%}

%%%%%%%%%%%%%%%%%%%%%%



\latexProblemContent{
\begin{problem}

Use the Fundamental Theorem of Calculus to find the derivative of the function.
\[g(t)=\int_{1}^{t} {{\left(x + 3\right)}^{4}}\;dx\]

\expandafter\input{\file@loc Integrals/2311-Compute-Integral-0006.HELP.tex}

\[\dfrac{d}{dt}(g(t))=\answer{{\left(t + 3\right)}^{4}}\]
\end{problem}}%}

%%%%%%%%%%%%%%%%%%%%%%



\latexProblemContent{
\begin{problem}

Use the Fundamental Theorem of Calculus to find the derivative of the function.
\[g(t)=\int_{2}^{t} {\sin\left(x + 8\right)}\;dx\]

\expandafter\input{\file@loc Integrals/2311-Compute-Integral-0006.HELP.tex}

\[\dfrac{d}{dt}(g(t))=\answer{\sin\left(t + 8\right)}\]
\end{problem}}%}

%%%%%%%%%%%%%%%%%%%%%%



\latexProblemContent{
\begin{problem}

Use the Fundamental Theorem of Calculus to find the derivative of the function.
\[g(t)=\int_{5}^{t} {e^{\left(x + 6\right)}}\;dx\]

\expandafter\input{\file@loc Integrals/2311-Compute-Integral-0006.HELP.tex}

\[\dfrac{d}{dt}(g(t))=\answer{e^{\left(t + 6\right)}}\]
\end{problem}}%}

%%%%%%%%%%%%%%%%%%%%%%



\latexProblemContent{
\begin{problem}

Use the Fundamental Theorem of Calculus to find the derivative of the function.
\[g(t)=\int_{1}^{t} {e^{\left(x + 6\right)}}\;dx\]

\expandafter\input{\file@loc Integrals/2311-Compute-Integral-0006.HELP.tex}

\[\dfrac{d}{dt}(g(t))=\answer{e^{\left(t + 6\right)}}\]
\end{problem}}%}

%%%%%%%%%%%%%%%%%%%%%%



\latexProblemContent{
\begin{problem}

Use the Fundamental Theorem of Calculus to find the derivative of the function.
\[g(t)=\int_{3}^{t} {{\left(x - 10\right)}^{2}}\;dx\]

\expandafter\input{\file@loc Integrals/2311-Compute-Integral-0006.HELP.tex}

\[\dfrac{d}{dt}(g(t))=\answer{{\left(t - 10\right)}^{2}}\]
\end{problem}}%}

%%%%%%%%%%%%%%%%%%%%%%



\latexProblemContent{
\begin{problem}

Use the Fundamental Theorem of Calculus to find the derivative of the function.
\[g(t)=\int_{2}^{t} {{\left(x + 1\right)}^{2}}\;dx\]

\expandafter\input{\file@loc Integrals/2311-Compute-Integral-0006.HELP.tex}

\[\dfrac{d}{dt}(g(t))=\answer{{\left(t + 1\right)}^{2}}\]
\end{problem}}%}

%%%%%%%%%%%%%%%%%%%%%%



\latexProblemContent{
\begin{problem}

Use the Fundamental Theorem of Calculus to find the derivative of the function.
\[g(t)=\int_{1}^{t} {\frac{1}{x - 10}}\;dx\]

\expandafter\input{\file@loc Integrals/2311-Compute-Integral-0006.HELP.tex}

\[\dfrac{d}{dt}(g(t))=\answer{\frac{1}{t - 10}}\]
\end{problem}}%}

%%%%%%%%%%%%%%%%%%%%%%



\latexProblemContent{
\begin{problem}

Use the Fundamental Theorem of Calculus to find the derivative of the function.
\[g(t)=\int_{5}^{t} {\sqrt{x - 8}}\;dx\]

\expandafter\input{\file@loc Integrals/2311-Compute-Integral-0006.HELP.tex}

\[\dfrac{d}{dt}(g(t))=\answer{\sqrt{t - 8}}\]
\end{problem}}%}

%%%%%%%%%%%%%%%%%%%%%%



%%%%%%%%%%%%%%%%%%%%%%



\latexProblemContent{
\begin{problem}

Use the Fundamental Theorem of Calculus to find the derivative of the function.
\[g(t)=\int_{5}^{t} {\frac{1}{{\left(x - 9\right)}^{3}}}\;dx\]

\expandafter\input{\file@loc Integrals/2311-Compute-Integral-0006.HELP.tex}

\[\dfrac{d}{dt}(g(t))=\answer{\frac{1}{{\left(t - 9\right)}^{3}}}\]
\end{problem}}%}

%%%%%%%%%%%%%%%%%%%%%%



%%%%%%%%%%%%%%%%%%%%%%



\latexProblemContent{
\begin{problem}

Use the Fundamental Theorem of Calculus to find the derivative of the function.
\[g(t)=\int_{2}^{t} {\sin\left(x - 4\right)}\;dx\]

\expandafter\input{\file@loc Integrals/2311-Compute-Integral-0006.HELP.tex}

\[\dfrac{d}{dt}(g(t))=\answer{\sin\left(t - 4\right)}\]
\end{problem}}%}

%%%%%%%%%%%%%%%%%%%%%%



\latexProblemContent{
\begin{problem}

Use the Fundamental Theorem of Calculus to find the derivative of the function.
\[g(t)=\int_{1}^{t} {\frac{1}{{\left(x + 7\right)}^{2}}}\;dx\]

\expandafter\input{\file@loc Integrals/2311-Compute-Integral-0006.HELP.tex}

\[\dfrac{d}{dt}(g(t))=\answer{\frac{1}{{\left(t + 7\right)}^{2}}}\]
\end{problem}}%}

%%%%%%%%%%%%%%%%%%%%%%



\latexProblemContent{
\begin{problem}

Use the Fundamental Theorem of Calculus to find the derivative of the function.
\[g(t)=\int_{5}^{t} {{\left(x - 3\right)}^{2}}\;dx\]

\expandafter\input{\file@loc Integrals/2311-Compute-Integral-0006.HELP.tex}

\[\dfrac{d}{dt}(g(t))=\answer{{\left(t - 3\right)}^{2}}\]
\end{problem}}%}

%%%%%%%%%%%%%%%%%%%%%%



\latexProblemContent{
\begin{problem}

Use the Fundamental Theorem of Calculus to find the derivative of the function.
\[g(t)=\int_{3}^{t} {\frac{1}{x - 4}}\;dx\]

\expandafter\input{\file@loc Integrals/2311-Compute-Integral-0006.HELP.tex}

\[\dfrac{d}{dt}(g(t))=\answer{\frac{1}{t - 4}}\]
\end{problem}}%}

%%%%%%%%%%%%%%%%%%%%%%



\latexProblemContent{
\begin{problem}

Use the Fundamental Theorem of Calculus to find the derivative of the function.
\[g(t)=\int_{1}^{t} {\sin\left(x - 7\right)}\;dx\]

\expandafter\input{\file@loc Integrals/2311-Compute-Integral-0006.HELP.tex}

\[\dfrac{d}{dt}(g(t))=\answer{\sin\left(t - 7\right)}\]
\end{problem}}%}

%%%%%%%%%%%%%%%%%%%%%%



\latexProblemContent{
\begin{problem}

Use the Fundamental Theorem of Calculus to find the derivative of the function.
\[g(t)=\int_{4}^{t} {\cos\left(x + 4\right)}\;dx\]

\expandafter\input{\file@loc Integrals/2311-Compute-Integral-0006.HELP.tex}

\[\dfrac{d}{dt}(g(t))=\answer{\cos\left(t + 4\right)}\]
\end{problem}}%}

%%%%%%%%%%%%%%%%%%%%%%



\latexProblemContent{
\begin{problem}

Use the Fundamental Theorem of Calculus to find the derivative of the function.
\[g(t)=\int_{4}^{t} {\frac{1}{x - 2}}\;dx\]

\expandafter\input{\file@loc Integrals/2311-Compute-Integral-0006.HELP.tex}

\[\dfrac{d}{dt}(g(t))=\answer{\frac{1}{t - 2}}\]
\end{problem}}%}

%%%%%%%%%%%%%%%%%%%%%%



\latexProblemContent{
\begin{problem}

Use the Fundamental Theorem of Calculus to find the derivative of the function.
\[g(t)=\int_{4}^{t} {\frac{1}{{\left(x - 1\right)}^{3}}}\;dx\]

\expandafter\input{\file@loc Integrals/2311-Compute-Integral-0006.HELP.tex}

\[\dfrac{d}{dt}(g(t))=\answer{\frac{1}{{\left(t - 1\right)}^{3}}}\]
\end{problem}}%}

%%%%%%%%%%%%%%%%%%%%%%



\latexProblemContent{
\begin{problem}

Use the Fundamental Theorem of Calculus to find the derivative of the function.
\[g(t)=\int_{5}^{t} {\log\left(x - 1\right)}\;dx\]

\expandafter\input{\file@loc Integrals/2311-Compute-Integral-0006.HELP.tex}

\[\dfrac{d}{dt}(g(t))=\answer{\log\left(t - 1\right)}\]
\end{problem}}%}

%%%%%%%%%%%%%%%%%%%%%%



\latexProblemContent{
\begin{problem}

Use the Fundamental Theorem of Calculus to find the derivative of the function.
\[g(t)=\int_{1}^{t} {\sin\left(x - 6\right)}\;dx\]

\expandafter\input{\file@loc Integrals/2311-Compute-Integral-0006.HELP.tex}

\[\dfrac{d}{dt}(g(t))=\answer{\sin\left(t - 6\right)}\]
\end{problem}}%}

%%%%%%%%%%%%%%%%%%%%%%



%%%%%%%%%%%%%%%%%%%%%%



\latexProblemContent{
\begin{problem}

Use the Fundamental Theorem of Calculus to find the derivative of the function.
\[g(t)=\int_{3}^{t} {\log\left(x - 2\right)}\;dx\]

\expandafter\input{\file@loc Integrals/2311-Compute-Integral-0006.HELP.tex}

\[\dfrac{d}{dt}(g(t))=\answer{\log\left(t - 2\right)}\]
\end{problem}}%}

%%%%%%%%%%%%%%%%%%%%%%



\latexProblemContent{
\begin{problem}

Use the Fundamental Theorem of Calculus to find the derivative of the function.
\[g(t)=\int_{5}^{t} {\frac{1}{x + 8}}\;dx\]

\expandafter\input{\file@loc Integrals/2311-Compute-Integral-0006.HELP.tex}

\[\dfrac{d}{dt}(g(t))=\answer{\frac{1}{t + 8}}\]
\end{problem}}%}

%%%%%%%%%%%%%%%%%%%%%%



\latexProblemContent{
\begin{problem}

Use the Fundamental Theorem of Calculus to find the derivative of the function.
\[g(t)=\int_{4}^{t} {x + 3}\;dx\]

\expandafter\input{\file@loc Integrals/2311-Compute-Integral-0006.HELP.tex}

\[\dfrac{d}{dt}(g(t))=\answer{t + 3}\]
\end{problem}}%}

%%%%%%%%%%%%%%%%%%%%%%



\latexProblemContent{
\begin{problem}

Use the Fundamental Theorem of Calculus to find the derivative of the function.
\[g(t)=\int_{4}^{t} {{\left(x - 2\right)}^{2}}\;dx\]

\expandafter\input{\file@loc Integrals/2311-Compute-Integral-0006.HELP.tex}

\[\dfrac{d}{dt}(g(t))=\answer{{\left(t - 2\right)}^{2}}\]
\end{problem}}%}

%%%%%%%%%%%%%%%%%%%%%%



\latexProblemContent{
\begin{problem}

Use the Fundamental Theorem of Calculus to find the derivative of the function.
\[g(t)=\int_{5}^{t} {\frac{1}{{\left(x - 4\right)}^{3}}}\;dx\]

\expandafter\input{\file@loc Integrals/2311-Compute-Integral-0006.HELP.tex}

\[\dfrac{d}{dt}(g(t))=\answer{\frac{1}{{\left(t - 4\right)}^{3}}}\]
\end{problem}}%}

%%%%%%%%%%%%%%%%%%%%%%



\latexProblemContent{
\begin{problem}

Use the Fundamental Theorem of Calculus to find the derivative of the function.
\[g(t)=\int_{1}^{t} {{\left(x + 3\right)}^{2}}\;dx\]

\expandafter\input{\file@loc Integrals/2311-Compute-Integral-0006.HELP.tex}

\[\dfrac{d}{dt}(g(t))=\answer{{\left(t + 3\right)}^{2}}\]
\end{problem}}%}

%%%%%%%%%%%%%%%%%%%%%%



%%%%%%%%%%%%%%%%%%%%%%



\latexProblemContent{
\begin{problem}

Use the Fundamental Theorem of Calculus to find the derivative of the function.
\[g(t)=\int_{1}^{t} {x + 5}\;dx\]

\expandafter\input{\file@loc Integrals/2311-Compute-Integral-0006.HELP.tex}

\[\dfrac{d}{dt}(g(t))=\answer{t + 5}\]
\end{problem}}%}

%%%%%%%%%%%%%%%%%%%%%%



\latexProblemContent{
\begin{problem}

Use the Fundamental Theorem of Calculus to find the derivative of the function.
\[g(t)=\int_{5}^{t} {\sin\left(x - 2\right)}\;dx\]

\expandafter\input{\file@loc Integrals/2311-Compute-Integral-0006.HELP.tex}

\[\dfrac{d}{dt}(g(t))=\answer{\sin\left(t - 2\right)}\]
\end{problem}}%}

%%%%%%%%%%%%%%%%%%%%%%



\latexProblemContent{
\begin{problem}

Use the Fundamental Theorem of Calculus to find the derivative of the function.
\[g(t)=\int_{5}^{t} {\frac{1}{x - 4}}\;dx\]

\expandafter\input{\file@loc Integrals/2311-Compute-Integral-0006.HELP.tex}

\[\dfrac{d}{dt}(g(t))=\answer{\frac{1}{t - 4}}\]
\end{problem}}%}

%%%%%%%%%%%%%%%%%%%%%%



\latexProblemContent{
\begin{problem}

Use the Fundamental Theorem of Calculus to find the derivative of the function.
\[g(t)=\int_{5}^{t} {{\left(x + 7\right)}^{2}}\;dx\]

\expandafter\input{\file@loc Integrals/2311-Compute-Integral-0006.HELP.tex}

\[\dfrac{d}{dt}(g(t))=\answer{{\left(t + 7\right)}^{2}}\]
\end{problem}}%}

%%%%%%%%%%%%%%%%%%%%%%



\latexProblemContent{
\begin{problem}

Use the Fundamental Theorem of Calculus to find the derivative of the function.
\[g(t)=\int_{3}^{t} {x - 5}\;dx\]

\expandafter\input{\file@loc Integrals/2311-Compute-Integral-0006.HELP.tex}

\[\dfrac{d}{dt}(g(t))=\answer{t - 5}\]
\end{problem}}%}

%%%%%%%%%%%%%%%%%%%%%%



\latexProblemContent{
\begin{problem}

Use the Fundamental Theorem of Calculus to find the derivative of the function.
\[g(t)=\int_{3}^{t} {{\left(x + 10\right)}^{2}}\;dx\]

\expandafter\input{\file@loc Integrals/2311-Compute-Integral-0006.HELP.tex}

\[\dfrac{d}{dt}(g(t))=\answer{{\left(t + 10\right)}^{2}}\]
\end{problem}}%}

%%%%%%%%%%%%%%%%%%%%%%



%%%%%%%%%%%%%%%%%%%%%%



\latexProblemContent{
\begin{problem}

Use the Fundamental Theorem of Calculus to find the derivative of the function.
\[g(t)=\int_{1}^{t} {\sqrt{x + 5}}\;dx\]

\expandafter\input{\file@loc Integrals/2311-Compute-Integral-0006.HELP.tex}

\[\dfrac{d}{dt}(g(t))=\answer{\sqrt{t + 5}}\]
\end{problem}}%}

%%%%%%%%%%%%%%%%%%%%%%



\latexProblemContent{
\begin{problem}

Use the Fundamental Theorem of Calculus to find the derivative of the function.
\[g(t)=\int_{3}^{t} {\sqrt{x + 9}}\;dx\]

\expandafter\input{\file@loc Integrals/2311-Compute-Integral-0006.HELP.tex}

\[\dfrac{d}{dt}(g(t))=\answer{\sqrt{t + 9}}\]
\end{problem}}%}

%%%%%%%%%%%%%%%%%%%%%%



\latexProblemContent{
\begin{problem}

Use the Fundamental Theorem of Calculus to find the derivative of the function.
\[g(t)=\int_{1}^{t} {\sin\left(x + 10\right)}\;dx\]

\expandafter\input{\file@loc Integrals/2311-Compute-Integral-0006.HELP.tex}

\[\dfrac{d}{dt}(g(t))=\answer{\sin\left(t + 10\right)}\]
\end{problem}}%}

%%%%%%%%%%%%%%%%%%%%%%



\latexProblemContent{
\begin{problem}

Use the Fundamental Theorem of Calculus to find the derivative of the function.
\[g(t)=\int_{3}^{t} {{\left(x + 7\right)}^{3}}\;dx\]

\expandafter\input{\file@loc Integrals/2311-Compute-Integral-0006.HELP.tex}

\[\dfrac{d}{dt}(g(t))=\answer{{\left(t + 7\right)}^{3}}\]
\end{problem}}%}

%%%%%%%%%%%%%%%%%%%%%%



\latexProblemContent{
\begin{problem}

Use the Fundamental Theorem of Calculus to find the derivative of the function.
\[g(t)=\int_{4}^{t} {\frac{1}{{\left(x + 1\right)}^{3}}}\;dx\]

\expandafter\input{\file@loc Integrals/2311-Compute-Integral-0006.HELP.tex}

\[\dfrac{d}{dt}(g(t))=\answer{\frac{1}{{\left(t + 1\right)}^{3}}}\]
\end{problem}}%}

%%%%%%%%%%%%%%%%%%%%%%



\latexProblemContent{
\begin{problem}

Use the Fundamental Theorem of Calculus to find the derivative of the function.
\[g(t)=\int_{2}^{t} {\frac{1}{x - 9}}\;dx\]

\expandafter\input{\file@loc Integrals/2311-Compute-Integral-0006.HELP.tex}

\[\dfrac{d}{dt}(g(t))=\answer{\frac{1}{t - 9}}\]
\end{problem}}%}

%%%%%%%%%%%%%%%%%%%%%%



\latexProblemContent{
\begin{problem}

Use the Fundamental Theorem of Calculus to find the derivative of the function.
\[g(t)=\int_{2}^{t} {e^{\left(x + 2\right)}}\;dx\]

\expandafter\input{\file@loc Integrals/2311-Compute-Integral-0006.HELP.tex}

\[\dfrac{d}{dt}(g(t))=\answer{e^{\left(t + 2\right)}}\]
\end{problem}}%}

%%%%%%%%%%%%%%%%%%%%%%



\latexProblemContent{
\begin{problem}

Use the Fundamental Theorem of Calculus to find the derivative of the function.
\[g(t)=\int_{1}^{t} {{\left(x + 10\right)}^{4}}\;dx\]

\expandafter\input{\file@loc Integrals/2311-Compute-Integral-0006.HELP.tex}

\[\dfrac{d}{dt}(g(t))=\answer{{\left(t + 10\right)}^{4}}\]
\end{problem}}%}

%%%%%%%%%%%%%%%%%%%%%%



\latexProblemContent{
\begin{problem}

Use the Fundamental Theorem of Calculus to find the derivative of the function.
\[g(t)=\int_{5}^{t} {\cos\left(x - 4\right)}\;dx\]

\expandafter\input{\file@loc Integrals/2311-Compute-Integral-0006.HELP.tex}

\[\dfrac{d}{dt}(g(t))=\answer{\cos\left(t - 4\right)}\]
\end{problem}}%}

%%%%%%%%%%%%%%%%%%%%%%



%%%%%%%%%%%%%%%%%%%%%%



\latexProblemContent{
\begin{problem}

Use the Fundamental Theorem of Calculus to find the derivative of the function.
\[g(t)=\int_{5}^{t} {{\left(x + 3\right)}^{2}}\;dx\]

\expandafter\input{\file@loc Integrals/2311-Compute-Integral-0006.HELP.tex}

\[\dfrac{d}{dt}(g(t))=\answer{{\left(t + 3\right)}^{2}}\]
\end{problem}}%}

%%%%%%%%%%%%%%%%%%%%%%



\latexProblemContent{
\begin{problem}

Use the Fundamental Theorem of Calculus to find the derivative of the function.
\[g(t)=\int_{2}^{t} {{\left(x - 4\right)}^{4}}\;dx\]

\expandafter\input{\file@loc Integrals/2311-Compute-Integral-0006.HELP.tex}

\[\dfrac{d}{dt}(g(t))=\answer{{\left(t - 4\right)}^{4}}\]
\end{problem}}%}

%%%%%%%%%%%%%%%%%%%%%%



\latexProblemContent{
\begin{problem}

Use the Fundamental Theorem of Calculus to find the derivative of the function.
\[g(t)=\int_{3}^{t} {\cos\left(x - 4\right)}\;dx\]

\expandafter\input{\file@loc Integrals/2311-Compute-Integral-0006.HELP.tex}

\[\dfrac{d}{dt}(g(t))=\answer{\cos\left(t - 4\right)}\]
\end{problem}}%}

%%%%%%%%%%%%%%%%%%%%%%



\latexProblemContent{
\begin{problem}

Use the Fundamental Theorem of Calculus to find the derivative of the function.
\[g(t)=\int_{3}^{t} {\sqrt{x - 1}}\;dx\]

\expandafter\input{\file@loc Integrals/2311-Compute-Integral-0006.HELP.tex}

\[\dfrac{d}{dt}(g(t))=\answer{\sqrt{t - 1}}\]
\end{problem}}%}

%%%%%%%%%%%%%%%%%%%%%%



\latexProblemContent{
\begin{problem}

Use the Fundamental Theorem of Calculus to find the derivative of the function.
\[g(t)=\int_{4}^{t} {\sin\left(x - 10\right)}\;dx\]

\expandafter\input{\file@loc Integrals/2311-Compute-Integral-0006.HELP.tex}

\[\dfrac{d}{dt}(g(t))=\answer{\sin\left(t - 10\right)}\]
\end{problem}}%}

%%%%%%%%%%%%%%%%%%%%%%



\latexProblemContent{
\begin{problem}

Use the Fundamental Theorem of Calculus to find the derivative of the function.
\[g(t)=\int_{3}^{t} {\frac{1}{{\left(x - 3\right)}^{3}}}\;dx\]

\expandafter\input{\file@loc Integrals/2311-Compute-Integral-0006.HELP.tex}

\[\dfrac{d}{dt}(g(t))=\answer{\frac{1}{{\left(t - 3\right)}^{3}}}\]
\end{problem}}%}

%%%%%%%%%%%%%%%%%%%%%%



\latexProblemContent{
\begin{problem}

Use the Fundamental Theorem of Calculus to find the derivative of the function.
\[g(t)=\int_{2}^{t} {\frac{1}{{\left(x - 3\right)}^{3}}}\;dx\]

\expandafter\input{\file@loc Integrals/2311-Compute-Integral-0006.HELP.tex}

\[\dfrac{d}{dt}(g(t))=\answer{\frac{1}{{\left(t - 3\right)}^{3}}}\]
\end{problem}}%}

%%%%%%%%%%%%%%%%%%%%%%



\latexProblemContent{
\begin{problem}

Use the Fundamental Theorem of Calculus to find the derivative of the function.
\[g(t)=\int_{2}^{t} {\log\left(x - 5\right)}\;dx\]

\expandafter\input{\file@loc Integrals/2311-Compute-Integral-0006.HELP.tex}

\[\dfrac{d}{dt}(g(t))=\answer{\log\left(t - 5\right)}\]
\end{problem}}%}

%%%%%%%%%%%%%%%%%%%%%%



\latexProblemContent{
\begin{problem}

Use the Fundamental Theorem of Calculus to find the derivative of the function.
\[g(t)=\int_{4}^{t} {\frac{1}{x - 7}}\;dx\]

\expandafter\input{\file@loc Integrals/2311-Compute-Integral-0006.HELP.tex}

\[\dfrac{d}{dt}(g(t))=\answer{\frac{1}{t - 7}}\]
\end{problem}}%}

%%%%%%%%%%%%%%%%%%%%%%



\latexProblemContent{
\begin{problem}

Use the Fundamental Theorem of Calculus to find the derivative of the function.
\[g(t)=\int_{5}^{t} {\sin\left(x + 2\right)}\;dx\]

\expandafter\input{\file@loc Integrals/2311-Compute-Integral-0006.HELP.tex}

\[\dfrac{d}{dt}(g(t))=\answer{\sin\left(t + 2\right)}\]
\end{problem}}%}

%%%%%%%%%%%%%%%%%%%%%%



\latexProblemContent{
\begin{problem}

Use the Fundamental Theorem of Calculus to find the derivative of the function.
\[g(t)=\int_{3}^{t} {\frac{1}{{\left(x - 6\right)}^{3}}}\;dx\]

\expandafter\input{\file@loc Integrals/2311-Compute-Integral-0006.HELP.tex}

\[\dfrac{d}{dt}(g(t))=\answer{\frac{1}{{\left(t - 6\right)}^{3}}}\]
\end{problem}}%}

%%%%%%%%%%%%%%%%%%%%%%



\latexProblemContent{
\begin{problem}

Use the Fundamental Theorem of Calculus to find the derivative of the function.
\[g(t)=\int_{4}^{t} {{\left(x - 2\right)}^{4}}\;dx\]

\expandafter\input{\file@loc Integrals/2311-Compute-Integral-0006.HELP.tex}

\[\dfrac{d}{dt}(g(t))=\answer{{\left(t - 2\right)}^{4}}\]
\end{problem}}%}

%%%%%%%%%%%%%%%%%%%%%%



\latexProblemContent{
\begin{problem}

Use the Fundamental Theorem of Calculus to find the derivative of the function.
\[g(t)=\int_{1}^{t} {\cos\left(x + 8\right)}\;dx\]

\expandafter\input{\file@loc Integrals/2311-Compute-Integral-0006.HELP.tex}

\[\dfrac{d}{dt}(g(t))=\answer{\cos\left(t + 8\right)}\]
\end{problem}}%}

%%%%%%%%%%%%%%%%%%%%%%



\latexProblemContent{
\begin{problem}

Use the Fundamental Theorem of Calculus to find the derivative of the function.
\[g(t)=\int_{2}^{t} {\cos\left(x - 2\right)}\;dx\]

\expandafter\input{\file@loc Integrals/2311-Compute-Integral-0006.HELP.tex}

\[\dfrac{d}{dt}(g(t))=\answer{\cos\left(t - 2\right)}\]
\end{problem}}%}

%%%%%%%%%%%%%%%%%%%%%%



\latexProblemContent{
\begin{problem}

Use the Fundamental Theorem of Calculus to find the derivative of the function.
\[g(t)=\int_{2}^{t} {{\left(x - 7\right)}^{2}}\;dx\]

\expandafter\input{\file@loc Integrals/2311-Compute-Integral-0006.HELP.tex}

\[\dfrac{d}{dt}(g(t))=\answer{{\left(t - 7\right)}^{2}}\]
\end{problem}}%}

%%%%%%%%%%%%%%%%%%%%%%



\latexProblemContent{
\begin{problem}

Use the Fundamental Theorem of Calculus to find the derivative of the function.
\[g(t)=\int_{4}^{t} {{\left(x - 3\right)}^{2}}\;dx\]

\expandafter\input{\file@loc Integrals/2311-Compute-Integral-0006.HELP.tex}

\[\dfrac{d}{dt}(g(t))=\answer{{\left(t - 3\right)}^{2}}\]
\end{problem}}%}

%%%%%%%%%%%%%%%%%%%%%%



\latexProblemContent{
\begin{problem}

Use the Fundamental Theorem of Calculus to find the derivative of the function.
\[g(t)=\int_{2}^{t} {\frac{1}{{\left(x + 7\right)}^{2}}}\;dx\]

\expandafter\input{\file@loc Integrals/2311-Compute-Integral-0006.HELP.tex}

\[\dfrac{d}{dt}(g(t))=\answer{\frac{1}{{\left(t + 7\right)}^{2}}}\]
\end{problem}}%}

%%%%%%%%%%%%%%%%%%%%%%



\latexProblemContent{
\begin{problem}

Use the Fundamental Theorem of Calculus to find the derivative of the function.
\[g(t)=\int_{4}^{t} {\cos\left(x - 9\right)}\;dx\]

\expandafter\input{\file@loc Integrals/2311-Compute-Integral-0006.HELP.tex}

\[\dfrac{d}{dt}(g(t))=\answer{\cos\left(t - 9\right)}\]
\end{problem}}%}

%%%%%%%%%%%%%%%%%%%%%%



\latexProblemContent{
\begin{problem}

Use the Fundamental Theorem of Calculus to find the derivative of the function.
\[g(t)=\int_{1}^{t} {\log\left(x - 3\right)}\;dx\]

\expandafter\input{\file@loc Integrals/2311-Compute-Integral-0006.HELP.tex}

\[\dfrac{d}{dt}(g(t))=\answer{\log\left(t - 3\right)}\]
\end{problem}}%}

%%%%%%%%%%%%%%%%%%%%%%



\latexProblemContent{
\begin{problem}

Use the Fundamental Theorem of Calculus to find the derivative of the function.
\[g(t)=\int_{4}^{t} {\frac{1}{x + 2}}\;dx\]

\expandafter\input{\file@loc Integrals/2311-Compute-Integral-0006.HELP.tex}

\[\dfrac{d}{dt}(g(t))=\answer{\frac{1}{t + 2}}\]
\end{problem}}%}

%%%%%%%%%%%%%%%%%%%%%%



\latexProblemContent{
\begin{problem}

Use the Fundamental Theorem of Calculus to find the derivative of the function.
\[g(t)=\int_{4}^{t} {x - 9}\;dx\]

\expandafter\input{\file@loc Integrals/2311-Compute-Integral-0006.HELP.tex}

\[\dfrac{d}{dt}(g(t))=\answer{t - 9}\]
\end{problem}}%}

%%%%%%%%%%%%%%%%%%%%%%



\latexProblemContent{
\begin{problem}

Use the Fundamental Theorem of Calculus to find the derivative of the function.
\[g(t)=\int_{4}^{t} {x - 10}\;dx\]

\expandafter\input{\file@loc Integrals/2311-Compute-Integral-0006.HELP.tex}

\[\dfrac{d}{dt}(g(t))=\answer{t - 10}\]
\end{problem}}%}

%%%%%%%%%%%%%%%%%%%%%%



\latexProblemContent{
\begin{problem}

Use the Fundamental Theorem of Calculus to find the derivative of the function.
\[g(t)=\int_{1}^{t} {{\left(x - 3\right)}^{4}}\;dx\]

\expandafter\input{\file@loc Integrals/2311-Compute-Integral-0006.HELP.tex}

\[\dfrac{d}{dt}(g(t))=\answer{{\left(t - 3\right)}^{4}}\]
\end{problem}}%}

%%%%%%%%%%%%%%%%%%%%%%



\latexProblemContent{
\begin{problem}

Use the Fundamental Theorem of Calculus to find the derivative of the function.
\[g(t)=\int_{5}^{t} {{\left(x + 3\right)}^{4}}\;dx\]

\expandafter\input{\file@loc Integrals/2311-Compute-Integral-0006.HELP.tex}

\[\dfrac{d}{dt}(g(t))=\answer{{\left(t + 3\right)}^{4}}\]
\end{problem}}%}

%%%%%%%%%%%%%%%%%%%%%%



\latexProblemContent{
\begin{problem}

Use the Fundamental Theorem of Calculus to find the derivative of the function.
\[g(t)=\int_{2}^{t} {\frac{1}{{\left(x - 2\right)}^{2}}}\;dx\]

\expandafter\input{\file@loc Integrals/2311-Compute-Integral-0006.HELP.tex}

\[\dfrac{d}{dt}(g(t))=\answer{\frac{1}{{\left(t - 2\right)}^{2}}}\]
\end{problem}}%}

%%%%%%%%%%%%%%%%%%%%%%



\latexProblemContent{
\begin{problem}

Use the Fundamental Theorem of Calculus to find the derivative of the function.
\[g(t)=\int_{1}^{t} {e^{\left(x + 10\right)}}\;dx\]

\expandafter\input{\file@loc Integrals/2311-Compute-Integral-0006.HELP.tex}

\[\dfrac{d}{dt}(g(t))=\answer{e^{\left(t + 10\right)}}\]
\end{problem}}%}

%%%%%%%%%%%%%%%%%%%%%%



\latexProblemContent{
\begin{problem}

Use the Fundamental Theorem of Calculus to find the derivative of the function.
\[g(t)=\int_{2}^{t} {x - 6}\;dx\]

\expandafter\input{\file@loc Integrals/2311-Compute-Integral-0006.HELP.tex}

\[\dfrac{d}{dt}(g(t))=\answer{t - 6}\]
\end{problem}}%}

%%%%%%%%%%%%%%%%%%%%%%



\latexProblemContent{
\begin{problem}

Use the Fundamental Theorem of Calculus to find the derivative of the function.
\[g(t)=\int_{2}^{t} {\frac{1}{{\left(x - 10\right)}^{2}}}\;dx\]

\expandafter\input{\file@loc Integrals/2311-Compute-Integral-0006.HELP.tex}

\[\dfrac{d}{dt}(g(t))=\answer{\frac{1}{{\left(t - 10\right)}^{2}}}\]
\end{problem}}%}

%%%%%%%%%%%%%%%%%%%%%%



%%%%%%%%%%%%%%%%%%%%%%



\latexProblemContent{
\begin{problem}

Use the Fundamental Theorem of Calculus to find the derivative of the function.
\[g(t)=\int_{1}^{t} {e^{\left(x + 4\right)}}\;dx\]

\expandafter\input{\file@loc Integrals/2311-Compute-Integral-0006.HELP.tex}

\[\dfrac{d}{dt}(g(t))=\answer{e^{\left(t + 4\right)}}\]
\end{problem}}%}

%%%%%%%%%%%%%%%%%%%%%%



\latexProblemContent{
\begin{problem}

Use the Fundamental Theorem of Calculus to find the derivative of the function.
\[g(t)=\int_{1}^{t} {e^{\left(x + 3\right)}}\;dx\]

\expandafter\input{\file@loc Integrals/2311-Compute-Integral-0006.HELP.tex}

\[\dfrac{d}{dt}(g(t))=\answer{e^{\left(t + 3\right)}}\]
\end{problem}}%}

%%%%%%%%%%%%%%%%%%%%%%



\latexProblemContent{
\begin{problem}

Use the Fundamental Theorem of Calculus to find the derivative of the function.
\[g(t)=\int_{3}^{t} {\frac{1}{{\left(x - 8\right)}^{2}}}\;dx\]

\expandafter\input{\file@loc Integrals/2311-Compute-Integral-0006.HELP.tex}

\[\dfrac{d}{dt}(g(t))=\answer{\frac{1}{{\left(t - 8\right)}^{2}}}\]
\end{problem}}%}

%%%%%%%%%%%%%%%%%%%%%%



\latexProblemContent{
\begin{problem}

Use the Fundamental Theorem of Calculus to find the derivative of the function.
\[g(t)=\int_{1}^{t} {\frac{1}{x + 6}}\;dx\]

\expandafter\input{\file@loc Integrals/2311-Compute-Integral-0006.HELP.tex}

\[\dfrac{d}{dt}(g(t))=\answer{\frac{1}{t + 6}}\]
\end{problem}}%}

%%%%%%%%%%%%%%%%%%%%%%



\latexProblemContent{
\begin{problem}

Use the Fundamental Theorem of Calculus to find the derivative of the function.
\[g(t)=\int_{5}^{t} {\frac{1}{x - 2}}\;dx\]

\expandafter\input{\file@loc Integrals/2311-Compute-Integral-0006.HELP.tex}

\[\dfrac{d}{dt}(g(t))=\answer{\frac{1}{t - 2}}\]
\end{problem}}%}

%%%%%%%%%%%%%%%%%%%%%%



\latexProblemContent{
\begin{problem}

Use the Fundamental Theorem of Calculus to find the derivative of the function.
\[g(t)=\int_{1}^{t} {{\left(x - 2\right)}^{4}}\;dx\]

\expandafter\input{\file@loc Integrals/2311-Compute-Integral-0006.HELP.tex}

\[\dfrac{d}{dt}(g(t))=\answer{{\left(t - 2\right)}^{4}}\]
\end{problem}}%}

%%%%%%%%%%%%%%%%%%%%%%



\latexProblemContent{
\begin{problem}

Use the Fundamental Theorem of Calculus to find the derivative of the function.
\[g(t)=\int_{2}^{t} {e^{\left(x + 10\right)}}\;dx\]

\expandafter\input{\file@loc Integrals/2311-Compute-Integral-0006.HELP.tex}

\[\dfrac{d}{dt}(g(t))=\answer{e^{\left(t + 10\right)}}\]
\end{problem}}%}

%%%%%%%%%%%%%%%%%%%%%%



\latexProblemContent{
\begin{problem}

Use the Fundamental Theorem of Calculus to find the derivative of the function.
\[g(t)=\int_{1}^{t} {{\left(x - 6\right)}^{3}}\;dx\]

\expandafter\input{\file@loc Integrals/2311-Compute-Integral-0006.HELP.tex}

\[\dfrac{d}{dt}(g(t))=\answer{{\left(t - 6\right)}^{3}}\]
\end{problem}}%}

%%%%%%%%%%%%%%%%%%%%%%



\latexProblemContent{
\begin{problem}

Use the Fundamental Theorem of Calculus to find the derivative of the function.
\[g(t)=\int_{3}^{t} {\frac{1}{x - 5}}\;dx\]

\expandafter\input{\file@loc Integrals/2311-Compute-Integral-0006.HELP.tex}

\[\dfrac{d}{dt}(g(t))=\answer{\frac{1}{t - 5}}\]
\end{problem}}%}

%%%%%%%%%%%%%%%%%%%%%%



\latexProblemContent{
\begin{problem}

Use the Fundamental Theorem of Calculus to find the derivative of the function.
\[g(t)=\int_{3}^{t} {e^{\left(x - 9\right)}}\;dx\]

\expandafter\input{\file@loc Integrals/2311-Compute-Integral-0006.HELP.tex}

\[\dfrac{d}{dt}(g(t))=\answer{e^{\left(t - 9\right)}}\]
\end{problem}}%}

%%%%%%%%%%%%%%%%%%%%%%



\latexProblemContent{
\begin{problem}

Use the Fundamental Theorem of Calculus to find the derivative of the function.
\[g(t)=\int_{4}^{t} {\sin\left(x + 7\right)}\;dx\]

\expandafter\input{\file@loc Integrals/2311-Compute-Integral-0006.HELP.tex}

\[\dfrac{d}{dt}(g(t))=\answer{\sin\left(t + 7\right)}\]
\end{problem}}%}

%%%%%%%%%%%%%%%%%%%%%%



\latexProblemContent{
\begin{problem}

Use the Fundamental Theorem of Calculus to find the derivative of the function.
\[g(t)=\int_{3}^{t} {e^{\left(x - 6\right)}}\;dx\]

\expandafter\input{\file@loc Integrals/2311-Compute-Integral-0006.HELP.tex}

\[\dfrac{d}{dt}(g(t))=\answer{e^{\left(t - 6\right)}}\]
\end{problem}}%}

%%%%%%%%%%%%%%%%%%%%%%



\latexProblemContent{
\begin{problem}

Use the Fundamental Theorem of Calculus to find the derivative of the function.
\[g(t)=\int_{3}^{t} {\cos\left(x - 5\right)}\;dx\]

\expandafter\input{\file@loc Integrals/2311-Compute-Integral-0006.HELP.tex}

\[\dfrac{d}{dt}(g(t))=\answer{\cos\left(t - 5\right)}\]
\end{problem}}%}

%%%%%%%%%%%%%%%%%%%%%%



\latexProblemContent{
\begin{problem}

Use the Fundamental Theorem of Calculus to find the derivative of the function.
\[g(t)=\int_{2}^{t} {x - 8}\;dx\]

\expandafter\input{\file@loc Integrals/2311-Compute-Integral-0006.HELP.tex}

\[\dfrac{d}{dt}(g(t))=\answer{t - 8}\]
\end{problem}}%}

%%%%%%%%%%%%%%%%%%%%%%



\latexProblemContent{
\begin{problem}

Use the Fundamental Theorem of Calculus to find the derivative of the function.
\[g(t)=\int_{4}^{t} {e^{\left(x + 4\right)}}\;dx\]

\expandafter\input{\file@loc Integrals/2311-Compute-Integral-0006.HELP.tex}

\[\dfrac{d}{dt}(g(t))=\answer{e^{\left(t + 4\right)}}\]
\end{problem}}%}

%%%%%%%%%%%%%%%%%%%%%%



\latexProblemContent{
\begin{problem}

Use the Fundamental Theorem of Calculus to find the derivative of the function.
\[g(t)=\int_{4}^{t} {\frac{1}{{\left(x - 7\right)}^{2}}}\;dx\]

\expandafter\input{\file@loc Integrals/2311-Compute-Integral-0006.HELP.tex}

\[\dfrac{d}{dt}(g(t))=\answer{\frac{1}{{\left(t - 7\right)}^{2}}}\]
\end{problem}}%}

%%%%%%%%%%%%%%%%%%%%%%



%%%%%%%%%%%%%%%%%%%%%%



%%%%%%%%%%%%%%%%%%%%%%



\latexProblemContent{
\begin{problem}

Use the Fundamental Theorem of Calculus to find the derivative of the function.
\[g(t)=\int_{2}^{t} {e^{\left(x + 6\right)}}\;dx\]

\expandafter\input{\file@loc Integrals/2311-Compute-Integral-0006.HELP.tex}

\[\dfrac{d}{dt}(g(t))=\answer{e^{\left(t + 6\right)}}\]
\end{problem}}%}

%%%%%%%%%%%%%%%%%%%%%%



\latexProblemContent{
\begin{problem}

Use the Fundamental Theorem of Calculus to find the derivative of the function.
\[g(t)=\int_{1}^{t} {{\left(x + 7\right)}^{3}}\;dx\]

\expandafter\input{\file@loc Integrals/2311-Compute-Integral-0006.HELP.tex}

\[\dfrac{d}{dt}(g(t))=\answer{{\left(t + 7\right)}^{3}}\]
\end{problem}}%}

%%%%%%%%%%%%%%%%%%%%%%



\latexProblemContent{
\begin{problem}

Use the Fundamental Theorem of Calculus to find the derivative of the function.
\[g(t)=\int_{1}^{t} {\sin\left(x + 2\right)}\;dx\]

\expandafter\input{\file@loc Integrals/2311-Compute-Integral-0006.HELP.tex}

\[\dfrac{d}{dt}(g(t))=\answer{\sin\left(t + 2\right)}\]
\end{problem}}%}

%%%%%%%%%%%%%%%%%%%%%%



\latexProblemContent{
\begin{problem}

Use the Fundamental Theorem of Calculus to find the derivative of the function.
\[g(t)=\int_{5}^{t} {\frac{1}{{\left(x - 1\right)}^{2}}}\;dx\]

\expandafter\input{\file@loc Integrals/2311-Compute-Integral-0006.HELP.tex}

\[\dfrac{d}{dt}(g(t))=\answer{\frac{1}{{\left(t - 1\right)}^{2}}}\]
\end{problem}}%}

%%%%%%%%%%%%%%%%%%%%%%



\latexProblemContent{
\begin{problem}

Use the Fundamental Theorem of Calculus to find the derivative of the function.
\[g(t)=\int_{4}^{t} {e^{\left(x - 1\right)}}\;dx\]

\expandafter\input{\file@loc Integrals/2311-Compute-Integral-0006.HELP.tex}

\[\dfrac{d}{dt}(g(t))=\answer{e^{\left(t - 1\right)}}\]
\end{problem}}%}

%%%%%%%%%%%%%%%%%%%%%%



\latexProblemContent{
\begin{problem}

Use the Fundamental Theorem of Calculus to find the derivative of the function.
\[g(t)=\int_{5}^{t} {x - 8}\;dx\]

\expandafter\input{\file@loc Integrals/2311-Compute-Integral-0006.HELP.tex}

\[\dfrac{d}{dt}(g(t))=\answer{t - 8}\]
\end{problem}}%}

%%%%%%%%%%%%%%%%%%%%%%



\latexProblemContent{
\begin{problem}

Use the Fundamental Theorem of Calculus to find the derivative of the function.
\[g(t)=\int_{2}^{t} {\frac{1}{{\left(x - 9\right)}^{3}}}\;dx\]

\expandafter\input{\file@loc Integrals/2311-Compute-Integral-0006.HELP.tex}

\[\dfrac{d}{dt}(g(t))=\answer{\frac{1}{{\left(t - 9\right)}^{3}}}\]
\end{problem}}%}

%%%%%%%%%%%%%%%%%%%%%%



\latexProblemContent{
\begin{problem}

Use the Fundamental Theorem of Calculus to find the derivative of the function.
\[g(t)=\int_{1}^{t} {\cos\left(x + 4\right)}\;dx\]

\expandafter\input{\file@loc Integrals/2311-Compute-Integral-0006.HELP.tex}

\[\dfrac{d}{dt}(g(t))=\answer{\cos\left(t + 4\right)}\]
\end{problem}}%}

%%%%%%%%%%%%%%%%%%%%%%



\latexProblemContent{
\begin{problem}

Use the Fundamental Theorem of Calculus to find the derivative of the function.
\[g(t)=\int_{2}^{t} {{\left(x - 9\right)}^{4}}\;dx\]

\expandafter\input{\file@loc Integrals/2311-Compute-Integral-0006.HELP.tex}

\[\dfrac{d}{dt}(g(t))=\answer{{\left(t - 9\right)}^{4}}\]
\end{problem}}%}

%%%%%%%%%%%%%%%%%%%%%%



%%%%%%%%%%%%%%%%%%%%%%



\latexProblemContent{
\begin{problem}

Use the Fundamental Theorem of Calculus to find the derivative of the function.
\[g(t)=\int_{4}^{t} {{\left(x + 5\right)}^{3}}\;dx\]

\expandafter\input{\file@loc Integrals/2311-Compute-Integral-0006.HELP.tex}

\[\dfrac{d}{dt}(g(t))=\answer{{\left(t + 5\right)}^{3}}\]
\end{problem}}%}

%%%%%%%%%%%%%%%%%%%%%%



%%%%%%%%%%%%%%%%%%%%%%



%%%%%%%%%%%%%%%%%%%%%%



\latexProblemContent{
\begin{problem}

Use the Fundamental Theorem of Calculus to find the derivative of the function.
\[g(t)=\int_{4}^{t} {\frac{1}{{\left(x + 6\right)}^{3}}}\;dx\]

\expandafter\input{\file@loc Integrals/2311-Compute-Integral-0006.HELP.tex}

\[\dfrac{d}{dt}(g(t))=\answer{\frac{1}{{\left(t + 6\right)}^{3}}}\]
\end{problem}}%}

%%%%%%%%%%%%%%%%%%%%%%



\latexProblemContent{
\begin{problem}

Use the Fundamental Theorem of Calculus to find the derivative of the function.
\[g(t)=\int_{5}^{t} {\frac{1}{x - 8}}\;dx\]

\expandafter\input{\file@loc Integrals/2311-Compute-Integral-0006.HELP.tex}

\[\dfrac{d}{dt}(g(t))=\answer{\frac{1}{t - 8}}\]
\end{problem}}%}

%%%%%%%%%%%%%%%%%%%%%%



\latexProblemContent{
\begin{problem}

Use the Fundamental Theorem of Calculus to find the derivative of the function.
\[g(t)=\int_{3}^{t} {\sin\left(x + 3\right)}\;dx\]

\expandafter\input{\file@loc Integrals/2311-Compute-Integral-0006.HELP.tex}

\[\dfrac{d}{dt}(g(t))=\answer{\sin\left(t + 3\right)}\]
\end{problem}}%}

%%%%%%%%%%%%%%%%%%%%%%



\latexProblemContent{
\begin{problem}

Use the Fundamental Theorem of Calculus to find the derivative of the function.
\[g(t)=\int_{2}^{t} {{\left(x - 5\right)}^{2}}\;dx\]

\expandafter\input{\file@loc Integrals/2311-Compute-Integral-0006.HELP.tex}

\[\dfrac{d}{dt}(g(t))=\answer{{\left(t - 5\right)}^{2}}\]
\end{problem}}%}

%%%%%%%%%%%%%%%%%%%%%%



%%%%%%%%%%%%%%%%%%%%%%



\latexProblemContent{
\begin{problem}

Use the Fundamental Theorem of Calculus to find the derivative of the function.
\[g(t)=\int_{4}^{t} {\log\left(x - 1\right)}\;dx\]

\expandafter\input{\file@loc Integrals/2311-Compute-Integral-0006.HELP.tex}

\[\dfrac{d}{dt}(g(t))=\answer{\log\left(t - 1\right)}\]
\end{problem}}%}

%%%%%%%%%%%%%%%%%%%%%%



%%%%%%%%%%%%%%%%%%%%%%



\latexProblemContent{
\begin{problem}

Use the Fundamental Theorem of Calculus to find the derivative of the function.
\[g(t)=\int_{5}^{t} {x + 2}\;dx\]

\expandafter\input{\file@loc Integrals/2311-Compute-Integral-0006.HELP.tex}

\[\dfrac{d}{dt}(g(t))=\answer{t + 2}\]
\end{problem}}%}

%%%%%%%%%%%%%%%%%%%%%%



\latexProblemContent{
\begin{problem}

Use the Fundamental Theorem of Calculus to find the derivative of the function.
\[g(t)=\int_{5}^{t} {e^{\left(x + 1\right)}}\;dx\]

\expandafter\input{\file@loc Integrals/2311-Compute-Integral-0006.HELP.tex}

\[\dfrac{d}{dt}(g(t))=\answer{e^{\left(t + 1\right)}}\]
\end{problem}}%}

%%%%%%%%%%%%%%%%%%%%%%



\latexProblemContent{
\begin{problem}

Use the Fundamental Theorem of Calculus to find the derivative of the function.
\[g(t)=\int_{4}^{t} {\cos\left(x + 3\right)}\;dx\]

\expandafter\input{\file@loc Integrals/2311-Compute-Integral-0006.HELP.tex}

\[\dfrac{d}{dt}(g(t))=\answer{\cos\left(t + 3\right)}\]
\end{problem}}%}

%%%%%%%%%%%%%%%%%%%%%%



\latexProblemContent{
\begin{problem}

Use the Fundamental Theorem of Calculus to find the derivative of the function.
\[g(t)=\int_{3}^{t} {{\left(x + 6\right)}^{2}}\;dx\]

\expandafter\input{\file@loc Integrals/2311-Compute-Integral-0006.HELP.tex}

\[\dfrac{d}{dt}(g(t))=\answer{{\left(t + 6\right)}^{2}}\]
\end{problem}}%}

%%%%%%%%%%%%%%%%%%%%%%



\latexProblemContent{
\begin{problem}

Use the Fundamental Theorem of Calculus to find the derivative of the function.
\[g(t)=\int_{5}^{t} {{\left(x + 4\right)}^{3}}\;dx\]

\expandafter\input{\file@loc Integrals/2311-Compute-Integral-0006.HELP.tex}

\[\dfrac{d}{dt}(g(t))=\answer{{\left(t + 4\right)}^{3}}\]
\end{problem}}%}

%%%%%%%%%%%%%%%%%%%%%%



%%%%%%%%%%%%%%%%%%%%%%



\latexProblemContent{
\begin{problem}

Use the Fundamental Theorem of Calculus to find the derivative of the function.
\[g(t)=\int_{3}^{t} {\frac{1}{x - 2}}\;dx\]

\expandafter\input{\file@loc Integrals/2311-Compute-Integral-0006.HELP.tex}

\[\dfrac{d}{dt}(g(t))=\answer{\frac{1}{t - 2}}\]
\end{problem}}%}

%%%%%%%%%%%%%%%%%%%%%%



\latexProblemContent{
\begin{problem}

Use the Fundamental Theorem of Calculus to find the derivative of the function.
\[g(t)=\int_{1}^{t} {{\left(x + 8\right)}^{4}}\;dx\]

\expandafter\input{\file@loc Integrals/2311-Compute-Integral-0006.HELP.tex}

\[\dfrac{d}{dt}(g(t))=\answer{{\left(t + 8\right)}^{4}}\]
\end{problem}}%}

%%%%%%%%%%%%%%%%%%%%%%



\latexProblemContent{
\begin{problem}

Use the Fundamental Theorem of Calculus to find the derivative of the function.
\[g(t)=\int_{1}^{t} {\cos\left(x + 10\right)}\;dx\]

\expandafter\input{\file@loc Integrals/2311-Compute-Integral-0006.HELP.tex}

\[\dfrac{d}{dt}(g(t))=\answer{\cos\left(t + 10\right)}\]
\end{problem}}%}

%%%%%%%%%%%%%%%%%%%%%%



\latexProblemContent{
\begin{problem}

Use the Fundamental Theorem of Calculus to find the derivative of the function.
\[g(t)=\int_{1}^{t} {\frac{1}{{\left(x - 5\right)}^{3}}}\;dx\]

\expandafter\input{\file@loc Integrals/2311-Compute-Integral-0006.HELP.tex}

\[\dfrac{d}{dt}(g(t))=\answer{\frac{1}{{\left(t - 5\right)}^{3}}}\]
\end{problem}}%}

%%%%%%%%%%%%%%%%%%%%%%



\latexProblemContent{
\begin{problem}

Use the Fundamental Theorem of Calculus to find the derivative of the function.
\[g(t)=\int_{2}^{t} {{\left(x - 4\right)}^{3}}\;dx\]

\expandafter\input{\file@loc Integrals/2311-Compute-Integral-0006.HELP.tex}

\[\dfrac{d}{dt}(g(t))=\answer{{\left(t - 4\right)}^{3}}\]
\end{problem}}%}

%%%%%%%%%%%%%%%%%%%%%%



\latexProblemContent{
\begin{problem}

Use the Fundamental Theorem of Calculus to find the derivative of the function.
\[g(t)=\int_{5}^{t} {\frac{1}{x - 9}}\;dx\]

\expandafter\input{\file@loc Integrals/2311-Compute-Integral-0006.HELP.tex}

\[\dfrac{d}{dt}(g(t))=\answer{\frac{1}{t - 9}}\]
\end{problem}}%}

%%%%%%%%%%%%%%%%%%%%%%



%%%%%%%%%%%%%%%%%%%%%%



\latexProblemContent{
\begin{problem}

Use the Fundamental Theorem of Calculus to find the derivative of the function.
\[g(t)=\int_{1}^{t} {\frac{1}{{\left(x + 5\right)}^{2}}}\;dx\]

\expandafter\input{\file@loc Integrals/2311-Compute-Integral-0006.HELP.tex}

\[\dfrac{d}{dt}(g(t))=\answer{\frac{1}{{\left(t + 5\right)}^{2}}}\]
\end{problem}}%}

%%%%%%%%%%%%%%%%%%%%%%



\latexProblemContent{
\begin{problem}

Use the Fundamental Theorem of Calculus to find the derivative of the function.
\[g(t)=\int_{1}^{t} {{\left(x - 3\right)}^{2}}\;dx\]

\expandafter\input{\file@loc Integrals/2311-Compute-Integral-0006.HELP.tex}

\[\dfrac{d}{dt}(g(t))=\answer{{\left(t - 3\right)}^{2}}\]
\end{problem}}%}

%%%%%%%%%%%%%%%%%%%%%%



\latexProblemContent{
\begin{problem}

Use the Fundamental Theorem of Calculus to find the derivative of the function.
\[g(t)=\int_{5}^{t} {x + 6}\;dx\]

\expandafter\input{\file@loc Integrals/2311-Compute-Integral-0006.HELP.tex}

\[\dfrac{d}{dt}(g(t))=\answer{t + 6}\]
\end{problem}}%}

%%%%%%%%%%%%%%%%%%%%%%



\latexProblemContent{
\begin{problem}

Use the Fundamental Theorem of Calculus to find the derivative of the function.
\[g(t)=\int_{3}^{t} {\frac{1}{{\left(x - 10\right)}^{3}}}\;dx\]

\expandafter\input{\file@loc Integrals/2311-Compute-Integral-0006.HELP.tex}

\[\dfrac{d}{dt}(g(t))=\answer{\frac{1}{{\left(t - 10\right)}^{3}}}\]
\end{problem}}%}

%%%%%%%%%%%%%%%%%%%%%%



\latexProblemContent{
\begin{problem}

Use the Fundamental Theorem of Calculus to find the derivative of the function.
\[g(t)=\int_{2}^{t} {{\left(x + 9\right)}^{2}}\;dx\]

\expandafter\input{\file@loc Integrals/2311-Compute-Integral-0006.HELP.tex}

\[\dfrac{d}{dt}(g(t))=\answer{{\left(t + 9\right)}^{2}}\]
\end{problem}}%}

%%%%%%%%%%%%%%%%%%%%%%



\latexProblemContent{
\begin{problem}

Use the Fundamental Theorem of Calculus to find the derivative of the function.
\[g(t)=\int_{5}^{t} {x - 5}\;dx\]

\expandafter\input{\file@loc Integrals/2311-Compute-Integral-0006.HELP.tex}

\[\dfrac{d}{dt}(g(t))=\answer{t - 5}\]
\end{problem}}%}

%%%%%%%%%%%%%%%%%%%%%%



\latexProblemContent{
\begin{problem}

Use the Fundamental Theorem of Calculus to find the derivative of the function.
\[g(t)=\int_{1}^{t} {{\left(x - 4\right)}^{2}}\;dx\]

\expandafter\input{\file@loc Integrals/2311-Compute-Integral-0006.HELP.tex}

\[\dfrac{d}{dt}(g(t))=\answer{{\left(t - 4\right)}^{2}}\]
\end{problem}}%}

%%%%%%%%%%%%%%%%%%%%%%



\latexProblemContent{
\begin{problem}

Use the Fundamental Theorem of Calculus to find the derivative of the function.
\[g(t)=\int_{4}^{t} {\sin\left(x - 9\right)}\;dx\]

\expandafter\input{\file@loc Integrals/2311-Compute-Integral-0006.HELP.tex}

\[\dfrac{d}{dt}(g(t))=\answer{\sin\left(t - 9\right)}\]
\end{problem}}%}

%%%%%%%%%%%%%%%%%%%%%%



\latexProblemContent{
\begin{problem}

Use the Fundamental Theorem of Calculus to find the derivative of the function.
\[g(t)=\int_{1}^{t} {\sin\left(x - 10\right)}\;dx\]

\expandafter\input{\file@loc Integrals/2311-Compute-Integral-0006.HELP.tex}

\[\dfrac{d}{dt}(g(t))=\answer{\sin\left(t - 10\right)}\]
\end{problem}}%}

%%%%%%%%%%%%%%%%%%%%%%



\latexProblemContent{
\begin{problem}

Use the Fundamental Theorem of Calculus to find the derivative of the function.
\[g(t)=\int_{2}^{t} {\log\left(x - 3\right)}\;dx\]

\expandafter\input{\file@loc Integrals/2311-Compute-Integral-0006.HELP.tex}

\[\dfrac{d}{dt}(g(t))=\answer{\log\left(t - 3\right)}\]
\end{problem}}%}

%%%%%%%%%%%%%%%%%%%%%%



\latexProblemContent{
\begin{problem}

Use the Fundamental Theorem of Calculus to find the derivative of the function.
\[g(t)=\int_{1}^{t} {\sqrt{x + 9}}\;dx\]

\expandafter\input{\file@loc Integrals/2311-Compute-Integral-0006.HELP.tex}

\[\dfrac{d}{dt}(g(t))=\answer{\sqrt{t + 9}}\]
\end{problem}}%}

%%%%%%%%%%%%%%%%%%%%%%



\latexProblemContent{
\begin{problem}

Use the Fundamental Theorem of Calculus to find the derivative of the function.
\[g(t)=\int_{1}^{t} {\sin\left(x - 1\right)}\;dx\]

\expandafter\input{\file@loc Integrals/2311-Compute-Integral-0006.HELP.tex}

\[\dfrac{d}{dt}(g(t))=\answer{\sin\left(t - 1\right)}\]
\end{problem}}%}

%%%%%%%%%%%%%%%%%%%%%%



%%%%%%%%%%%%%%%%%%%%%%



%%%%%%%%%%%%%%%%%%%%%%



\latexProblemContent{
\begin{problem}

Use the Fundamental Theorem of Calculus to find the derivative of the function.
\[g(t)=\int_{1}^{t} {{\left(x + 4\right)}^{2}}\;dx\]

\expandafter\input{\file@loc Integrals/2311-Compute-Integral-0006.HELP.tex}

\[\dfrac{d}{dt}(g(t))=\answer{{\left(t + 4\right)}^{2}}\]
\end{problem}}%}

%%%%%%%%%%%%%%%%%%%%%%



\latexProblemContent{
\begin{problem}

Use the Fundamental Theorem of Calculus to find the derivative of the function.
\[g(t)=\int_{3}^{t} {\log\left(x - 3\right)}\;dx\]

\expandafter\input{\file@loc Integrals/2311-Compute-Integral-0006.HELP.tex}

\[\dfrac{d}{dt}(g(t))=\answer{\log\left(t - 3\right)}\]
\end{problem}}%}

%%%%%%%%%%%%%%%%%%%%%%



%%%%%%%%%%%%%%%%%%%%%%



\latexProblemContent{
\begin{problem}

Use the Fundamental Theorem of Calculus to find the derivative of the function.
\[g(t)=\int_{4}^{t} {\cos\left(x + 6\right)}\;dx\]

\expandafter\input{\file@loc Integrals/2311-Compute-Integral-0006.HELP.tex}

\[\dfrac{d}{dt}(g(t))=\answer{\cos\left(t + 6\right)}\]
\end{problem}}%}

%%%%%%%%%%%%%%%%%%%%%%



\latexProblemContent{
\begin{problem}

Use the Fundamental Theorem of Calculus to find the derivative of the function.
\[g(t)=\int_{5}^{t} {\frac{1}{x + 10}}\;dx\]

\expandafter\input{\file@loc Integrals/2311-Compute-Integral-0006.HELP.tex}

\[\dfrac{d}{dt}(g(t))=\answer{\frac{1}{t + 10}}\]
\end{problem}}%}

%%%%%%%%%%%%%%%%%%%%%%



%%%%%%%%%%%%%%%%%%%%%%



\latexProblemContent{
\begin{problem}

Use the Fundamental Theorem of Calculus to find the derivative of the function.
\[g(t)=\int_{4}^{t} {{\left(x + 3\right)}^{2}}\;dx\]

\expandafter\input{\file@loc Integrals/2311-Compute-Integral-0006.HELP.tex}

\[\dfrac{d}{dt}(g(t))=\answer{{\left(t + 3\right)}^{2}}\]
\end{problem}}%}

%%%%%%%%%%%%%%%%%%%%%%



\latexProblemContent{
\begin{problem}

Use the Fundamental Theorem of Calculus to find the derivative of the function.
\[g(t)=\int_{2}^{t} {\sin\left(x - 6\right)}\;dx\]

\expandafter\input{\file@loc Integrals/2311-Compute-Integral-0006.HELP.tex}

\[\dfrac{d}{dt}(g(t))=\answer{\sin\left(t - 6\right)}\]
\end{problem}}%}

%%%%%%%%%%%%%%%%%%%%%%



\latexProblemContent{
\begin{problem}

Use the Fundamental Theorem of Calculus to find the derivative of the function.
\[g(t)=\int_{1}^{t} {\sin\left(x + 1\right)}\;dx\]

\expandafter\input{\file@loc Integrals/2311-Compute-Integral-0006.HELP.tex}

\[\dfrac{d}{dt}(g(t))=\answer{\sin\left(t + 1\right)}\]
\end{problem}}%}

%%%%%%%%%%%%%%%%%%%%%%



\latexProblemContent{
\begin{problem}

Use the Fundamental Theorem of Calculus to find the derivative of the function.
\[g(t)=\int_{4}^{t} {\sqrt{x + 1}}\;dx\]

\expandafter\input{\file@loc Integrals/2311-Compute-Integral-0006.HELP.tex}

\[\dfrac{d}{dt}(g(t))=\answer{\sqrt{t + 1}}\]
\end{problem}}%}

%%%%%%%%%%%%%%%%%%%%%%



\latexProblemContent{
\begin{problem}

Use the Fundamental Theorem of Calculus to find the derivative of the function.
\[g(t)=\int_{1}^{t} {\frac{1}{{\left(x - 7\right)}^{3}}}\;dx\]

\expandafter\input{\file@loc Integrals/2311-Compute-Integral-0006.HELP.tex}

\[\dfrac{d}{dt}(g(t))=\answer{\frac{1}{{\left(t - 7\right)}^{3}}}\]
\end{problem}}%}

%%%%%%%%%%%%%%%%%%%%%%



\latexProblemContent{
\begin{problem}

Use the Fundamental Theorem of Calculus to find the derivative of the function.
\[g(t)=\int_{2}^{t} {\frac{1}{{\left(x + 6\right)}^{3}}}\;dx\]

\expandafter\input{\file@loc Integrals/2311-Compute-Integral-0006.HELP.tex}

\[\dfrac{d}{dt}(g(t))=\answer{\frac{1}{{\left(t + 6\right)}^{3}}}\]
\end{problem}}%}

%%%%%%%%%%%%%%%%%%%%%%



\latexProblemContent{
\begin{problem}

Use the Fundamental Theorem of Calculus to find the derivative of the function.
\[g(t)=\int_{3}^{t} {{\left(x + 8\right)}^{2}}\;dx\]

\expandafter\input{\file@loc Integrals/2311-Compute-Integral-0006.HELP.tex}

\[\dfrac{d}{dt}(g(t))=\answer{{\left(t + 8\right)}^{2}}\]
\end{problem}}%}

%%%%%%%%%%%%%%%%%%%%%%



\latexProblemContent{
\begin{problem}

Use the Fundamental Theorem of Calculus to find the derivative of the function.
\[g(t)=\int_{2}^{t} {\log\left(x - 1\right)}\;dx\]

\expandafter\input{\file@loc Integrals/2311-Compute-Integral-0006.HELP.tex}

\[\dfrac{d}{dt}(g(t))=\answer{\log\left(t - 1\right)}\]
\end{problem}}%}

%%%%%%%%%%%%%%%%%%%%%%



\latexProblemContent{
\begin{problem}

Use the Fundamental Theorem of Calculus to find the derivative of the function.
\[g(t)=\int_{4}^{t} {\cos\left(x + 9\right)}\;dx\]

\expandafter\input{\file@loc Integrals/2311-Compute-Integral-0006.HELP.tex}

\[\dfrac{d}{dt}(g(t))=\answer{\cos\left(t + 9\right)}\]
\end{problem}}%}

%%%%%%%%%%%%%%%%%%%%%%



\latexProblemContent{
\begin{problem}

Use the Fundamental Theorem of Calculus to find the derivative of the function.
\[g(t)=\int_{1}^{t} {{\left(x - 9\right)}^{2}}\;dx\]

\expandafter\input{\file@loc Integrals/2311-Compute-Integral-0006.HELP.tex}

\[\dfrac{d}{dt}(g(t))=\answer{{\left(t - 9\right)}^{2}}\]
\end{problem}}%}

%%%%%%%%%%%%%%%%%%%%%%



\latexProblemContent{
\begin{problem}

Use the Fundamental Theorem of Calculus to find the derivative of the function.
\[g(t)=\int_{3}^{t} {x + 2}\;dx\]

\expandafter\input{\file@loc Integrals/2311-Compute-Integral-0006.HELP.tex}

\[\dfrac{d}{dt}(g(t))=\answer{t + 2}\]
\end{problem}}%}

%%%%%%%%%%%%%%%%%%%%%%



%%%%%%%%%%%%%%%%%%%%%%



\latexProblemContent{
\begin{problem}

Use the Fundamental Theorem of Calculus to find the derivative of the function.
\[g(t)=\int_{1}^{t} {\sin\left(x + 9\right)}\;dx\]

\expandafter\input{\file@loc Integrals/2311-Compute-Integral-0006.HELP.tex}

\[\dfrac{d}{dt}(g(t))=\answer{\sin\left(t + 9\right)}\]
\end{problem}}%}

%%%%%%%%%%%%%%%%%%%%%%



\latexProblemContent{
\begin{problem}

Use the Fundamental Theorem of Calculus to find the derivative of the function.
\[g(t)=\int_{1}^{t} {\frac{1}{x - 4}}\;dx\]

\expandafter\input{\file@loc Integrals/2311-Compute-Integral-0006.HELP.tex}

\[\dfrac{d}{dt}(g(t))=\answer{\frac{1}{t - 4}}\]
\end{problem}}%}

%%%%%%%%%%%%%%%%%%%%%%



\latexProblemContent{
\begin{problem}

Use the Fundamental Theorem of Calculus to find the derivative of the function.
\[g(t)=\int_{5}^{t} {e^{\left(x + 2\right)}}\;dx\]

\expandafter\input{\file@loc Integrals/2311-Compute-Integral-0006.HELP.tex}

\[\dfrac{d}{dt}(g(t))=\answer{e^{\left(t + 2\right)}}\]
\end{problem}}%}

%%%%%%%%%%%%%%%%%%%%%%



\latexProblemContent{
\begin{problem}

Use the Fundamental Theorem of Calculus to find the derivative of the function.
\[g(t)=\int_{2}^{t} {{\left(x + 6\right)}^{2}}\;dx\]

\expandafter\input{\file@loc Integrals/2311-Compute-Integral-0006.HELP.tex}

\[\dfrac{d}{dt}(g(t))=\answer{{\left(t + 6\right)}^{2}}\]
\end{problem}}%}

%%%%%%%%%%%%%%%%%%%%%%



\latexProblemContent{
\begin{problem}

Use the Fundamental Theorem of Calculus to find the derivative of the function.
\[g(t)=\int_{4}^{t} {{\left(x - 5\right)}^{2}}\;dx\]

\expandafter\input{\file@loc Integrals/2311-Compute-Integral-0006.HELP.tex}

\[\dfrac{d}{dt}(g(t))=\answer{{\left(t - 5\right)}^{2}}\]
\end{problem}}%}

%%%%%%%%%%%%%%%%%%%%%%



\latexProblemContent{
\begin{problem}

Use the Fundamental Theorem of Calculus to find the derivative of the function.
\[g(t)=\int_{1}^{t} {\cos\left(x - 7\right)}\;dx\]

\expandafter\input{\file@loc Integrals/2311-Compute-Integral-0006.HELP.tex}

\[\dfrac{d}{dt}(g(t))=\answer{\cos\left(t - 7\right)}\]
\end{problem}}%}

%%%%%%%%%%%%%%%%%%%%%%



\latexProblemContent{
\begin{problem}

Use the Fundamental Theorem of Calculus to find the derivative of the function.
\[g(t)=\int_{3}^{t} {\sin\left(x + 9\right)}\;dx\]

\expandafter\input{\file@loc Integrals/2311-Compute-Integral-0006.HELP.tex}

\[\dfrac{d}{dt}(g(t))=\answer{\sin\left(t + 9\right)}\]
\end{problem}}%}

%%%%%%%%%%%%%%%%%%%%%%



\latexProblemContent{
\begin{problem}

Use the Fundamental Theorem of Calculus to find the derivative of the function.
\[g(t)=\int_{4}^{t} {\cos\left(x - 4\right)}\;dx\]

\expandafter\input{\file@loc Integrals/2311-Compute-Integral-0006.HELP.tex}

\[\dfrac{d}{dt}(g(t))=\answer{\cos\left(t - 4\right)}\]
\end{problem}}%}

%%%%%%%%%%%%%%%%%%%%%%



\latexProblemContent{
\begin{problem}

Use the Fundamental Theorem of Calculus to find the derivative of the function.
\[g(t)=\int_{3}^{t} {x + 8}\;dx\]

\expandafter\input{\file@loc Integrals/2311-Compute-Integral-0006.HELP.tex}

\[\dfrac{d}{dt}(g(t))=\answer{t + 8}\]
\end{problem}}%}

%%%%%%%%%%%%%%%%%%%%%%



\latexProblemContent{
\begin{problem}

Use the Fundamental Theorem of Calculus to find the derivative of the function.
\[g(t)=\int_{2}^{t} {\sqrt{x - 5}}\;dx\]

\expandafter\input{\file@loc Integrals/2311-Compute-Integral-0006.HELP.tex}

\[\dfrac{d}{dt}(g(t))=\answer{\sqrt{t - 5}}\]
\end{problem}}%}

%%%%%%%%%%%%%%%%%%%%%%



\latexProblemContent{
\begin{problem}

Use the Fundamental Theorem of Calculus to find the derivative of the function.
\[g(t)=\int_{2}^{t} {x - 9}\;dx\]

\expandafter\input{\file@loc Integrals/2311-Compute-Integral-0006.HELP.tex}

\[\dfrac{d}{dt}(g(t))=\answer{t - 9}\]
\end{problem}}%}

%%%%%%%%%%%%%%%%%%%%%%



\latexProblemContent{
\begin{problem}

Use the Fundamental Theorem of Calculus to find the derivative of the function.
\[g(t)=\int_{1}^{t} {\sqrt{x + 2}}\;dx\]

\expandafter\input{\file@loc Integrals/2311-Compute-Integral-0006.HELP.tex}

\[\dfrac{d}{dt}(g(t))=\answer{\sqrt{t + 2}}\]
\end{problem}}%}

%%%%%%%%%%%%%%%%%%%%%%



\latexProblemContent{
\begin{problem}

Use the Fundamental Theorem of Calculus to find the derivative of the function.
\[g(t)=\int_{4}^{t} {{\left(x - 8\right)}^{2}}\;dx\]

\expandafter\input{\file@loc Integrals/2311-Compute-Integral-0006.HELP.tex}

\[\dfrac{d}{dt}(g(t))=\answer{{\left(t - 8\right)}^{2}}\]
\end{problem}}%}

%%%%%%%%%%%%%%%%%%%%%%



\latexProblemContent{
\begin{problem}

Use the Fundamental Theorem of Calculus to find the derivative of the function.
\[g(t)=\int_{2}^{t} {\cos\left(x + 9\right)}\;dx\]

\expandafter\input{\file@loc Integrals/2311-Compute-Integral-0006.HELP.tex}

\[\dfrac{d}{dt}(g(t))=\answer{\cos\left(t + 9\right)}\]
\end{problem}}%}

%%%%%%%%%%%%%%%%%%%%%%



\latexProblemContent{
\begin{problem}

Use the Fundamental Theorem of Calculus to find the derivative of the function.
\[g(t)=\int_{3}^{t} {\frac{1}{{\left(x - 8\right)}^{3}}}\;dx\]

\expandafter\input{\file@loc Integrals/2311-Compute-Integral-0006.HELP.tex}

\[\dfrac{d}{dt}(g(t))=\answer{\frac{1}{{\left(t - 8\right)}^{3}}}\]
\end{problem}}%}

%%%%%%%%%%%%%%%%%%%%%%



%%%%%%%%%%%%%%%%%%%%%%



\latexProblemContent{
\begin{problem}

Use the Fundamental Theorem of Calculus to find the derivative of the function.
\[g(t)=\int_{4}^{t} {e^{\left(x + 2\right)}}\;dx\]

\expandafter\input{\file@loc Integrals/2311-Compute-Integral-0006.HELP.tex}

\[\dfrac{d}{dt}(g(t))=\answer{e^{\left(t + 2\right)}}\]
\end{problem}}%}

%%%%%%%%%%%%%%%%%%%%%%



\latexProblemContent{
\begin{problem}

Use the Fundamental Theorem of Calculus to find the derivative of the function.
\[g(t)=\int_{2}^{t} {\sin\left(x + 3\right)}\;dx\]

\expandafter\input{\file@loc Integrals/2311-Compute-Integral-0006.HELP.tex}

\[\dfrac{d}{dt}(g(t))=\answer{\sin\left(t + 3\right)}\]
\end{problem}}%}

%%%%%%%%%%%%%%%%%%%%%%



%%%%%%%%%%%%%%%%%%%%%%



\latexProblemContent{
\begin{problem}

Use the Fundamental Theorem of Calculus to find the derivative of the function.
\[g(t)=\int_{1}^{t} {x + 3}\;dx\]

\expandafter\input{\file@loc Integrals/2311-Compute-Integral-0006.HELP.tex}

\[\dfrac{d}{dt}(g(t))=\answer{t + 3}\]
\end{problem}}%}

%%%%%%%%%%%%%%%%%%%%%%



\latexProblemContent{
\begin{problem}

Use the Fundamental Theorem of Calculus to find the derivative of the function.
\[g(t)=\int_{1}^{t} {\frac{1}{{\left(x + 9\right)}^{2}}}\;dx\]

\expandafter\input{\file@loc Integrals/2311-Compute-Integral-0006.HELP.tex}

\[\dfrac{d}{dt}(g(t))=\answer{\frac{1}{{\left(t + 9\right)}^{2}}}\]
\end{problem}}%}

%%%%%%%%%%%%%%%%%%%%%%



\latexProblemContent{
\begin{problem}

Use the Fundamental Theorem of Calculus to find the derivative of the function.
\[g(t)=\int_{3}^{t} {x + 9}\;dx\]

\expandafter\input{\file@loc Integrals/2311-Compute-Integral-0006.HELP.tex}

\[\dfrac{d}{dt}(g(t))=\answer{t + 9}\]
\end{problem}}%}

%%%%%%%%%%%%%%%%%%%%%%



\latexProblemContent{
\begin{problem}

Use the Fundamental Theorem of Calculus to find the derivative of the function.
\[g(t)=\int_{1}^{t} {\frac{1}{{\left(x + 10\right)}^{2}}}\;dx\]

\expandafter\input{\file@loc Integrals/2311-Compute-Integral-0006.HELP.tex}

\[\dfrac{d}{dt}(g(t))=\answer{\frac{1}{{\left(t + 10\right)}^{2}}}\]
\end{problem}}%}

%%%%%%%%%%%%%%%%%%%%%%



\latexProblemContent{
\begin{problem}

Use the Fundamental Theorem of Calculus to find the derivative of the function.
\[g(t)=\int_{2}^{t} {{\left(x - 2\right)}^{4}}\;dx\]

\expandafter\input{\file@loc Integrals/2311-Compute-Integral-0006.HELP.tex}

\[\dfrac{d}{dt}(g(t))=\answer{{\left(t - 2\right)}^{4}}\]
\end{problem}}%}

%%%%%%%%%%%%%%%%%%%%%%



%%%%%%%%%%%%%%%%%%%%%%



\latexProblemContent{
\begin{problem}

Use the Fundamental Theorem of Calculus to find the derivative of the function.
\[g(t)=\int_{2}^{t} {\frac{1}{{\left(x + 7\right)}^{3}}}\;dx\]

\expandafter\input{\file@loc Integrals/2311-Compute-Integral-0006.HELP.tex}

\[\dfrac{d}{dt}(g(t))=\answer{\frac{1}{{\left(t + 7\right)}^{3}}}\]
\end{problem}}%}

%%%%%%%%%%%%%%%%%%%%%%



%%%%%%%%%%%%%%%%%%%%%%



%%%%%%%%%%%%%%%%%%%%%%



\latexProblemContent{
\begin{problem}

Use the Fundamental Theorem of Calculus to find the derivative of the function.
\[g(t)=\int_{3}^{t} {e^{\left(x + 4\right)}}\;dx\]

\expandafter\input{\file@loc Integrals/2311-Compute-Integral-0006.HELP.tex}

\[\dfrac{d}{dt}(g(t))=\answer{e^{\left(t + 4\right)}}\]
\end{problem}}%}

%%%%%%%%%%%%%%%%%%%%%%



\latexProblemContent{
\begin{problem}

Use the Fundamental Theorem of Calculus to find the derivative of the function.
\[g(t)=\int_{5}^{t} {{\left(x + 6\right)}^{3}}\;dx\]

\expandafter\input{\file@loc Integrals/2311-Compute-Integral-0006.HELP.tex}

\[\dfrac{d}{dt}(g(t))=\answer{{\left(t + 6\right)}^{3}}\]
\end{problem}}%}

%%%%%%%%%%%%%%%%%%%%%%



\latexProblemContent{
\begin{problem}

Use the Fundamental Theorem of Calculus to find the derivative of the function.
\[g(t)=\int_{2}^{t} {{\left(x + 9\right)}^{4}}\;dx\]

\expandafter\input{\file@loc Integrals/2311-Compute-Integral-0006.HELP.tex}

\[\dfrac{d}{dt}(g(t))=\answer{{\left(t + 9\right)}^{4}}\]
\end{problem}}%}

%%%%%%%%%%%%%%%%%%%%%%



%%%%%%%%%%%%%%%%%%%%%%



%%%%%%%%%%%%%%%%%%%%%%



\latexProblemContent{
\begin{problem}

Use the Fundamental Theorem of Calculus to find the derivative of the function.
\[g(t)=\int_{4}^{t} {\sin\left(x + 10\right)}\;dx\]

\expandafter\input{\file@loc Integrals/2311-Compute-Integral-0006.HELP.tex}

\[\dfrac{d}{dt}(g(t))=\answer{\sin\left(t + 10\right)}\]
\end{problem}}%}

%%%%%%%%%%%%%%%%%%%%%%



\latexProblemContent{
\begin{problem}

Use the Fundamental Theorem of Calculus to find the derivative of the function.
\[g(t)=\int_{4}^{t} {\sqrt{x - 7}}\;dx\]

\expandafter\input{\file@loc Integrals/2311-Compute-Integral-0006.HELP.tex}

\[\dfrac{d}{dt}(g(t))=\answer{\sqrt{t - 7}}\]
\end{problem}}%}

%%%%%%%%%%%%%%%%%%%%%%



\latexProblemContent{
\begin{problem}

Use the Fundamental Theorem of Calculus to find the derivative of the function.
\[g(t)=\int_{3}^{t} {\frac{1}{x + 2}}\;dx\]

\expandafter\input{\file@loc Integrals/2311-Compute-Integral-0006.HELP.tex}

\[\dfrac{d}{dt}(g(t))=\answer{\frac{1}{t + 2}}\]
\end{problem}}%}

%%%%%%%%%%%%%%%%%%%%%%



\latexProblemContent{
\begin{problem}

Use the Fundamental Theorem of Calculus to find the derivative of the function.
\[g(t)=\int_{1}^{t} {\frac{1}{{\left(x - 8\right)}^{3}}}\;dx\]

\expandafter\input{\file@loc Integrals/2311-Compute-Integral-0006.HELP.tex}

\[\dfrac{d}{dt}(g(t))=\answer{\frac{1}{{\left(t - 8\right)}^{3}}}\]
\end{problem}}%}

%%%%%%%%%%%%%%%%%%%%%%



\latexProblemContent{
\begin{problem}

Use the Fundamental Theorem of Calculus to find the derivative of the function.
\[g(t)=\int_{4}^{t} {\frac{1}{{\left(x - 5\right)}^{3}}}\;dx\]

\expandafter\input{\file@loc Integrals/2311-Compute-Integral-0006.HELP.tex}

\[\dfrac{d}{dt}(g(t))=\answer{\frac{1}{{\left(t - 5\right)}^{3}}}\]
\end{problem}}%}

%%%%%%%%%%%%%%%%%%%%%%



\latexProblemContent{
\begin{problem}

Use the Fundamental Theorem of Calculus to find the derivative of the function.
\[g(t)=\int_{5}^{t} {\frac{1}{{\left(x + 4\right)}^{3}}}\;dx\]

\expandafter\input{\file@loc Integrals/2311-Compute-Integral-0006.HELP.tex}

\[\dfrac{d}{dt}(g(t))=\answer{\frac{1}{{\left(t + 4\right)}^{3}}}\]
\end{problem}}%}

%%%%%%%%%%%%%%%%%%%%%%



%%%%%%%%%%%%%%%%%%%%%%



\latexProblemContent{
\begin{problem}

Use the Fundamental Theorem of Calculus to find the derivative of the function.
\[g(t)=\int_{1}^{t} {x - 1}\;dx\]

\expandafter\input{\file@loc Integrals/2311-Compute-Integral-0006.HELP.tex}

\[\dfrac{d}{dt}(g(t))=\answer{t - 1}\]
\end{problem}}%}

%%%%%%%%%%%%%%%%%%%%%%



\latexProblemContent{
\begin{problem}

Use the Fundamental Theorem of Calculus to find the derivative of the function.
\[g(t)=\int_{3}^{t} {\frac{1}{{\left(x - 10\right)}^{2}}}\;dx\]

\expandafter\input{\file@loc Integrals/2311-Compute-Integral-0006.HELP.tex}

\[\dfrac{d}{dt}(g(t))=\answer{\frac{1}{{\left(t - 10\right)}^{2}}}\]
\end{problem}}%}

%%%%%%%%%%%%%%%%%%%%%%



\latexProblemContent{
\begin{problem}

Use the Fundamental Theorem of Calculus to find the derivative of the function.
\[g(t)=\int_{2}^{t} {\sin\left(x - 2\right)}\;dx\]

\expandafter\input{\file@loc Integrals/2311-Compute-Integral-0006.HELP.tex}

\[\dfrac{d}{dt}(g(t))=\answer{\sin\left(t - 2\right)}\]
\end{problem}}%}

%%%%%%%%%%%%%%%%%%%%%%



\latexProblemContent{
\begin{problem}

Use the Fundamental Theorem of Calculus to find the derivative of the function.
\[g(t)=\int_{4}^{t} {e^{\left(x - 8\right)}}\;dx\]

\expandafter\input{\file@loc Integrals/2311-Compute-Integral-0006.HELP.tex}

\[\dfrac{d}{dt}(g(t))=\answer{e^{\left(t - 8\right)}}\]
\end{problem}}%}

%%%%%%%%%%%%%%%%%%%%%%



\latexProblemContent{
\begin{problem}

Use the Fundamental Theorem of Calculus to find the derivative of the function.
\[g(t)=\int_{2}^{t} {e^{\left(x - 7\right)}}\;dx\]

\expandafter\input{\file@loc Integrals/2311-Compute-Integral-0006.HELP.tex}

\[\dfrac{d}{dt}(g(t))=\answer{e^{\left(t - 7\right)}}\]
\end{problem}}%}

%%%%%%%%%%%%%%%%%%%%%%



\latexProblemContent{
\begin{problem}

Use the Fundamental Theorem of Calculus to find the derivative of the function.
\[g(t)=\int_{3}^{t} {x - 6}\;dx\]

\expandafter\input{\file@loc Integrals/2311-Compute-Integral-0006.HELP.tex}

\[\dfrac{d}{dt}(g(t))=\answer{t - 6}\]
\end{problem}}%}

%%%%%%%%%%%%%%%%%%%%%%



\latexProblemContent{
\begin{problem}

Use the Fundamental Theorem of Calculus to find the derivative of the function.
\[g(t)=\int_{4}^{t} {\cos\left(x - 3\right)}\;dx\]

\expandafter\input{\file@loc Integrals/2311-Compute-Integral-0006.HELP.tex}

\[\dfrac{d}{dt}(g(t))=\answer{\cos\left(t - 3\right)}\]
\end{problem}}%}

%%%%%%%%%%%%%%%%%%%%%%



\latexProblemContent{
\begin{problem}

Use the Fundamental Theorem of Calculus to find the derivative of the function.
\[g(t)=\int_{5}^{t} {{\left(x - 9\right)}^{3}}\;dx\]

\expandafter\input{\file@loc Integrals/2311-Compute-Integral-0006.HELP.tex}

\[\dfrac{d}{dt}(g(t))=\answer{{\left(t - 9\right)}^{3}}\]
\end{problem}}%}

%%%%%%%%%%%%%%%%%%%%%%



\latexProblemContent{
\begin{problem}

Use the Fundamental Theorem of Calculus to find the derivative of the function.
\[g(t)=\int_{2}^{t} {\frac{1}{{\left(x + 4\right)}^{2}}}\;dx\]

\expandafter\input{\file@loc Integrals/2311-Compute-Integral-0006.HELP.tex}

\[\dfrac{d}{dt}(g(t))=\answer{\frac{1}{{\left(t + 4\right)}^{2}}}\]
\end{problem}}%}

%%%%%%%%%%%%%%%%%%%%%%



\latexProblemContent{
\begin{problem}

Use the Fundamental Theorem of Calculus to find the derivative of the function.
\[g(t)=\int_{2}^{t} {{\left(x - 6\right)}^{4}}\;dx\]

\expandafter\input{\file@loc Integrals/2311-Compute-Integral-0006.HELP.tex}

\[\dfrac{d}{dt}(g(t))=\answer{{\left(t - 6\right)}^{4}}\]
\end{problem}}%}

%%%%%%%%%%%%%%%%%%%%%%



\latexProblemContent{
\begin{problem}

Use the Fundamental Theorem of Calculus to find the derivative of the function.
\[g(t)=\int_{4}^{t} {{\left(x - 1\right)}^{2}}\;dx\]

\expandafter\input{\file@loc Integrals/2311-Compute-Integral-0006.HELP.tex}

\[\dfrac{d}{dt}(g(t))=\answer{{\left(t - 1\right)}^{2}}\]
\end{problem}}%}

%%%%%%%%%%%%%%%%%%%%%%



%%%%%%%%%%%%%%%%%%%%%%



\latexProblemContent{
\begin{problem}

Use the Fundamental Theorem of Calculus to find the derivative of the function.
\[g(t)=\int_{1}^{t} {{\left(x + 8\right)}^{2}}\;dx\]

\expandafter\input{\file@loc Integrals/2311-Compute-Integral-0006.HELP.tex}

\[\dfrac{d}{dt}(g(t))=\answer{{\left(t + 8\right)}^{2}}\]
\end{problem}}%}

%%%%%%%%%%%%%%%%%%%%%%



\latexProblemContent{
\begin{problem}

Use the Fundamental Theorem of Calculus to find the derivative of the function.
\[g(t)=\int_{4}^{t} {{\left(x - 2\right)}^{3}}\;dx\]

\expandafter\input{\file@loc Integrals/2311-Compute-Integral-0006.HELP.tex}

\[\dfrac{d}{dt}(g(t))=\answer{{\left(t - 2\right)}^{3}}\]
\end{problem}}%}

%%%%%%%%%%%%%%%%%%%%%%



\latexProblemContent{
\begin{problem}

Use the Fundamental Theorem of Calculus to find the derivative of the function.
\[g(t)=\int_{4}^{t} {{\left(x - 5\right)}^{4}}\;dx\]

\expandafter\input{\file@loc Integrals/2311-Compute-Integral-0006.HELP.tex}

\[\dfrac{d}{dt}(g(t))=\answer{{\left(t - 5\right)}^{4}}\]
\end{problem}}%}

%%%%%%%%%%%%%%%%%%%%%%



\latexProblemContent{
\begin{problem}

Use the Fundamental Theorem of Calculus to find the derivative of the function.
\[g(t)=\int_{3}^{t} {\sqrt{x - 2}}\;dx\]

\expandafter\input{\file@loc Integrals/2311-Compute-Integral-0006.HELP.tex}

\[\dfrac{d}{dt}(g(t))=\answer{\sqrt{t - 2}}\]
\end{problem}}%}

%%%%%%%%%%%%%%%%%%%%%%



\latexProblemContent{
\begin{problem}

Use the Fundamental Theorem of Calculus to find the derivative of the function.
\[g(t)=\int_{5}^{t} {{\left(x - 7\right)}^{2}}\;dx\]

\expandafter\input{\file@loc Integrals/2311-Compute-Integral-0006.HELP.tex}

\[\dfrac{d}{dt}(g(t))=\answer{{\left(t - 7\right)}^{2}}\]
\end{problem}}%}

%%%%%%%%%%%%%%%%%%%%%%



\latexProblemContent{
\begin{problem}

Use the Fundamental Theorem of Calculus to find the derivative of the function.
\[g(t)=\int_{2}^{t} {{\left(x - 5\right)}^{3}}\;dx\]

\expandafter\input{\file@loc Integrals/2311-Compute-Integral-0006.HELP.tex}

\[\dfrac{d}{dt}(g(t))=\answer{{\left(t - 5\right)}^{3}}\]
\end{problem}}%}

%%%%%%%%%%%%%%%%%%%%%%



%%%%%%%%%%%%%%%%%%%%%%



\latexProblemContent{
\begin{problem}

Use the Fundamental Theorem of Calculus to find the derivative of the function.
\[g(t)=\int_{3}^{t} {\frac{1}{x - 10}}\;dx\]

\expandafter\input{\file@loc Integrals/2311-Compute-Integral-0006.HELP.tex}

\[\dfrac{d}{dt}(g(t))=\answer{\frac{1}{t - 10}}\]
\end{problem}}%}

%%%%%%%%%%%%%%%%%%%%%%



\latexProblemContent{
\begin{problem}

Use the Fundamental Theorem of Calculus to find the derivative of the function.
\[g(t)=\int_{4}^{t} {x + 6}\;dx\]

\expandafter\input{\file@loc Integrals/2311-Compute-Integral-0006.HELP.tex}

\[\dfrac{d}{dt}(g(t))=\answer{t + 6}\]
\end{problem}}%}

%%%%%%%%%%%%%%%%%%%%%%



\latexProblemContent{
\begin{problem}

Use the Fundamental Theorem of Calculus to find the derivative of the function.
\[g(t)=\int_{4}^{t} {\sin\left(x - 6\right)}\;dx\]

\expandafter\input{\file@loc Integrals/2311-Compute-Integral-0006.HELP.tex}

\[\dfrac{d}{dt}(g(t))=\answer{\sin\left(t - 6\right)}\]
\end{problem}}%}

%%%%%%%%%%%%%%%%%%%%%%



\latexProblemContent{
\begin{problem}

Use the Fundamental Theorem of Calculus to find the derivative of the function.
\[g(t)=\int_{3}^{t} {\sin\left(x + 8\right)}\;dx\]

\expandafter\input{\file@loc Integrals/2311-Compute-Integral-0006.HELP.tex}

\[\dfrac{d}{dt}(g(t))=\answer{\sin\left(t + 8\right)}\]
\end{problem}}%}

%%%%%%%%%%%%%%%%%%%%%%



\latexProblemContent{
\begin{problem}

Use the Fundamental Theorem of Calculus to find the derivative of the function.
\[g(t)=\int_{2}^{t} {x - 4}\;dx\]

\expandafter\input{\file@loc Integrals/2311-Compute-Integral-0006.HELP.tex}

\[\dfrac{d}{dt}(g(t))=\answer{t - 4}\]
\end{problem}}%}

%%%%%%%%%%%%%%%%%%%%%%



\latexProblemContent{
\begin{problem}

Use the Fundamental Theorem of Calculus to find the derivative of the function.
\[g(t)=\int_{5}^{t} {\sqrt{x - 6}}\;dx\]

\expandafter\input{\file@loc Integrals/2311-Compute-Integral-0006.HELP.tex}

\[\dfrac{d}{dt}(g(t))=\answer{\sqrt{t - 6}}\]
\end{problem}}%}

%%%%%%%%%%%%%%%%%%%%%%



\latexProblemContent{
\begin{problem}

Use the Fundamental Theorem of Calculus to find the derivative of the function.
\[g(t)=\int_{4}^{t} {{\left(x + 8\right)}^{3}}\;dx\]

\expandafter\input{\file@loc Integrals/2311-Compute-Integral-0006.HELP.tex}

\[\dfrac{d}{dt}(g(t))=\answer{{\left(t + 8\right)}^{3}}\]
\end{problem}}%}

%%%%%%%%%%%%%%%%%%%%%%



\latexProblemContent{
\begin{problem}

Use the Fundamental Theorem of Calculus to find the derivative of the function.
\[g(t)=\int_{3}^{t} {x - 8}\;dx\]

\expandafter\input{\file@loc Integrals/2311-Compute-Integral-0006.HELP.tex}

\[\dfrac{d}{dt}(g(t))=\answer{t - 8}\]
\end{problem}}%}

%%%%%%%%%%%%%%%%%%%%%%



\latexProblemContent{
\begin{problem}

Use the Fundamental Theorem of Calculus to find the derivative of the function.
\[g(t)=\int_{4}^{t} {{\left(x + 10\right)}^{2}}\;dx\]

\expandafter\input{\file@loc Integrals/2311-Compute-Integral-0006.HELP.tex}

\[\dfrac{d}{dt}(g(t))=\answer{{\left(t + 10\right)}^{2}}\]
\end{problem}}%}

%%%%%%%%%%%%%%%%%%%%%%



\latexProblemContent{
\begin{problem}

Use the Fundamental Theorem of Calculus to find the derivative of the function.
\[g(t)=\int_{1}^{t} {{\left(x + 5\right)}^{3}}\;dx\]

\expandafter\input{\file@loc Integrals/2311-Compute-Integral-0006.HELP.tex}

\[\dfrac{d}{dt}(g(t))=\answer{{\left(t + 5\right)}^{3}}\]
\end{problem}}%}

%%%%%%%%%%%%%%%%%%%%%%



\latexProblemContent{
\begin{problem}

Use the Fundamental Theorem of Calculus to find the derivative of the function.
\[g(t)=\int_{2}^{t} {\frac{1}{{\left(x - 5\right)}^{2}}}\;dx\]

\expandafter\input{\file@loc Integrals/2311-Compute-Integral-0006.HELP.tex}

\[\dfrac{d}{dt}(g(t))=\answer{\frac{1}{{\left(t - 5\right)}^{2}}}\]
\end{problem}}%}

%%%%%%%%%%%%%%%%%%%%%%



\latexProblemContent{
\begin{problem}

Use the Fundamental Theorem of Calculus to find the derivative of the function.
\[g(t)=\int_{2}^{t} {\frac{1}{{\left(x - 8\right)}^{3}}}\;dx\]

\expandafter\input{\file@loc Integrals/2311-Compute-Integral-0006.HELP.tex}

\[\dfrac{d}{dt}(g(t))=\answer{\frac{1}{{\left(t - 8\right)}^{3}}}\]
\end{problem}}%}

%%%%%%%%%%%%%%%%%%%%%%



\latexProblemContent{
\begin{problem}

Use the Fundamental Theorem of Calculus to find the derivative of the function.
\[g(t)=\int_{4}^{t} {\frac{1}{x + 5}}\;dx\]

\expandafter\input{\file@loc Integrals/2311-Compute-Integral-0006.HELP.tex}

\[\dfrac{d}{dt}(g(t))=\answer{\frac{1}{t + 5}}\]
\end{problem}}%}

%%%%%%%%%%%%%%%%%%%%%%



%%%%%%%%%%%%%%%%%%%%%%



\latexProblemContent{
\begin{problem}

Use the Fundamental Theorem of Calculus to find the derivative of the function.
\[g(t)=\int_{4}^{t} {\frac{1}{{\left(x + 3\right)}^{2}}}\;dx\]

\expandafter\input{\file@loc Integrals/2311-Compute-Integral-0006.HELP.tex}

\[\dfrac{d}{dt}(g(t))=\answer{\frac{1}{{\left(t + 3\right)}^{2}}}\]
\end{problem}}%}

%%%%%%%%%%%%%%%%%%%%%%



\latexProblemContent{
\begin{problem}

Use the Fundamental Theorem of Calculus to find the derivative of the function.
\[g(t)=\int_{4}^{t} {\frac{1}{{\left(x + 5\right)}^{2}}}\;dx\]

\expandafter\input{\file@loc Integrals/2311-Compute-Integral-0006.HELP.tex}

\[\dfrac{d}{dt}(g(t))=\answer{\frac{1}{{\left(t + 5\right)}^{2}}}\]
\end{problem}}%}

%%%%%%%%%%%%%%%%%%%%%%



\latexProblemContent{
\begin{problem}

Use the Fundamental Theorem of Calculus to find the derivative of the function.
\[g(t)=\int_{4}^{t} {\log\left(x - 2\right)}\;dx\]

\expandafter\input{\file@loc Integrals/2311-Compute-Integral-0006.HELP.tex}

\[\dfrac{d}{dt}(g(t))=\answer{\log\left(t - 2\right)}\]
\end{problem}}%}

%%%%%%%%%%%%%%%%%%%%%%



%%%%%%%%%%%%%%%%%%%%%%



\latexProblemContent{
\begin{problem}

Use the Fundamental Theorem of Calculus to find the derivative of the function.
\[g(t)=\int_{2}^{t} {\cos\left(x - 10\right)}\;dx\]

\expandafter\input{\file@loc Integrals/2311-Compute-Integral-0006.HELP.tex}

\[\dfrac{d}{dt}(g(t))=\answer{\cos\left(t - 10\right)}\]
\end{problem}}%}

%%%%%%%%%%%%%%%%%%%%%%



%%%%%%%%%%%%%%%%%%%%%%



\latexProblemContent{
\begin{problem}

Use the Fundamental Theorem of Calculus to find the derivative of the function.
\[g(t)=\int_{5}^{t} {e^{\left(x + 5\right)}}\;dx\]

\expandafter\input{\file@loc Integrals/2311-Compute-Integral-0006.HELP.tex}

\[\dfrac{d}{dt}(g(t))=\answer{e^{\left(t + 5\right)}}\]
\end{problem}}%}

%%%%%%%%%%%%%%%%%%%%%%



\latexProblemContent{
\begin{problem}

Use the Fundamental Theorem of Calculus to find the derivative of the function.
\[g(t)=\int_{3}^{t} {\log\left(x - 5\right)}\;dx\]

\expandafter\input{\file@loc Integrals/2311-Compute-Integral-0006.HELP.tex}

\[\dfrac{d}{dt}(g(t))=\answer{\log\left(t - 5\right)}\]
\end{problem}}%}

%%%%%%%%%%%%%%%%%%%%%%



\latexProblemContent{
\begin{problem}

Use the Fundamental Theorem of Calculus to find the derivative of the function.
\[g(t)=\int_{3}^{t} {e^{\left(x + 2\right)}}\;dx\]

\expandafter\input{\file@loc Integrals/2311-Compute-Integral-0006.HELP.tex}

\[\dfrac{d}{dt}(g(t))=\answer{e^{\left(t + 2\right)}}\]
\end{problem}}%}

%%%%%%%%%%%%%%%%%%%%%%



%%%%%%%%%%%%%%%%%%%%%%



%%%%%%%%%%%%%%%%%%%%%%



%%%%%%%%%%%%%%%%%%%%%%



\latexProblemContent{
\begin{problem}

Use the Fundamental Theorem of Calculus to find the derivative of the function.
\[g(t)=\int_{5}^{t} {\cos\left(x - 9\right)}\;dx\]

\expandafter\input{\file@loc Integrals/2311-Compute-Integral-0006.HELP.tex}

\[\dfrac{d}{dt}(g(t))=\answer{\cos\left(t - 9\right)}\]
\end{problem}}%}

%%%%%%%%%%%%%%%%%%%%%%



\latexProblemContent{
\begin{problem}

Use the Fundamental Theorem of Calculus to find the derivative of the function.
\[g(t)=\int_{2}^{t} {e^{\left(x + 5\right)}}\;dx\]

\expandafter\input{\file@loc Integrals/2311-Compute-Integral-0006.HELP.tex}

\[\dfrac{d}{dt}(g(t))=\answer{e^{\left(t + 5\right)}}\]
\end{problem}}%}

%%%%%%%%%%%%%%%%%%%%%%



%%%%%%%%%%%%%%%%%%%%%%



\latexProblemContent{
\begin{problem}

Use the Fundamental Theorem of Calculus to find the derivative of the function.
\[g(t)=\int_{4}^{t} {{\left(x + 2\right)}^{3}}\;dx\]

\expandafter\input{\file@loc Integrals/2311-Compute-Integral-0006.HELP.tex}

\[\dfrac{d}{dt}(g(t))=\answer{{\left(t + 2\right)}^{3}}\]
\end{problem}}%}

%%%%%%%%%%%%%%%%%%%%%%



%%%%%%%%%%%%%%%%%%%%%%



\latexProblemContent{
\begin{problem}

Use the Fundamental Theorem of Calculus to find the derivative of the function.
\[g(t)=\int_{2}^{t} {\frac{1}{{\left(x + 9\right)}^{2}}}\;dx\]

\expandafter\input{\file@loc Integrals/2311-Compute-Integral-0006.HELP.tex}

\[\dfrac{d}{dt}(g(t))=\answer{\frac{1}{{\left(t + 9\right)}^{2}}}\]
\end{problem}}%}

%%%%%%%%%%%%%%%%%%%%%%



\latexProblemContent{
\begin{problem}

Use the Fundamental Theorem of Calculus to find the derivative of the function.
\[g(t)=\int_{5}^{t} {\sin\left(x + 10\right)}\;dx\]

\expandafter\input{\file@loc Integrals/2311-Compute-Integral-0006.HELP.tex}

\[\dfrac{d}{dt}(g(t))=\answer{\sin\left(t + 10\right)}\]
\end{problem}}%}

%%%%%%%%%%%%%%%%%%%%%%



\latexProblemContent{
\begin{problem}

Use the Fundamental Theorem of Calculus to find the derivative of the function.
\[g(t)=\int_{1}^{t} {{\left(x - 2\right)}^{2}}\;dx\]

\expandafter\input{\file@loc Integrals/2311-Compute-Integral-0006.HELP.tex}

\[\dfrac{d}{dt}(g(t))=\answer{{\left(t - 2\right)}^{2}}\]
\end{problem}}%}

%%%%%%%%%%%%%%%%%%%%%%



%%%%%%%%%%%%%%%%%%%%%%



%%%%%%%%%%%%%%%%%%%%%%



\latexProblemContent{
\begin{problem}

Use the Fundamental Theorem of Calculus to find the derivative of the function.
\[g(t)=\int_{4}^{t} {{\left(x + 10\right)}^{4}}\;dx\]

\expandafter\input{\file@loc Integrals/2311-Compute-Integral-0006.HELP.tex}

\[\dfrac{d}{dt}(g(t))=\answer{{\left(t + 10\right)}^{4}}\]
\end{problem}}%}

%%%%%%%%%%%%%%%%%%%%%%



%%%%%%%%%%%%%%%%%%%%%%



%%%%%%%%%%%%%%%%%%%%%%



\latexProblemContent{
\begin{problem}

Use the Fundamental Theorem of Calculus to find the derivative of the function.
\[g(t)=\int_{3}^{t} {\sqrt{x - 8}}\;dx\]

\expandafter\input{\file@loc Integrals/2311-Compute-Integral-0006.HELP.tex}

\[\dfrac{d}{dt}(g(t))=\answer{\sqrt{t - 8}}\]
\end{problem}}%}

%%%%%%%%%%%%%%%%%%%%%%



\latexProblemContent{
\begin{problem}

Use the Fundamental Theorem of Calculus to find the derivative of the function.
\[g(t)=\int_{1}^{t} {\sqrt{x + 8}}\;dx\]

\expandafter\input{\file@loc Integrals/2311-Compute-Integral-0006.HELP.tex}

\[\dfrac{d}{dt}(g(t))=\answer{\sqrt{t + 8}}\]
\end{problem}}%}

%%%%%%%%%%%%%%%%%%%%%%



\latexProblemContent{
\begin{problem}

Use the Fundamental Theorem of Calculus to find the derivative of the function.
\[g(t)=\int_{4}^{t} {\frac{1}{{\left(x - 9\right)}^{2}}}\;dx\]

\expandafter\input{\file@loc Integrals/2311-Compute-Integral-0006.HELP.tex}

\[\dfrac{d}{dt}(g(t))=\answer{\frac{1}{{\left(t - 9\right)}^{2}}}\]
\end{problem}}%}

%%%%%%%%%%%%%%%%%%%%%%



\latexProblemContent{
\begin{problem}

Use the Fundamental Theorem of Calculus to find the derivative of the function.
\[g(t)=\int_{4}^{t} {{\left(x - 7\right)}^{2}}\;dx\]

\expandafter\input{\file@loc Integrals/2311-Compute-Integral-0006.HELP.tex}

\[\dfrac{d}{dt}(g(t))=\answer{{\left(t - 7\right)}^{2}}\]
\end{problem}}%}

%%%%%%%%%%%%%%%%%%%%%%



\latexProblemContent{
\begin{problem}

Use the Fundamental Theorem of Calculus to find the derivative of the function.
\[g(t)=\int_{3}^{t} {\cos\left(x - 8\right)}\;dx\]

\expandafter\input{\file@loc Integrals/2311-Compute-Integral-0006.HELP.tex}

\[\dfrac{d}{dt}(g(t))=\answer{\cos\left(t - 8\right)}\]
\end{problem}}%}

%%%%%%%%%%%%%%%%%%%%%%



\latexProblemContent{
\begin{problem}

Use the Fundamental Theorem of Calculus to find the derivative of the function.
\[g(t)=\int_{4}^{t} {\sqrt{x + 9}}\;dx\]

\expandafter\input{\file@loc Integrals/2311-Compute-Integral-0006.HELP.tex}

\[\dfrac{d}{dt}(g(t))=\answer{\sqrt{t + 9}}\]
\end{problem}}%}

%%%%%%%%%%%%%%%%%%%%%%



\latexProblemContent{
\begin{problem}

Use the Fundamental Theorem of Calculus to find the derivative of the function.
\[g(t)=\int_{4}^{t} {\sin\left(x + 4\right)}\;dx\]

\expandafter\input{\file@loc Integrals/2311-Compute-Integral-0006.HELP.tex}

\[\dfrac{d}{dt}(g(t))=\answer{\sin\left(t + 4\right)}\]
\end{problem}}%}

%%%%%%%%%%%%%%%%%%%%%%



\latexProblemContent{
\begin{problem}

Use the Fundamental Theorem of Calculus to find the derivative of the function.
\[g(t)=\int_{5}^{t} {\frac{1}{{\left(x + 9\right)}^{3}}}\;dx\]

\expandafter\input{\file@loc Integrals/2311-Compute-Integral-0006.HELP.tex}

\[\dfrac{d}{dt}(g(t))=\answer{\frac{1}{{\left(t + 9\right)}^{3}}}\]
\end{problem}}%}

%%%%%%%%%%%%%%%%%%%%%%



\latexProblemContent{
\begin{problem}

Use the Fundamental Theorem of Calculus to find the derivative of the function.
\[g(t)=\int_{2}^{t} {e^{\left(x - 3\right)}}\;dx\]

\expandafter\input{\file@loc Integrals/2311-Compute-Integral-0006.HELP.tex}

\[\dfrac{d}{dt}(g(t))=\answer{e^{\left(t - 3\right)}}\]
\end{problem}}%}

%%%%%%%%%%%%%%%%%%%%%%



\latexProblemContent{
\begin{problem}

Use the Fundamental Theorem of Calculus to find the derivative of the function.
\[g(t)=\int_{1}^{t} {\frac{1}{{\left(x - 7\right)}^{2}}}\;dx\]

\expandafter\input{\file@loc Integrals/2311-Compute-Integral-0006.HELP.tex}

\[\dfrac{d}{dt}(g(t))=\answer{\frac{1}{{\left(t - 7\right)}^{2}}}\]
\end{problem}}%}

%%%%%%%%%%%%%%%%%%%%%%



\latexProblemContent{
\begin{problem}

Use the Fundamental Theorem of Calculus to find the derivative of the function.
\[g(t)=\int_{5}^{t} {\frac{1}{x + 9}}\;dx\]

\expandafter\input{\file@loc Integrals/2311-Compute-Integral-0006.HELP.tex}

\[\dfrac{d}{dt}(g(t))=\answer{\frac{1}{t + 9}}\]
\end{problem}}%}

%%%%%%%%%%%%%%%%%%%%%%



%%%%%%%%%%%%%%%%%%%%%%



\latexProblemContent{
\begin{problem}

Use the Fundamental Theorem of Calculus to find the derivative of the function.
\[g(t)=\int_{1}^{t} {x - 7}\;dx\]

\expandafter\input{\file@loc Integrals/2311-Compute-Integral-0006.HELP.tex}

\[\dfrac{d}{dt}(g(t))=\answer{t - 7}\]
\end{problem}}%}

%%%%%%%%%%%%%%%%%%%%%%



\latexProblemContent{
\begin{problem}

Use the Fundamental Theorem of Calculus to find the derivative of the function.
\[g(t)=\int_{5}^{t} {\cos\left(x + 2\right)}\;dx\]

\expandafter\input{\file@loc Integrals/2311-Compute-Integral-0006.HELP.tex}

\[\dfrac{d}{dt}(g(t))=\answer{\cos\left(t + 2\right)}\]
\end{problem}}%}

%%%%%%%%%%%%%%%%%%%%%%



%%%%%%%%%%%%%%%%%%%%%%



%%%%%%%%%%%%%%%%%%%%%%



\latexProblemContent{
\begin{problem}

Use the Fundamental Theorem of Calculus to find the derivative of the function.
\[g(t)=\int_{1}^{t} {{\left(x - 6\right)}^{2}}\;dx\]

\expandafter\input{\file@loc Integrals/2311-Compute-Integral-0006.HELP.tex}

\[\dfrac{d}{dt}(g(t))=\answer{{\left(t - 6\right)}^{2}}\]
\end{problem}}%}

%%%%%%%%%%%%%%%%%%%%%%



\latexProblemContent{
\begin{problem}

Use the Fundamental Theorem of Calculus to find the derivative of the function.
\[g(t)=\int_{3}^{t} {{\left(x - 9\right)}^{3}}\;dx\]

\expandafter\input{\file@loc Integrals/2311-Compute-Integral-0006.HELP.tex}

\[\dfrac{d}{dt}(g(t))=\answer{{\left(t - 9\right)}^{3}}\]
\end{problem}}%}

%%%%%%%%%%%%%%%%%%%%%%



\latexProblemContent{
\begin{problem}

Use the Fundamental Theorem of Calculus to find the derivative of the function.
\[g(t)=\int_{4}^{t} {\frac{1}{{\left(x - 5\right)}^{2}}}\;dx\]

\expandafter\input{\file@loc Integrals/2311-Compute-Integral-0006.HELP.tex}

\[\dfrac{d}{dt}(g(t))=\answer{\frac{1}{{\left(t - 5\right)}^{2}}}\]
\end{problem}}%}

%%%%%%%%%%%%%%%%%%%%%%



%%%%%%%%%%%%%%%%%%%%%%



\latexProblemContent{
\begin{problem}

Use the Fundamental Theorem of Calculus to find the derivative of the function.
\[g(t)=\int_{1}^{t} {\frac{1}{{\left(x + 2\right)}^{2}}}\;dx\]

\expandafter\input{\file@loc Integrals/2311-Compute-Integral-0006.HELP.tex}

\[\dfrac{d}{dt}(g(t))=\answer{\frac{1}{{\left(t + 2\right)}^{2}}}\]
\end{problem}}%}

%%%%%%%%%%%%%%%%%%%%%%



%%%%%%%%%%%%%%%%%%%%%%



%%%%%%%%%%%%%%%%%%%%%%



\latexProblemContent{
\begin{problem}

Use the Fundamental Theorem of Calculus to find the derivative of the function.
\[g(t)=\int_{5}^{t} {{\left(x - 7\right)}^{3}}\;dx\]

\expandafter\input{\file@loc Integrals/2311-Compute-Integral-0006.HELP.tex}

\[\dfrac{d}{dt}(g(t))=\answer{{\left(t - 7\right)}^{3}}\]
\end{problem}}%}

%%%%%%%%%%%%%%%%%%%%%%



\latexProblemContent{
\begin{problem}

Use the Fundamental Theorem of Calculus to find the derivative of the function.
\[g(t)=\int_{5}^{t} {\frac{1}{x + 1}}\;dx\]

\expandafter\input{\file@loc Integrals/2311-Compute-Integral-0006.HELP.tex}

\[\dfrac{d}{dt}(g(t))=\answer{\frac{1}{t + 1}}\]
\end{problem}}%}

%%%%%%%%%%%%%%%%%%%%%%



\latexProblemContent{
\begin{problem}

Use the Fundamental Theorem of Calculus to find the derivative of the function.
\[g(t)=\int_{4}^{t} {\frac{1}{{\left(x - 6\right)}^{3}}}\;dx\]

\expandafter\input{\file@loc Integrals/2311-Compute-Integral-0006.HELP.tex}

\[\dfrac{d}{dt}(g(t))=\answer{\frac{1}{{\left(t - 6\right)}^{3}}}\]
\end{problem}}%}

%%%%%%%%%%%%%%%%%%%%%%



%%%%%%%%%%%%%%%%%%%%%%



\latexProblemContent{
\begin{problem}

Use the Fundamental Theorem of Calculus to find the derivative of the function.
\[g(t)=\int_{5}^{t} {\sqrt{x + 9}}\;dx\]

\expandafter\input{\file@loc Integrals/2311-Compute-Integral-0006.HELP.tex}

\[\dfrac{d}{dt}(g(t))=\answer{\sqrt{t + 9}}\]
\end{problem}}%}

%%%%%%%%%%%%%%%%%%%%%%



%%%%%%%%%%%%%%%%%%%%%%



\latexProblemContent{
\begin{problem}

Use the Fundamental Theorem of Calculus to find the derivative of the function.
\[g(t)=\int_{1}^{t} {\frac{1}{{\left(x + 9\right)}^{3}}}\;dx\]

\expandafter\input{\file@loc Integrals/2311-Compute-Integral-0006.HELP.tex}

\[\dfrac{d}{dt}(g(t))=\answer{\frac{1}{{\left(t + 9\right)}^{3}}}\]
\end{problem}}%}

%%%%%%%%%%%%%%%%%%%%%%



\latexProblemContent{
\begin{problem}

Use the Fundamental Theorem of Calculus to find the derivative of the function.
\[g(t)=\int_{5}^{t} {{\left(x + 1\right)}^{4}}\;dx\]

\expandafter\input{\file@loc Integrals/2311-Compute-Integral-0006.HELP.tex}

\[\dfrac{d}{dt}(g(t))=\answer{{\left(t + 1\right)}^{4}}\]
\end{problem}}%}

%%%%%%%%%%%%%%%%%%%%%%



\latexProblemContent{
\begin{problem}

Use the Fundamental Theorem of Calculus to find the derivative of the function.
\[g(t)=\int_{1}^{t} {e^{\left(x - 10\right)}}\;dx\]

\expandafter\input{\file@loc Integrals/2311-Compute-Integral-0006.HELP.tex}

\[\dfrac{d}{dt}(g(t))=\answer{e^{\left(t - 10\right)}}\]
\end{problem}}%}

%%%%%%%%%%%%%%%%%%%%%%



\latexProblemContent{
\begin{problem}

Use the Fundamental Theorem of Calculus to find the derivative of the function.
\[g(t)=\int_{3}^{t} {\frac{1}{{\left(x - 3\right)}^{2}}}\;dx\]

\expandafter\input{\file@loc Integrals/2311-Compute-Integral-0006.HELP.tex}

\[\dfrac{d}{dt}(g(t))=\answer{\frac{1}{{\left(t - 3\right)}^{2}}}\]
\end{problem}}%}

%%%%%%%%%%%%%%%%%%%%%%



%%%%%%%%%%%%%%%%%%%%%%



\latexProblemContent{
\begin{problem}

Use the Fundamental Theorem of Calculus to find the derivative of the function.
\[g(t)=\int_{3}^{t} {\log\left(x - 1\right)}\;dx\]

\expandafter\input{\file@loc Integrals/2311-Compute-Integral-0006.HELP.tex}

\[\dfrac{d}{dt}(g(t))=\answer{\log\left(t - 1\right)}\]
\end{problem}}%}

%%%%%%%%%%%%%%%%%%%%%%



\latexProblemContent{
\begin{problem}

Use the Fundamental Theorem of Calculus to find the derivative of the function.
\[g(t)=\int_{2}^{t} {x - 2}\;dx\]

\expandafter\input{\file@loc Integrals/2311-Compute-Integral-0006.HELP.tex}

\[\dfrac{d}{dt}(g(t))=\answer{t - 2}\]
\end{problem}}%}

%%%%%%%%%%%%%%%%%%%%%%



\latexProblemContent{
\begin{problem}

Use the Fundamental Theorem of Calculus to find the derivative of the function.
\[g(t)=\int_{1}^{t} {\frac{1}{{\left(x - 10\right)}^{3}}}\;dx\]

\expandafter\input{\file@loc Integrals/2311-Compute-Integral-0006.HELP.tex}

\[\dfrac{d}{dt}(g(t))=\answer{\frac{1}{{\left(t - 10\right)}^{3}}}\]
\end{problem}}%}

%%%%%%%%%%%%%%%%%%%%%%



\latexProblemContent{
\begin{problem}

Use the Fundamental Theorem of Calculus to find the derivative of the function.
\[g(t)=\int_{2}^{t} {{\left(x + 5\right)}^{2}}\;dx\]

\expandafter\input{\file@loc Integrals/2311-Compute-Integral-0006.HELP.tex}

\[\dfrac{d}{dt}(g(t))=\answer{{\left(t + 5\right)}^{2}}\]
\end{problem}}%}

%%%%%%%%%%%%%%%%%%%%%%



%%%%%%%%%%%%%%%%%%%%%%



\latexProblemContent{
\begin{problem}

Use the Fundamental Theorem of Calculus to find the derivative of the function.
\[g(t)=\int_{2}^{t} {e^{\left(x + 7\right)}}\;dx\]

\expandafter\input{\file@loc Integrals/2311-Compute-Integral-0006.HELP.tex}

\[\dfrac{d}{dt}(g(t))=\answer{e^{\left(t + 7\right)}}\]
\end{problem}}%}

%%%%%%%%%%%%%%%%%%%%%%



%%%%%%%%%%%%%%%%%%%%%%



\latexProblemContent{
\begin{problem}

Use the Fundamental Theorem of Calculus to find the derivative of the function.
\[g(t)=\int_{2}^{t} {\sqrt{x - 1}}\;dx\]

\expandafter\input{\file@loc Integrals/2311-Compute-Integral-0006.HELP.tex}

\[\dfrac{d}{dt}(g(t))=\answer{\sqrt{t - 1}}\]
\end{problem}}%}

%%%%%%%%%%%%%%%%%%%%%%



%%%%%%%%%%%%%%%%%%%%%%



\latexProblemContent{
\begin{problem}

Use the Fundamental Theorem of Calculus to find the derivative of the function.
\[g(t)=\int_{2}^{t} {e^{\left(x - 2\right)}}\;dx\]

\expandafter\input{\file@loc Integrals/2311-Compute-Integral-0006.HELP.tex}

\[\dfrac{d}{dt}(g(t))=\answer{e^{\left(t - 2\right)}}\]
\end{problem}}%}

%%%%%%%%%%%%%%%%%%%%%%



\latexProblemContent{
\begin{problem}

Use the Fundamental Theorem of Calculus to find the derivative of the function.
\[g(t)=\int_{4}^{t} {e^{\left(x - 7\right)}}\;dx\]

\expandafter\input{\file@loc Integrals/2311-Compute-Integral-0006.HELP.tex}

\[\dfrac{d}{dt}(g(t))=\answer{e^{\left(t - 7\right)}}\]
\end{problem}}%}

%%%%%%%%%%%%%%%%%%%%%%



%%%%%%%%%%%%%%%%%%%%%%



%%%%%%%%%%%%%%%%%%%%%%



\latexProblemContent{
\begin{problem}

Use the Fundamental Theorem of Calculus to find the derivative of the function.
\[g(t)=\int_{1}^{t} {\frac{1}{x + 7}}\;dx\]

\expandafter\input{\file@loc Integrals/2311-Compute-Integral-0006.HELP.tex}

\[\dfrac{d}{dt}(g(t))=\answer{\frac{1}{t + 7}}\]
\end{problem}}%}

%%%%%%%%%%%%%%%%%%%%%%



\latexProblemContent{
\begin{problem}

Use the Fundamental Theorem of Calculus to find the derivative of the function.
\[g(t)=\int_{3}^{t} {{\left(x + 8\right)}^{3}}\;dx\]

\expandafter\input{\file@loc Integrals/2311-Compute-Integral-0006.HELP.tex}

\[\dfrac{d}{dt}(g(t))=\answer{{\left(t + 8\right)}^{3}}\]
\end{problem}}%}

%%%%%%%%%%%%%%%%%%%%%%



%%%%%%%%%%%%%%%%%%%%%%



%%%%%%%%%%%%%%%%%%%%%%



%%%%%%%%%%%%%%%%%%%%%%



\latexProblemContent{
\begin{problem}

Use the Fundamental Theorem of Calculus to find the derivative of the function.
\[g(t)=\int_{4}^{t} {{\left(x + 9\right)}^{3}}\;dx\]

\expandafter\input{\file@loc Integrals/2311-Compute-Integral-0006.HELP.tex}

\[\dfrac{d}{dt}(g(t))=\answer{{\left(t + 9\right)}^{3}}\]
\end{problem}}%}

%%%%%%%%%%%%%%%%%%%%%%



\latexProblemContent{
\begin{problem}

Use the Fundamental Theorem of Calculus to find the derivative of the function.
\[g(t)=\int_{5}^{t} {{\left(x - 5\right)}^{2}}\;dx\]

\expandafter\input{\file@loc Integrals/2311-Compute-Integral-0006.HELP.tex}

\[\dfrac{d}{dt}(g(t))=\answer{{\left(t - 5\right)}^{2}}\]
\end{problem}}%}

%%%%%%%%%%%%%%%%%%%%%%



%%%%%%%%%%%%%%%%%%%%%%



\latexProblemContent{
\begin{problem}

Use the Fundamental Theorem of Calculus to find the derivative of the function.
\[g(t)=\int_{1}^{t} {\cos\left(x + 1\right)}\;dx\]

\expandafter\input{\file@loc Integrals/2311-Compute-Integral-0006.HELP.tex}

\[\dfrac{d}{dt}(g(t))=\answer{\cos\left(t + 1\right)}\]
\end{problem}}%}

%%%%%%%%%%%%%%%%%%%%%%



\latexProblemContent{
\begin{problem}

Use the Fundamental Theorem of Calculus to find the derivative of the function.
\[g(t)=\int_{5}^{t} {{\left(x - 6\right)}^{3}}\;dx\]

\expandafter\input{\file@loc Integrals/2311-Compute-Integral-0006.HELP.tex}

\[\dfrac{d}{dt}(g(t))=\answer{{\left(t - 6\right)}^{3}}\]
\end{problem}}%}

%%%%%%%%%%%%%%%%%%%%%%



\latexProblemContent{
\begin{problem}

Use the Fundamental Theorem of Calculus to find the derivative of the function.
\[g(t)=\int_{5}^{t} {e^{\left(x - 7\right)}}\;dx\]

\expandafter\input{\file@loc Integrals/2311-Compute-Integral-0006.HELP.tex}

\[\dfrac{d}{dt}(g(t))=\answer{e^{\left(t - 7\right)}}\]
\end{problem}}%}

%%%%%%%%%%%%%%%%%%%%%%



\latexProblemContent{
\begin{problem}

Use the Fundamental Theorem of Calculus to find the derivative of the function.
\[g(t)=\int_{5}^{t} {\log\left(x - 2\right)}\;dx\]

\expandafter\input{\file@loc Integrals/2311-Compute-Integral-0006.HELP.tex}

\[\dfrac{d}{dt}(g(t))=\answer{\log\left(t - 2\right)}\]
\end{problem}}%}

%%%%%%%%%%%%%%%%%%%%%%



%%%%%%%%%%%%%%%%%%%%%%



\latexProblemContent{
\begin{problem}

Use the Fundamental Theorem of Calculus to find the derivative of the function.
\[g(t)=\int_{4}^{t} {\sqrt{x - 1}}\;dx\]

\expandafter\input{\file@loc Integrals/2311-Compute-Integral-0006.HELP.tex}

\[\dfrac{d}{dt}(g(t))=\answer{\sqrt{t - 1}}\]
\end{problem}}%}

%%%%%%%%%%%%%%%%%%%%%%



\latexProblemContent{
\begin{problem}

Use the Fundamental Theorem of Calculus to find the derivative of the function.
\[g(t)=\int_{1}^{t} {x + 6}\;dx\]

\expandafter\input{\file@loc Integrals/2311-Compute-Integral-0006.HELP.tex}

\[\dfrac{d}{dt}(g(t))=\answer{t + 6}\]
\end{problem}}%}

%%%%%%%%%%%%%%%%%%%%%%



\latexProblemContent{
\begin{problem}

Use the Fundamental Theorem of Calculus to find the derivative of the function.
\[g(t)=\int_{1}^{t} {{\left(x + 5\right)}^{4}}\;dx\]

\expandafter\input{\file@loc Integrals/2311-Compute-Integral-0006.HELP.tex}

\[\dfrac{d}{dt}(g(t))=\answer{{\left(t + 5\right)}^{4}}\]
\end{problem}}%}

%%%%%%%%%%%%%%%%%%%%%%



%%%%%%%%%%%%%%%%%%%%%%



\latexProblemContent{
\begin{problem}

Use the Fundamental Theorem of Calculus to find the derivative of the function.
\[g(t)=\int_{3}^{t} {{\left(x - 7\right)}^{3}}\;dx\]

\expandafter\input{\file@loc Integrals/2311-Compute-Integral-0006.HELP.tex}

\[\dfrac{d}{dt}(g(t))=\answer{{\left(t - 7\right)}^{3}}\]
\end{problem}}%}

%%%%%%%%%%%%%%%%%%%%%%



\latexProblemContent{
\begin{problem}

Use the Fundamental Theorem of Calculus to find the derivative of the function.
\[g(t)=\int_{1}^{t} {\frac{1}{{\left(x - 9\right)}^{2}}}\;dx\]

\expandafter\input{\file@loc Integrals/2311-Compute-Integral-0006.HELP.tex}

\[\dfrac{d}{dt}(g(t))=\answer{\frac{1}{{\left(t - 9\right)}^{2}}}\]
\end{problem}}%}

%%%%%%%%%%%%%%%%%%%%%%



\latexProblemContent{
\begin{problem}

Use the Fundamental Theorem of Calculus to find the derivative of the function.
\[g(t)=\int_{5}^{t} {{\left(x - 8\right)}^{3}}\;dx\]

\expandafter\input{\file@loc Integrals/2311-Compute-Integral-0006.HELP.tex}

\[\dfrac{d}{dt}(g(t))=\answer{{\left(t - 8\right)}^{3}}\]
\end{problem}}%}

%%%%%%%%%%%%%%%%%%%%%%



\latexProblemContent{
\begin{problem}

Use the Fundamental Theorem of Calculus to find the derivative of the function.
\[g(t)=\int_{5}^{t} {{\left(x + 6\right)}^{4}}\;dx\]

\expandafter\input{\file@loc Integrals/2311-Compute-Integral-0006.HELP.tex}

\[\dfrac{d}{dt}(g(t))=\answer{{\left(t + 6\right)}^{4}}\]
\end{problem}}%}

%%%%%%%%%%%%%%%%%%%%%%



\latexProblemContent{
\begin{problem}

Use the Fundamental Theorem of Calculus to find the derivative of the function.
\[g(t)=\int_{1}^{t} {\frac{1}{{\left(x + 4\right)}^{2}}}\;dx\]

\expandafter\input{\file@loc Integrals/2311-Compute-Integral-0006.HELP.tex}

\[\dfrac{d}{dt}(g(t))=\answer{\frac{1}{{\left(t + 4\right)}^{2}}}\]
\end{problem}}%}

%%%%%%%%%%%%%%%%%%%%%%



\latexProblemContent{
\begin{problem}

Use the Fundamental Theorem of Calculus to find the derivative of the function.
\[g(t)=\int_{4}^{t} {{\left(x + 8\right)}^{2}}\;dx\]

\expandafter\input{\file@loc Integrals/2311-Compute-Integral-0006.HELP.tex}

\[\dfrac{d}{dt}(g(t))=\answer{{\left(t + 8\right)}^{2}}\]
\end{problem}}%}

%%%%%%%%%%%%%%%%%%%%%%



\latexProblemContent{
\begin{problem}

Use the Fundamental Theorem of Calculus to find the derivative of the function.
\[g(t)=\int_{1}^{t} {x - 2}\;dx\]

\expandafter\input{\file@loc Integrals/2311-Compute-Integral-0006.HELP.tex}

\[\dfrac{d}{dt}(g(t))=\answer{t - 2}\]
\end{problem}}%}

%%%%%%%%%%%%%%%%%%%%%%



\latexProblemContent{
\begin{problem}

Use the Fundamental Theorem of Calculus to find the derivative of the function.
\[g(t)=\int_{2}^{t} {\frac{1}{{\left(x - 1\right)}^{2}}}\;dx\]

\expandafter\input{\file@loc Integrals/2311-Compute-Integral-0006.HELP.tex}

\[\dfrac{d}{dt}(g(t))=\answer{\frac{1}{{\left(t - 1\right)}^{2}}}\]
\end{problem}}%}

%%%%%%%%%%%%%%%%%%%%%%



%%%%%%%%%%%%%%%%%%%%%%



%%%%%%%%%%%%%%%%%%%%%%



\latexProblemContent{
\begin{problem}

Use the Fundamental Theorem of Calculus to find the derivative of the function.
\[g(t)=\int_{3}^{t} {\sin\left(x - 1\right)}\;dx\]

\expandafter\input{\file@loc Integrals/2311-Compute-Integral-0006.HELP.tex}

\[\dfrac{d}{dt}(g(t))=\answer{\sin\left(t - 1\right)}\]
\end{problem}}%}

%%%%%%%%%%%%%%%%%%%%%%



%%%%%%%%%%%%%%%%%%%%%%



%%%%%%%%%%%%%%%%%%%%%%



%%%%%%%%%%%%%%%%%%%%%%



\latexProblemContent{
\begin{problem}

Use the Fundamental Theorem of Calculus to find the derivative of the function.
\[g(t)=\int_{1}^{t} {\log\left(x - 4\right)}\;dx\]

\expandafter\input{\file@loc Integrals/2311-Compute-Integral-0006.HELP.tex}

\[\dfrac{d}{dt}(g(t))=\answer{\log\left(t - 4\right)}\]
\end{problem}}%}

%%%%%%%%%%%%%%%%%%%%%%



\latexProblemContent{
\begin{problem}

Use the Fundamental Theorem of Calculus to find the derivative of the function.
\[g(t)=\int_{4}^{t} {\frac{1}{{\left(x + 9\right)}^{2}}}\;dx\]

\expandafter\input{\file@loc Integrals/2311-Compute-Integral-0006.HELP.tex}

\[\dfrac{d}{dt}(g(t))=\answer{\frac{1}{{\left(t + 9\right)}^{2}}}\]
\end{problem}}%}

%%%%%%%%%%%%%%%%%%%%%%



%%%%%%%%%%%%%%%%%%%%%%



\latexProblemContent{
\begin{problem}

Use the Fundamental Theorem of Calculus to find the derivative of the function.
\[g(t)=\int_{3}^{t} {{\left(x + 4\right)}^{4}}\;dx\]

\expandafter\input{\file@loc Integrals/2311-Compute-Integral-0006.HELP.tex}

\[\dfrac{d}{dt}(g(t))=\answer{{\left(t + 4\right)}^{4}}\]
\end{problem}}%}

%%%%%%%%%%%%%%%%%%%%%%



\latexProblemContent{
\begin{problem}

Use the Fundamental Theorem of Calculus to find the derivative of the function.
\[g(t)=\int_{4}^{t} {\frac{1}{{\left(x + 10\right)}^{2}}}\;dx\]

\expandafter\input{\file@loc Integrals/2311-Compute-Integral-0006.HELP.tex}

\[\dfrac{d}{dt}(g(t))=\answer{\frac{1}{{\left(t + 10\right)}^{2}}}\]
\end{problem}}%}

%%%%%%%%%%%%%%%%%%%%%%



\latexProblemContent{
\begin{problem}

Use the Fundamental Theorem of Calculus to find the derivative of the function.
\[g(t)=\int_{4}^{t} {\sqrt{x + 7}}\;dx\]

\expandafter\input{\file@loc Integrals/2311-Compute-Integral-0006.HELP.tex}

\[\dfrac{d}{dt}(g(t))=\answer{\sqrt{t + 7}}\]
\end{problem}}%}

%%%%%%%%%%%%%%%%%%%%%%



\latexProblemContent{
\begin{problem}

Use the Fundamental Theorem of Calculus to find the derivative of the function.
\[g(t)=\int_{4}^{t} {{\left(x - 9\right)}^{4}}\;dx\]

\expandafter\input{\file@loc Integrals/2311-Compute-Integral-0006.HELP.tex}

\[\dfrac{d}{dt}(g(t))=\answer{{\left(t - 9\right)}^{4}}\]
\end{problem}}%}

%%%%%%%%%%%%%%%%%%%%%%



\latexProblemContent{
\begin{problem}

Use the Fundamental Theorem of Calculus to find the derivative of the function.
\[g(t)=\int_{1}^{t} {\sin\left(x - 5\right)}\;dx\]

\expandafter\input{\file@loc Integrals/2311-Compute-Integral-0006.HELP.tex}

\[\dfrac{d}{dt}(g(t))=\answer{\sin\left(t - 5\right)}\]
\end{problem}}%}

%%%%%%%%%%%%%%%%%%%%%%



\latexProblemContent{
\begin{problem}

Use the Fundamental Theorem of Calculus to find the derivative of the function.
\[g(t)=\int_{1}^{t} {e^{\left(x - 2\right)}}\;dx\]

\expandafter\input{\file@loc Integrals/2311-Compute-Integral-0006.HELP.tex}

\[\dfrac{d}{dt}(g(t))=\answer{e^{\left(t - 2\right)}}\]
\end{problem}}%}

%%%%%%%%%%%%%%%%%%%%%%



%%%%%%%%%%%%%%%%%%%%%%



%%%%%%%%%%%%%%%%%%%%%%



\latexProblemContent{
\begin{problem}

Use the Fundamental Theorem of Calculus to find the derivative of the function.
\[g(t)=\int_{4}^{t} {\frac{1}{{\left(x + 2\right)}^{2}}}\;dx\]

\expandafter\input{\file@loc Integrals/2311-Compute-Integral-0006.HELP.tex}

\[\dfrac{d}{dt}(g(t))=\answer{\frac{1}{{\left(t + 2\right)}^{2}}}\]
\end{problem}}%}

%%%%%%%%%%%%%%%%%%%%%%



%%%%%%%%%%%%%%%%%%%%%%



%%%%%%%%%%%%%%%%%%%%%%



\latexProblemContent{
\begin{problem}

Use the Fundamental Theorem of Calculus to find the derivative of the function.
\[g(t)=\int_{1}^{t} {\frac{1}{x - 1}}\;dx\]

\expandafter\input{\file@loc Integrals/2311-Compute-Integral-0006.HELP.tex}

\[\dfrac{d}{dt}(g(t))=\answer{\frac{1}{t - 1}}\]
\end{problem}}%}

%%%%%%%%%%%%%%%%%%%%%%



\latexProblemContent{
\begin{problem}

Use the Fundamental Theorem of Calculus to find the derivative of the function.
\[g(t)=\int_{5}^{t} {{\left(x + 5\right)}^{2}}\;dx\]

\expandafter\input{\file@loc Integrals/2311-Compute-Integral-0006.HELP.tex}

\[\dfrac{d}{dt}(g(t))=\answer{{\left(t + 5\right)}^{2}}\]
\end{problem}}%}

%%%%%%%%%%%%%%%%%%%%%%



\latexProblemContent{
\begin{problem}

Use the Fundamental Theorem of Calculus to find the derivative of the function.
\[g(t)=\int_{2}^{t} {\frac{1}{{\left(x + 10\right)}^{3}}}\;dx\]

\expandafter\input{\file@loc Integrals/2311-Compute-Integral-0006.HELP.tex}

\[\dfrac{d}{dt}(g(t))=\answer{\frac{1}{{\left(t + 10\right)}^{3}}}\]
\end{problem}}%}

%%%%%%%%%%%%%%%%%%%%%%



\latexProblemContent{
\begin{problem}

Use the Fundamental Theorem of Calculus to find the derivative of the function.
\[g(t)=\int_{2}^{t} {{\left(x - 10\right)}^{2}}\;dx\]

\expandafter\input{\file@loc Integrals/2311-Compute-Integral-0006.HELP.tex}

\[\dfrac{d}{dt}(g(t))=\answer{{\left(t - 10\right)}^{2}}\]
\end{problem}}%}

%%%%%%%%%%%%%%%%%%%%%%



\latexProblemContent{
\begin{problem}

Use the Fundamental Theorem of Calculus to find the derivative of the function.
\[g(t)=\int_{2}^{t} {{\left(x - 2\right)}^{3}}\;dx\]

\expandafter\input{\file@loc Integrals/2311-Compute-Integral-0006.HELP.tex}

\[\dfrac{d}{dt}(g(t))=\answer{{\left(t - 2\right)}^{3}}\]
\end{problem}}%}

%%%%%%%%%%%%%%%%%%%%%%



\latexProblemContent{
\begin{problem}

Use the Fundamental Theorem of Calculus to find the derivative of the function.
\[g(t)=\int_{1}^{t} {{\left(x + 6\right)}^{3}}\;dx\]

\expandafter\input{\file@loc Integrals/2311-Compute-Integral-0006.HELP.tex}

\[\dfrac{d}{dt}(g(t))=\answer{{\left(t + 6\right)}^{3}}\]
\end{problem}}%}

%%%%%%%%%%%%%%%%%%%%%%



%%%%%%%%%%%%%%%%%%%%%%



%%%%%%%%%%%%%%%%%%%%%%



%%%%%%%%%%%%%%%%%%%%%%



\latexProblemContent{
\begin{problem}

Use the Fundamental Theorem of Calculus to find the derivative of the function.
\[g(t)=\int_{5}^{t} {{\left(x + 10\right)}^{3}}\;dx\]

\expandafter\input{\file@loc Integrals/2311-Compute-Integral-0006.HELP.tex}

\[\dfrac{d}{dt}(g(t))=\answer{{\left(t + 10\right)}^{3}}\]
\end{problem}}%}

%%%%%%%%%%%%%%%%%%%%%%



%%%%%%%%%%%%%%%%%%%%%%



\latexProblemContent{
\begin{problem}

Use the Fundamental Theorem of Calculus to find the derivative of the function.
\[g(t)=\int_{4}^{t} {\frac{1}{{\left(x - 9\right)}^{3}}}\;dx\]

\expandafter\input{\file@loc Integrals/2311-Compute-Integral-0006.HELP.tex}

\[\dfrac{d}{dt}(g(t))=\answer{\frac{1}{{\left(t - 9\right)}^{3}}}\]
\end{problem}}%}

%%%%%%%%%%%%%%%%%%%%%%



%%%%%%%%%%%%%%%%%%%%%%



%%%%%%%%%%%%%%%%%%%%%%



%%%%%%%%%%%%%%%%%%%%%%



\latexProblemContent{
\begin{problem}

Use the Fundamental Theorem of Calculus to find the derivative of the function.
\[g(t)=\int_{3}^{t} {{\left(x - 9\right)}^{4}}\;dx\]

\expandafter\input{\file@loc Integrals/2311-Compute-Integral-0006.HELP.tex}

\[\dfrac{d}{dt}(g(t))=\answer{{\left(t - 9\right)}^{4}}\]
\end{problem}}%}

%%%%%%%%%%%%%%%%%%%%%%



%%%%%%%%%%%%%%%%%%%%%%



%%%%%%%%%%%%%%%%%%%%%%



\latexProblemContent{
\begin{problem}

Use the Fundamental Theorem of Calculus to find the derivative of the function.
\[g(t)=\int_{2}^{t} {\sqrt{x + 8}}\;dx\]

\expandafter\input{\file@loc Integrals/2311-Compute-Integral-0006.HELP.tex}

\[\dfrac{d}{dt}(g(t))=\answer{\sqrt{t + 8}}\]
\end{problem}}%}

%%%%%%%%%%%%%%%%%%%%%%



\latexProblemContent{
\begin{problem}

Use the Fundamental Theorem of Calculus to find the derivative of the function.
\[g(t)=\int_{5}^{t} {\frac{1}{{\left(x + 5\right)}^{2}}}\;dx\]

\expandafter\input{\file@loc Integrals/2311-Compute-Integral-0006.HELP.tex}

\[\dfrac{d}{dt}(g(t))=\answer{\frac{1}{{\left(t + 5\right)}^{2}}}\]
\end{problem}}%}

%%%%%%%%%%%%%%%%%%%%%%



%%%%%%%%%%%%%%%%%%%%%%



\latexProblemContent{
\begin{problem}

Use the Fundamental Theorem of Calculus to find the derivative of the function.
\[g(t)=\int_{3}^{t} {\frac{1}{{\left(x - 1\right)}^{3}}}\;dx\]

\expandafter\input{\file@loc Integrals/2311-Compute-Integral-0006.HELP.tex}

\[\dfrac{d}{dt}(g(t))=\answer{\frac{1}{{\left(t - 1\right)}^{3}}}\]
\end{problem}}%}

%%%%%%%%%%%%%%%%%%%%%%



\latexProblemContent{
\begin{problem}

Use the Fundamental Theorem of Calculus to find the derivative of the function.
\[g(t)=\int_{3}^{t} {\sin\left(x + 5\right)}\;dx\]

\expandafter\input{\file@loc Integrals/2311-Compute-Integral-0006.HELP.tex}

\[\dfrac{d}{dt}(g(t))=\answer{\sin\left(t + 5\right)}\]
\end{problem}}%}

%%%%%%%%%%%%%%%%%%%%%%



%%%%%%%%%%%%%%%%%%%%%%



%%%%%%%%%%%%%%%%%%%%%%



%%%%%%%%%%%%%%%%%%%%%%



\latexProblemContent{
\begin{problem}

Use the Fundamental Theorem of Calculus to find the derivative of the function.
\[g(t)=\int_{3}^{t} {{\left(x - 10\right)}^{4}}\;dx\]

\expandafter\input{\file@loc Integrals/2311-Compute-Integral-0006.HELP.tex}

\[\dfrac{d}{dt}(g(t))=\answer{{\left(t - 10\right)}^{4}}\]
\end{problem}}%}

%%%%%%%%%%%%%%%%%%%%%%



\latexProblemContent{
\begin{problem}

Use the Fundamental Theorem of Calculus to find the derivative of the function.
\[g(t)=\int_{3}^{t} {{\left(x - 2\right)}^{4}}\;dx\]

\expandafter\input{\file@loc Integrals/2311-Compute-Integral-0006.HELP.tex}

\[\dfrac{d}{dt}(g(t))=\answer{{\left(t - 2\right)}^{4}}\]
\end{problem}}%}

%%%%%%%%%%%%%%%%%%%%%%



\latexProblemContent{
\begin{problem}

Use the Fundamental Theorem of Calculus to find the derivative of the function.
\[g(t)=\int_{2}^{t} {{\left(x - 8\right)}^{4}}\;dx\]

\expandafter\input{\file@loc Integrals/2311-Compute-Integral-0006.HELP.tex}

\[\dfrac{d}{dt}(g(t))=\answer{{\left(t - 8\right)}^{4}}\]
\end{problem}}%}

%%%%%%%%%%%%%%%%%%%%%%



\latexProblemContent{
\begin{problem}

Use the Fundamental Theorem of Calculus to find the derivative of the function.
\[g(t)=\int_{2}^{t} {\log\left(x - 2\right)}\;dx\]

\expandafter\input{\file@loc Integrals/2311-Compute-Integral-0006.HELP.tex}

\[\dfrac{d}{dt}(g(t))=\answer{\log\left(t - 2\right)}\]
\end{problem}}%}

%%%%%%%%%%%%%%%%%%%%%%



%%%%%%%%%%%%%%%%%%%%%%



\latexProblemContent{
\begin{problem}

Use the Fundamental Theorem of Calculus to find the derivative of the function.
\[g(t)=\int_{1}^{t} {\cos\left(x - 2\right)}\;dx\]

\expandafter\input{\file@loc Integrals/2311-Compute-Integral-0006.HELP.tex}

\[\dfrac{d}{dt}(g(t))=\answer{\cos\left(t - 2\right)}\]
\end{problem}}%}

%%%%%%%%%%%%%%%%%%%%%%



\latexProblemContent{
\begin{problem}

Use the Fundamental Theorem of Calculus to find the derivative of the function.
\[g(t)=\int_{1}^{t} {{\left(x - 5\right)}^{4}}\;dx\]

\expandafter\input{\file@loc Integrals/2311-Compute-Integral-0006.HELP.tex}

\[\dfrac{d}{dt}(g(t))=\answer{{\left(t - 5\right)}^{4}}\]
\end{problem}}%}

%%%%%%%%%%%%%%%%%%%%%%



\latexProblemContent{
\begin{problem}

Use the Fundamental Theorem of Calculus to find the derivative of the function.
\[g(t)=\int_{5}^{t} {x - 4}\;dx\]

\expandafter\input{\file@loc Integrals/2311-Compute-Integral-0006.HELP.tex}

\[\dfrac{d}{dt}(g(t))=\answer{t - 4}\]
\end{problem}}%}

%%%%%%%%%%%%%%%%%%%%%%



\latexProblemContent{
\begin{problem}

Use the Fundamental Theorem of Calculus to find the derivative of the function.
\[g(t)=\int_{5}^{t} {\cos\left(x - 8\right)}\;dx\]

\expandafter\input{\file@loc Integrals/2311-Compute-Integral-0006.HELP.tex}

\[\dfrac{d}{dt}(g(t))=\answer{\cos\left(t - 8\right)}\]
\end{problem}}%}

%%%%%%%%%%%%%%%%%%%%%%



\latexProblemContent{
\begin{problem}

Use the Fundamental Theorem of Calculus to find the derivative of the function.
\[g(t)=\int_{1}^{t} {x - 9}\;dx\]

\expandafter\input{\file@loc Integrals/2311-Compute-Integral-0006.HELP.tex}

\[\dfrac{d}{dt}(g(t))=\answer{t - 9}\]
\end{problem}}%}

%%%%%%%%%%%%%%%%%%%%%%



%%%%%%%%%%%%%%%%%%%%%%



%%%%%%%%%%%%%%%%%%%%%%



\latexProblemContent{
\begin{problem}

Use the Fundamental Theorem of Calculus to find the derivative of the function.
\[g(t)=\int_{3}^{t} {\sqrt{x + 4}}\;dx\]

\expandafter\input{\file@loc Integrals/2311-Compute-Integral-0006.HELP.tex}

\[\dfrac{d}{dt}(g(t))=\answer{\sqrt{t + 4}}\]
\end{problem}}%}

%%%%%%%%%%%%%%%%%%%%%%



%%%%%%%%%%%%%%%%%%%%%%


