%%%%%%%%%%%%%%%%%%%%%%%
%%\tagged{Cat@One, Cat@Two, Cat@Three, Cat@Four, Cat@Five, Ans@ShortAns, Type@Compute, Topic@Integral, Sub@Definite, Sub@Theorems_FTC, Sub@Poly}{

\latexProblemContent{
\begin{problem}

Use the Fundamental Theorem of Calculus to evaluate the integral.

\expandafter\input{\file@loc Integrals/2311-Compute-Integral-0011.HELP.tex}

\[\int_{1}^{13} {\frac{x^{2} + 18 \, x + 80}{6 \, \sqrt{x}}}\;dx=\answer{\frac{959}{15} \, \sqrt{13} - \frac{431}{15}}\]
\end{problem}}%}

%%%%%%%%%%%%%%%%%%%%%%


\latexProblemContent{
\begin{problem}

Use the Fundamental Theorem of Calculus to evaluate the integral.

\expandafter\input{\file@loc Integrals/2311-Compute-Integral-0011.HELP.tex}

\[\int_{2}^{15} {\frac{x - 9}{5 \, \sqrt{x}}}\;dx=\answer{-\frac{8}{5} \, \sqrt{15} + \frac{10}{3} \, \sqrt{2}}\]
\end{problem}}%}

%%%%%%%%%%%%%%%%%%%%%%


\latexProblemContent{
\begin{problem}

Use the Fundamental Theorem of Calculus to evaluate the integral.

\expandafter\input{\file@loc Integrals/2311-Compute-Integral-0011.HELP.tex}

\[\int_{5}^{13} {-\frac{x - 8}{3 \, \sqrt{x}}}\;dx=\answer{\frac{22}{9} \, \sqrt{13} - \frac{38}{9} \, \sqrt{5}}\]
\end{problem}}%}

%%%%%%%%%%%%%%%%%%%%%%


\latexProblemContent{
\begin{problem}

Use the Fundamental Theorem of Calculus to evaluate the integral.

\expandafter\input{\file@loc Integrals/2311-Compute-Integral-0011.HELP.tex}

\[\int_{7}^{12} {\frac{x - 6}{4 \, \sqrt{x}}}\;dx=\answer{\frac{11}{6} \, \sqrt{7} - 2 \, \sqrt{3}}\]
\end{problem}}%}

%%%%%%%%%%%%%%%%%%%%%%


\latexProblemContent{
\begin{problem}

Use the Fundamental Theorem of Calculus to evaluate the integral.

\expandafter\input{\file@loc Integrals/2311-Compute-Integral-0011.HELP.tex}

\[\int_{6}^{11} {-\frac{x - 5}{2 \, \sqrt{x}}}\;dx=\answer{\frac{4}{3} \, \sqrt{11} - 3 \, \sqrt{6}}\]
\end{problem}}%}

%%%%%%%%%%%%%%%%%%%%%%


\latexProblemContent{
\begin{problem}

Use the Fundamental Theorem of Calculus to evaluate the integral.

\expandafter\input{\file@loc Integrals/2311-Compute-Integral-0011.HELP.tex}

\[\int_{7}^{10} {-\frac{x - 7}{9 \, \sqrt{x}}}\;dx=\answer{\frac{22}{27} \, \sqrt{10} - \frac{28}{27} \, \sqrt{7}}\]
\end{problem}}%}

%%%%%%%%%%%%%%%%%%%%%%


\latexProblemContent{
\begin{problem}

Use the Fundamental Theorem of Calculus to evaluate the integral.

\expandafter\input{\file@loc Integrals/2311-Compute-Integral-0011.HELP.tex}

\[\int_{7}^{10} {\frac{x^{2} - 25}{4 \, \sqrt{x}}}\;dx=\answer{-\frac{5}{2} \, \sqrt{10} + \frac{38}{5} \, \sqrt{7}}\]
\end{problem}}%}

%%%%%%%%%%%%%%%%%%%%%%


\latexProblemContent{
\begin{problem}

Use the Fundamental Theorem of Calculus to evaluate the integral.

\expandafter\input{\file@loc Integrals/2311-Compute-Integral-0011.HELP.tex}

\[\int_{7}^{9} {-\frac{x^{2} - 4}{9 \, \sqrt{x}}}\;dx=\answer{\frac{58}{45} \, \sqrt{7} - \frac{122}{15}}\]
\end{problem}}%}

%%%%%%%%%%%%%%%%%%%%%%


\latexProblemContent{
\begin{problem}

Use the Fundamental Theorem of Calculus to evaluate the integral.

\expandafter\input{\file@loc Integrals/2311-Compute-Integral-0011.HELP.tex}

\[\int_{8}^{14} {-\frac{x^{2} - 4}{9 \, \sqrt{x}}}\;dx=\answer{-\frac{352}{45} \, \sqrt{14} + \frac{176}{45} \, \sqrt{2}}\]
\end{problem}}%}

%%%%%%%%%%%%%%%%%%%%%%


\latexProblemContent{
\begin{problem}

Use the Fundamental Theorem of Calculus to evaluate the integral.

\expandafter\input{\file@loc Integrals/2311-Compute-Integral-0011.HELP.tex}

\[\int_{2}^{13} {\frac{1}{2} \, \sqrt{x}}\;dx=\answer{\frac{13}{3} \, \sqrt{13} - \frac{2}{3} \, \sqrt{2}}\]
\end{problem}}%}

%%%%%%%%%%%%%%%%%%%%%%


\latexProblemContent{
\begin{problem}

Use the Fundamental Theorem of Calculus to evaluate the integral.

\expandafter\input{\file@loc Integrals/2311-Compute-Integral-0011.HELP.tex}

\[\int_{2}^{13} {-\frac{x - 2}{3 \, \sqrt{x}}}\;dx=\answer{-\frac{1}{9} \, \left(8 \, \sqrt{2}\right) - \frac{14}{9} \, \sqrt{13}}\]
\end{problem}}%}

%%%%%%%%%%%%%%%%%%%%%%


\latexProblemContent{
\begin{problem}

Use the Fundamental Theorem of Calculus to evaluate the integral.

\expandafter\input{\file@loc Integrals/2311-Compute-Integral-0011.HELP.tex}

\[\int_{4}^{14} {\frac{x - 4}{2 \, \sqrt{x}}}\;dx=\answer{\frac{2}{3} \, \sqrt{14} + \frac{16}{3}}\]
\end{problem}}%}

%%%%%%%%%%%%%%%%%%%%%%


\latexProblemContent{
\begin{problem}

Use the Fundamental Theorem of Calculus to evaluate the integral.

\expandafter\input{\file@loc Integrals/2311-Compute-Integral-0011.HELP.tex}

\[\int_{4}^{9} {\frac{x^{2} - 5 \, x - 24}{2 \, \sqrt{x}}}\;dx=\answer{-\frac{202}{15}}\]
\end{problem}}%}

%%%%%%%%%%%%%%%%%%%%%%


\latexProblemContent{
\begin{problem}

Use the Fundamental Theorem of Calculus to evaluate the integral.

\expandafter\input{\file@loc Integrals/2311-Compute-Integral-0011.HELP.tex}

\[\int_{4}^{15} {\frac{x + 4}{\sqrt{x}}}\;dx=\answer{18 \, \sqrt{15} - \frac{64}{3}}\]
\end{problem}}%}

%%%%%%%%%%%%%%%%%%%%%%


\latexProblemContent{
\begin{problem}

Use the Fundamental Theorem of Calculus to evaluate the integral.

\expandafter\input{\file@loc Integrals/2311-Compute-Integral-0011.HELP.tex}

\[\int_{1}^{11} {-\frac{x^{2} - 81}{2 \, \sqrt{x}}}\;dx=\answer{\frac{284}{5} \, \sqrt{11} - \frac{404}{5}}\]
\end{problem}}%}

%%%%%%%%%%%%%%%%%%%%%%


\latexProblemContent{
\begin{problem}

Use the Fundamental Theorem of Calculus to evaluate the integral.

\expandafter\input{\file@loc Integrals/2311-Compute-Integral-0011.HELP.tex}

\[\int_{2}^{8} {-\frac{x^{2} - 100}{\sqrt{x}}}\;dx=\answer{\frac{47}{5} \, \left(16 \, \sqrt{2}\right)}\]
\end{problem}}%}

%%%%%%%%%%%%%%%%%%%%%%


\latexProblemContent{
\begin{problem}

Use the Fundamental Theorem of Calculus to evaluate the integral.

\expandafter\input{\file@loc Integrals/2311-Compute-Integral-0011.HELP.tex}

\[\int_{5}^{11} {-\frac{x^{2} - 9}{4 \, \sqrt{x}}}\;dx=\answer{-\frac{38}{5} \, \sqrt{11} - 2 \, \sqrt{5}}\]
\end{problem}}%}

%%%%%%%%%%%%%%%%%%%%%%


\latexProblemContent{
\begin{problem}

Use the Fundamental Theorem of Calculus to evaluate the integral.

\expandafter\input{\file@loc Integrals/2311-Compute-Integral-0011.HELP.tex}

\[\int_{3}^{14} {-\frac{x^{2} - 9}{9 \, \sqrt{x}}}\;dx=\answer{-\frac{302}{45} \, \sqrt{14} - \frac{8}{5} \, \sqrt{3}}\]
\end{problem}}%}

%%%%%%%%%%%%%%%%%%%%%%


\latexProblemContent{
\begin{problem}

Use the Fundamental Theorem of Calculus to evaluate the integral.

\expandafter\input{\file@loc Integrals/2311-Compute-Integral-0011.HELP.tex}

\[\int_{2}^{10} {-\frac{x^{2} - 16}{7 \, \sqrt{x}}}\;dx=\answer{-\frac{8}{7} \, \sqrt{10} - \frac{152}{35} \, \sqrt{2}}\]
\end{problem}}%}

%%%%%%%%%%%%%%%%%%%%%%


\latexProblemContent{
\begin{problem}

Use the Fundamental Theorem of Calculus to evaluate the integral.

\expandafter\input{\file@loc Integrals/2311-Compute-Integral-0011.HELP.tex}

\[\int_{1}^{15} {-\frac{x^{2} - 49}{7 \, \sqrt{x}}}\;dx=\answer{\frac{8}{7} \, \sqrt{15} - \frac{488}{35}}\]
\end{problem}}%}

%%%%%%%%%%%%%%%%%%%%%%


\latexProblemContent{
\begin{problem}

Use the Fundamental Theorem of Calculus to evaluate the integral.

\expandafter\input{\file@loc Integrals/2311-Compute-Integral-0011.HELP.tex}

\[\int_{3}^{13} {\frac{x^{2} - 5 \, x - 6}{2 \, \sqrt{x}}}\;dx=\answer{\frac{92}{15} \, \sqrt{13} + \frac{46}{5} \, \sqrt{3}}\]
\end{problem}}%}

%%%%%%%%%%%%%%%%%%%%%%


\latexProblemContent{
\begin{problem}

Use the Fundamental Theorem of Calculus to evaluate the integral.

\expandafter\input{\file@loc Integrals/2311-Compute-Integral-0011.HELP.tex}

\[\int_{3}^{9} {-\frac{x - 9}{\sqrt{x}}}\;dx=\answer{-16 \, \sqrt{3} + 36}\]
\end{problem}}%}

%%%%%%%%%%%%%%%%%%%%%%


\latexProblemContent{
\begin{problem}

Use the Fundamental Theorem of Calculus to evaluate the integral.

\expandafter\input{\file@loc Integrals/2311-Compute-Integral-0011.HELP.tex}

\[\int_{8}^{9} {\frac{x^{2} - 16}{2 \, \sqrt{x}}}\;dx=\answer{\frac{1}{10} \, \left(64 \, \sqrt{2}\right) + \frac{3}{5}}\]
\end{problem}}%}

%%%%%%%%%%%%%%%%%%%%%%


\latexProblemContent{
\begin{problem}

Use the Fundamental Theorem of Calculus to evaluate the integral.

\expandafter\input{\file@loc Integrals/2311-Compute-Integral-0011.HELP.tex}

\[\int_{8}^{12} {-\frac{x^{2} - 1}{5 \, \sqrt{x}}}\;dx=\answer{-\frac{556}{25} \, \sqrt{3} + \frac{236}{25} \, \sqrt{2}}\]
\end{problem}}%}

%%%%%%%%%%%%%%%%%%%%%%


\latexProblemContent{
\begin{problem}

Use the Fundamental Theorem of Calculus to evaluate the integral.

\expandafter\input{\file@loc Integrals/2311-Compute-Integral-0011.HELP.tex}

\[\int_{4}^{11} {-\frac{x + 4}{8 \, \sqrt{x}}}\;dx=\answer{-\frac{23}{12} \, \sqrt{11} + \frac{8}{3}}\]
\end{problem}}%}

%%%%%%%%%%%%%%%%%%%%%%


\latexProblemContent{
\begin{problem}

Use the Fundamental Theorem of Calculus to evaluate the integral.

\expandafter\input{\file@loc Integrals/2311-Compute-Integral-0011.HELP.tex}

\[\int_{6}^{13} {\frac{x^{2} - 81}{10 \, \sqrt{x}}}\;dx=\answer{-\frac{236}{25} \, \sqrt{13} + \frac{369}{25} \, \sqrt{6}}\]
\end{problem}}%}

%%%%%%%%%%%%%%%%%%%%%%


\latexProblemContent{
\begin{problem}

Use the Fundamental Theorem of Calculus to evaluate the integral.

\expandafter\input{\file@loc Integrals/2311-Compute-Integral-0011.HELP.tex}

\[\int_{2}^{14} {-\frac{x^{2} + 11 \, x + 10}{4 \, \sqrt{x}}}\;dx=\answer{-\frac{754}{15} \, \sqrt{14} + \frac{136}{15} \, \sqrt{2}}\]
\end{problem}}%}

%%%%%%%%%%%%%%%%%%%%%%


\latexProblemContent{
\begin{problem}

Use the Fundamental Theorem of Calculus to evaluate the integral.

\expandafter\input{\file@loc Integrals/2311-Compute-Integral-0011.HELP.tex}

\[\int_{6}^{11} {\frac{x + 3}{5 \, \sqrt{x}}}\;dx=\answer{\frac{8}{3} \, \sqrt{11} - 2 \, \sqrt{6}}\]
\end{problem}}%}

%%%%%%%%%%%%%%%%%%%%%%


\latexProblemContent{
\begin{problem}

Use the Fundamental Theorem of Calculus to evaluate the integral.

\expandafter\input{\file@loc Integrals/2311-Compute-Integral-0011.HELP.tex}

\[\int_{1}^{3} {-\frac{x^{2} - 49}{6 \, \sqrt{x}}}\;dx=\answer{\frac{236}{15} \, \sqrt{3} - \frac{244}{15}}\]
\end{problem}}%}

%%%%%%%%%%%%%%%%%%%%%%


\latexProblemContent{
\begin{problem}

Use the Fundamental Theorem of Calculus to evaluate the integral.

\expandafter\input{\file@loc Integrals/2311-Compute-Integral-0011.HELP.tex}

\[\int_{4}^{11} {-\frac{x^{2} - 49}{5 \, \sqrt{x}}}\;dx=\answer{\frac{248}{25} \, \sqrt{11} - \frac{916}{25}}\]
\end{problem}}%}

%%%%%%%%%%%%%%%%%%%%%%


\latexProblemContent{
\begin{problem}

Use the Fundamental Theorem of Calculus to evaluate the integral.

\expandafter\input{\file@loc Integrals/2311-Compute-Integral-0011.HELP.tex}

\[\int_{7}^{13} {-\frac{x - 5}{5 \, \sqrt{x}}}\;dx=\answer{\frac{4}{15} \, \sqrt{13} - \frac{16}{15} \, \sqrt{7}}\]
\end{problem}}%}

%%%%%%%%%%%%%%%%%%%%%%


\latexProblemContent{
\begin{problem}

Use the Fundamental Theorem of Calculus to evaluate the integral.

\expandafter\input{\file@loc Integrals/2311-Compute-Integral-0011.HELP.tex}

\[\int_{5}^{15} {\frac{x^{2} - 4}{5 \, \sqrt{x}}}\;dx=\answer{\frac{82}{5} \, \sqrt{15} - \frac{2}{5} \, \sqrt{5}}\]
\end{problem}}%}

%%%%%%%%%%%%%%%%%%%%%%


\latexProblemContent{
\begin{problem}

Use the Fundamental Theorem of Calculus to evaluate the integral.

\expandafter\input{\file@loc Integrals/2311-Compute-Integral-0011.HELP.tex}

\[\int_{8}^{9} {-\frac{x^{2} - 10 \, x}{\sqrt{x}}}\;dx=\answer{-\frac{832}{15} \, \sqrt{2} + \frac{414}{5}}\]
\end{problem}}%}

%%%%%%%%%%%%%%%%%%%%%%


\latexProblemContent{
\begin{problem}

Use the Fundamental Theorem of Calculus to evaluate the integral.

\expandafter\input{\file@loc Integrals/2311-Compute-Integral-0011.HELP.tex}

\[\int_{4}^{13} {\frac{x^{2} - 6 \, x - 27}{9 \, \sqrt{x}}}\;dx=\answer{-\frac{64}{15} \, \sqrt{13} + \frac{212}{15}}\]
\end{problem}}%}

%%%%%%%%%%%%%%%%%%%%%%


\latexProblemContent{
\begin{problem}

Use the Fundamental Theorem of Calculus to evaluate the integral.

\expandafter\input{\file@loc Integrals/2311-Compute-Integral-0011.HELP.tex}

\[\int_{3}^{11} {\frac{x^{2} + 12 \, x + 20}{5 \, \sqrt{x}}}\;dx=\answer{\frac{882}{25} \, \sqrt{11} - \frac{338}{25} \, \sqrt{3}}\]
\end{problem}}%}

%%%%%%%%%%%%%%%%%%%%%%


\latexProblemContent{
\begin{problem}

Use the Fundamental Theorem of Calculus to evaluate the integral.

\expandafter\input{\file@loc Integrals/2311-Compute-Integral-0011.HELP.tex}

\[\int_{2}^{7} {\frac{x^{2} - 2 \, x - 8}{\sqrt{x}}}\;dx=\answer{-\frac{86}{15} \, \sqrt{7} + \frac{256}{15} \, \sqrt{2}}\]
\end{problem}}%}

%%%%%%%%%%%%%%%%%%%%%%


\latexProblemContent{
\begin{problem}

Use the Fundamental Theorem of Calculus to evaluate the integral.

\expandafter\input{\file@loc Integrals/2311-Compute-Integral-0011.HELP.tex}

\[\int_{1}^{5} {-\frac{x^{2} - 36}{10 \, \sqrt{x}}}\;dx=\answer{\frac{31}{5} \, \sqrt{5} - \frac{179}{25}}\]
\end{problem}}%}

%%%%%%%%%%%%%%%%%%%%%%


\latexProblemContent{
\begin{problem}

Use the Fundamental Theorem of Calculus to evaluate the integral.

\expandafter\input{\file@loc Integrals/2311-Compute-Integral-0011.HELP.tex}

\[\int_{7}^{13} {-\frac{x^{2} + 4 \, x - 12}{10 \, \sqrt{x}}}\;dx=\answer{-\frac{587}{75} \, \sqrt{13} + \frac{107}{75} \, \sqrt{7}}\]
\end{problem}}%}

%%%%%%%%%%%%%%%%%%%%%%


\latexProblemContent{
\begin{problem}

Use the Fundamental Theorem of Calculus to evaluate the integral.

\expandafter\input{\file@loc Integrals/2311-Compute-Integral-0011.HELP.tex}

\[\int_{1}^{11} {\frac{x^{2} - 3 \, x - 40}{4 \, \sqrt{x}}}\;dx=\answer{-\frac{67}{5} \, \sqrt{11} + \frac{102}{5}}\]
\end{problem}}%}

%%%%%%%%%%%%%%%%%%%%%%


\latexProblemContent{
\begin{problem}

Use the Fundamental Theorem of Calculus to evaluate the integral.

\expandafter\input{\file@loc Integrals/2311-Compute-Integral-0011.HELP.tex}

\[\int_{6}^{13} {\frac{x^{2} + 3 \, x}{10 \, \sqrt{x}}}\;dx=\answer{\frac{234}{25} \, \sqrt{13} - \frac{66}{25} \, \sqrt{6}}\]
\end{problem}}%}

%%%%%%%%%%%%%%%%%%%%%%


\latexProblemContent{
\begin{problem}

Use the Fundamental Theorem of Calculus to evaluate the integral.

\expandafter\input{\file@loc Integrals/2311-Compute-Integral-0011.HELP.tex}

\[\int_{5}^{6} {-\frac{x^{2} - 64}{9 \, \sqrt{x}}}\;dx=\answer{\frac{568}{45} \, \sqrt{6} - \frac{118}{9} \, \sqrt{5}}\]
\end{problem}}%}

%%%%%%%%%%%%%%%%%%%%%%


\latexProblemContent{
\begin{problem}

Use the Fundamental Theorem of Calculus to evaluate the integral.

\expandafter\input{\file@loc Integrals/2311-Compute-Integral-0011.HELP.tex}

\[\int_{8}^{13} {\frac{x^{2} + 2 \, x - 3}{5 \, \sqrt{x}}}\;dx=\answer{\frac{1184}{75} \, \sqrt{13} - \frac{908}{75} \, \sqrt{2}}\]
\end{problem}}%}

%%%%%%%%%%%%%%%%%%%%%%


\latexProblemContent{
\begin{problem}

Use the Fundamental Theorem of Calculus to evaluate the integral.

\expandafter\input{\file@loc Integrals/2311-Compute-Integral-0011.HELP.tex}

\[\int_{7}^{10} {\frac{x^{2} - 64}{8 \, \sqrt{x}}}\;dx=\answer{-11 \, \sqrt{10} + \frac{271}{20} \, \sqrt{7}}\]
\end{problem}}%}

%%%%%%%%%%%%%%%%%%%%%%


\latexProblemContent{
\begin{problem}

Use the Fundamental Theorem of Calculus to evaluate the integral.

\expandafter\input{\file@loc Integrals/2311-Compute-Integral-0011.HELP.tex}

\[\int_{6}^{13} {\frac{x^{2} - 6 \, x + 9}{8 \, \sqrt{x}}}\;dx=\answer{\frac{21}{5} \, \sqrt{13} - \frac{21}{20} \, \sqrt{6}}\]
\end{problem}}%}

%%%%%%%%%%%%%%%%%%%%%%


\latexProblemContent{
\begin{problem}

Use the Fundamental Theorem of Calculus to evaluate the integral.

\expandafter\input{\file@loc Integrals/2311-Compute-Integral-0011.HELP.tex}

\[\int_{4}^{15} {\frac{x^{2} + x - 2}{6 \, \sqrt{x}}}\;dx=\answer{16 \, \sqrt{15} - \frac{76}{45}}\]
\end{problem}}%}

%%%%%%%%%%%%%%%%%%%%%%


\latexProblemContent{
\begin{problem}

Use the Fundamental Theorem of Calculus to evaluate the integral.

\expandafter\input{\file@loc Integrals/2311-Compute-Integral-0011.HELP.tex}

\[\int_{5}^{14} {\frac{x^{2} + 4 \, x - 5}{2 \, \sqrt{x}}}\;dx=\answer{\frac{793}{15} \, \sqrt{14} - \frac{20}{3} \, \sqrt{5}}\]
\end{problem}}%}

%%%%%%%%%%%%%%%%%%%%%%


\latexProblemContent{
\begin{problem}

Use the Fundamental Theorem of Calculus to evaluate the integral.

\expandafter\input{\file@loc Integrals/2311-Compute-Integral-0011.HELP.tex}

\[\int_{8}^{13} {\frac{x^{2} - 9}{5 \, \sqrt{x}}}\;dx=\answer{\frac{248}{25} \, \sqrt{13} - \frac{76}{25} \, \sqrt{2}}\]
\end{problem}}%}

%%%%%%%%%%%%%%%%%%%%%%


\latexProblemContent{
\begin{problem}

Use the Fundamental Theorem of Calculus to evaluate the integral.

\expandafter\input{\file@loc Integrals/2311-Compute-Integral-0011.HELP.tex}

\[\int_{4}^{5} {\frac{x^{2} - 16}{7 \, \sqrt{x}}}\;dx=\answer{-\frac{22}{7} \, \sqrt{5} + \frac{256}{35}}\]
\end{problem}}%}

%%%%%%%%%%%%%%%%%%%%%%


\latexProblemContent{
\begin{problem}

Use the Fundamental Theorem of Calculus to evaluate the integral.

\expandafter\input{\file@loc Integrals/2311-Compute-Integral-0011.HELP.tex}

\[\int_{3}^{13} {\frac{x + 4}{8 \, \sqrt{x}}}\;dx=\answer{\frac{25}{12} \, \sqrt{13} - \frac{5}{4} \, \sqrt{3}}\]
\end{problem}}%}

%%%%%%%%%%%%%%%%%%%%%%


\latexProblemContent{
\begin{problem}

Use the Fundamental Theorem of Calculus to evaluate the integral.

\expandafter\input{\file@loc Integrals/2311-Compute-Integral-0011.HELP.tex}

\[\int_{6}^{8} {\frac{x - 1}{9 \, \sqrt{x}}}\;dx=\answer{-\frac{1}{27} \, \left(6 \, \sqrt{6}\right) + \frac{20}{27} \, \sqrt{2}}\]
\end{problem}}%}

%%%%%%%%%%%%%%%%%%%%%%


\latexProblemContent{
\begin{problem}

Use the Fundamental Theorem of Calculus to evaluate the integral.

\expandafter\input{\file@loc Integrals/2311-Compute-Integral-0011.HELP.tex}

\[\int_{2}^{6} {\frac{x - 1}{\sqrt{x}}}\;dx=\answer{\frac{1}{3} \, \left(6 \, \sqrt{6}\right) + \frac{1}{3} \, \left(2 \, \sqrt{2}\right)}\]
\end{problem}}%}

%%%%%%%%%%%%%%%%%%%%%%


\latexProblemContent{
\begin{problem}

Use the Fundamental Theorem of Calculus to evaluate the integral.

\expandafter\input{\file@loc Integrals/2311-Compute-Integral-0011.HELP.tex}

\[\int_{7}^{10} {-\frac{x^{2} - 9}{3 \, \sqrt{x}}}\;dx=\answer{-\frac{22}{3} \, \sqrt{10} + \frac{8}{15} \, \sqrt{7}}\]
\end{problem}}%}

%%%%%%%%%%%%%%%%%%%%%%


\latexProblemContent{
\begin{problem}

Use the Fundamental Theorem of Calculus to evaluate the integral.

\expandafter\input{\file@loc Integrals/2311-Compute-Integral-0011.HELP.tex}

\[\int_{5}^{11} {-\frac{1}{5} \, x^{\frac{3}{2}}}\;dx=\answer{-\frac{242}{25} \, \sqrt{11} + 2 \, \sqrt{5}}\]
\end{problem}}%}

%%%%%%%%%%%%%%%%%%%%%%


\latexProblemContent{
\begin{problem}

Use the Fundamental Theorem of Calculus to evaluate the integral.

\expandafter\input{\file@loc Integrals/2311-Compute-Integral-0011.HELP.tex}

\[\int_{4}^{14} {\frac{x^{2} + 10 \, x + 16}{8 \, \sqrt{x}}}\;dx=\answer{\frac{382}{15} \, \sqrt{14} - \frac{244}{15}}\]
\end{problem}}%}

%%%%%%%%%%%%%%%%%%%%%%


\latexProblemContent{
\begin{problem}

Use the Fundamental Theorem of Calculus to evaluate the integral.

\expandafter\input{\file@loc Integrals/2311-Compute-Integral-0011.HELP.tex}

\[\int_{3}^{8} {\frac{x^{2} - 13 \, x + 30}{3 \, \sqrt{x}}}\;dx=\answer{-\frac{188}{15} \, \sqrt{3} + \frac{488}{45} \, \sqrt{2}}\]
\end{problem}}%}

%%%%%%%%%%%%%%%%%%%%%%


\latexProblemContent{
\begin{problem}

Use the Fundamental Theorem of Calculus to evaluate the integral.

\expandafter\input{\file@loc Integrals/2311-Compute-Integral-0011.HELP.tex}

\[\int_{2}^{9} {\frac{x^{2} + 7 \, x - 18}{2 \, \sqrt{x}}}\;dx=\answer{\frac{188}{15} \, \sqrt{2} + \frac{288}{5}}\]
\end{problem}}%}

%%%%%%%%%%%%%%%%%%%%%%


\latexProblemContent{
\begin{problem}

Use the Fundamental Theorem of Calculus to evaluate the integral.

\expandafter\input{\file@loc Integrals/2311-Compute-Integral-0011.HELP.tex}

\[\int_{3}^{12} {-\frac{x + 8}{\sqrt{x}}}\;dx=\answer{-10 \, \left(3 \, \sqrt{3}\right)}\]
\end{problem}}%}

%%%%%%%%%%%%%%%%%%%%%%


\latexProblemContent{
\begin{problem}

Use the Fundamental Theorem of Calculus to evaluate the integral.

\expandafter\input{\file@loc Integrals/2311-Compute-Integral-0011.HELP.tex}

\[\int_{6}^{8} {-\frac{x^{2} - 100}{7 \, \sqrt{x}}}\;dx=\answer{-\frac{928}{35} \, \sqrt{6} + \frac{1744}{35} \, \sqrt{2}}\]
\end{problem}}%}

%%%%%%%%%%%%%%%%%%%%%%


\latexProblemContent{
\begin{problem}

Use the Fundamental Theorem of Calculus to evaluate the integral.

\expandafter\input{\file@loc Integrals/2311-Compute-Integral-0011.HELP.tex}

\[\int_{5}^{15} {\frac{x^{2} + 6 \, x + 9}{9 \, \sqrt{x}}}\;dx=\answer{\frac{56}{3} \, \sqrt{15} - \frac{16}{3} \, \sqrt{5}}\]
\end{problem}}%}

%%%%%%%%%%%%%%%%%%%%%%


\latexProblemContent{
\begin{problem}

Use the Fundamental Theorem of Calculus to evaluate the integral.

\expandafter\input{\file@loc Integrals/2311-Compute-Integral-0011.HELP.tex}

\[\int_{1}^{6} {\frac{x - 5}{10 \, \sqrt{x}}}\;dx=\answer{-\frac{3}{5} \, \sqrt{6} + \frac{14}{15}}\]
\end{problem}}%}

%%%%%%%%%%%%%%%%%%%%%%


\latexProblemContent{
\begin{problem}

Use the Fundamental Theorem of Calculus to evaluate the integral.

\expandafter\input{\file@loc Integrals/2311-Compute-Integral-0011.HELP.tex}

\[\int_{5}^{9} {\frac{x - 8}{4 \, \sqrt{x}}}\;dx=\answer{\frac{19}{6} \, \sqrt{5} - \frac{15}{2}}\]
\end{problem}}%}

%%%%%%%%%%%%%%%%%%%%%%


\latexProblemContent{
\begin{problem}

Use the Fundamental Theorem of Calculus to evaluate the integral.

\expandafter\input{\file@loc Integrals/2311-Compute-Integral-0011.HELP.tex}

\[\int_{6}^{9} {-\frac{x^{2} - 36}{8 \, \sqrt{x}}}\;dx=\answer{-\frac{36}{5} \, \sqrt{6} + \frac{297}{20}}\]
\end{problem}}%}

%%%%%%%%%%%%%%%%%%%%%%


\latexProblemContent{
\begin{problem}

Use the Fundamental Theorem of Calculus to evaluate the integral.

\expandafter\input{\file@loc Integrals/2311-Compute-Integral-0011.HELP.tex}

\[\int_{6}^{14} {\frac{x + 3}{5 \, \sqrt{x}}}\;dx=\answer{\frac{46}{15} \, \sqrt{14} - 2 \, \sqrt{6}}\]
\end{problem}}%}

%%%%%%%%%%%%%%%%%%%%%%


\latexProblemContent{
\begin{problem}

Use the Fundamental Theorem of Calculus to evaluate the integral.

\expandafter\input{\file@loc Integrals/2311-Compute-Integral-0011.HELP.tex}

\[\int_{4}^{14} {-\frac{x - 10}{6 \, \sqrt{x}}}\;dx=\answer{\frac{16}{9} \, \sqrt{14} - \frac{52}{9}}\]
\end{problem}}%}

%%%%%%%%%%%%%%%%%%%%%%


\latexProblemContent{
\begin{problem}

Use the Fundamental Theorem of Calculus to evaluate the integral.

\expandafter\input{\file@loc Integrals/2311-Compute-Integral-0011.HELP.tex}

\[\int_{4}^{11} {-\frac{x - 10}{8 \, \sqrt{x}}}\;dx=\answer{\frac{19}{12} \, \sqrt{11} - \frac{13}{3}}\]
\end{problem}}%}

%%%%%%%%%%%%%%%%%%%%%%


\latexProblemContent{
\begin{problem}

Use the Fundamental Theorem of Calculus to evaluate the integral.

\expandafter\input{\file@loc Integrals/2311-Compute-Integral-0011.HELP.tex}

\[\int_{8}^{13} {-\frac{x^{2} - 1}{8 \, \sqrt{x}}}\;dx=\answer{-\frac{41}{5} \, \sqrt{13} + \frac{59}{10} \, \sqrt{2}}\]
\end{problem}}%}

%%%%%%%%%%%%%%%%%%%%%%


\latexProblemContent{
\begin{problem}

Use the Fundamental Theorem of Calculus to evaluate the integral.

\expandafter\input{\file@loc Integrals/2311-Compute-Integral-0011.HELP.tex}

\[\int_{1}^{15} {\frac{x^{2} - 64}{2 \, \sqrt{x}}}\;dx=\answer{-19 \, \sqrt{15} + \frac{319}{5}}\]
\end{problem}}%}

%%%%%%%%%%%%%%%%%%%%%%


\latexProblemContent{
\begin{problem}

Use the Fundamental Theorem of Calculus to evaluate the integral.

\expandafter\input{\file@loc Integrals/2311-Compute-Integral-0011.HELP.tex}

\[\int_{3}^{5} {\frac{x^{2} + 10 \, x}{\sqrt{x}}}\;dx=\answer{\frac{130}{3} \, \sqrt{5} - \frac{118}{5} \, \sqrt{3}}\]
\end{problem}}%}

%%%%%%%%%%%%%%%%%%%%%%


\latexProblemContent{
\begin{problem}

Use the Fundamental Theorem of Calculus to evaluate the integral.

\expandafter\input{\file@loc Integrals/2311-Compute-Integral-0011.HELP.tex}

\[\int_{8}^{11} {-\frac{x^{2} - 10 \, x + 25}{2 \, \sqrt{x}}}\;dx=\answer{-\frac{188}{15} \, \sqrt{11} + \frac{334}{15} \, \sqrt{2}}\]
\end{problem}}%}

%%%%%%%%%%%%%%%%%%%%%%


\latexProblemContent{
\begin{problem}

Use the Fundamental Theorem of Calculus to evaluate the integral.

\expandafter\input{\file@loc Integrals/2311-Compute-Integral-0011.HELP.tex}

\[\int_{6}^{9} {\frac{x^{2} - 25}{6 \, \sqrt{x}}}\;dx=\answer{\frac{89}{15} \, \sqrt{6} - \frac{44}{5}}\]
\end{problem}}%}

%%%%%%%%%%%%%%%%%%%%%%


\latexProblemContent{
\begin{problem}

Use the Fundamental Theorem of Calculus to evaluate the integral.

\expandafter\input{\file@loc Integrals/2311-Compute-Integral-0011.HELP.tex}

\[\int_{7}^{8} {-\frac{x + 8}{6 \, \sqrt{x}}}\;dx=\answer{\frac{31}{9} \, \sqrt{7} - \frac{64}{9} \, \sqrt{2}}\]
\end{problem}}%}

%%%%%%%%%%%%%%%%%%%%%%


\latexProblemContent{
\begin{problem}

Use the Fundamental Theorem of Calculus to evaluate the integral.

\expandafter\input{\file@loc Integrals/2311-Compute-Integral-0011.HELP.tex}

\[\int_{7}^{9} {-\frac{x^{2} - 11 \, x + 24}{7 \, \sqrt{x}}}\;dx=\answer{\frac{244}{105} \, \sqrt{7} - \frac{216}{35}}\]
\end{problem}}%}

%%%%%%%%%%%%%%%%%%%%%%


\latexProblemContent{
\begin{problem}

Use the Fundamental Theorem of Calculus to evaluate the integral.

\expandafter\input{\file@loc Integrals/2311-Compute-Integral-0011.HELP.tex}

\[\int_{4}^{9} {-\frac{x^{2} - 5 \, x - 24}{7 \, \sqrt{x}}}\;dx=\answer{\frac{404}{105}}\]
\end{problem}}%}

%%%%%%%%%%%%%%%%%%%%%%


\latexProblemContent{
\begin{problem}

Use the Fundamental Theorem of Calculus to evaluate the integral.

\expandafter\input{\file@loc Integrals/2311-Compute-Integral-0011.HELP.tex}

\[\int_{8}^{13} {-\frac{x + 4}{4 \, \sqrt{x}}}\;dx=\answer{-\frac{25}{6} \, \sqrt{13} + \frac{20}{3} \, \sqrt{2}}\]
\end{problem}}%}

%%%%%%%%%%%%%%%%%%%%%%


\latexProblemContent{
\begin{problem}

Use the Fundamental Theorem of Calculus to evaluate the integral.

\expandafter\input{\file@loc Integrals/2311-Compute-Integral-0011.HELP.tex}

\[\int_{5}^{6} {\frac{x - 2}{2 \, \sqrt{x}}}\;dx=\answer{\frac{1}{3} \, \sqrt{5}}\]
\end{problem}}%}

%%%%%%%%%%%%%%%%%%%%%%


\latexProblemContent{
\begin{problem}

Use the Fundamental Theorem of Calculus to evaluate the integral.

\expandafter\input{\file@loc Integrals/2311-Compute-Integral-0011.HELP.tex}

\[\int_{2}^{3} {\frac{x^{2} + 17 \, x + 70}{4 \, \sqrt{x}}}\;dx=\answer{\frac{222}{5} \, \sqrt{3} - \frac{616}{15} \, \sqrt{2}}\]
\end{problem}}%}

%%%%%%%%%%%%%%%%%%%%%%


\latexProblemContent{
\begin{problem}

Use the Fundamental Theorem of Calculus to evaluate the integral.

\expandafter\input{\file@loc Integrals/2311-Compute-Integral-0011.HELP.tex}

\[\int_{3}^{11} {\frac{x^{2} - 36}{9 \, \sqrt{x}}}\;dx=\answer{-\frac{118}{45} \, \sqrt{11} + \frac{38}{5} \, \sqrt{3}}\]
\end{problem}}%}

%%%%%%%%%%%%%%%%%%%%%%


\latexProblemContent{
\begin{problem}

Use the Fundamental Theorem of Calculus to evaluate the integral.

\expandafter\input{\file@loc Integrals/2311-Compute-Integral-0011.HELP.tex}

\[\int_{8}^{12} {-\frac{x^{2} - 25}{7 \, \sqrt{x}}}\;dx=\answer{-\frac{76}{35} \, \sqrt{3} - \frac{244}{35} \, \sqrt{2}}\]
\end{problem}}%}

%%%%%%%%%%%%%%%%%%%%%%


\latexProblemContent{
\begin{problem}

Use the Fundamental Theorem of Calculus to evaluate the integral.

\expandafter\input{\file@loc Integrals/2311-Compute-Integral-0011.HELP.tex}

\[\int_{1}^{3} {-\frac{x^{2} - 16}{7 \, \sqrt{x}}}\;dx=\answer{\frac{142}{35} \, \sqrt{3} - \frac{158}{35}}\]
\end{problem}}%}

%%%%%%%%%%%%%%%%%%%%%%


\latexProblemContent{
\begin{problem}

Use the Fundamental Theorem of Calculus to evaluate the integral.

\expandafter\input{\file@loc Integrals/2311-Compute-Integral-0011.HELP.tex}

\[\int_{7}^{14} {-\frac{x^{2} - 17 \, x + 72}{2 \, \sqrt{x}}}\;dx=\answer{-\frac{478}{15} \, \sqrt{14} + \frac{632}{15} \, \sqrt{7}}\]
\end{problem}}%}

%%%%%%%%%%%%%%%%%%%%%%


\latexProblemContent{
\begin{problem}

Use the Fundamental Theorem of Calculus to evaluate the integral.

\expandafter\input{\file@loc Integrals/2311-Compute-Integral-0011.HELP.tex}

\[\int_{6}^{13} {-\frac{x^{2} - 4 \, x - 32}{6 \, \sqrt{x}}}\;dx=\answer{\frac{233}{45} \, \sqrt{13} - \frac{164}{15} \, \sqrt{6}}\]
\end{problem}}%}

%%%%%%%%%%%%%%%%%%%%%%


\latexProblemContent{
\begin{problem}

Use the Fundamental Theorem of Calculus to evaluate the integral.

\expandafter\input{\file@loc Integrals/2311-Compute-Integral-0011.HELP.tex}

\[\int_{3}^{6} {\frac{x - 6}{3 \, \sqrt{x}}}\;dx=\answer{-\frac{8}{3} \, \sqrt{6} + \frac{10}{3} \, \sqrt{3}}\]
\end{problem}}%}

%%%%%%%%%%%%%%%%%%%%%%


\latexProblemContent{
\begin{problem}

Use the Fundamental Theorem of Calculus to evaluate the integral.

\expandafter\input{\file@loc Integrals/2311-Compute-Integral-0011.HELP.tex}

\[\int_{3}^{9} {-\frac{1}{8} \, x^{\frac{3}{2}}}\;dx=\answer{\frac{9}{20} \, \sqrt{3} - \frac{243}{20}}\]
\end{problem}}%}

%%%%%%%%%%%%%%%%%%%%%%


\latexProblemContent{
\begin{problem}

Use the Fundamental Theorem of Calculus to evaluate the integral.

\expandafter\input{\file@loc Integrals/2311-Compute-Integral-0011.HELP.tex}

\[\int_{1}^{4} {-\frac{x - 10}{5 \, \sqrt{x}}}\;dx=\answer{\frac{46}{15}}\]
\end{problem}}%}

%%%%%%%%%%%%%%%%%%%%%%


\latexProblemContent{
\begin{problem}

Use the Fundamental Theorem of Calculus to evaluate the integral.

\expandafter\input{\file@loc Integrals/2311-Compute-Integral-0011.HELP.tex}

\[\int_{6}^{13} {-\frac{x^{2} + x - 90}{7 \, \sqrt{x}}}\;dx=\answer{\frac{1556}{105} \, \sqrt{13} - \frac{808}{35} \, \sqrt{6}}\]
\end{problem}}%}

%%%%%%%%%%%%%%%%%%%%%%


\latexProblemContent{
\begin{problem}

Use the Fundamental Theorem of Calculus to evaluate the integral.

\expandafter\input{\file@loc Integrals/2311-Compute-Integral-0011.HELP.tex}

\[\int_{6}^{13} {\frac{x^{2} - 100}{10 \, \sqrt{x}}}\;dx=\answer{-\frac{331}{25} \, \sqrt{13} + \frac{464}{25} \, \sqrt{6}}\]
\end{problem}}%}

%%%%%%%%%%%%%%%%%%%%%%


\latexProblemContent{
\begin{problem}

Use the Fundamental Theorem of Calculus to evaluate the integral.

\expandafter\input{\file@loc Integrals/2311-Compute-Integral-0011.HELP.tex}

\[\int_{8}^{13} {-\frac{x + 1}{4 \, \sqrt{x}}}\;dx=\answer{-\frac{8}{3} \, \sqrt{13} + \frac{11}{3} \, \sqrt{2}}\]
\end{problem}}%}

%%%%%%%%%%%%%%%%%%%%%%


\latexProblemContent{
\begin{problem}

Use the Fundamental Theorem of Calculus to evaluate the integral.

\expandafter\input{\file@loc Integrals/2311-Compute-Integral-0011.HELP.tex}

\[\int_{5}^{6} {-\frac{x^{2} + 8 \, x + 12}{6 \, \sqrt{x}}}\;dx=\answer{-\frac{176}{15} \, \sqrt{6} + \frac{91}{9} \, \sqrt{5}}\]
\end{problem}}%}

%%%%%%%%%%%%%%%%%%%%%%


\latexProblemContent{
\begin{problem}

Use the Fundamental Theorem of Calculus to evaluate the integral.

\expandafter\input{\file@loc Integrals/2311-Compute-Integral-0011.HELP.tex}

\[\int_{7}^{8} {\frac{x^{2} - 4 \, x + 3}{5 \, \sqrt{x}}}\;dx=\answer{-\frac{104}{75} \, \sqrt{7} + \frac{308}{75} \, \sqrt{2}}\]
\end{problem}}%}

%%%%%%%%%%%%%%%%%%%%%%


\latexProblemContent{
\begin{problem}

Use the Fundamental Theorem of Calculus to evaluate the integral.

\expandafter\input{\file@loc Integrals/2311-Compute-Integral-0011.HELP.tex}

\[\int_{4}^{7} {\frac{x - 2}{\sqrt{x}}}\;dx=\answer{\frac{2}{3} \, \sqrt{7} + \frac{8}{3}}\]
\end{problem}}%}

%%%%%%%%%%%%%%%%%%%%%%


\latexProblemContent{
\begin{problem}

Use the Fundamental Theorem of Calculus to evaluate the integral.

\expandafter\input{\file@loc Integrals/2311-Compute-Integral-0011.HELP.tex}

\[\int_{8}^{15} {\frac{x^{2} + x - 30}{9 \, \sqrt{x}}}\;dx=\answer{\frac{40}{9} \, \sqrt{15} + \frac{872}{135} \, \sqrt{2}}\]
\end{problem}}%}

%%%%%%%%%%%%%%%%%%%%%%


\latexProblemContent{
\begin{problem}

Use the Fundamental Theorem of Calculus to evaluate the integral.

\expandafter\input{\file@loc Integrals/2311-Compute-Integral-0011.HELP.tex}

\[\int_{6}^{9} {-\frac{x^{2} - x - 6}{9 \, \sqrt{x}}}\;dx=\answer{-\frac{8}{45} \, \sqrt{6} - \frac{24}{5}}\]
\end{problem}}%}

%%%%%%%%%%%%%%%%%%%%%%


\latexProblemContent{
\begin{problem}

Use the Fundamental Theorem of Calculus to evaluate the integral.

\expandafter\input{\file@loc Integrals/2311-Compute-Integral-0011.HELP.tex}

\[\int_{4}^{8} {\frac{x - 7}{9 \, \sqrt{x}}}\;dx=\answer{-\frac{52}{27} \, \sqrt{2} + \frac{68}{27}}\]
\end{problem}}%}

%%%%%%%%%%%%%%%%%%%%%%


\latexProblemContent{
\begin{problem}

Use the Fundamental Theorem of Calculus to evaluate the integral.

\expandafter\input{\file@loc Integrals/2311-Compute-Integral-0011.HELP.tex}

\[\int_{8}^{13} {\frac{x^{2} - 6 \, x - 7}{4 \, \sqrt{x}}}\;dx=\answer{\frac{2}{5} \, \sqrt{13} + \frac{51}{5} \, \sqrt{2}}\]
\end{problem}}%}

%%%%%%%%%%%%%%%%%%%%%%


\latexProblemContent{
\begin{problem}

Use the Fundamental Theorem of Calculus to evaluate the integral.

\expandafter\input{\file@loc Integrals/2311-Compute-Integral-0011.HELP.tex}

\[\int_{2}^{9} {\frac{x - 3}{6 \, \sqrt{x}}}\;dx=\answer{\frac{7}{9} \, \sqrt{2}}\]
\end{problem}}%}

%%%%%%%%%%%%%%%%%%%%%%


\latexProblemContent{
\begin{problem}

Use the Fundamental Theorem of Calculus to evaluate the integral.

\expandafter\input{\file@loc Integrals/2311-Compute-Integral-0011.HELP.tex}

\[\int_{2}^{15} {-\frac{x^{2} - 10 \, x + 24}{6 \, \sqrt{x}}}\;dx=\answer{-\frac{19}{3} \, \sqrt{15} + \frac{272}{45} \, \sqrt{2}}\]
\end{problem}}%}

%%%%%%%%%%%%%%%%%%%%%%


\latexProblemContent{
\begin{problem}

Use the Fundamental Theorem of Calculus to evaluate the integral.

\expandafter\input{\file@loc Integrals/2311-Compute-Integral-0011.HELP.tex}

\[\int_{5}^{14} {-\frac{x^{2} - 8 \, x - 9}{3 \, \sqrt{x}}}\;dx=\answer{\frac{214}{45} \, \sqrt{14} - \frac{104}{9} \, \sqrt{5}}\]
\end{problem}}%}

%%%%%%%%%%%%%%%%%%%%%%


\latexProblemContent{
\begin{problem}

Use the Fundamental Theorem of Calculus to evaluate the integral.

\expandafter\input{\file@loc Integrals/2311-Compute-Integral-0011.HELP.tex}

\[\int_{6}^{10} {\frac{x - 1}{6 \, \sqrt{x}}}\;dx=\answer{\frac{7}{9} \, \sqrt{10} - \frac{1}{3} \, \sqrt{6}}\]
\end{problem}}%}

%%%%%%%%%%%%%%%%%%%%%%


\latexProblemContent{
\begin{problem}

Use the Fundamental Theorem of Calculus to evaluate the integral.

\expandafter\input{\file@loc Integrals/2311-Compute-Integral-0011.HELP.tex}

\[\int_{4}^{6} {\frac{x + 8}{4 \, \sqrt{x}}}\;dx=\answer{5 \, \sqrt{6} - \frac{28}{3}}\]
\end{problem}}%}

%%%%%%%%%%%%%%%%%%%%%%


\latexProblemContent{
\begin{problem}

Use the Fundamental Theorem of Calculus to evaluate the integral.

\expandafter\input{\file@loc Integrals/2311-Compute-Integral-0011.HELP.tex}

\[\int_{6}^{11} {\frac{x + 6}{8 \, \sqrt{x}}}\;dx=\answer{\frac{29}{12} \, \sqrt{11} - 2 \, \sqrt{6}}\]
\end{problem}}%}

%%%%%%%%%%%%%%%%%%%%%%


\latexProblemContent{
\begin{problem}

Use the Fundamental Theorem of Calculus to evaluate the integral.

\expandafter\input{\file@loc Integrals/2311-Compute-Integral-0011.HELP.tex}

\[\int_{2}^{11} {-\frac{x^{2} - 15 \, x + 54}{4 \, \sqrt{x}}}\;dx=\answer{-\frac{58}{5} \, \sqrt{11} + \frac{112}{5} \, \sqrt{2}}\]
\end{problem}}%}

%%%%%%%%%%%%%%%%%%%%%%


\latexProblemContent{
\begin{problem}

Use the Fundamental Theorem of Calculus to evaluate the integral.

\expandafter\input{\file@loc Integrals/2311-Compute-Integral-0011.HELP.tex}

\[\int_{2}^{13} {\frac{x - 8}{6 \, \sqrt{x}}}\;dx=\answer{-\frac{11}{9} \, \sqrt{13} + \frac{22}{9} \, \sqrt{2}}\]
\end{problem}}%}

%%%%%%%%%%%%%%%%%%%%%%


\latexProblemContent{
\begin{problem}

Use the Fundamental Theorem of Calculus to evaluate the integral.

\expandafter\input{\file@loc Integrals/2311-Compute-Integral-0011.HELP.tex}

\[\int_{8}^{12} {-\frac{x - 10}{5 \, \sqrt{x}}}\;dx=\answer{\frac{24}{5} \, \sqrt{3} - \frac{88}{15} \, \sqrt{2}}\]
\end{problem}}%}

%%%%%%%%%%%%%%%%%%%%%%


\latexProblemContent{
\begin{problem}

Use the Fundamental Theorem of Calculus to evaluate the integral.

\expandafter\input{\file@loc Integrals/2311-Compute-Integral-0011.HELP.tex}

\[\int_{6}^{15} {\frac{x - 6}{5 \, \sqrt{x}}}\;dx=\answer{-\frac{2}{5} \, \sqrt{15} + \frac{8}{5} \, \sqrt{6}}\]
\end{problem}}%}

%%%%%%%%%%%%%%%%%%%%%%


\latexProblemContent{
\begin{problem}

Use the Fundamental Theorem of Calculus to evaluate the integral.

\expandafter\input{\file@loc Integrals/2311-Compute-Integral-0011.HELP.tex}

\[\int_{1}^{2} {\frac{x^{2} - 25}{3 \, \sqrt{x}}}\;dx=\answer{-\frac{242}{15} \, \sqrt{2} + \frac{248}{15}}\]
\end{problem}}%}

%%%%%%%%%%%%%%%%%%%%%%


\latexProblemContent{
\begin{problem}

Use the Fundamental Theorem of Calculus to evaluate the integral.

\expandafter\input{\file@loc Integrals/2311-Compute-Integral-0011.HELP.tex}

\[\int_{3}^{12} {-\frac{x^{2} - 100}{10 \, \sqrt{x}}}\;dx=\answer{\frac{221}{25} \, \sqrt{3}}\]
\end{problem}}%}

%%%%%%%%%%%%%%%%%%%%%%


\latexProblemContent{
\begin{problem}

Use the Fundamental Theorem of Calculus to evaluate the integral.

\expandafter\input{\file@loc Integrals/2311-Compute-Integral-0011.HELP.tex}

\[\int_{8}^{13} {\frac{x^{2} - 16}{8 \, \sqrt{x}}}\;dx=\answer{\frac{1}{40} \, \left(64 \, \sqrt{2}\right) + \frac{89}{20} \, \sqrt{13}}\]
\end{problem}}%}

%%%%%%%%%%%%%%%%%%%%%%


\latexProblemContent{
\begin{problem}

Use the Fundamental Theorem of Calculus to evaluate the integral.

\expandafter\input{\file@loc Integrals/2311-Compute-Integral-0011.HELP.tex}

\[\int_{5}^{9} {-\frac{x^{2} - 4}{2 \, \sqrt{x}}}\;dx=\answer{\sqrt{5} - \frac{183}{5}}\]
\end{problem}}%}

%%%%%%%%%%%%%%%%%%%%%%


\latexProblemContent{
\begin{problem}

Use the Fundamental Theorem of Calculus to evaluate the integral.

\expandafter\input{\file@loc Integrals/2311-Compute-Integral-0011.HELP.tex}

\[\int_{6}^{8} {\frac{x^{2} - 1}{5 \, \sqrt{x}}}\;dx=\answer{-\frac{62}{25} \, \sqrt{6} + \frac{236}{25} \, \sqrt{2}}\]
\end{problem}}%}

%%%%%%%%%%%%%%%%%%%%%%


\latexProblemContent{
\begin{problem}

Use the Fundamental Theorem of Calculus to evaluate the integral.

\expandafter\input{\file@loc Integrals/2311-Compute-Integral-0011.HELP.tex}

\[\int_{1}^{10} {\frac{x^{2} - 9 \, x + 8}{4 \, \sqrt{x}}}\;dx=\answer{-\sqrt{10} - \frac{13}{5}}\]
\end{problem}}%}

%%%%%%%%%%%%%%%%%%%%%%


\latexProblemContent{
\begin{problem}

Use the Fundamental Theorem of Calculus to evaluate the integral.

\expandafter\input{\file@loc Integrals/2311-Compute-Integral-0011.HELP.tex}

\[\int_{4}^{15} {-\frac{x - 1}{9 \, \sqrt{x}}}\;dx=\answer{-\frac{8}{9} \, \sqrt{15} + \frac{4}{27}}\]
\end{problem}}%}

%%%%%%%%%%%%%%%%%%%%%%


\latexProblemContent{
\begin{problem}

Use the Fundamental Theorem of Calculus to evaluate the integral.

\expandafter\input{\file@loc Integrals/2311-Compute-Integral-0011.HELP.tex}

\[\int_{2}^{7} {\frac{x - 7}{2 \, \sqrt{x}}}\;dx=\answer{-\frac{14}{3} \, \sqrt{7} + \frac{19}{3} \, \sqrt{2}}\]
\end{problem}}%}

%%%%%%%%%%%%%%%%%%%%%%


\latexProblemContent{
\begin{problem}

Use the Fundamental Theorem of Calculus to evaluate the integral.

\expandafter\input{\file@loc Integrals/2311-Compute-Integral-0011.HELP.tex}

\[\int_{3}^{6} {-\frac{x^{2} - 9}{5 \, \sqrt{x}}}\;dx=\answer{\frac{18}{25} \, \sqrt{6} - \frac{72}{25} \, \sqrt{3}}\]
\end{problem}}%}

%%%%%%%%%%%%%%%%%%%%%%


\latexProblemContent{
\begin{problem}

Use the Fundamental Theorem of Calculus to evaluate the integral.

\expandafter\input{\file@loc Integrals/2311-Compute-Integral-0011.HELP.tex}

\[\int_{6}^{11} {-\frac{1}{9} \, x^{\frac{3}{2}}}\;dx=\answer{-\frac{242}{45} \, \sqrt{11} + \frac{8}{5} \, \sqrt{6}}\]
\end{problem}}%}

%%%%%%%%%%%%%%%%%%%%%%


\latexProblemContent{
\begin{problem}

Use the Fundamental Theorem of Calculus to evaluate the integral.

\expandafter\input{\file@loc Integrals/2311-Compute-Integral-0011.HELP.tex}

\[\int_{6}^{14} {\frac{x^{2} - 81}{9 \, \sqrt{x}}}\;dx=\answer{-\frac{418}{45} \, \sqrt{14} + \frac{82}{5} \, \sqrt{6}}\]
\end{problem}}%}

%%%%%%%%%%%%%%%%%%%%%%


\latexProblemContent{
\begin{problem}

Use the Fundamental Theorem of Calculus to evaluate the integral.

\expandafter\input{\file@loc Integrals/2311-Compute-Integral-0011.HELP.tex}

\[\int_{1}^{4} {\frac{x + 8}{5 \, \sqrt{x}}}\;dx=\answer{\frac{62}{15}}\]
\end{problem}}%}

%%%%%%%%%%%%%%%%%%%%%%


\latexProblemContent{
\begin{problem}

Use the Fundamental Theorem of Calculus to evaluate the integral.

\expandafter\input{\file@loc Integrals/2311-Compute-Integral-0011.HELP.tex}

\[\int_{1}^{2} {\frac{x^{2} - 100}{6 \, \sqrt{x}}}\;dx=\answer{-\frac{496}{15} \, \sqrt{2} + \frac{499}{15}}\]
\end{problem}}%}

%%%%%%%%%%%%%%%%%%%%%%


\latexProblemContent{
\begin{problem}

Use the Fundamental Theorem of Calculus to evaluate the integral.

\expandafter\input{\file@loc Integrals/2311-Compute-Integral-0011.HELP.tex}

\[\int_{5}^{7} {\frac{x + 4}{7 \, \sqrt{x}}}\;dx=\answer{\frac{38}{21} \, \sqrt{7} - \frac{34}{21} \, \sqrt{5}}\]
\end{problem}}%}

%%%%%%%%%%%%%%%%%%%%%%


\latexProblemContent{
\begin{problem}

Use the Fundamental Theorem of Calculus to evaluate the integral.

\expandafter\input{\file@loc Integrals/2311-Compute-Integral-0011.HELP.tex}

\[\int_{4}^{8} {-\frac{x^{2} - 13 \, x + 30}{6 \, \sqrt{x}}}\;dx=\answer{-\frac{244}{45} \, \sqrt{2} + \frac{476}{45}}\]
\end{problem}}%}

%%%%%%%%%%%%%%%%%%%%%%


\latexProblemContent{
\begin{problem}

Use the Fundamental Theorem of Calculus to evaluate the integral.

\expandafter\input{\file@loc Integrals/2311-Compute-Integral-0011.HELP.tex}

\[\int_{2}^{8} {\frac{x^{2} - 64}{7 \, \sqrt{x}}}\;dx=\answer{-\frac{7}{5} \, \left(8 \, \sqrt{2}\right)}\]
\end{problem}}%}

%%%%%%%%%%%%%%%%%%%%%%


\latexProblemContent{
\begin{problem}

Use the Fundamental Theorem of Calculus to evaluate the integral.

\expandafter\input{\file@loc Integrals/2311-Compute-Integral-0011.HELP.tex}

\[\int_{5}^{8} {-\frac{x^{2} - 16}{4 \, \sqrt{x}}}\;dx=\answer{-\frac{11}{2} \, \sqrt{5} + \frac{16}{5} \, \sqrt{2}}\]
\end{problem}}%}

%%%%%%%%%%%%%%%%%%%%%%


\latexProblemContent{
\begin{problem}

Use the Fundamental Theorem of Calculus to evaluate the integral.

\expandafter\input{\file@loc Integrals/2311-Compute-Integral-0011.HELP.tex}

\[\int_{3}^{14} {\frac{x^{2} - 20 \, x + 100}{2 \, \sqrt{x}}}\;dx=\answer{\frac{688}{15} \, \sqrt{14} - \frac{409}{5} \, \sqrt{3}}\]
\end{problem}}%}

%%%%%%%%%%%%%%%%%%%%%%


\latexProblemContent{
\begin{problem}

Use the Fundamental Theorem of Calculus to evaluate the integral.

\expandafter\input{\file@loc Integrals/2311-Compute-Integral-0011.HELP.tex}

\[\int_{5}^{7} {\frac{x^{2} + 7 \, x - 8}{8 \, \sqrt{x}}}\;dx=\answer{\frac{68}{15} \, \sqrt{7} - \frac{13}{6} \, \sqrt{5}}\]
\end{problem}}%}

%%%%%%%%%%%%%%%%%%%%%%


\latexProblemContent{
\begin{problem}

Use the Fundamental Theorem of Calculus to evaluate the integral.

\expandafter\input{\file@loc Integrals/2311-Compute-Integral-0011.HELP.tex}

\[\int_{3}^{12} {-\frac{x^{2} - 49}{6 \, \sqrt{x}}}\;dx=\answer{-\frac{34}{15} \, \sqrt{3}}\]
\end{problem}}%}

%%%%%%%%%%%%%%%%%%%%%%


\latexProblemContent{
\begin{problem}

Use the Fundamental Theorem of Calculus to evaluate the integral.

\expandafter\input{\file@loc Integrals/2311-Compute-Integral-0011.HELP.tex}

\[\int_{2}^{11} {\frac{x^{2} + 6 \, x - 27}{10 \, \sqrt{x}}}\;dx=\answer{\frac{96}{25} \, \sqrt{11} + \frac{111}{25} \, \sqrt{2}}\]
\end{problem}}%}

%%%%%%%%%%%%%%%%%%%%%%


\latexProblemContent{
\begin{problem}

Use the Fundamental Theorem of Calculus to evaluate the integral.

\expandafter\input{\file@loc Integrals/2311-Compute-Integral-0011.HELP.tex}

\[\int_{5}^{7} {\frac{x^{2} - 81}{7 \, \sqrt{x}}}\;dx=\answer{-\frac{712}{35} \, \sqrt{7} + \frac{152}{7} \, \sqrt{5}}\]
\end{problem}}%}

%%%%%%%%%%%%%%%%%%%%%%


\latexProblemContent{
\begin{problem}

Use the Fundamental Theorem of Calculus to evaluate the integral.

\expandafter\input{\file@loc Integrals/2311-Compute-Integral-0011.HELP.tex}

\[\int_{8}^{10} {\frac{x + 7}{\sqrt{x}}}\;dx=\answer{\frac{62}{3} \, \sqrt{10} - \frac{116}{3} \, \sqrt{2}}\]
\end{problem}}%}

%%%%%%%%%%%%%%%%%%%%%%


\latexProblemContent{
\begin{problem}

Use the Fundamental Theorem of Calculus to evaluate the integral.

\expandafter\input{\file@loc Integrals/2311-Compute-Integral-0011.HELP.tex}

\[\int_{5}^{15} {-\frac{x^{2} - 6 \, x - 7}{2 \, \sqrt{x}}}\;dx=\answer{-8 \, \sqrt{15} - 12 \, \sqrt{5}}\]
\end{problem}}%}

%%%%%%%%%%%%%%%%%%%%%%


\latexProblemContent{
\begin{problem}

Use the Fundamental Theorem of Calculus to evaluate the integral.

\expandafter\input{\file@loc Integrals/2311-Compute-Integral-0011.HELP.tex}

\[\int_{5}^{12} {-\frac{x^{2} - 100}{4 \, \sqrt{x}}}\;dx=\answer{-\frac{95}{2} \, \sqrt{5} + \frac{356}{5} \, \sqrt{3}}\]
\end{problem}}%}

%%%%%%%%%%%%%%%%%%%%%%


\latexProblemContent{
\begin{problem}

Use the Fundamental Theorem of Calculus to evaluate the integral.

\expandafter\input{\file@loc Integrals/2311-Compute-Integral-0011.HELP.tex}

\[\int_{1}^{13} {\frac{x^{2} - 100}{6 \, \sqrt{x}}}\;dx=\answer{-\frac{331}{15} \, \sqrt{13} + \frac{499}{15}}\]
\end{problem}}%}

%%%%%%%%%%%%%%%%%%%%%%


\latexProblemContent{
\begin{problem}

Use the Fundamental Theorem of Calculus to evaluate the integral.

\expandafter\input{\file@loc Integrals/2311-Compute-Integral-0011.HELP.tex}

\[\int_{1}^{12} {\frac{x - 4}{10 \, \sqrt{x}}}\;dx=\answer{\frac{11}{15}}\]
\end{problem}}%}

%%%%%%%%%%%%%%%%%%%%%%


\latexProblemContent{
\begin{problem}

Use the Fundamental Theorem of Calculus to evaluate the integral.

\expandafter\input{\file@loc Integrals/2311-Compute-Integral-0011.HELP.tex}

\[\int_{3}^{5} {-\frac{x^{2} - 1}{4 \, \sqrt{x}}}\;dx=\answer{-2 \, \sqrt{5} + \frac{2}{5} \, \sqrt{3}}\]
\end{problem}}%}

%%%%%%%%%%%%%%%%%%%%%%


\latexProblemContent{
\begin{problem}

Use the Fundamental Theorem of Calculus to evaluate the integral.

\expandafter\input{\file@loc Integrals/2311-Compute-Integral-0011.HELP.tex}

\[\int_{6}^{11} {\frac{x + 6}{\sqrt{x}}}\;dx=\answer{\frac{58}{3} \, \sqrt{11} - 16 \, \sqrt{6}}\]
\end{problem}}%}

%%%%%%%%%%%%%%%%%%%%%%


\latexProblemContent{
\begin{problem}

Use the Fundamental Theorem of Calculus to evaluate the integral.

\expandafter\input{\file@loc Integrals/2311-Compute-Integral-0011.HELP.tex}

\[\int_{8}^{15} {-\frac{x^{2} - 6 \, x - 7}{10 \, \sqrt{x}}}\;dx=\answer{-\frac{8}{5} \, \sqrt{15} - \frac{102}{25} \, \sqrt{2}}\]
\end{problem}}%}

%%%%%%%%%%%%%%%%%%%%%%


\latexProblemContent{
\begin{problem}

Use the Fundamental Theorem of Calculus to evaluate the integral.

\expandafter\input{\file@loc Integrals/2311-Compute-Integral-0011.HELP.tex}

\[\int_{7}^{14} {\frac{x^{2} - 36}{4 \, \sqrt{x}}}\;dx=\answer{\frac{8}{5} \, \sqrt{14} + \frac{131}{10} \, \sqrt{7}}\]
\end{problem}}%}

%%%%%%%%%%%%%%%%%%%%%%


\latexProblemContent{
\begin{problem}

Use the Fundamental Theorem of Calculus to evaluate the integral.

\expandafter\input{\file@loc Integrals/2311-Compute-Integral-0011.HELP.tex}

\[\int_{6}^{10} {\frac{x^{2} - 9 \, x}{\sqrt{x}}}\;dx=\answer{-20 \, \sqrt{10} + \frac{108}{5} \, \sqrt{6}}\]
\end{problem}}%}

%%%%%%%%%%%%%%%%%%%%%%


\latexProblemContent{
\begin{problem}

Use the Fundamental Theorem of Calculus to evaluate the integral.

\expandafter\input{\file@loc Integrals/2311-Compute-Integral-0011.HELP.tex}

\[\int_{7}^{11} {\frac{x^{2} + x - 12}{5 \, \sqrt{x}}}\;dx=\answer{\frac{476}{75} \, \sqrt{11} - \frac{4}{75} \, \sqrt{7}}\]
\end{problem}}%}

%%%%%%%%%%%%%%%%%%%%%%


\latexProblemContent{
\begin{problem}

Use the Fundamental Theorem of Calculus to evaluate the integral.

\expandafter\input{\file@loc Integrals/2311-Compute-Integral-0011.HELP.tex}

\[\int_{7}^{14} {\frac{x^{2} + 17 \, x + 70}{5 \, \sqrt{x}}}\;dx=\answer{\frac{5656}{75} \, \sqrt{14} - \frac{3584}{75} \, \sqrt{7}}\]
\end{problem}}%}

%%%%%%%%%%%%%%%%%%%%%%


\latexProblemContent{
\begin{problem}

Use the Fundamental Theorem of Calculus to evaluate the integral.

\expandafter\input{\file@loc Integrals/2311-Compute-Integral-0011.HELP.tex}

\[\int_{3}^{9} {-\frac{x + 7}{4 \, \sqrt{x}}}\;dx=\answer{4 \, \sqrt{3} - 15}\]
\end{problem}}%}

%%%%%%%%%%%%%%%%%%%%%%


\latexProblemContent{
\begin{problem}

Use the Fundamental Theorem of Calculus to evaluate the integral.

\expandafter\input{\file@loc Integrals/2311-Compute-Integral-0011.HELP.tex}

\[\int_{3}^{6} {\frac{x^{2} - 1}{5 \, \sqrt{x}}}\;dx=\answer{\frac{62}{25} \, \sqrt{6} - \frac{8}{25} \, \sqrt{3}}\]
\end{problem}}%}

%%%%%%%%%%%%%%%%%%%%%%


\latexProblemContent{
\begin{problem}

Use the Fundamental Theorem of Calculus to evaluate the integral.

\expandafter\input{\file@loc Integrals/2311-Compute-Integral-0011.HELP.tex}

\[\int_{8}^{12} {\frac{x^{2} - 2 \, x - 15}{4 \, \sqrt{x}}}\;dx=\answer{\frac{29}{5} \, \sqrt{3} + \frac{113}{15} \, \sqrt{2}}\]
\end{problem}}%}

%%%%%%%%%%%%%%%%%%%%%%


\latexProblemContent{
\begin{problem}

Use the Fundamental Theorem of Calculus to evaluate the integral.

\expandafter\input{\file@loc Integrals/2311-Compute-Integral-0011.HELP.tex}

\[\int_{3}^{15} {-\frac{x + 6}{9 \, \sqrt{x}}}\;dx=\answer{-\frac{22}{9} \, \sqrt{15} + \frac{14}{9} \, \sqrt{3}}\]
\end{problem}}%}

%%%%%%%%%%%%%%%%%%%%%%


\latexProblemContent{
\begin{problem}

Use the Fundamental Theorem of Calculus to evaluate the integral.

\expandafter\input{\file@loc Integrals/2311-Compute-Integral-0011.HELP.tex}

\[\int_{2}^{7} {-\frac{x + 6}{10 \, \sqrt{x}}}\;dx=\answer{-\frac{5}{3} \, \sqrt{7} + \frac{4}{3} \, \sqrt{2}}\]
\end{problem}}%}

%%%%%%%%%%%%%%%%%%%%%%


\latexProblemContent{
\begin{problem}

Use the Fundamental Theorem of Calculus to evaluate the integral.

\expandafter\input{\file@loc Integrals/2311-Compute-Integral-0011.HELP.tex}

\[\int_{1}^{15} {-\frac{x - 10}{10 \, \sqrt{x}}}\;dx=\answer{\sqrt{15} - \frac{29}{15}}\]
\end{problem}}%}

%%%%%%%%%%%%%%%%%%%%%%


\latexProblemContent{
\begin{problem}

Use the Fundamental Theorem of Calculus to evaluate the integral.

\expandafter\input{\file@loc Integrals/2311-Compute-Integral-0011.HELP.tex}

\[\int_{3}^{10} {-\frac{x + 6}{7 \, \sqrt{x}}}\;dx=\answer{-\frac{8}{3} \, \sqrt{10} + 2 \, \sqrt{3}}\]
\end{problem}}%}

%%%%%%%%%%%%%%%%%%%%%%


\latexProblemContent{
\begin{problem}

Use the Fundamental Theorem of Calculus to evaluate the integral.

\expandafter\input{\file@loc Integrals/2311-Compute-Integral-0011.HELP.tex}

\[\int_{3}^{8} {\frac{x^{2} - 81}{3 \, \sqrt{x}}}\;dx=\answer{\frac{264}{5} \, \sqrt{3} - \frac{1364}{15} \, \sqrt{2}}\]
\end{problem}}%}

%%%%%%%%%%%%%%%%%%%%%%


\latexProblemContent{
\begin{problem}

Use the Fundamental Theorem of Calculus to evaluate the integral.

\expandafter\input{\file@loc Integrals/2311-Compute-Integral-0011.HELP.tex}

\[\int_{3}^{4} {-\frac{x^{2} - 6 \, x + 5}{10 \, \sqrt{x}}}\;dx=\answer{\frac{4}{25} \, \sqrt{3} - \frac{2}{25}}\]
\end{problem}}%}

%%%%%%%%%%%%%%%%%%%%%%


\latexProblemContent{
\begin{problem}

Use the Fundamental Theorem of Calculus to evaluate the integral.

\expandafter\input{\file@loc Integrals/2311-Compute-Integral-0011.HELP.tex}

\[\int_{5}^{7} {-\frac{x - 6}{6 \, \sqrt{x}}}\;dx=\answer{\frac{11}{9} \, \sqrt{7} - \frac{13}{9} \, \sqrt{5}}\]
\end{problem}}%}

%%%%%%%%%%%%%%%%%%%%%%


\latexProblemContent{
\begin{problem}

Use the Fundamental Theorem of Calculus to evaluate the integral.

\expandafter\input{\file@loc Integrals/2311-Compute-Integral-0011.HELP.tex}

\[\int_{2}^{10} {\frac{x^{2} - 36}{6 \, \sqrt{x}}}\;dx=\answer{-\frac{16}{3} \, \sqrt{10} + \frac{176}{15} \, \sqrt{2}}\]
\end{problem}}%}

%%%%%%%%%%%%%%%%%%%%%%


\latexProblemContent{
\begin{problem}

Use the Fundamental Theorem of Calculus to evaluate the integral.

\expandafter\input{\file@loc Integrals/2311-Compute-Integral-0011.HELP.tex}

\[\int_{5}^{8} {\frac{x + 7}{6 \, \sqrt{x}}}\;dx=\answer{-\frac{26}{9} \, \sqrt{5} + \frac{58}{9} \, \sqrt{2}}\]
\end{problem}}%}

%%%%%%%%%%%%%%%%%%%%%%


\latexProblemContent{
\begin{problem}

Use the Fundamental Theorem of Calculus to evaluate the integral.

\expandafter\input{\file@loc Integrals/2311-Compute-Integral-0011.HELP.tex}

\[\int_{5}^{13} {-\frac{x^{2} - 9 \, x + 20}{9 \, \sqrt{x}}}\;dx=\answer{-\frac{148}{45} \, \sqrt{13} + \frac{20}{9} \, \sqrt{5}}\]
\end{problem}}%}

%%%%%%%%%%%%%%%%%%%%%%


\latexProblemContent{
\begin{problem}

Use the Fundamental Theorem of Calculus to evaluate the integral.

\expandafter\input{\file@loc Integrals/2311-Compute-Integral-0011.HELP.tex}

\[\int_{3}^{12} {-\frac{x - 6}{10 \, \sqrt{x}}}\;dx=\answer{-\frac{1}{5} \, \sqrt{3}}\]
\end{problem}}%}

%%%%%%%%%%%%%%%%%%%%%%


\latexProblemContent{
\begin{problem}

Use the Fundamental Theorem of Calculus to evaluate the integral.

\expandafter\input{\file@loc Integrals/2311-Compute-Integral-0011.HELP.tex}

\[\int_{7}^{9} {\frac{x^{2} + 8 \, x}{7 \, \sqrt{x}}}\;dx=\answer{-\frac{122}{15} \, \sqrt{7} + \frac{1206}{35}}\]
\end{problem}}%}

%%%%%%%%%%%%%%%%%%%%%%


\latexProblemContent{
\begin{problem}

Use the Fundamental Theorem of Calculus to evaluate the integral.

\expandafter\input{\file@loc Integrals/2311-Compute-Integral-0011.HELP.tex}

\[\int_{6}^{7} {-\frac{x^{2} - 4 \, x - 45}{9 \, \sqrt{x}}}\;dx=\answer{\frac{1336}{135} \, \sqrt{7} - \frac{458}{45} \, \sqrt{6}}\]
\end{problem}}%}

%%%%%%%%%%%%%%%%%%%%%%


\latexProblemContent{
\begin{problem}

Use the Fundamental Theorem of Calculus to evaluate the integral.

\expandafter\input{\file@loc Integrals/2311-Compute-Integral-0011.HELP.tex}

\[\int_{6}^{9} {-\frac{x^{2} - 100}{5 \, \sqrt{x}}}\;dx=\answer{-\frac{928}{25} \, \sqrt{6} + \frac{2514}{25}}\]
\end{problem}}%}

%%%%%%%%%%%%%%%%%%%%%%


\latexProblemContent{
\begin{problem}

Use the Fundamental Theorem of Calculus to evaluate the integral.

\expandafter\input{\file@loc Integrals/2311-Compute-Integral-0011.HELP.tex}

\[\int_{7}^{13} {\frac{1}{6} \, x^{\frac{3}{2}}}\;dx=\answer{\frac{169}{15} \, \sqrt{13} - \frac{49}{15} \, \sqrt{7}}\]
\end{problem}}%}

%%%%%%%%%%%%%%%%%%%%%%


\latexProblemContent{
\begin{problem}

Use the Fundamental Theorem of Calculus to evaluate the integral.

\expandafter\input{\file@loc Integrals/2311-Compute-Integral-0011.HELP.tex}

\[\int_{6}^{12} {\frac{x^{2} + 10 \, x + 16}{\sqrt{x}}}\;dx=\answer{-\frac{432}{5} \, \sqrt{6} + \frac{1696}{5} \, \sqrt{3}}\]
\end{problem}}%}

%%%%%%%%%%%%%%%%%%%%%%


\latexProblemContent{
\begin{problem}

Use the Fundamental Theorem of Calculus to evaluate the integral.

\expandafter\input{\file@loc Integrals/2311-Compute-Integral-0011.HELP.tex}

\[\int_{7}^{9} {\frac{x^{2} - x - 56}{2 \, \sqrt{x}}}\;dx=\answer{\frac{728}{15} \, \sqrt{7} - \frac{642}{5}}\]
\end{problem}}%}

%%%%%%%%%%%%%%%%%%%%%%


\latexProblemContent{
\begin{problem}

Use the Fundamental Theorem of Calculus to evaluate the integral.

\expandafter\input{\file@loc Integrals/2311-Compute-Integral-0011.HELP.tex}

\[\int_{7}^{15} {\frac{x^{2} - 16}{4 \, \sqrt{x}}}\;dx=\answer{\frac{29}{2} \, \sqrt{15} + \frac{31}{10} \, \sqrt{7}}\]
\end{problem}}%}

%%%%%%%%%%%%%%%%%%%%%%


\latexProblemContent{
\begin{problem}

Use the Fundamental Theorem of Calculus to evaluate the integral.

\expandafter\input{\file@loc Integrals/2311-Compute-Integral-0011.HELP.tex}

\[\int_{7}^{12} {\frac{x^{2} + 4 \, x}{5 \, \sqrt{x}}}\;dx=\answer{-\frac{574}{75} \, \sqrt{7} + \frac{896}{25} \, \sqrt{3}}\]
\end{problem}}%}

%%%%%%%%%%%%%%%%%%%%%%


\latexProblemContent{
\begin{problem}

Use the Fundamental Theorem of Calculus to evaluate the integral.

\expandafter\input{\file@loc Integrals/2311-Compute-Integral-0011.HELP.tex}

\[\int_{6}^{15} {-\frac{x + 1}{5 \, \sqrt{x}}}\;dx=\answer{-\frac{12}{5} \, \sqrt{15} + \frac{6}{5} \, \sqrt{6}}\]
\end{problem}}%}

%%%%%%%%%%%%%%%%%%%%%%


\latexProblemContent{
\begin{problem}

Use the Fundamental Theorem of Calculus to evaluate the integral.

\expandafter\input{\file@loc Integrals/2311-Compute-Integral-0011.HELP.tex}

\[\int_{1}^{3} {\frac{x + 9}{2 \, \sqrt{x}}}\;dx=\answer{10 \, \sqrt{3} - \frac{28}{3}}\]
\end{problem}}%}

%%%%%%%%%%%%%%%%%%%%%%


\latexProblemContent{
\begin{problem}

Use the Fundamental Theorem of Calculus to evaluate the integral.

\expandafter\input{\file@loc Integrals/2311-Compute-Integral-0011.HELP.tex}

\[\int_{5}^{12} {\frac{x^{2} - 7 \, x + 12}{5 \, \sqrt{x}}}\;dx=\answer{-\frac{32}{15} \, \sqrt{5} + \frac{256}{25} \, \sqrt{3}}\]
\end{problem}}%}

%%%%%%%%%%%%%%%%%%%%%%


\latexProblemContent{
\begin{problem}

Use the Fundamental Theorem of Calculus to evaluate the integral.

\expandafter\input{\file@loc Integrals/2311-Compute-Integral-0011.HELP.tex}

\[\int_{2}^{14} {\frac{x^{2} - 11 \, x + 30}{3 \, \sqrt{x}}}\;dx=\answer{\frac{536}{45} \, \sqrt{14} - \frac{704}{45} \, \sqrt{2}}\]
\end{problem}}%}

%%%%%%%%%%%%%%%%%%%%%%


\latexProblemContent{
\begin{problem}

Use the Fundamental Theorem of Calculus to evaluate the integral.

\expandafter\input{\file@loc Integrals/2311-Compute-Integral-0011.HELP.tex}

\[\int_{8}^{13} {\frac{x^{2} - 9}{9 \, \sqrt{x}}}\;dx=\answer{\frac{248}{45} \, \sqrt{13} - \frac{76}{45} \, \sqrt{2}}\]
\end{problem}}%}

%%%%%%%%%%%%%%%%%%%%%%


\latexProblemContent{
\begin{problem}

Use the Fundamental Theorem of Calculus to evaluate the integral.

\expandafter\input{\file@loc Integrals/2311-Compute-Integral-0011.HELP.tex}

\[\int_{4}^{13} {-\frac{x^{2} - 64}{\sqrt{x}}}\;dx=\answer{\frac{302}{5} \, \sqrt{13} - \frac{1216}{5}}\]
\end{problem}}%}

%%%%%%%%%%%%%%%%%%%%%%


\latexProblemContent{
\begin{problem}

Use the Fundamental Theorem of Calculus to evaluate the integral.

\expandafter\input{\file@loc Integrals/2311-Compute-Integral-0011.HELP.tex}

\[\int_{4}^{6} {\frac{x^{2} - 64}{7 \, \sqrt{x}}}\;dx=\answer{-\frac{568}{35} \, \sqrt{6} + \frac{1216}{35}}\]
\end{problem}}%}

%%%%%%%%%%%%%%%%%%%%%%


\latexProblemContent{
\begin{problem}

Use the Fundamental Theorem of Calculus to evaluate the integral.

\expandafter\input{\file@loc Integrals/2311-Compute-Integral-0011.HELP.tex}

\[\int_{5}^{12} {-\frac{x^{2} + 8 \, x + 12}{10 \, \sqrt{x}}}\;dx=\answer{\frac{91}{15} \, \sqrt{5} - \frac{728}{25} \, \sqrt{3}}\]
\end{problem}}%}

%%%%%%%%%%%%%%%%%%%%%%


\latexProblemContent{
\begin{problem}

Use the Fundamental Theorem of Calculus to evaluate the integral.

\expandafter\input{\file@loc Integrals/2311-Compute-Integral-0011.HELP.tex}

\[\int_{3}^{5} {-\frac{x - 1}{9 \, \sqrt{x}}}\;dx=\answer{-\frac{4}{27} \, \sqrt{5}}\]
\end{problem}}%}

%%%%%%%%%%%%%%%%%%%%%%


\latexProblemContent{
\begin{problem}

Use the Fundamental Theorem of Calculus to evaluate the integral.

\expandafter\input{\file@loc Integrals/2311-Compute-Integral-0011.HELP.tex}

\[\int_{3}^{15} {\frac{x + 5}{5 \, \sqrt{x}}}\;dx=\answer{4 \, \sqrt{15} - \frac{12}{5} \, \sqrt{3}}\]
\end{problem}}%}

%%%%%%%%%%%%%%%%%%%%%%


\latexProblemContent{
\begin{problem}

Use the Fundamental Theorem of Calculus to evaluate the integral.

\expandafter\input{\file@loc Integrals/2311-Compute-Integral-0011.HELP.tex}

\[\int_{6}^{8} {\frac{x^{2} - 9}{\sqrt{x}}}\;dx=\answer{\frac{18}{5} \, \sqrt{6} + \frac{76}{5} \, \sqrt{2}}\]
\end{problem}}%}

%%%%%%%%%%%%%%%%%%%%%%


\latexProblemContent{
\begin{problem}

Use the Fundamental Theorem of Calculus to evaluate the integral.

\expandafter\input{\file@loc Integrals/2311-Compute-Integral-0011.HELP.tex}

\[\int_{5}^{15} {\frac{x^{2} - 25}{3 \, \sqrt{x}}}\;dx=\answer{\frac{40}{3} \, \sqrt{15} + \frac{40}{3} \, \sqrt{5}}\]
\end{problem}}%}

%%%%%%%%%%%%%%%%%%%%%%


\latexProblemContent{
\begin{problem}

Use the Fundamental Theorem of Calculus to evaluate the integral.

\expandafter\input{\file@loc Integrals/2311-Compute-Integral-0011.HELP.tex}

\[\int_{2}^{5} {-\frac{x^{2} - 16}{4 \, \sqrt{x}}}\;dx=\answer{\frac{11}{2} \, \sqrt{5} - \frac{38}{5} \, \sqrt{2}}\]
\end{problem}}%}

%%%%%%%%%%%%%%%%%%%%%%


\latexProblemContent{
\begin{problem}

Use the Fundamental Theorem of Calculus to evaluate the integral.

\expandafter\input{\file@loc Integrals/2311-Compute-Integral-0011.HELP.tex}

\[\int_{4}^{15} {\frac{x^{2} - 49}{6 \, \sqrt{x}}}\;dx=\answer{-\frac{4}{3} \, \sqrt{15} + \frac{458}{15}}\]
\end{problem}}%}

%%%%%%%%%%%%%%%%%%%%%%


\latexProblemContent{
\begin{problem}

Use the Fundamental Theorem of Calculus to evaluate the integral.

\expandafter\input{\file@loc Integrals/2311-Compute-Integral-0011.HELP.tex}

\[\int_{8}^{9} {\frac{x^{2} - 9 \, x + 14}{6 \, \sqrt{x}}}\;dx=\answer{-\frac{28}{15} \, \sqrt{2} + \frac{16}{5}}\]
\end{problem}}%}

%%%%%%%%%%%%%%%%%%%%%%


\latexProblemContent{
\begin{problem}

Use the Fundamental Theorem of Calculus to evaluate the integral.

\expandafter\input{\file@loc Integrals/2311-Compute-Integral-0011.HELP.tex}

\[\int_{7}^{8} {-\frac{x^{2} - 1}{8 \, \sqrt{x}}}\;dx=\answer{\frac{11}{5} \, \sqrt{7} - \frac{59}{10} \, \sqrt{2}}\]
\end{problem}}%}

%%%%%%%%%%%%%%%%%%%%%%


\latexProblemContent{
\begin{problem}

Use the Fundamental Theorem of Calculus to evaluate the integral.

\expandafter\input{\file@loc Integrals/2311-Compute-Integral-0011.HELP.tex}

\[\int_{4}^{11} {\frac{x^{2} + 16 \, x + 60}{5 \, \sqrt{x}}}\;dx=\answer{\frac{4286}{75} \, \sqrt{11} - \frac{5072}{75}}\]
\end{problem}}%}

%%%%%%%%%%%%%%%%%%%%%%


\latexProblemContent{
\begin{problem}

Use the Fundamental Theorem of Calculus to evaluate the integral.

\expandafter\input{\file@loc Integrals/2311-Compute-Integral-0011.HELP.tex}

\[\int_{7}^{9} {\frac{1}{6} \, \sqrt{x}}\;dx=\answer{-\frac{7}{9} \, \sqrt{7} + 3}\]
\end{problem}}%}

%%%%%%%%%%%%%%%%%%%%%%


\latexProblemContent{
\begin{problem}

Use the Fundamental Theorem of Calculus to evaluate the integral.

\expandafter\input{\file@loc Integrals/2311-Compute-Integral-0011.HELP.tex}

\[\int_{8}^{13} {-\frac{x^{2} - 16 \, x + 63}{6 \, \sqrt{x}}}\;dx=\answer{-\frac{412}{45} \, \sqrt{13} + \frac{994}{45} \, \sqrt{2}}\]
\end{problem}}%}

%%%%%%%%%%%%%%%%%%%%%%


\latexProblemContent{
\begin{problem}

Use the Fundamental Theorem of Calculus to evaluate the integral.

\expandafter\input{\file@loc Integrals/2311-Compute-Integral-0011.HELP.tex}

\[\int_{5}^{10} {\frac{x - 9}{10 \, \sqrt{x}}}\;dx=\answer{-\frac{17}{15} \, \sqrt{10} + \frac{22}{15} \, \sqrt{5}}\]
\end{problem}}%}

%%%%%%%%%%%%%%%%%%%%%%


\latexProblemContent{
\begin{problem}

Use the Fundamental Theorem of Calculus to evaluate the integral.

\expandafter\input{\file@loc Integrals/2311-Compute-Integral-0011.HELP.tex}

\[\int_{4}^{12} {\frac{x^{2} - 4}{6 \, \sqrt{x}}}\;dx=\answer{\frac{248}{15} \, \sqrt{3} + \frac{8}{15}}\]
\end{problem}}%}

%%%%%%%%%%%%%%%%%%%%%%


\latexProblemContent{
\begin{problem}

Use the Fundamental Theorem of Calculus to evaluate the integral.

\expandafter\input{\file@loc Integrals/2311-Compute-Integral-0011.HELP.tex}

\[\int_{1}^{8} {-\frac{x^{2} - 64}{3 \, \sqrt{x}}}\;dx=\answer{\frac{1}{15} \, \left(1024 \, \sqrt{2}\right) - \frac{638}{15}}\]
\end{problem}}%}

%%%%%%%%%%%%%%%%%%%%%%


\latexProblemContent{
\begin{problem}

Use the Fundamental Theorem of Calculus to evaluate the integral.

\expandafter\input{\file@loc Integrals/2311-Compute-Integral-0011.HELP.tex}

\[\int_{5}^{7} {-\frac{x^{2} + 4 \, x}{7 \, \sqrt{x}}}\;dx=\answer{-\frac{82}{15} \, \sqrt{7} + \frac{10}{3} \, \sqrt{5}}\]
\end{problem}}%}

%%%%%%%%%%%%%%%%%%%%%%


\latexProblemContent{
\begin{problem}

Use the Fundamental Theorem of Calculus to evaluate the integral.

\expandafter\input{\file@loc Integrals/2311-Compute-Integral-0011.HELP.tex}

\[\int_{4}^{14} {-\frac{x^{2} - 49}{8 \, \sqrt{x}}}\;dx=\answer{\frac{49}{20} \, \sqrt{14} - \frac{229}{10}}\]
\end{problem}}%}

%%%%%%%%%%%%%%%%%%%%%%


\latexProblemContent{
\begin{problem}

Use the Fundamental Theorem of Calculus to evaluate the integral.

\expandafter\input{\file@loc Integrals/2311-Compute-Integral-0011.HELP.tex}

\[\int_{7}^{14} {\frac{x^{2} - 11 \, x + 10}{2 \, \sqrt{x}}}\;dx=\answer{-\frac{32}{15} \, \sqrt{14} + \frac{88}{15} \, \sqrt{7}}\]
\end{problem}}%}

%%%%%%%%%%%%%%%%%%%%%%


\latexProblemContent{
\begin{problem}

Use the Fundamental Theorem of Calculus to evaluate the integral.

\expandafter\input{\file@loc Integrals/2311-Compute-Integral-0011.HELP.tex}

\[\int_{3}^{4} {-\frac{1}{5} \, \sqrt{x}}\;dx=\answer{\frac{2}{5} \, \sqrt{3} - \frac{16}{15}}\]
\end{problem}}%}

%%%%%%%%%%%%%%%%%%%%%%


\latexProblemContent{
\begin{problem}

Use the Fundamental Theorem of Calculus to evaluate the integral.

\expandafter\input{\file@loc Integrals/2311-Compute-Integral-0011.HELP.tex}

\[\int_{1}^{10} {-\frac{x - 7}{5 \, \sqrt{x}}}\;dx=\answer{\frac{22}{15} \, \sqrt{10} - \frac{8}{3}}\]
\end{problem}}%}

%%%%%%%%%%%%%%%%%%%%%%


\latexProblemContent{
\begin{problem}

Use the Fundamental Theorem of Calculus to evaluate the integral.

\expandafter\input{\file@loc Integrals/2311-Compute-Integral-0011.HELP.tex}

\[\int_{6}^{8} {\frac{x + 4}{3 \, \sqrt{x}}}\;dx=\answer{-\frac{1}{9} \, \left(36 \, \sqrt{6}\right) + \frac{80}{9} \, \sqrt{2}}\]
\end{problem}}%}

%%%%%%%%%%%%%%%%%%%%%%


\latexProblemContent{
\begin{problem}

Use the Fundamental Theorem of Calculus to evaluate the integral.

\expandafter\input{\file@loc Integrals/2311-Compute-Integral-0011.HELP.tex}

\[\int_{7}^{10} {-\frac{x - 5}{5 \, \sqrt{x}}}\;dx=\answer{\frac{1}{15} \, \left(10 \, \sqrt{10}\right) - \frac{16}{15} \, \sqrt{7}}\]
\end{problem}}%}

%%%%%%%%%%%%%%%%%%%%%%


\latexProblemContent{
\begin{problem}

Use the Fundamental Theorem of Calculus to evaluate the integral.

\expandafter\input{\file@loc Integrals/2311-Compute-Integral-0011.HELP.tex}

\[\int_{1}^{11} {\frac{x^{2} - 25}{8 \, \sqrt{x}}}\;dx=\answer{-\frac{1}{5} \, \sqrt{11} + \frac{31}{5}}\]
\end{problem}}%}

%%%%%%%%%%%%%%%%%%%%%%


\latexProblemContent{
\begin{problem}

Use the Fundamental Theorem of Calculus to evaluate the integral.

\expandafter\input{\file@loc Integrals/2311-Compute-Integral-0011.HELP.tex}

\[\int_{1}^{2} {\frac{x + 7}{4 \, \sqrt{x}}}\;dx=\answer{\frac{23}{6} \, \sqrt{2} - \frac{11}{3}}\]
\end{problem}}%}

%%%%%%%%%%%%%%%%%%%%%%


\latexProblemContent{
\begin{problem}

Use the Fundamental Theorem of Calculus to evaluate the integral.

\expandafter\input{\file@loc Integrals/2311-Compute-Integral-0011.HELP.tex}

\[\int_{1}^{4} {-\frac{x + 7}{2 \, \sqrt{x}}}\;dx=\answer{-\frac{28}{3}}\]
\end{problem}}%}

%%%%%%%%%%%%%%%%%%%%%%


\latexProblemContent{
\begin{problem}

Use the Fundamental Theorem of Calculus to evaluate the integral.

\expandafter\input{\file@loc Integrals/2311-Compute-Integral-0011.HELP.tex}

\[\int_{5}^{10} {\frac{x + 10}{5 \, \sqrt{x}}}\;dx=\answer{\frac{16}{3} \, \sqrt{10} - \frac{14}{3} \, \sqrt{5}}\]
\end{problem}}%}

%%%%%%%%%%%%%%%%%%%%%%


\latexProblemContent{
\begin{problem}

Use the Fundamental Theorem of Calculus to evaluate the integral.

\expandafter\input{\file@loc Integrals/2311-Compute-Integral-0011.HELP.tex}

\[\int_{1}^{6} {\frac{x^{2} - 16}{8 \, \sqrt{x}}}\;dx=\answer{-\frac{11}{5} \, \sqrt{6} + \frac{79}{20}}\]
\end{problem}}%}

%%%%%%%%%%%%%%%%%%%%%%


\latexProblemContent{
\begin{problem}

Use the Fundamental Theorem of Calculus to evaluate the integral.

\expandafter\input{\file@loc Integrals/2311-Compute-Integral-0011.HELP.tex}

\[\int_{7}^{15} {\frac{x^{2} + 7 \, x - 30}{9 \, \sqrt{x}}}\;dx=\answer{\frac{100}{9} \, \sqrt{15} + \frac{116}{135} \, \sqrt{7}}\]
\end{problem}}%}

%%%%%%%%%%%%%%%%%%%%%%


\latexProblemContent{
\begin{problem}

Use the Fundamental Theorem of Calculus to evaluate the integral.

\expandafter\input{\file@loc Integrals/2311-Compute-Integral-0011.HELP.tex}

\[\int_{6}^{13} {-\frac{x^{2} - 81}{5 \, \sqrt{x}}}\;dx=\answer{\frac{472}{25} \, \sqrt{13} - \frac{738}{25} \, \sqrt{6}}\]
\end{problem}}%}

%%%%%%%%%%%%%%%%%%%%%%


\latexProblemContent{
\begin{problem}

Use the Fundamental Theorem of Calculus to evaluate the integral.

\expandafter\input{\file@loc Integrals/2311-Compute-Integral-0011.HELP.tex}

\[\int_{7}^{10} {\frac{x^{2} - 9}{7 \, \sqrt{x}}}\;dx=\answer{\frac{22}{7} \, \sqrt{10} - \frac{8}{35} \, \sqrt{7}}\]
\end{problem}}%}

%%%%%%%%%%%%%%%%%%%%%%


\latexProblemContent{
\begin{problem}

Use the Fundamental Theorem of Calculus to evaluate the integral.

\expandafter\input{\file@loc Integrals/2311-Compute-Integral-0011.HELP.tex}

\[\int_{1}^{14} {\frac{x - 1}{7 \, \sqrt{x}}}\;dx=\answer{\frac{22}{21} \, \sqrt{14} + \frac{4}{21}}\]
\end{problem}}%}

%%%%%%%%%%%%%%%%%%%%%%


\latexProblemContent{
\begin{problem}

Use the Fundamental Theorem of Calculus to evaluate the integral.

\expandafter\input{\file@loc Integrals/2311-Compute-Integral-0011.HELP.tex}

\[\int_{1}^{3} {-\frac{x^{2} + 8 \, x + 16}{2 \, \sqrt{x}}}\;dx=\answer{-\frac{129}{5} \, \sqrt{3} + \frac{283}{15}}\]
\end{problem}}%}

%%%%%%%%%%%%%%%%%%%%%%


\latexProblemContent{
\begin{problem}

Use the Fundamental Theorem of Calculus to evaluate the integral.

\expandafter\input{\file@loc Integrals/2311-Compute-Integral-0011.HELP.tex}

\[\int_{4}^{12} {-\frac{x + 9}{8 \, \sqrt{x}}}\;dx=\answer{-\frac{13}{2} \, \sqrt{3} + \frac{31}{6}}\]
\end{problem}}%}

%%%%%%%%%%%%%%%%%%%%%%


\latexProblemContent{
\begin{problem}

Use the Fundamental Theorem of Calculus to evaluate the integral.

\expandafter\input{\file@loc Integrals/2311-Compute-Integral-0011.HELP.tex}

\[\int_{7}^{10} {-\frac{x^{2} - 100}{7 \, \sqrt{x}}}\;dx=\answer{\frac{160}{7} \, \sqrt{10} - \frac{902}{35} \, \sqrt{7}}\]
\end{problem}}%}

%%%%%%%%%%%%%%%%%%%%%%


\latexProblemContent{
\begin{problem}

Use the Fundamental Theorem of Calculus to evaluate the integral.

\expandafter\input{\file@loc Integrals/2311-Compute-Integral-0011.HELP.tex}

\[\int_{8}^{15} {-\frac{x + 3}{4 \, \sqrt{x}}}\;dx=\answer{-4 \, \sqrt{15} + \frac{17}{3} \, \sqrt{2}}\]
\end{problem}}%}

%%%%%%%%%%%%%%%%%%%%%%


\latexProblemContent{
\begin{problem}

Use the Fundamental Theorem of Calculus to evaluate the integral.

\expandafter\input{\file@loc Integrals/2311-Compute-Integral-0011.HELP.tex}

\[\int_{7}^{8} {\frac{x^{2} + 17 \, x + 70}{5 \, \sqrt{x}}}\;dx=\answer{-\frac{3584}{75} \, \sqrt{7} + \frac{7688}{75} \, \sqrt{2}}\]
\end{problem}}%}

%%%%%%%%%%%%%%%%%%%%%%


\latexProblemContent{
\begin{problem}

Use the Fundamental Theorem of Calculus to evaluate the integral.

\expandafter\input{\file@loc Integrals/2311-Compute-Integral-0011.HELP.tex}

\[\int_{7}^{13} {-\frac{x^{2} - 64}{6 \, \sqrt{x}}}\;dx=\answer{\frac{151}{15} \, \sqrt{13} - \frac{271}{15} \, \sqrt{7}}\]
\end{problem}}%}

%%%%%%%%%%%%%%%%%%%%%%


\latexProblemContent{
\begin{problem}

Use the Fundamental Theorem of Calculus to evaluate the integral.

\expandafter\input{\file@loc Integrals/2311-Compute-Integral-0011.HELP.tex}

\[\int_{2}^{4} {\frac{x^{2} - 100}{\sqrt{x}}}\;dx=\answer{\frac{992}{5} \, \sqrt{2} - \frac{1936}{5}}\]
\end{problem}}%}

%%%%%%%%%%%%%%%%%%%%%%


\latexProblemContent{
\begin{problem}

Use the Fundamental Theorem of Calculus to evaluate the integral.

\expandafter\input{\file@loc Integrals/2311-Compute-Integral-0011.HELP.tex}

\[\int_{7}^{15} {\frac{x + 4}{8 \, \sqrt{x}}}\;dx=\answer{\frac{9}{4} \, \sqrt{15} - \frac{19}{12} \, \sqrt{7}}\]
\end{problem}}%}

%%%%%%%%%%%%%%%%%%%%%%


\latexProblemContent{
\begin{problem}

Use the Fundamental Theorem of Calculus to evaluate the integral.

\expandafter\input{\file@loc Integrals/2311-Compute-Integral-0011.HELP.tex}

\[\int_{8}^{11} {-\frac{x - 2}{7 \, \sqrt{x}}}\;dx=\answer{-\frac{10}{21} \, \sqrt{11} + \frac{8}{21} \, \sqrt{2}}\]
\end{problem}}%}

%%%%%%%%%%%%%%%%%%%%%%


\latexProblemContent{
\begin{problem}

Use the Fundamental Theorem of Calculus to evaluate the integral.

\expandafter\input{\file@loc Integrals/2311-Compute-Integral-0011.HELP.tex}

\[\int_{4}^{11} {\frac{x - 7}{3 \, \sqrt{x}}}\;dx=\answer{-\frac{20}{9} \, \sqrt{11} + \frac{68}{9}}\]
\end{problem}}%}

%%%%%%%%%%%%%%%%%%%%%%


\latexProblemContent{
\begin{problem}

Use the Fundamental Theorem of Calculus to evaluate the integral.

\expandafter\input{\file@loc Integrals/2311-Compute-Integral-0011.HELP.tex}

\[\int_{2}^{9} {-\frac{1}{4} \, \sqrt{x}}\;dx=\answer{\frac{1}{3} \, \sqrt{2} - \frac{9}{2}}\]
\end{problem}}%}

%%%%%%%%%%%%%%%%%%%%%%


\latexProblemContent{
\begin{problem}

Use the Fundamental Theorem of Calculus to evaluate the integral.

\expandafter\input{\file@loc Integrals/2311-Compute-Integral-0011.HELP.tex}

\[\int_{6}^{8} {-\frac{x^{2} - 4 \, x - 45}{8 \, \sqrt{x}}}\;dx=\answer{-\frac{229}{20} \, \sqrt{6} + \frac{643}{30} \, \sqrt{2}}\]
\end{problem}}%}

%%%%%%%%%%%%%%%%%%%%%%


\latexProblemContent{
\begin{problem}

Use the Fundamental Theorem of Calculus to evaluate the integral.

\expandafter\input{\file@loc Integrals/2311-Compute-Integral-0011.HELP.tex}

\[\int_{8}^{13} {-\frac{x^{2} - 64}{2 \, \sqrt{x}}}\;dx=\answer{\frac{151}{5} \, \sqrt{13} - \frac{512}{5} \, \sqrt{2}}\]
\end{problem}}%}

%%%%%%%%%%%%%%%%%%%%%%


\latexProblemContent{
\begin{problem}

Use the Fundamental Theorem of Calculus to evaluate the integral.

\expandafter\input{\file@loc Integrals/2311-Compute-Integral-0011.HELP.tex}

\[\int_{1}^{3} {\frac{x - 8}{10 \, \sqrt{x}}}\;dx=\answer{-\frac{7}{5} \, \sqrt{3} + \frac{23}{15}}\]
\end{problem}}%}

%%%%%%%%%%%%%%%%%%%%%%


\latexProblemContent{
\begin{problem}

Use the Fundamental Theorem of Calculus to evaluate the integral.

\expandafter\input{\file@loc Integrals/2311-Compute-Integral-0011.HELP.tex}

\[\int_{4}^{12} {-\frac{x - 2}{4 \, \sqrt{x}}}\;dx=\answer{-2 \, \sqrt{3} - \frac{2}{3}}\]
\end{problem}}%}

%%%%%%%%%%%%%%%%%%%%%%


\latexProblemContent{
\begin{problem}

Use the Fundamental Theorem of Calculus to evaluate the integral.

\expandafter\input{\file@loc Integrals/2311-Compute-Integral-0011.HELP.tex}

\[\int_{5}^{14} {-\frac{x + 10}{7 \, \sqrt{x}}}\;dx=\answer{-\frac{88}{21} \, \sqrt{14} + \frac{10}{3} \, \sqrt{5}}\]
\end{problem}}%}

%%%%%%%%%%%%%%%%%%%%%%


\latexProblemContent{
\begin{problem}

Use the Fundamental Theorem of Calculus to evaluate the integral.

\expandafter\input{\file@loc Integrals/2311-Compute-Integral-0011.HELP.tex}

\[\int_{1}^{14} {\frac{x^{2} - 5 \, x - 50}{10 \, \sqrt{x}}}\;dx=\answer{-\frac{512}{75} \, \sqrt{14} + \frac{772}{75}}\]
\end{problem}}%}

%%%%%%%%%%%%%%%%%%%%%%


\latexProblemContent{
\begin{problem}

Use the Fundamental Theorem of Calculus to evaluate the integral.

\expandafter\input{\file@loc Integrals/2311-Compute-Integral-0011.HELP.tex}

\[\int_{4}^{11} {-\frac{x^{2} - 16}{6 \, \sqrt{x}}}\;dx=\answer{-\frac{41}{15} \, \sqrt{11} - \frac{128}{15}}\]
\end{problem}}%}

%%%%%%%%%%%%%%%%%%%%%%


\latexProblemContent{
\begin{problem}

Use the Fundamental Theorem of Calculus to evaluate the integral.

\expandafter\input{\file@loc Integrals/2311-Compute-Integral-0011.HELP.tex}

\[\int_{5}^{7} {\frac{x^{2} - 2 \, x - 48}{4 \, \sqrt{x}}}\;dx=\answer{-\frac{643}{30} \, \sqrt{7} + \frac{139}{6} \, \sqrt{5}}\]
\end{problem}}%}

%%%%%%%%%%%%%%%%%%%%%%


\latexProblemContent{
\begin{problem}

Use the Fundamental Theorem of Calculus to evaluate the integral.

\expandafter\input{\file@loc Integrals/2311-Compute-Integral-0011.HELP.tex}

\[\int_{4}^{13} {-\frac{x^{2} - 36}{8 \, \sqrt{x}}}\;dx=\answer{\frac{11}{20} \, \sqrt{13} - \frac{82}{5}}\]
\end{problem}}%}

%%%%%%%%%%%%%%%%%%%%%%


\latexProblemContent{
\begin{problem}

Use the Fundamental Theorem of Calculus to evaluate the integral.

\expandafter\input{\file@loc Integrals/2311-Compute-Integral-0011.HELP.tex}

\[\int_{4}^{8} {\frac{x^{2} + 2 \, x - 63}{6 \, \sqrt{x}}}\;dx=\answer{-\frac{1346}{45} \, \sqrt{2} + \frac{1714}{45}}\]
\end{problem}}%}

%%%%%%%%%%%%%%%%%%%%%%


\latexProblemContent{
\begin{problem}

Use the Fundamental Theorem of Calculus to evaluate the integral.

\expandafter\input{\file@loc Integrals/2311-Compute-Integral-0011.HELP.tex}

\[\int_{5}^{6} {\frac{x^{2} - 16}{3 \, \sqrt{x}}}\;dx=\answer{-\frac{88}{15} \, \sqrt{6} + \frac{22}{3} \, \sqrt{5}}\]
\end{problem}}%}

%%%%%%%%%%%%%%%%%%%%%%


\latexProblemContent{
\begin{problem}

Use the Fundamental Theorem of Calculus to evaluate the integral.

\expandafter\input{\file@loc Integrals/2311-Compute-Integral-0011.HELP.tex}

\[\int_{1}^{10} {\frac{x - 9}{2 \, \sqrt{x}}}\;dx=\answer{-\frac{17}{3} \, \sqrt{10} + \frac{26}{3}}\]
\end{problem}}%}

%%%%%%%%%%%%%%%%%%%%%%


\latexProblemContent{
\begin{problem}

Use the Fundamental Theorem of Calculus to evaluate the integral.

\expandafter\input{\file@loc Integrals/2311-Compute-Integral-0011.HELP.tex}

\[\int_{5}^{6} {-\frac{x^{2} - 81}{7 \, \sqrt{x}}}\;dx=\answer{\frac{738}{35} \, \sqrt{6} - \frac{152}{7} \, \sqrt{5}}\]
\end{problem}}%}

%%%%%%%%%%%%%%%%%%%%%%


\latexProblemContent{
\begin{problem}

Use the Fundamental Theorem of Calculus to evaluate the integral.

\expandafter\input{\file@loc Integrals/2311-Compute-Integral-0011.HELP.tex}

\[\int_{4}^{13} {\frac{x + 9}{5 \, \sqrt{x}}}\;dx=\answer{\frac{16}{3} \, \sqrt{13} - \frac{124}{15}}\]
\end{problem}}%}

%%%%%%%%%%%%%%%%%%%%%%


\latexProblemContent{
\begin{problem}

Use the Fundamental Theorem of Calculus to evaluate the integral.

\expandafter\input{\file@loc Integrals/2311-Compute-Integral-0011.HELP.tex}

\[\int_{3}^{14} {\frac{x^{2} - 9}{4 \, \sqrt{x}}}\;dx=\answer{\frac{151}{10} \, \sqrt{14} + \frac{18}{5} \, \sqrt{3}}\]
\end{problem}}%}

%%%%%%%%%%%%%%%%%%%%%%


\latexProblemContent{
\begin{problem}

Use the Fundamental Theorem of Calculus to evaluate the integral.

\expandafter\input{\file@loc Integrals/2311-Compute-Integral-0011.HELP.tex}

\[\int_{8}^{15} {-\frac{x^{2} - 1}{10 \, \sqrt{x}}}\;dx=\answer{-\frac{44}{5} \, \sqrt{15} + \frac{118}{25} \, \sqrt{2}}\]
\end{problem}}%}

%%%%%%%%%%%%%%%%%%%%%%


\latexProblemContent{
\begin{problem}

Use the Fundamental Theorem of Calculus to evaluate the integral.

\expandafter\input{\file@loc Integrals/2311-Compute-Integral-0011.HELP.tex}

\[\int_{1}^{11} {\frac{x^{2} + 5 \, x + 4}{4 \, \sqrt{x}}}\;dx=\answer{\frac{349}{15} \, \sqrt{11} - \frac{44}{15}}\]
\end{problem}}%}

%%%%%%%%%%%%%%%%%%%%%%


\latexProblemContent{
\begin{problem}

Use the Fundamental Theorem of Calculus to evaluate the integral.

\expandafter\input{\file@loc Integrals/2311-Compute-Integral-0011.HELP.tex}

\[\int_{4}^{11} {\frac{x^{2} + 8 \, x + 15}{8 \, \sqrt{x}}}\;dx=\answer{\frac{257}{15} \, \sqrt{11} - \frac{433}{30}}\]
\end{problem}}%}

%%%%%%%%%%%%%%%%%%%%%%


\latexProblemContent{
\begin{problem}

Use the Fundamental Theorem of Calculus to evaluate the integral.

\expandafter\input{\file@loc Integrals/2311-Compute-Integral-0011.HELP.tex}

\[\int_{8}^{13} {-\frac{x^{2} - 9}{3 \, \sqrt{x}}}\;dx=\answer{-\frac{248}{15} \, \sqrt{13} + \frac{76}{15} \, \sqrt{2}}\]
\end{problem}}%}

%%%%%%%%%%%%%%%%%%%%%%


\latexProblemContent{
\begin{problem}

Use the Fundamental Theorem of Calculus to evaluate the integral.

\expandafter\input{\file@loc Integrals/2311-Compute-Integral-0011.HELP.tex}

\[\int_{4}^{11} {-\frac{x^{2} - 16}{\sqrt{x}}}\;dx=\answer{-\frac{82}{5} \, \sqrt{11} - \frac{256}{5}}\]
\end{problem}}%}

%%%%%%%%%%%%%%%%%%%%%%


\latexProblemContent{
\begin{problem}

Use the Fundamental Theorem of Calculus to evaluate the integral.

\expandafter\input{\file@loc Integrals/2311-Compute-Integral-0011.HELP.tex}

\[\int_{1}^{2} {-\frac{x^{2} + 4 \, x + 3}{6 \, \sqrt{x}}}\;dx=\answer{-\frac{97}{45} \, \sqrt{2} + \frac{68}{45}}\]
\end{problem}}%}

%%%%%%%%%%%%%%%%%%%%%%


\latexProblemContent{
\begin{problem}

Use the Fundamental Theorem of Calculus to evaluate the integral.

\expandafter\input{\file@loc Integrals/2311-Compute-Integral-0011.HELP.tex}

\[\int_{4}^{12} {-\frac{x^{2} - 8 \, x - 9}{4 \, \sqrt{x}}}\;dx=\answer{\frac{61}{5} \, \sqrt{3} - \frac{247}{15}}\]
\end{problem}}%}

%%%%%%%%%%%%%%%%%%%%%%


\latexProblemContent{
\begin{problem}

Use the Fundamental Theorem of Calculus to evaluate the integral.

\expandafter\input{\file@loc Integrals/2311-Compute-Integral-0011.HELP.tex}

\[\int_{6}^{15} {-\frac{x - 4}{2 \, \sqrt{x}}}\;dx=\answer{-\sqrt{15} - 2 \, \sqrt{6}}\]
\end{problem}}%}

%%%%%%%%%%%%%%%%%%%%%%


\latexProblemContent{
\begin{problem}

Use the Fundamental Theorem of Calculus to evaluate the integral.

\expandafter\input{\file@loc Integrals/2311-Compute-Integral-0011.HELP.tex}

\[\int_{4}^{6} {\frac{x + 8}{10 \, \sqrt{x}}}\;dx=\answer{2 \, \sqrt{6} - \frac{56}{15}}\]
\end{problem}}%}

%%%%%%%%%%%%%%%%%%%%%%


\latexProblemContent{
\begin{problem}

Use the Fundamental Theorem of Calculus to evaluate the integral.

\expandafter\input{\file@loc Integrals/2311-Compute-Integral-0011.HELP.tex}

\[\int_{5}^{6} {\frac{x^{2} + 3 \, x - 28}{3 \, \sqrt{x}}}\;dx=\answer{-\frac{148}{15} \, \sqrt{6} + 12 \, \sqrt{5}}\]
\end{problem}}%}

%%%%%%%%%%%%%%%%%%%%%%


\latexProblemContent{
\begin{problem}

Use the Fundamental Theorem of Calculus to evaluate the integral.

\expandafter\input{\file@loc Integrals/2311-Compute-Integral-0011.HELP.tex}

\[\int_{8}^{13} {-\frac{x + 6}{8 \, \sqrt{x}}}\;dx=\answer{-\frac{31}{12} \, \sqrt{13} + \frac{13}{3} \, \sqrt{2}}\]
\end{problem}}%}

%%%%%%%%%%%%%%%%%%%%%%


\latexProblemContent{
\begin{problem}

Use the Fundamental Theorem of Calculus to evaluate the integral.

\expandafter\input{\file@loc Integrals/2311-Compute-Integral-0011.HELP.tex}

\[\int_{4}^{7} {\frac{x^{2} - 1}{6 \, \sqrt{x}}}\;dx=\answer{\frac{44}{15} \, \sqrt{7} - \frac{22}{15}}\]
\end{problem}}%}

%%%%%%%%%%%%%%%%%%%%%%


\latexProblemContent{
\begin{problem}

Use the Fundamental Theorem of Calculus to evaluate the integral.

\expandafter\input{\file@loc Integrals/2311-Compute-Integral-0011.HELP.tex}

\[\int_{5}^{8} {-\frac{x^{2} - 1}{3 \, \sqrt{x}}}\;dx=\answer{\frac{8}{3} \, \sqrt{5} - \frac{236}{15} \, \sqrt{2}}\]
\end{problem}}%}

%%%%%%%%%%%%%%%%%%%%%%


\latexProblemContent{
\begin{problem}

Use the Fundamental Theorem of Calculus to evaluate the integral.

\expandafter\input{\file@loc Integrals/2311-Compute-Integral-0011.HELP.tex}

\[\int_{3}^{9} {-\frac{1}{9} \, x^{\frac{3}{2}}}\;dx=\answer{\frac{2}{5} \, \sqrt{3} - \frac{54}{5}}\]
\end{problem}}%}

%%%%%%%%%%%%%%%%%%%%%%


\latexProblemContent{
\begin{problem}

Use the Fundamental Theorem of Calculus to evaluate the integral.

\expandafter\input{\file@loc Integrals/2311-Compute-Integral-0011.HELP.tex}

\[\int_{8}^{11} {\frac{x - 2}{7 \, \sqrt{x}}}\;dx=\answer{\frac{10}{21} \, \sqrt{11} - \frac{8}{21} \, \sqrt{2}}\]
\end{problem}}%}

%%%%%%%%%%%%%%%%%%%%%%


\latexProblemContent{
\begin{problem}

Use the Fundamental Theorem of Calculus to evaluate the integral.

\expandafter\input{\file@loc Integrals/2311-Compute-Integral-0011.HELP.tex}

\[\int_{2}^{13} {-\frac{x^{2} - 10 \, x + 24}{4 \, \sqrt{x}}}\;dx=\answer{-\frac{217}{30} \, \sqrt{13} + \frac{136}{15} \, \sqrt{2}}\]
\end{problem}}%}

%%%%%%%%%%%%%%%%%%%%%%


\latexProblemContent{
\begin{problem}

Use the Fundamental Theorem of Calculus to evaluate the integral.

\expandafter\input{\file@loc Integrals/2311-Compute-Integral-0011.HELP.tex}

\[\int_{4}^{14} {\frac{x - 6}{4 \, \sqrt{x}}}\;dx=\answer{-\frac{2}{3} \, \sqrt{14} + \frac{14}{3}}\]
\end{problem}}%}

%%%%%%%%%%%%%%%%%%%%%%


\latexProblemContent{
\begin{problem}

Use the Fundamental Theorem of Calculus to evaluate the integral.

\expandafter\input{\file@loc Integrals/2311-Compute-Integral-0011.HELP.tex}

\[\int_{1}^{9} {-\frac{x^{2} - 6 \, x}{9 \, \sqrt{x}}}\;dx=\answer{\frac{4}{5}}\]
\end{problem}}%}

%%%%%%%%%%%%%%%%%%%%%%


\latexProblemContent{
\begin{problem}

Use the Fundamental Theorem of Calculus to evaluate the integral.

\expandafter\input{\file@loc Integrals/2311-Compute-Integral-0011.HELP.tex}

\[\int_{5}^{14} {\frac{x^{2} - 1}{7 \, \sqrt{x}}}\;dx=\answer{\frac{382}{35} \, \sqrt{14} - \frac{8}{7} \, \sqrt{5}}\]
\end{problem}}%}

%%%%%%%%%%%%%%%%%%%%%%


\latexProblemContent{
\begin{problem}

Use the Fundamental Theorem of Calculus to evaluate the integral.

\expandafter\input{\file@loc Integrals/2311-Compute-Integral-0011.HELP.tex}

\[\int_{5}^{15} {\frac{x^{2} - 81}{7 \, \sqrt{x}}}\;dx=\answer{-\frac{72}{7} \, \sqrt{15} + \frac{152}{7} \, \sqrt{5}}\]
\end{problem}}%}

%%%%%%%%%%%%%%%%%%%%%%


\latexProblemContent{
\begin{problem}

Use the Fundamental Theorem of Calculus to evaluate the integral.

\expandafter\input{\file@loc Integrals/2311-Compute-Integral-0011.HELP.tex}

\[\int_{1}^{4} {\frac{x + 1}{10 \, \sqrt{x}}}\;dx=\answer{\frac{2}{3}}\]
\end{problem}}%}

%%%%%%%%%%%%%%%%%%%%%%


\latexProblemContent{
\begin{problem}

Use the Fundamental Theorem of Calculus to evaluate the integral.

\expandafter\input{\file@loc Integrals/2311-Compute-Integral-0011.HELP.tex}

\[\int_{8}^{9} {\frac{x^{2} - 7 \, x + 10}{8 \, \sqrt{x}}}\;dx=\answer{-\frac{31}{15} \, \sqrt{2} + \frac{39}{10}}\]
\end{problem}}%}

%%%%%%%%%%%%%%%%%%%%%%


\latexProblemContent{
\begin{problem}

Use the Fundamental Theorem of Calculus to evaluate the integral.

\expandafter\input{\file@loc Integrals/2311-Compute-Integral-0011.HELP.tex}

\[\int_{2}^{4} {\frac{x - 6}{5 \, \sqrt{x}}}\;dx=\answer{\frac{1}{15} \, \left(32 \, \sqrt{2}\right) - \frac{56}{15}}\]
\end{problem}}%}

%%%%%%%%%%%%%%%%%%%%%%


\latexProblemContent{
\begin{problem}

Use the Fundamental Theorem of Calculus to evaluate the integral.

\expandafter\input{\file@loc Integrals/2311-Compute-Integral-0011.HELP.tex}

\[\int_{3}^{7} {-\frac{x^{2} - 4}{10 \, \sqrt{x}}}\;dx=\answer{-\frac{29}{25} \, \sqrt{7} - \frac{11}{25} \, \sqrt{3}}\]
\end{problem}}%}

%%%%%%%%%%%%%%%%%%%%%%


\latexProblemContent{
\begin{problem}

Use the Fundamental Theorem of Calculus to evaluate the integral.

\expandafter\input{\file@loc Integrals/2311-Compute-Integral-0011.HELP.tex}

\[\int_{6}^{14} {-\frac{x^{2} + 3 \, x + 2}{7 \, \sqrt{x}}}\;dx=\answer{-\frac{552}{35} \, \sqrt{14} + \frac{152}{35} \, \sqrt{6}}\]
\end{problem}}%}

%%%%%%%%%%%%%%%%%%%%%%


\latexProblemContent{
\begin{problem}

Use the Fundamental Theorem of Calculus to evaluate the integral.

\expandafter\input{\file@loc Integrals/2311-Compute-Integral-0011.HELP.tex}

\[\int_{1}^{14} {\frac{x + 8}{2 \, \sqrt{x}}}\;dx=\answer{\frac{38}{3} \, \sqrt{14} - \frac{25}{3}}\]
\end{problem}}%}

%%%%%%%%%%%%%%%%%%%%%%


\latexProblemContent{
\begin{problem}

Use the Fundamental Theorem of Calculus to evaluate the integral.

\expandafter\input{\file@loc Integrals/2311-Compute-Integral-0011.HELP.tex}

\[\int_{1}^{2} {-\frac{x^{2} - 16}{7 \, \sqrt{x}}}\;dx=\answer{\frac{152}{35} \, \sqrt{2} - \frac{158}{35}}\]
\end{problem}}%}

%%%%%%%%%%%%%%%%%%%%%%


\latexProblemContent{
\begin{problem}

Use the Fundamental Theorem of Calculus to evaluate the integral.

\expandafter\input{\file@loc Integrals/2311-Compute-Integral-0011.HELP.tex}

\[\int_{7}^{12} {-\frac{x^{2} - 7 \, x + 12}{10 \, \sqrt{x}}}\;dx=\answer{\frac{82}{75} \, \sqrt{7} - \frac{128}{25} \, \sqrt{3}}\]
\end{problem}}%}

%%%%%%%%%%%%%%%%%%%%%%


\latexProblemContent{
\begin{problem}

Use the Fundamental Theorem of Calculus to evaluate the integral.

\expandafter\input{\file@loc Integrals/2311-Compute-Integral-0011.HELP.tex}

\[\int_{5}^{6} {\frac{x^{2} - 64}{\sqrt{x}}}\;dx=\answer{-\frac{568}{5} \, \sqrt{6} + 118 \, \sqrt{5}}\]
\end{problem}}%}

%%%%%%%%%%%%%%%%%%%%%%


\latexProblemContent{
\begin{problem}

Use the Fundamental Theorem of Calculus to evaluate the integral.

\expandafter\input{\file@loc Integrals/2311-Compute-Integral-0011.HELP.tex}

\[\int_{2}^{14} {\frac{x^{2} + 11 \, x + 10}{\sqrt{x}}}\;dx=\answer{\frac{3016}{15} \, \sqrt{14} - \frac{544}{15} \, \sqrt{2}}\]
\end{problem}}%}

%%%%%%%%%%%%%%%%%%%%%%


\latexProblemContent{
\begin{problem}

Use the Fundamental Theorem of Calculus to evaluate the integral.

\expandafter\input{\file@loc Integrals/2311-Compute-Integral-0011.HELP.tex}

\[\int_{3}^{15} {-\frac{x^{2} - 64}{\sqrt{x}}}\;dx=\answer{38 \, \sqrt{15} - \frac{622}{5} \, \sqrt{3}}\]
\end{problem}}%}

%%%%%%%%%%%%%%%%%%%%%%


\latexProblemContent{
\begin{problem}

Use the Fundamental Theorem of Calculus to evaluate the integral.

\expandafter\input{\file@loc Integrals/2311-Compute-Integral-0011.HELP.tex}

\[\int_{5}^{6} {-\frac{x^{2} - 25}{5 \, \sqrt{x}}}\;dx=\answer{\frac{178}{25} \, \sqrt{6} - 8 \, \sqrt{5}}\]
\end{problem}}%}

%%%%%%%%%%%%%%%%%%%%%%


\latexProblemContent{
\begin{problem}

Use the Fundamental Theorem of Calculus to evaluate the integral.

\expandafter\input{\file@loc Integrals/2311-Compute-Integral-0011.HELP.tex}

\[\int_{4}^{9} {\frac{x^{2} - 16}{6 \, \sqrt{x}}}\;dx=\answer{\frac{131}{15}}\]
\end{problem}}%}

%%%%%%%%%%%%%%%%%%%%%%


\latexProblemContent{
\begin{problem}

Use the Fundamental Theorem of Calculus to evaluate the integral.

\expandafter\input{\file@loc Integrals/2311-Compute-Integral-0011.HELP.tex}

\[\int_{6}^{14} {\frac{x - 7}{9 \, \sqrt{x}}}\;dx=\answer{-\frac{1}{27} \, \left(14 \, \sqrt{14}\right) + \frac{10}{9} \, \sqrt{6}}\]
\end{problem}}%}

%%%%%%%%%%%%%%%%%%%%%%


\latexProblemContent{
\begin{problem}

Use the Fundamental Theorem of Calculus to evaluate the integral.

\expandafter\input{\file@loc Integrals/2311-Compute-Integral-0011.HELP.tex}

\[\int_{1}^{2} {-\frac{x^{2} + 16 \, x + 60}{\sqrt{x}}}\;dx=\answer{-\frac{2144}{15} \, \sqrt{2} + \frac{1966}{15}}\]
\end{problem}}%}

%%%%%%%%%%%%%%%%%%%%%%


\latexProblemContent{
\begin{problem}

Use the Fundamental Theorem of Calculus to evaluate the integral.

\expandafter\input{\file@loc Integrals/2311-Compute-Integral-0011.HELP.tex}

\[\int_{6}^{11} {\frac{x - 4}{4 \, \sqrt{x}}}\;dx=\answer{-\frac{1}{6} \, \sqrt{11} + \sqrt{6}}\]
\end{problem}}%}

%%%%%%%%%%%%%%%%%%%%%%


\latexProblemContent{
\begin{problem}

Use the Fundamental Theorem of Calculus to evaluate the integral.

\expandafter\input{\file@loc Integrals/2311-Compute-Integral-0011.HELP.tex}

\[\int_{4}^{11} {\frac{x - 9}{\sqrt{x}}}\;dx=\answer{-\frac{32}{3} \, \sqrt{11} + \frac{92}{3}}\]
\end{problem}}%}

%%%%%%%%%%%%%%%%%%%%%%


\latexProblemContent{
\begin{problem}

Use the Fundamental Theorem of Calculus to evaluate the integral.

\expandafter\input{\file@loc Integrals/2311-Compute-Integral-0011.HELP.tex}

\[\int_{7}^{12} {-\frac{x + 9}{8 \, \sqrt{x}}}\;dx=\answer{\frac{17}{6} \, \sqrt{7} - \frac{13}{2} \, \sqrt{3}}\]
\end{problem}}%}

%%%%%%%%%%%%%%%%%%%%%%


\latexProblemContent{
\begin{problem}

Use the Fundamental Theorem of Calculus to evaluate the integral.

\expandafter\input{\file@loc Integrals/2311-Compute-Integral-0011.HELP.tex}

\[\int_{1}^{8} {-\frac{x - 5}{10 \, \sqrt{x}}}\;dx=\answer{\frac{14}{15} \, \sqrt{2} - \frac{14}{15}}\]
\end{problem}}%}

%%%%%%%%%%%%%%%%%%%%%%


\latexProblemContent{
\begin{problem}

Use the Fundamental Theorem of Calculus to evaluate the integral.

\expandafter\input{\file@loc Integrals/2311-Compute-Integral-0011.HELP.tex}

\[\int_{6}^{12} {\frac{x + 1}{10 \, \sqrt{x}}}\;dx=\answer{-\frac{3}{5} \, \sqrt{6} + 2 \, \sqrt{3}}\]
\end{problem}}%}

%%%%%%%%%%%%%%%%%%%%%%


\latexProblemContent{
\begin{problem}

Use the Fundamental Theorem of Calculus to evaluate the integral.

\expandafter\input{\file@loc Integrals/2311-Compute-Integral-0011.HELP.tex}

\[\int_{5}^{13} {\frac{x - 5}{6 \, \sqrt{x}}}\;dx=\answer{-\frac{2}{9} \, \sqrt{13} + \frac{10}{9} \, \sqrt{5}}\]
\end{problem}}%}

%%%%%%%%%%%%%%%%%%%%%%


\latexProblemContent{
\begin{problem}

Use the Fundamental Theorem of Calculus to evaluate the integral.

\expandafter\input{\file@loc Integrals/2311-Compute-Integral-0011.HELP.tex}

\[\int_{4}^{10} {\frac{x^{2} - 64}{8 \, \sqrt{x}}}\;dx=\answer{-11 \, \sqrt{10} + \frac{152}{5}}\]
\end{problem}}%}

%%%%%%%%%%%%%%%%%%%%%%


\latexProblemContent{
\begin{problem}

Use the Fundamental Theorem of Calculus to evaluate the integral.

\expandafter\input{\file@loc Integrals/2311-Compute-Integral-0011.HELP.tex}

\[\int_{1}^{14} {-\frac{1}{3} \, x^{\frac{3}{2}}}\;dx=\answer{-\frac{392}{15} \, \sqrt{14} + \frac{2}{15}}\]
\end{problem}}%}

%%%%%%%%%%%%%%%%%%%%%%


\latexProblemContent{
\begin{problem}

Use the Fundamental Theorem of Calculus to evaluate the integral.

\expandafter\input{\file@loc Integrals/2311-Compute-Integral-0011.HELP.tex}

\[\int_{8}^{9} {\frac{x + 6}{10 \, \sqrt{x}}}\;dx=\answer{-\frac{52}{15} \, \sqrt{2} + \frac{27}{5}}\]
\end{problem}}%}

%%%%%%%%%%%%%%%%%%%%%%


\latexProblemContent{
\begin{problem}

Use the Fundamental Theorem of Calculus to evaluate the integral.

\expandafter\input{\file@loc Integrals/2311-Compute-Integral-0011.HELP.tex}

\[\int_{4}^{15} {-\frac{x^{2} - 25}{5 \, \sqrt{x}}}\;dx=\answer{-8 \, \sqrt{15} - \frac{436}{25}}\]
\end{problem}}%}

%%%%%%%%%%%%%%%%%%%%%%


\latexProblemContent{
\begin{problem}

Use the Fundamental Theorem of Calculus to evaluate the integral.

\expandafter\input{\file@loc Integrals/2311-Compute-Integral-0011.HELP.tex}

\[\int_{1}^{11} {-\frac{x^{2} - 64}{2 \, \sqrt{x}}}\;dx=\answer{\frac{199}{5} \, \sqrt{11} - \frac{319}{5}}\]
\end{problem}}%}

%%%%%%%%%%%%%%%%%%%%%%


\latexProblemContent{
\begin{problem}

Use the Fundamental Theorem of Calculus to evaluate the integral.

\expandafter\input{\file@loc Integrals/2311-Compute-Integral-0011.HELP.tex}

\[\int_{4}^{10} {\frac{x - 9}{3 \, \sqrt{x}}}\;dx=\answer{-\frac{34}{9} \, \sqrt{10} + \frac{92}{9}}\]
\end{problem}}%}

%%%%%%%%%%%%%%%%%%%%%%


\latexProblemContent{
\begin{problem}

Use the Fundamental Theorem of Calculus to evaluate the integral.

\expandafter\input{\file@loc Integrals/2311-Compute-Integral-0011.HELP.tex}

\[\int_{8}^{14} {-\frac{x^{2} - 17 \, x + 72}{5 \, \sqrt{x}}}\;dx=\answer{-\frac{956}{75} \, \sqrt{14} + \frac{2368}{75} \, \sqrt{2}}\]
\end{problem}}%}

%%%%%%%%%%%%%%%%%%%%%%


\latexProblemContent{
\begin{problem}

Use the Fundamental Theorem of Calculus to evaluate the integral.

\expandafter\input{\file@loc Integrals/2311-Compute-Integral-0011.HELP.tex}

\[\int_{1}^{7} {-\frac{x^{2} - 100}{\sqrt{x}}}\;dx=\answer{\frac{902}{5} \, \sqrt{7} - \frac{998}{5}}\]
\end{problem}}%}

%%%%%%%%%%%%%%%%%%%%%%


\latexProblemContent{
\begin{problem}

Use the Fundamental Theorem of Calculus to evaluate the integral.

\expandafter\input{\file@loc Integrals/2311-Compute-Integral-0011.HELP.tex}

\[\int_{2}^{14} {-\frac{x^{2} - 4}{9 \, \sqrt{x}}}\;dx=\answer{-\frac{1}{45} \, \left(32 \, \sqrt{2}\right) - \frac{352}{45} \, \sqrt{14}}\]
\end{problem}}%}

%%%%%%%%%%%%%%%%%%%%%%


\latexProblemContent{
\begin{problem}

Use the Fundamental Theorem of Calculus to evaluate the integral.

\expandafter\input{\file@loc Integrals/2311-Compute-Integral-0011.HELP.tex}

\[\int_{8}^{11} {\frac{x^{2} - 81}{2 \, \sqrt{x}}}\;dx=\answer{-\frac{284}{5} \, \sqrt{11} + \frac{682}{5} \, \sqrt{2}}\]
\end{problem}}%}

%%%%%%%%%%%%%%%%%%%%%%


\latexProblemContent{
\begin{problem}

Use the Fundamental Theorem of Calculus to evaluate the integral.

\expandafter\input{\file@loc Integrals/2311-Compute-Integral-0011.HELP.tex}

\[\int_{4}^{13} {\frac{x^{2} - 9}{2 \, \sqrt{x}}}\;dx=\answer{\frac{124}{5} \, \sqrt{13} + \frac{58}{5}}\]
\end{problem}}%}

%%%%%%%%%%%%%%%%%%%%%%


\latexProblemContent{
\begin{problem}

Use the Fundamental Theorem of Calculus to evaluate the integral.

\expandafter\input{\file@loc Integrals/2311-Compute-Integral-0011.HELP.tex}

\[\int_{7}^{9} {\frac{x + 6}{7 \, \sqrt{x}}}\;dx=\answer{-\frac{50}{21} \, \sqrt{7} + \frac{54}{7}}\]
\end{problem}}%}

%%%%%%%%%%%%%%%%%%%%%%


\latexProblemContent{
\begin{problem}

Use the Fundamental Theorem of Calculus to evaluate the integral.

\expandafter\input{\file@loc Integrals/2311-Compute-Integral-0011.HELP.tex}

\[\int_{5}^{6} {-\frac{x + 3}{7 \, \sqrt{x}}}\;dx=\answer{-\frac{10}{7} \, \sqrt{6} + \frac{4}{3} \, \sqrt{5}}\]
\end{problem}}%}

%%%%%%%%%%%%%%%%%%%%%%


\latexProblemContent{
\begin{problem}

Use the Fundamental Theorem of Calculus to evaluate the integral.

\expandafter\input{\file@loc Integrals/2311-Compute-Integral-0011.HELP.tex}

\[\int_{3}^{13} {-\frac{x^{2} + 7 \, x - 30}{\sqrt{x}}}\;dx=\answer{-\frac{1024}{15} \, \sqrt{13} - \frac{212}{5} \, \sqrt{3}}\]
\end{problem}}%}

%%%%%%%%%%%%%%%%%%%%%%


\latexProblemContent{
\begin{problem}

Use the Fundamental Theorem of Calculus to evaluate the integral.

\expandafter\input{\file@loc Integrals/2311-Compute-Integral-0011.HELP.tex}

\[\int_{6}^{11} {-\frac{x^{2} - 49}{5 \, \sqrt{x}}}\;dx=\answer{\frac{248}{25} \, \sqrt{11} - \frac{418}{25} \, \sqrt{6}}\]
\end{problem}}%}

%%%%%%%%%%%%%%%%%%%%%%


\latexProblemContent{
\begin{problem}

Use the Fundamental Theorem of Calculus to evaluate the integral.

\expandafter\input{\file@loc Integrals/2311-Compute-Integral-0011.HELP.tex}

\[\int_{5}^{15} {\frac{x + 7}{5 \, \sqrt{x}}}\;dx=\answer{\frac{24}{5} \, \sqrt{15} - \frac{52}{15} \, \sqrt{5}}\]
\end{problem}}%}

%%%%%%%%%%%%%%%%%%%%%%


\latexProblemContent{
\begin{problem}

Use the Fundamental Theorem of Calculus to evaluate the integral.

\expandafter\input{\file@loc Integrals/2311-Compute-Integral-0011.HELP.tex}

\[\int_{6}^{10} {-\frac{x^{2} - 16}{3 \, \sqrt{x}}}\;dx=\answer{-\frac{8}{3} \, \sqrt{10} - \frac{88}{15} \, \sqrt{6}}\]
\end{problem}}%}

%%%%%%%%%%%%%%%%%%%%%%


\latexProblemContent{
\begin{problem}

Use the Fundamental Theorem of Calculus to evaluate the integral.

\expandafter\input{\file@loc Integrals/2311-Compute-Integral-0011.HELP.tex}

\[\int_{5}^{14} {-\frac{x^{2} - 6 \, x - 27}{10 \, \sqrt{x}}}\;dx=\answer{\frac{79}{25} \, \sqrt{14} - \frac{32}{5} \, \sqrt{5}}\]
\end{problem}}%}

%%%%%%%%%%%%%%%%%%%%%%


\latexProblemContent{
\begin{problem}

Use the Fundamental Theorem of Calculus to evaluate the integral.

\expandafter\input{\file@loc Integrals/2311-Compute-Integral-0011.HELP.tex}

\[\int_{1}^{4} {\frac{x^{2} - 7 \, x + 6}{\sqrt{x}}}\;dx=\answer{-\frac{124}{15}}\]
\end{problem}}%}

%%%%%%%%%%%%%%%%%%%%%%


\latexProblemContent{
\begin{problem}

Use the Fundamental Theorem of Calculus to evaluate the integral.

\expandafter\input{\file@loc Integrals/2311-Compute-Integral-0011.HELP.tex}

\[\int_{8}^{11} {\frac{x^{2} - 1}{4 \, \sqrt{x}}}\;dx=\answer{\frac{58}{5} \, \sqrt{11} - \frac{59}{5} \, \sqrt{2}}\]
\end{problem}}%}

%%%%%%%%%%%%%%%%%%%%%%


\latexProblemContent{
\begin{problem}

Use the Fundamental Theorem of Calculus to evaluate the integral.

\expandafter\input{\file@loc Integrals/2311-Compute-Integral-0011.HELP.tex}

\[\int_{4}^{8} {\frac{x^{2} - 36}{2 \, \sqrt{x}}}\;dx=\answer{-\frac{232}{5} \, \sqrt{2} + \frac{328}{5}}\]
\end{problem}}%}

%%%%%%%%%%%%%%%%%%%%%%


\latexProblemContent{
\begin{problem}

Use the Fundamental Theorem of Calculus to evaluate the integral.

\expandafter\input{\file@loc Integrals/2311-Compute-Integral-0011.HELP.tex}

\[\int_{2}^{10} {\frac{x^{2} + 2 \, x - 15}{2 \, \sqrt{x}}}\;dx=\answer{\frac{35}{3} \, \sqrt{10} + \frac{193}{15} \, \sqrt{2}}\]
\end{problem}}%}

%%%%%%%%%%%%%%%%%%%%%%


\latexProblemContent{
\begin{problem}

Use the Fundamental Theorem of Calculus to evaluate the integral.

\expandafter\input{\file@loc Integrals/2311-Compute-Integral-0011.HELP.tex}

\[\int_{1}^{2} {-\frac{x^{2} - 4}{10 \, \sqrt{x}}}\;dx=\answer{\frac{1}{50} \, \left(32 \, \sqrt{2}\right) - \frac{19}{25}}\]
\end{problem}}%}

%%%%%%%%%%%%%%%%%%%%%%


\latexProblemContent{
\begin{problem}

Use the Fundamental Theorem of Calculus to evaluate the integral.

\expandafter\input{\file@loc Integrals/2311-Compute-Integral-0011.HELP.tex}

\[\int_{2}^{8} {-\frac{x - 9}{7 \, \sqrt{x}}}\;dx=\answer{\frac{13}{21} \, \left(2 \, \sqrt{2}\right)}\]
\end{problem}}%}

%%%%%%%%%%%%%%%%%%%%%%


\latexProblemContent{
\begin{problem}

Use the Fundamental Theorem of Calculus to evaluate the integral.

\expandafter\input{\file@loc Integrals/2311-Compute-Integral-0011.HELP.tex}

\[\int_{8}^{15} {-\frac{x^{2} + 5 \, x + 4}{9 \, \sqrt{x}}}\;dx=\answer{-\frac{148}{9} \, \sqrt{15} + \frac{1808}{135} \, \sqrt{2}}\]
\end{problem}}%}

%%%%%%%%%%%%%%%%%%%%%%


\latexProblemContent{
\begin{problem}

Use the Fundamental Theorem of Calculus to evaluate the integral.

\expandafter\input{\file@loc Integrals/2311-Compute-Integral-0011.HELP.tex}

\[\int_{1}^{3} {\frac{x^{2} - 9}{4 \, \sqrt{x}}}\;dx=\answer{-\frac{18}{5} \, \sqrt{3} + \frac{22}{5}}\]
\end{problem}}%}

%%%%%%%%%%%%%%%%%%%%%%


\latexProblemContent{
\begin{problem}

Use the Fundamental Theorem of Calculus to evaluate the integral.

\expandafter\input{\file@loc Integrals/2311-Compute-Integral-0011.HELP.tex}

\[\int_{2}^{3} {-\frac{x - 3}{9 \, \sqrt{x}}}\;dx=\answer{\frac{4}{9} \, \sqrt{3} - \frac{14}{27} \, \sqrt{2}}\]
\end{problem}}%}

%%%%%%%%%%%%%%%%%%%%%%


\latexProblemContent{
\begin{problem}

Use the Fundamental Theorem of Calculus to evaluate the integral.

\expandafter\input{\file@loc Integrals/2311-Compute-Integral-0011.HELP.tex}

\[\int_{3}^{10} {-\frac{x^{2} - 9}{6 \, \sqrt{x}}}\;dx=\answer{-\frac{11}{3} \, \sqrt{10} - \frac{12}{5} \, \sqrt{3}}\]
\end{problem}}%}

%%%%%%%%%%%%%%%%%%%%%%


\latexProblemContent{
\begin{problem}

Use the Fundamental Theorem of Calculus to evaluate the integral.

\expandafter\input{\file@loc Integrals/2311-Compute-Integral-0011.HELP.tex}

\[\int_{4}^{14} {\frac{x - 9}{7 \, \sqrt{x}}}\;dx=\answer{-\frac{26}{21} \, \sqrt{14} + \frac{92}{21}}\]
\end{problem}}%}

%%%%%%%%%%%%%%%%%%%%%%


\latexProblemContent{
\begin{problem}

Use the Fundamental Theorem of Calculus to evaluate the integral.

\expandafter\input{\file@loc Integrals/2311-Compute-Integral-0011.HELP.tex}

\[\int_{8}^{15} {\frac{x^{2} - 25}{7 \, \sqrt{x}}}\;dx=\answer{\frac{40}{7} \, \sqrt{15} + \frac{244}{35} \, \sqrt{2}}\]
\end{problem}}%}

%%%%%%%%%%%%%%%%%%%%%%


\latexProblemContent{
\begin{problem}

Use the Fundamental Theorem of Calculus to evaluate the integral.

\expandafter\input{\file@loc Integrals/2311-Compute-Integral-0011.HELP.tex}

\[\int_{4}^{11} {-\frac{x^{2} - 4}{9 \, \sqrt{x}}}\;dx=\answer{-\frac{202}{45} \, \sqrt{11} - \frac{16}{45}}\]
\end{problem}}%}

%%%%%%%%%%%%%%%%%%%%%%


\latexProblemContent{
\begin{problem}

Use the Fundamental Theorem of Calculus to evaluate the integral.

\expandafter\input{\file@loc Integrals/2311-Compute-Integral-0011.HELP.tex}

\[\int_{7}^{15} {-\frac{x + 4}{5 \, \sqrt{x}}}\;dx=\answer{-\frac{18}{5} \, \sqrt{15} + \frac{38}{15} \, \sqrt{7}}\]
\end{problem}}%}

%%%%%%%%%%%%%%%%%%%%%%


\latexProblemContent{
\begin{problem}

Use the Fundamental Theorem of Calculus to evaluate the integral.

\expandafter\input{\file@loc Integrals/2311-Compute-Integral-0011.HELP.tex}

\[\int_{7}^{8} {-\frac{x^{2} - 25}{5 \, \sqrt{x}}}\;dx=\answer{-\frac{152}{25} \, \sqrt{7} + \frac{244}{25} \, \sqrt{2}}\]
\end{problem}}%}

%%%%%%%%%%%%%%%%%%%%%%


\latexProblemContent{
\begin{problem}

Use the Fundamental Theorem of Calculus to evaluate the integral.

\expandafter\input{\file@loc Integrals/2311-Compute-Integral-0011.HELP.tex}

\[\int_{5}^{8} {-\frac{1}{5} \, x^{\frac{3}{2}}}\;dx=\answer{2 \, \sqrt{5} - \frac{256}{25} \, \sqrt{2}}\]
\end{problem}}%}

%%%%%%%%%%%%%%%%%%%%%%


\latexProblemContent{
\begin{problem}

Use the Fundamental Theorem of Calculus to evaluate the integral.

\expandafter\input{\file@loc Integrals/2311-Compute-Integral-0011.HELP.tex}

\[\int_{7}^{12} {\frac{x^{2} - 4}{3 \, \sqrt{x}}}\;dx=\answer{-\frac{58}{15} \, \sqrt{7} + \frac{496}{15} \, \sqrt{3}}\]
\end{problem}}%}

%%%%%%%%%%%%%%%%%%%%%%


\latexProblemContent{
\begin{problem}

Use the Fundamental Theorem of Calculus to evaluate the integral.

\expandafter\input{\file@loc Integrals/2311-Compute-Integral-0011.HELP.tex}

\[\int_{2}^{9} {\frac{x + 6}{9 \, \sqrt{x}}}\;dx=\answer{-\frac{40}{27} \, \sqrt{2} + 6}\]
\end{problem}}%}

%%%%%%%%%%%%%%%%%%%%%%


\latexProblemContent{
\begin{problem}

Use the Fundamental Theorem of Calculus to evaluate the integral.

\expandafter\input{\file@loc Integrals/2311-Compute-Integral-0011.HELP.tex}

\[\int_{3}^{14} {\frac{x^{2} + 7 \, x - 30}{9 \, \sqrt{x}}}\;dx=\answer{\frac{1256}{135} \, \sqrt{14} + \frac{212}{45} \, \sqrt{3}}\]
\end{problem}}%}

%%%%%%%%%%%%%%%%%%%%%%


\latexProblemContent{
\begin{problem}

Use the Fundamental Theorem of Calculus to evaluate the integral.

\expandafter\input{\file@loc Integrals/2311-Compute-Integral-0011.HELP.tex}

\[\int_{6}^{14} {\frac{x - 3}{2 \, \sqrt{x}}}\;dx=\answer{\frac{1}{6} \, \left(6 \, \sqrt{6}\right) + \frac{5}{3} \, \sqrt{14}}\]
\end{problem}}%}

%%%%%%%%%%%%%%%%%%%%%%


\latexProblemContent{
\begin{problem}

Use the Fundamental Theorem of Calculus to evaluate the integral.

\expandafter\input{\file@loc Integrals/2311-Compute-Integral-0011.HELP.tex}

\[\int_{3}^{9} {-\frac{x + 4}{4 \, \sqrt{x}}}\;dx=\answer{\frac{5}{2} \, \sqrt{3} - \frac{21}{2}}\]
\end{problem}}%}

%%%%%%%%%%%%%%%%%%%%%%


\latexProblemContent{
\begin{problem}

Use the Fundamental Theorem of Calculus to evaluate the integral.

\expandafter\input{\file@loc Integrals/2311-Compute-Integral-0011.HELP.tex}

\[\int_{3}^{4} {-\frac{x^{2} - 3 \, x + 2}{9 \, \sqrt{x}}}\;dx=\answer{\frac{8}{45} \, \sqrt{3} - \frac{8}{15}}\]
\end{problem}}%}

%%%%%%%%%%%%%%%%%%%%%%


\latexProblemContent{
\begin{problem}

Use the Fundamental Theorem of Calculus to evaluate the integral.

\expandafter\input{\file@loc Integrals/2311-Compute-Integral-0011.HELP.tex}

\[\int_{6}^{13} {-\frac{x^{2} + 2 \, x - 80}{\sqrt{x}}}\;dx=\answer{\frac{1126}{15} \, \sqrt{13} - \frac{688}{5} \, \sqrt{6}}\]
\end{problem}}%}

%%%%%%%%%%%%%%%%%%%%%%


\latexProblemContent{
\begin{problem}

Use the Fundamental Theorem of Calculus to evaluate the integral.

\expandafter\input{\file@loc Integrals/2311-Compute-Integral-0011.HELP.tex}

\[\int_{3}^{13} {\frac{x^{2} - 100}{5 \, \sqrt{x}}}\;dx=\answer{-\frac{662}{25} \, \sqrt{13} + \frac{982}{25} \, \sqrt{3}}\]
\end{problem}}%}

%%%%%%%%%%%%%%%%%%%%%%


\latexProblemContent{
\begin{problem}

Use the Fundamental Theorem of Calculus to evaluate the integral.

\expandafter\input{\file@loc Integrals/2311-Compute-Integral-0011.HELP.tex}

\[\int_{8}^{9} {\frac{x^{2} - 36}{\sqrt{x}}}\;dx=\answer{\frac{464}{5} \, \sqrt{2} - \frac{594}{5}}\]
\end{problem}}%}

%%%%%%%%%%%%%%%%%%%%%%


\latexProblemContent{
\begin{problem}

Use the Fundamental Theorem of Calculus to evaluate the integral.

\expandafter\input{\file@loc Integrals/2311-Compute-Integral-0011.HELP.tex}

\[\int_{4}^{6} {\frac{x^{2} - 8 \, x + 12}{6 \, \sqrt{x}}}\;dx=\answer{\frac{16}{15} \, \sqrt{6} - \frac{136}{45}}\]
\end{problem}}%}

%%%%%%%%%%%%%%%%%%%%%%


\latexProblemContent{
\begin{problem}

Use the Fundamental Theorem of Calculus to evaluate the integral.

\expandafter\input{\file@loc Integrals/2311-Compute-Integral-0011.HELP.tex}

\[\int_{7}^{14} {-\frac{x^{2} - 1}{3 \, \sqrt{x}}}\;dx=\answer{-\frac{382}{15} \, \sqrt{14} + \frac{88}{15} \, \sqrt{7}}\]
\end{problem}}%}

%%%%%%%%%%%%%%%%%%%%%%


\latexProblemContent{
\begin{problem}

Use the Fundamental Theorem of Calculus to evaluate the integral.

\expandafter\input{\file@loc Integrals/2311-Compute-Integral-0011.HELP.tex}

\[\int_{1}^{7} {\frac{x^{2} - 9}{9 \, \sqrt{x}}}\;dx=\answer{\frac{8}{45} \, \sqrt{7} + \frac{88}{45}}\]
\end{problem}}%}

%%%%%%%%%%%%%%%%%%%%%%


\latexProblemContent{
\begin{problem}

Use the Fundamental Theorem of Calculus to evaluate the integral.

\expandafter\input{\file@loc Integrals/2311-Compute-Integral-0011.HELP.tex}

\[\int_{4}^{15} {-\frac{x^{2} - 4}{2 \, \sqrt{x}}}\;dx=\answer{-41 \, \sqrt{15} - \frac{8}{5}}\]
\end{problem}}%}

%%%%%%%%%%%%%%%%%%%%%%


\latexProblemContent{
\begin{problem}

Use the Fundamental Theorem of Calculus to evaluate the integral.

\expandafter\input{\file@loc Integrals/2311-Compute-Integral-0011.HELP.tex}

\[\int_{8}^{12} {-\frac{x + 2}{4 \, \sqrt{x}}}\;dx=\answer{-6 \, \sqrt{3} + \frac{14}{3} \, \sqrt{2}}\]
\end{problem}}%}

%%%%%%%%%%%%%%%%%%%%%%


\latexProblemContent{
\begin{problem}

Use the Fundamental Theorem of Calculus to evaluate the integral.

\expandafter\input{\file@loc Integrals/2311-Compute-Integral-0011.HELP.tex}

\[\int_{2}^{14} {-\frac{x^{2} - 16}{4 \, \sqrt{x}}}\;dx=\answer{-\frac{58}{5} \, \sqrt{14} - \frac{38}{5} \, \sqrt{2}}\]
\end{problem}}%}

%%%%%%%%%%%%%%%%%%%%%%


\latexProblemContent{
\begin{problem}

Use the Fundamental Theorem of Calculus to evaluate the integral.

\expandafter\input{\file@loc Integrals/2311-Compute-Integral-0011.HELP.tex}

\[\int_{5}^{13} {-\frac{x + 4}{8 \, \sqrt{x}}}\;dx=\answer{-\frac{25}{12} \, \sqrt{13} + \frac{17}{12} \, \sqrt{5}}\]
\end{problem}}%}

%%%%%%%%%%%%%%%%%%%%%%


\latexProblemContent{
\begin{problem}

Use the Fundamental Theorem of Calculus to evaluate the integral.

\expandafter\input{\file@loc Integrals/2311-Compute-Integral-0011.HELP.tex}

\[\int_{7}^{15} {-\frac{x + 1}{9 \, \sqrt{x}}}\;dx=\answer{-\frac{4}{3} \, \sqrt{15} + \frac{20}{27} \, \sqrt{7}}\]
\end{problem}}%}

%%%%%%%%%%%%%%%%%%%%%%


\latexProblemContent{
\begin{problem}

Use the Fundamental Theorem of Calculus to evaluate the integral.

\expandafter\input{\file@loc Integrals/2311-Compute-Integral-0011.HELP.tex}

\[\int_{2}^{6} {\frac{x^{2} + 13 \, x + 36}{3 \, \sqrt{x}}}\;dx=\answer{\frac{692}{15} \, \sqrt{6} - \frac{1364}{45} \, \sqrt{2}}\]
\end{problem}}%}

%%%%%%%%%%%%%%%%%%%%%%


\latexProblemContent{
\begin{problem}

Use the Fundamental Theorem of Calculus to evaluate the integral.

\expandafter\input{\file@loc Integrals/2311-Compute-Integral-0011.HELP.tex}

\[\int_{7}^{12} {\frac{x^{2} - x - 2}{7 \, \sqrt{x}}}\;dx=\answer{-\frac{164}{105} \, \sqrt{7} + \frac{456}{35} \, \sqrt{3}}\]
\end{problem}}%}

%%%%%%%%%%%%%%%%%%%%%%


\latexProblemContent{
\begin{problem}

Use the Fundamental Theorem of Calculus to evaluate the integral.

\expandafter\input{\file@loc Integrals/2311-Compute-Integral-0011.HELP.tex}

\[\int_{1}^{11} {-\frac{x - 10}{2 \, \sqrt{x}}}\;dx=\answer{\frac{19}{3} \, \sqrt{11} - \frac{29}{3}}\]
\end{problem}}%}

%%%%%%%%%%%%%%%%%%%%%%


\latexProblemContent{
\begin{problem}

Use the Fundamental Theorem of Calculus to evaluate the integral.

\expandafter\input{\file@loc Integrals/2311-Compute-Integral-0011.HELP.tex}

\[\int_{4}^{13} {-\frac{x^{2} - 25}{2 \, \sqrt{x}}}\;dx=\answer{-\frac{44}{5} \, \sqrt{13} - \frac{218}{5}}\]
\end{problem}}%}

%%%%%%%%%%%%%%%%%%%%%%


\latexProblemContent{
\begin{problem}

Use the Fundamental Theorem of Calculus to evaluate the integral.

\expandafter\input{\file@loc Integrals/2311-Compute-Integral-0011.HELP.tex}

\[\int_{3}^{15} {\frac{x^{2} - 49}{5 \, \sqrt{x}}}\;dx=\answer{-\frac{8}{5} \, \sqrt{15} + \frac{472}{25} \, \sqrt{3}}\]
\end{problem}}%}

%%%%%%%%%%%%%%%%%%%%%%


\latexProblemContent{
\begin{problem}

Use the Fundamental Theorem of Calculus to evaluate the integral.

\expandafter\input{\file@loc Integrals/2311-Compute-Integral-0011.HELP.tex}

\[\int_{3}^{14} {\frac{x^{2} - 81}{9 \, \sqrt{x}}}\;dx=\answer{-\frac{418}{45} \, \sqrt{14} + \frac{88}{5} \, \sqrt{3}}\]
\end{problem}}%}

%%%%%%%%%%%%%%%%%%%%%%


\latexProblemContent{
\begin{problem}

Use the Fundamental Theorem of Calculus to evaluate the integral.

\expandafter\input{\file@loc Integrals/2311-Compute-Integral-0011.HELP.tex}

\[\int_{2}^{9} {-\frac{x + 2}{3 \, \sqrt{x}}}\;dx=\answer{\frac{1}{9} \, \left(16 \, \sqrt{2}\right) - 10}\]
\end{problem}}%}

%%%%%%%%%%%%%%%%%%%%%%


\latexProblemContent{
\begin{problem}

Use the Fundamental Theorem of Calculus to evaluate the integral.

\expandafter\input{\file@loc Integrals/2311-Compute-Integral-0011.HELP.tex}

\[\int_{4}^{13} {\frac{x^{2} + x - 20}{5 \, \sqrt{x}}}\;dx=\answer{\frac{544}{75} \, \sqrt{13} + \frac{928}{75}}\]
\end{problem}}%}

%%%%%%%%%%%%%%%%%%%%%%


\latexProblemContent{
\begin{problem}

Use the Fundamental Theorem of Calculus to evaluate the integral.

\expandafter\input{\file@loc Integrals/2311-Compute-Integral-0011.HELP.tex}

\[\int_{6}^{9} {\frac{1}{5} \, \sqrt{x}}\;dx=\answer{-\frac{4}{5} \, \sqrt{6} + \frac{18}{5}}\]
\end{problem}}%}

%%%%%%%%%%%%%%%%%%%%%%


\latexProblemContent{
\begin{problem}

Use the Fundamental Theorem of Calculus to evaluate the integral.

\expandafter\input{\file@loc Integrals/2311-Compute-Integral-0011.HELP.tex}

\[\int_{7}^{8} {\frac{x^{2} + x - 90}{7 \, \sqrt{x}}}\;dx=\answer{\frac{2336}{105} \, \sqrt{7} - \frac{4472}{105} \, \sqrt{2}}\]
\end{problem}}%}

%%%%%%%%%%%%%%%%%%%%%%


\latexProblemContent{
\begin{problem}

Use the Fundamental Theorem of Calculus to evaluate the integral.

\expandafter\input{\file@loc Integrals/2311-Compute-Integral-0011.HELP.tex}

\[\int_{5}^{10} {-\frac{1}{8} \, \sqrt{x}}\;dx=\answer{-\frac{5}{6} \, \sqrt{10} + \frac{5}{12} \, \sqrt{5}}\]
\end{problem}}%}

%%%%%%%%%%%%%%%%%%%%%%


\latexProblemContent{
\begin{problem}

Use the Fundamental Theorem of Calculus to evaluate the integral.

\expandafter\input{\file@loc Integrals/2311-Compute-Integral-0011.HELP.tex}

\[\int_{6}^{8} {-\frac{x + 6}{6 \, \sqrt{x}}}\;dx=\answer{\frac{8}{3} \, \sqrt{6} - \frac{52}{9} \, \sqrt{2}}\]
\end{problem}}%}

%%%%%%%%%%%%%%%%%%%%%%


\latexProblemContent{
\begin{problem}

Use the Fundamental Theorem of Calculus to evaluate the integral.

\expandafter\input{\file@loc Integrals/2311-Compute-Integral-0011.HELP.tex}

\[\int_{4}^{13} {\frac{x^{2} + 15 \, x + 54}{7 \, \sqrt{x}}}\;dx=\answer{\frac{1528}{35} \, \sqrt{13} - \frac{1544}{35}}\]
\end{problem}}%}

%%%%%%%%%%%%%%%%%%%%%%


\latexProblemContent{
\begin{problem}

Use the Fundamental Theorem of Calculus to evaluate the integral.

\expandafter\input{\file@loc Integrals/2311-Compute-Integral-0011.HELP.tex}

\[\int_{5}^{10} {\frac{x - 4}{10 \, \sqrt{x}}}\;dx=\answer{-\frac{2}{15} \, \sqrt{10} + \frac{7}{15} \, \sqrt{5}}\]
\end{problem}}%}

%%%%%%%%%%%%%%%%%%%%%%


\latexProblemContent{
\begin{problem}

Use the Fundamental Theorem of Calculus to evaluate the integral.

\expandafter\input{\file@loc Integrals/2311-Compute-Integral-0011.HELP.tex}

\[\int_{7}^{11} {-\frac{x^{2} - 1}{7 \, \sqrt{x}}}\;dx=\answer{-\frac{232}{35} \, \sqrt{11} + \frac{88}{35} \, \sqrt{7}}\]
\end{problem}}%}

%%%%%%%%%%%%%%%%%%%%%%


\latexProblemContent{
\begin{problem}

Use the Fundamental Theorem of Calculus to evaluate the integral.

\expandafter\input{\file@loc Integrals/2311-Compute-Integral-0011.HELP.tex}

\[\int_{8}^{10} {\frac{x^{2} + x - 56}{2 \, \sqrt{x}}}\;dx=\answer{-\frac{98}{3} \, \sqrt{10} + \frac{1216}{15} \, \sqrt{2}}\]
\end{problem}}%}

%%%%%%%%%%%%%%%%%%%%%%


\latexProblemContent{
\begin{problem}

Use the Fundamental Theorem of Calculus to evaluate the integral.

\expandafter\input{\file@loc Integrals/2311-Compute-Integral-0011.HELP.tex}

\[\int_{5}^{7} {\frac{x^{2} - 64}{5 \, \sqrt{x}}}\;dx=\answer{-\frac{542}{25} \, \sqrt{7} + \frac{118}{5} \, \sqrt{5}}\]
\end{problem}}%}

%%%%%%%%%%%%%%%%%%%%%%


\latexProblemContent{
\begin{problem}

Use the Fundamental Theorem of Calculus to evaluate the integral.

\expandafter\input{\file@loc Integrals/2311-Compute-Integral-0011.HELP.tex}

\[\int_{8}^{9} {\frac{x^{2} + x - 56}{7 \, \sqrt{x}}}\;dx=\answer{\frac{2432}{105} \, \sqrt{2} - \frac{1104}{35}}\]
\end{problem}}%}

%%%%%%%%%%%%%%%%%%%%%%


\latexProblemContent{
\begin{problem}

Use the Fundamental Theorem of Calculus to evaluate the integral.

\expandafter\input{\file@loc Integrals/2311-Compute-Integral-0011.HELP.tex}

\[\int_{7}^{13} {\frac{x^{2} - 100}{5 \, \sqrt{x}}}\;dx=\answer{-\frac{662}{25} \, \sqrt{13} + \frac{902}{25} \, \sqrt{7}}\]
\end{problem}}%}

%%%%%%%%%%%%%%%%%%%%%%


\latexProblemContent{
\begin{problem}

Use the Fundamental Theorem of Calculus to evaluate the integral.

\expandafter\input{\file@loc Integrals/2311-Compute-Integral-0011.HELP.tex}

\[\int_{4}^{10} {-\frac{x + 4}{8 \, \sqrt{x}}}\;dx=\answer{-\frac{11}{6} \, \sqrt{10} + \frac{8}{3}}\]
\end{problem}}%}

%%%%%%%%%%%%%%%%%%%%%%


\latexProblemContent{
\begin{problem}

Use the Fundamental Theorem of Calculus to evaluate the integral.

\expandafter\input{\file@loc Integrals/2311-Compute-Integral-0011.HELP.tex}

\[\int_{8}^{10} {-\frac{x - 3}{6 \, \sqrt{x}}}\;dx=\answer{-\frac{1}{18} \, \left(4 \, \sqrt{2}\right) - \frac{1}{9} \, \sqrt{10}}\]
\end{problem}}%}

%%%%%%%%%%%%%%%%%%%%%%


\latexProblemContent{
\begin{problem}

Use the Fundamental Theorem of Calculus to evaluate the integral.

\expandafter\input{\file@loc Integrals/2311-Compute-Integral-0011.HELP.tex}

\[\int_{1}^{10} {-\frac{x^{2} - 64}{8 \, \sqrt{x}}}\;dx=\answer{11 \, \sqrt{10} - \frac{319}{20}}\]
\end{problem}}%}

%%%%%%%%%%%%%%%%%%%%%%


\latexProblemContent{
\begin{problem}

Use the Fundamental Theorem of Calculus to evaluate the integral.

\expandafter\input{\file@loc Integrals/2311-Compute-Integral-0011.HELP.tex}

\[\int_{4}^{6} {-\frac{x^{2} - 4}{3 \, \sqrt{x}}}\;dx=\answer{-\frac{32}{15} \, \sqrt{6} - \frac{16}{15}}\]
\end{problem}}%}

%%%%%%%%%%%%%%%%%%%%%%


\latexProblemContent{
\begin{problem}

Use the Fundamental Theorem of Calculus to evaluate the integral.

\expandafter\input{\file@loc Integrals/2311-Compute-Integral-0011.HELP.tex}

\[\int_{1}^{15} {-\frac{x^{2} - 36}{5 \, \sqrt{x}}}\;dx=\answer{-\frac{18}{5} \, \sqrt{15} - \frac{358}{25}}\]
\end{problem}}%}

%%%%%%%%%%%%%%%%%%%%%%


\latexProblemContent{
\begin{problem}

Use the Fundamental Theorem of Calculus to evaluate the integral.

\expandafter\input{\file@loc Integrals/2311-Compute-Integral-0011.HELP.tex}

\[\int_{3}^{7} {\frac{x^{2} + 2 \, x - 35}{8 \, \sqrt{x}}}\;dx=\answer{-\frac{77}{15} \, \sqrt{7} + \frac{39}{5} \, \sqrt{3}}\]
\end{problem}}%}

%%%%%%%%%%%%%%%%%%%%%%


\latexProblemContent{
\begin{problem}

Use the Fundamental Theorem of Calculus to evaluate the integral.

\expandafter\input{\file@loc Integrals/2311-Compute-Integral-0011.HELP.tex}

\[\int_{3}^{8} {\frac{x^{2} + x - 20}{\sqrt{x}}}\;dx=\answer{\frac{172}{5} \, \sqrt{3} - \frac{272}{15} \, \sqrt{2}}\]
\end{problem}}%}

%%%%%%%%%%%%%%%%%%%%%%


\latexProblemContent{
\begin{problem}

Use the Fundamental Theorem of Calculus to evaluate the integral.

\expandafter\input{\file@loc Integrals/2311-Compute-Integral-0011.HELP.tex}

\[\int_{5}^{8} {-\frac{x^{2} - 49}{\sqrt{x}}}\;dx=\answer{-88 \, \sqrt{5} + \frac{724}{5} \, \sqrt{2}}\]
\end{problem}}%}

%%%%%%%%%%%%%%%%%%%%%%


\latexProblemContent{
\begin{problem}

Use the Fundamental Theorem of Calculus to evaluate the integral.

\expandafter\input{\file@loc Integrals/2311-Compute-Integral-0011.HELP.tex}

\[\int_{7}^{9} {\frac{x^{2} - 7 \, x + 6}{10 \, \sqrt{x}}}\;dx=\answer{\frac{8}{75} \, \sqrt{7} + \frac{18}{25}}\]
\end{problem}}%}

%%%%%%%%%%%%%%%%%%%%%%


\latexProblemContent{
\begin{problem}

Use the Fundamental Theorem of Calculus to evaluate the integral.

\expandafter\input{\file@loc Integrals/2311-Compute-Integral-0011.HELP.tex}

\[\int_{4}^{12} {\frac{x^{2} + 3 \, x - 28}{9 \, \sqrt{x}}}\;dx=\answer{\frac{256}{45} \, \sqrt{3} + \frac{416}{45}}\]
\end{problem}}%}

%%%%%%%%%%%%%%%%%%%%%%


\latexProblemContent{
\begin{problem}

Use the Fundamental Theorem of Calculus to evaluate the integral.

\expandafter\input{\file@loc Integrals/2311-Compute-Integral-0011.HELP.tex}

\[\int_{4}^{6} {\frac{x - 7}{5 \, \sqrt{x}}}\;dx=\answer{-2 \, \sqrt{6} + \frac{68}{15}}\]
\end{problem}}%}

%%%%%%%%%%%%%%%%%%%%%%


\latexProblemContent{
\begin{problem}

Use the Fundamental Theorem of Calculus to evaluate the integral.

\expandafter\input{\file@loc Integrals/2311-Compute-Integral-0011.HELP.tex}

\[\int_{1}^{12} {-\frac{x^{2} - 4 \, x - 32}{\sqrt{x}}}\;dx=\answer{\frac{384}{5} \, \sqrt{3} - \frac{994}{15}}\]
\end{problem}}%}

%%%%%%%%%%%%%%%%%%%%%%


\latexProblemContent{
\begin{problem}

Use the Fundamental Theorem of Calculus to evaluate the integral.

\expandafter\input{\file@loc Integrals/2311-Compute-Integral-0011.HELP.tex}

\[\int_{5}^{9} {-\frac{x^{2} - 13 \, x + 40}{9 \, \sqrt{x}}}\;dx=\answer{\frac{140}{27} \, \sqrt{5} - \frac{172}{15}}\]
\end{problem}}%}

%%%%%%%%%%%%%%%%%%%%%%


\latexProblemContent{
\begin{problem}

Use the Fundamental Theorem of Calculus to evaluate the integral.

\expandafter\input{\file@loc Integrals/2311-Compute-Integral-0011.HELP.tex}

\[\int_{7}^{13} {\frac{x - 10}{10 \, \sqrt{x}}}\;dx=\answer{-\frac{17}{15} \, \sqrt{13} + \frac{23}{15} \, \sqrt{7}}\]
\end{problem}}%}

%%%%%%%%%%%%%%%%%%%%%%


\latexProblemContent{
\begin{problem}

Use the Fundamental Theorem of Calculus to evaluate the integral.

\expandafter\input{\file@loc Integrals/2311-Compute-Integral-0011.HELP.tex}

\[\int_{2}^{11} {\frac{x^{2} + 14 \, x + 49}{3 \, \sqrt{x}}}\;dx=\answer{\frac{3736}{45} \, \sqrt{11} - \frac{1774}{45} \, \sqrt{2}}\]
\end{problem}}%}

%%%%%%%%%%%%%%%%%%%%%%


\latexProblemContent{
\begin{problem}

Use the Fundamental Theorem of Calculus to evaluate the integral.

\expandafter\input{\file@loc Integrals/2311-Compute-Integral-0011.HELP.tex}

\[\int_{6}^{9} {-\frac{x^{2} - 64}{4 \, \sqrt{x}}}\;dx=\answer{-\frac{142}{5} \, \sqrt{6} + \frac{717}{10}}\]
\end{problem}}%}

%%%%%%%%%%%%%%%%%%%%%%


\latexProblemContent{
\begin{problem}

Use the Fundamental Theorem of Calculus to evaluate the integral.

\expandafter\input{\file@loc Integrals/2311-Compute-Integral-0011.HELP.tex}

\[\int_{1}^{3} {-\frac{x + 2}{2 \, \sqrt{x}}}\;dx=\answer{-3 \, \sqrt{3} + \frac{7}{3}}\]
\end{problem}}%}

%%%%%%%%%%%%%%%%%%%%%%


\latexProblemContent{
\begin{problem}

Use the Fundamental Theorem of Calculus to evaluate the integral.

\expandafter\input{\file@loc Integrals/2311-Compute-Integral-0011.HELP.tex}

\[\int_{6}^{11} {\frac{x^{2} - 100}{2 \, \sqrt{x}}}\;dx=\answer{-\frac{379}{5} \, \sqrt{11} + \frac{464}{5} \, \sqrt{6}}\]
\end{problem}}%}

%%%%%%%%%%%%%%%%%%%%%%


\latexProblemContent{
\begin{problem}

Use the Fundamental Theorem of Calculus to evaluate the integral.

\expandafter\input{\file@loc Integrals/2311-Compute-Integral-0011.HELP.tex}

\[\int_{7}^{15} {-\frac{x^{2} - 36}{10 \, \sqrt{x}}}\;dx=\answer{-\frac{9}{5} \, \sqrt{15} - \frac{131}{25} \, \sqrt{7}}\]
\end{problem}}%}

%%%%%%%%%%%%%%%%%%%%%%


\latexProblemContent{
\begin{problem}

Use the Fundamental Theorem of Calculus to evaluate the integral.

\expandafter\input{\file@loc Integrals/2311-Compute-Integral-0011.HELP.tex}

\[\int_{5}^{8} {\frac{x + 1}{4 \, \sqrt{x}}}\;dx=\answer{-\frac{4}{3} \, \sqrt{5} + \frac{11}{3} \, \sqrt{2}}\]
\end{problem}}%}

%%%%%%%%%%%%%%%%%%%%%%


\latexProblemContent{
\begin{problem}

Use the Fundamental Theorem of Calculus to evaluate the integral.

\expandafter\input{\file@loc Integrals/2311-Compute-Integral-0011.HELP.tex}

\[\int_{3}^{6} {\frac{x^{2} - 16}{10 \, \sqrt{x}}}\;dx=\answer{-\frac{44}{25} \, \sqrt{6} + \frac{71}{25} \, \sqrt{3}}\]
\end{problem}}%}

%%%%%%%%%%%%%%%%%%%%%%


\latexProblemContent{
\begin{problem}

Use the Fundamental Theorem of Calculus to evaluate the integral.

\expandafter\input{\file@loc Integrals/2311-Compute-Integral-0011.HELP.tex}

\[\int_{1}^{13} {-\frac{x + 4}{\sqrt{x}}}\;dx=\answer{-\frac{50}{3} \, \sqrt{13} + \frac{26}{3}}\]
\end{problem}}%}

%%%%%%%%%%%%%%%%%%%%%%


\latexProblemContent{
\begin{problem}

Use the Fundamental Theorem of Calculus to evaluate the integral.

\expandafter\input{\file@loc Integrals/2311-Compute-Integral-0011.HELP.tex}

\[\int_{4}^{5} {-\frac{x^{2} - 9 \, x + 20}{10 \, \sqrt{x}}}\;dx=\answer{-2 \, \sqrt{5} + \frac{112}{25}}\]
\end{problem}}%}

%%%%%%%%%%%%%%%%%%%%%%


\latexProblemContent{
\begin{problem}

Use the Fundamental Theorem of Calculus to evaluate the integral.

\expandafter\input{\file@loc Integrals/2311-Compute-Integral-0011.HELP.tex}

\[\int_{8}^{9} {\frac{x^{2} + 10 \, x + 21}{5 \, \sqrt{x}}}\;dx=\answer{-\frac{3628}{75} \, \sqrt{2} + \frac{2016}{25}}\]
\end{problem}}%}

%%%%%%%%%%%%%%%%%%%%%%


\latexProblemContent{
\begin{problem}

Use the Fundamental Theorem of Calculus to evaluate the integral.

\expandafter\input{\file@loc Integrals/2311-Compute-Integral-0011.HELP.tex}

\[\int_{6}^{10} {\frac{x + 2}{5 \, \sqrt{x}}}\;dx=\answer{\frac{32}{15} \, \sqrt{10} - \frac{8}{5} \, \sqrt{6}}\]
\end{problem}}%}

%%%%%%%%%%%%%%%%%%%%%%


\latexProblemContent{
\begin{problem}

Use the Fundamental Theorem of Calculus to evaluate the integral.

\expandafter\input{\file@loc Integrals/2311-Compute-Integral-0011.HELP.tex}

\[\int_{2}^{9} {\frac{x^{2} - 81}{9 \, \sqrt{x}}}\;dx=\answer{\frac{802}{45} \, \sqrt{2} - \frac{216}{5}}\]
\end{problem}}%}

%%%%%%%%%%%%%%%%%%%%%%


\latexProblemContent{
\begin{problem}

Use the Fundamental Theorem of Calculus to evaluate the integral.

\expandafter\input{\file@loc Integrals/2311-Compute-Integral-0011.HELP.tex}

\[\int_{2}^{5} {\frac{x^{2} + 8 \, x + 12}{8 \, \sqrt{x}}}\;dx=\answer{\frac{91}{12} \, \sqrt{5} - \frac{68}{15} \, \sqrt{2}}\]
\end{problem}}%}

%%%%%%%%%%%%%%%%%%%%%%


\latexProblemContent{
\begin{problem}

Use the Fundamental Theorem of Calculus to evaluate the integral.

\expandafter\input{\file@loc Integrals/2311-Compute-Integral-0011.HELP.tex}

\[\int_{1}^{3} {\frac{x^{2} - 100}{3 \, \sqrt{x}}}\;dx=\answer{-\frac{982}{15} \, \sqrt{3} + \frac{998}{15}}\]
\end{problem}}%}

%%%%%%%%%%%%%%%%%%%%%%


\latexProblemContent{
\begin{problem}

Use the Fundamental Theorem of Calculus to evaluate the integral.

\expandafter\input{\file@loc Integrals/2311-Compute-Integral-0011.HELP.tex}

\[\int_{6}^{12} {-\frac{x^{2} - 15 \, x + 56}{2 \, \sqrt{x}}}\;dx=\answer{\frac{166}{5} \, \sqrt{6} - \frac{248}{5} \, \sqrt{3}}\]
\end{problem}}%}

%%%%%%%%%%%%%%%%%%%%%%


\latexProblemContent{
\begin{problem}

Use the Fundamental Theorem of Calculus to evaluate the integral.

\expandafter\input{\file@loc Integrals/2311-Compute-Integral-0011.HELP.tex}

\[\int_{6}^{7} {\frac{x + 2}{9 \, \sqrt{x}}}\;dx=\answer{\frac{26}{27} \, \sqrt{7} - \frac{8}{9} \, \sqrt{6}}\]
\end{problem}}%}

%%%%%%%%%%%%%%%%%%%%%%


\latexProblemContent{
\begin{problem}

Use the Fundamental Theorem of Calculus to evaluate the integral.

\expandafter\input{\file@loc Integrals/2311-Compute-Integral-0011.HELP.tex}

\[\int_{4}^{11} {\frac{x^{2} + 15 \, x + 54}{4 \, \sqrt{x}}}\;dx=\answer{\frac{333}{5} \, \sqrt{11} - \frac{386}{5}}\]
\end{problem}}%}

%%%%%%%%%%%%%%%%%%%%%%


\latexProblemContent{
\begin{problem}

Use the Fundamental Theorem of Calculus to evaluate the integral.

\expandafter\input{\file@loc Integrals/2311-Compute-Integral-0011.HELP.tex}

\[\int_{8}^{13} {-\frac{x^{2} - 3 \, x - 10}{\sqrt{x}}}\;dx=\answer{-\frac{108}{5} \, \sqrt{13} - \frac{104}{5} \, \sqrt{2}}\]
\end{problem}}%}

%%%%%%%%%%%%%%%%%%%%%%


\latexProblemContent{
\begin{problem}

Use the Fundamental Theorem of Calculus to evaluate the integral.

\expandafter\input{\file@loc Integrals/2311-Compute-Integral-0011.HELP.tex}

\[\int_{2}^{5} {-\frac{x^{2} - 49}{6 \, \sqrt{x}}}\;dx=\answer{\frac{44}{3} \, \sqrt{5} - \frac{241}{15} \, \sqrt{2}}\]
\end{problem}}%}

%%%%%%%%%%%%%%%%%%%%%%


\latexProblemContent{
\begin{problem}

Use the Fundamental Theorem of Calculus to evaluate the integral.

\expandafter\input{\file@loc Integrals/2311-Compute-Integral-0011.HELP.tex}

\[\int_{8}^{14} {\frac{x^{2} + 13 \, x + 36}{3 \, \sqrt{x}}}\;dx=\answer{\frac{4076}{45} \, \sqrt{14} - \frac{5008}{45} \, \sqrt{2}}\]
\end{problem}}%}

%%%%%%%%%%%%%%%%%%%%%%


\latexProblemContent{
\begin{problem}

Use the Fundamental Theorem of Calculus to evaluate the integral.

\expandafter\input{\file@loc Integrals/2311-Compute-Integral-0011.HELP.tex}

\[\int_{1}^{15} {\frac{x^{2} - 1}{4 \, \sqrt{x}}}\;dx=\answer{22 \, \sqrt{15} + \frac{2}{5}}\]
\end{problem}}%}

%%%%%%%%%%%%%%%%%%%%%%


\latexProblemContent{
\begin{problem}

Use the Fundamental Theorem of Calculus to evaluate the integral.

\expandafter\input{\file@loc Integrals/2311-Compute-Integral-0011.HELP.tex}

\[\int_{5}^{14} {-\frac{x^{2} - 16}{\sqrt{x}}}\;dx=\answer{-\frac{232}{5} \, \sqrt{14} - 22 \, \sqrt{5}}\]
\end{problem}}%}

%%%%%%%%%%%%%%%%%%%%%%


\latexProblemContent{
\begin{problem}

Use the Fundamental Theorem of Calculus to evaluate the integral.

\expandafter\input{\file@loc Integrals/2311-Compute-Integral-0011.HELP.tex}

\[\int_{5}^{8} {-\frac{x^{2} - 6 \, x - 40}{5 \, \sqrt{x}}}\;dx=\answer{-18 \, \sqrt{5} + \frac{864}{25} \, \sqrt{2}}\]
\end{problem}}%}

%%%%%%%%%%%%%%%%%%%%%%


\latexProblemContent{
\begin{problem}

Use the Fundamental Theorem of Calculus to evaluate the integral.

\expandafter\input{\file@loc Integrals/2311-Compute-Integral-0011.HELP.tex}

\[\int_{1}^{6} {\frac{x + 7}{10 \, \sqrt{x}}}\;dx=\answer{\frac{9}{5} \, \sqrt{6} - \frac{22}{15}}\]
\end{problem}}%}

%%%%%%%%%%%%%%%%%%%%%%


\latexProblemContent{
\begin{problem}

Use the Fundamental Theorem of Calculus to evaluate the integral.

\expandafter\input{\file@loc Integrals/2311-Compute-Integral-0011.HELP.tex}

\[\int_{3}^{13} {-\frac{x + 7}{10 \, \sqrt{x}}}\;dx=\answer{-\frac{34}{15} \, \sqrt{13} + \frac{8}{5} \, \sqrt{3}}\]
\end{problem}}%}

%%%%%%%%%%%%%%%%%%%%%%


\latexProblemContent{
\begin{problem}

Use the Fundamental Theorem of Calculus to evaluate the integral.

\expandafter\input{\file@loc Integrals/2311-Compute-Integral-0011.HELP.tex}

\[\int_{2}^{5} {\frac{x^{2} - 4 \, x - 12}{5 \, \sqrt{x}}}\;dx=\answer{-\frac{82}{15} \, \sqrt{5} + \frac{416}{75} \, \sqrt{2}}\]
\end{problem}}%}

%%%%%%%%%%%%%%%%%%%%%%


\latexProblemContent{
\begin{problem}

Use the Fundamental Theorem of Calculus to evaluate the integral.

\expandafter\input{\file@loc Integrals/2311-Compute-Integral-0011.HELP.tex}

\[\int_{8}^{12} {\frac{x + 7}{9 \, \sqrt{x}}}\;dx=\answer{\frac{44}{9} \, \sqrt{3} - \frac{116}{27} \, \sqrt{2}}\]
\end{problem}}%}

%%%%%%%%%%%%%%%%%%%%%%


\latexProblemContent{
\begin{problem}

Use the Fundamental Theorem of Calculus to evaluate the integral.

\expandafter\input{\file@loc Integrals/2311-Compute-Integral-0011.HELP.tex}

\[\int_{2}^{10} {-\frac{x + 6}{2 \, \sqrt{x}}}\;dx=\answer{-\frac{28}{3} \, \sqrt{10} + \frac{20}{3} \, \sqrt{2}}\]
\end{problem}}%}

%%%%%%%%%%%%%%%%%%%%%%


\latexProblemContent{
\begin{problem}

Use the Fundamental Theorem of Calculus to evaluate the integral.

\expandafter\input{\file@loc Integrals/2311-Compute-Integral-0011.HELP.tex}

\[\int_{2}^{13} {\frac{x^{2} - 64}{5 \, \sqrt{x}}}\;dx=\answer{-\frac{302}{25} \, \sqrt{13} + \frac{632}{25} \, \sqrt{2}}\]
\end{problem}}%}

%%%%%%%%%%%%%%%%%%%%%%


\latexProblemContent{
\begin{problem}

Use the Fundamental Theorem of Calculus to evaluate the integral.

\expandafter\input{\file@loc Integrals/2311-Compute-Integral-0011.HELP.tex}

\[\int_{1}^{12} {\frac{x^{2} - 4 \, x - 60}{10 \, \sqrt{x}}}\;dx=\answer{-\frac{472}{25} \, \sqrt{3} + \frac{917}{75}}\]
\end{problem}}%}

%%%%%%%%%%%%%%%%%%%%%%


\latexProblemContent{
\begin{problem}

Use the Fundamental Theorem of Calculus to evaluate the integral.

\expandafter\input{\file@loc Integrals/2311-Compute-Integral-0011.HELP.tex}

\[\int_{6}^{15} {-\frac{x^{2} - 36}{\sqrt{x}}}\;dx=\answer{-18 \, \sqrt{15} - \frac{288}{5} \, \sqrt{6}}\]
\end{problem}}%}

%%%%%%%%%%%%%%%%%%%%%%


\latexProblemContent{
\begin{problem}

Use the Fundamental Theorem of Calculus to evaluate the integral.

\expandafter\input{\file@loc Integrals/2311-Compute-Integral-0011.HELP.tex}

\[\int_{6}^{12} {-\frac{x^{2} - 10 \, x + 21}{7 \, \sqrt{x}}}\;dx=\answer{\frac{82}{35} \, \sqrt{6} - \frac{28}{5} \, \sqrt{3}}\]
\end{problem}}%}

%%%%%%%%%%%%%%%%%%%%%%


\latexProblemContent{
\begin{problem}

Use the Fundamental Theorem of Calculus to evaluate the integral.

\expandafter\input{\file@loc Integrals/2311-Compute-Integral-0011.HELP.tex}

\[\int_{3}^{6} {-\frac{x^{2} - 49}{10 \, \sqrt{x}}}\;dx=\answer{\frac{209}{25} \, \sqrt{6} - \frac{236}{25} \, \sqrt{3}}\]
\end{problem}}%}

%%%%%%%%%%%%%%%%%%%%%%


\latexProblemContent{
\begin{problem}

Use the Fundamental Theorem of Calculus to evaluate the integral.

\expandafter\input{\file@loc Integrals/2311-Compute-Integral-0011.HELP.tex}

\[\int_{8}^{14} {-\frac{x + 9}{9 \, \sqrt{x}}}\;dx=\answer{-\frac{82}{27} \, \sqrt{14} + \frac{140}{27} \, \sqrt{2}}\]
\end{problem}}%}

%%%%%%%%%%%%%%%%%%%%%%


\latexProblemContent{
\begin{problem}

Use the Fundamental Theorem of Calculus to evaluate the integral.

\expandafter\input{\file@loc Integrals/2311-Compute-Integral-0011.HELP.tex}

\[\int_{7}^{14} {-\frac{1}{2} \, \sqrt{x}}\;dx=\answer{-\frac{14}{3} \, \sqrt{14} + \frac{7}{3} \, \sqrt{7}}\]
\end{problem}}%}

%%%%%%%%%%%%%%%%%%%%%%


\latexProblemContent{
\begin{problem}

Use the Fundamental Theorem of Calculus to evaluate the integral.

\expandafter\input{\file@loc Integrals/2311-Compute-Integral-0011.HELP.tex}

\[\int_{1}^{10} {-\frac{x^{2} - 64}{3 \, \sqrt{x}}}\;dx=\answer{\frac{88}{3} \, \sqrt{10} - \frac{638}{15}}\]
\end{problem}}%}

%%%%%%%%%%%%%%%%%%%%%%


\latexProblemContent{
\begin{problem}

Use the Fundamental Theorem of Calculus to evaluate the integral.

\expandafter\input{\file@loc Integrals/2311-Compute-Integral-0011.HELP.tex}

\[\int_{4}^{14} {\frac{x + 3}{5 \, \sqrt{x}}}\;dx=\answer{\frac{46}{15} \, \sqrt{14} - \frac{52}{15}}\]
\end{problem}}%}

%%%%%%%%%%%%%%%%%%%%%%


%%%%%%%%%%%%%%%%%%%%%%


\latexProblemContent{
\begin{problem}

Use the Fundamental Theorem of Calculus to evaluate the integral.

\expandafter\input{\file@loc Integrals/2311-Compute-Integral-0011.HELP.tex}

\[\int_{3}^{13} {\frac{x + 4}{6 \, \sqrt{x}}}\;dx=\answer{\frac{25}{9} \, \sqrt{13} - \frac{5}{3} \, \sqrt{3}}\]
\end{problem}}%}

%%%%%%%%%%%%%%%%%%%%%%


\latexProblemContent{
\begin{problem}

Use the Fundamental Theorem of Calculus to evaluate the integral.

\expandafter\input{\file@loc Integrals/2311-Compute-Integral-0011.HELP.tex}

\[\int_{3}^{10} {-\frac{x^{2} - 17 \, x + 72}{7 \, \sqrt{x}}}\;dx=\answer{-\frac{212}{21} \, \sqrt{10} + \frac{568}{35} \, \sqrt{3}}\]
\end{problem}}%}

%%%%%%%%%%%%%%%%%%%%%%


\latexProblemContent{
\begin{problem}

Use the Fundamental Theorem of Calculus to evaluate the integral.

\expandafter\input{\file@loc Integrals/2311-Compute-Integral-0011.HELP.tex}

\[\int_{8}^{13} {-\frac{x - 6}{8 \, \sqrt{x}}}\;dx=\answer{\frac{5}{12} \, \sqrt{13} - \frac{5}{3} \, \sqrt{2}}\]
\end{problem}}%}

%%%%%%%%%%%%%%%%%%%%%%


\latexProblemContent{
\begin{problem}

Use the Fundamental Theorem of Calculus to evaluate the integral.

\expandafter\input{\file@loc Integrals/2311-Compute-Integral-0011.HELP.tex}

\[\int_{5}^{10} {\frac{x^{2} + 10 \, x + 16}{9 \, \sqrt{x}}}\;dx=\answer{\frac{416}{27} \, \sqrt{10} - \frac{226}{27} \, \sqrt{5}}\]
\end{problem}}%}

%%%%%%%%%%%%%%%%%%%%%%


\latexProblemContent{
\begin{problem}

Use the Fundamental Theorem of Calculus to evaluate the integral.

\expandafter\input{\file@loc Integrals/2311-Compute-Integral-0011.HELP.tex}

\[\int_{3}^{15} {\frac{x^{2} - 36}{8 \, \sqrt{x}}}\;dx=\answer{\frac{9}{4} \, \sqrt{15} + \frac{171}{20} \, \sqrt{3}}\]
\end{problem}}%}

%%%%%%%%%%%%%%%%%%%%%%


\latexProblemContent{
\begin{problem}

Use the Fundamental Theorem of Calculus to evaluate the integral.

\expandafter\input{\file@loc Integrals/2311-Compute-Integral-0011.HELP.tex}

\[\int_{6}^{13} {\frac{x + 9}{4 \, \sqrt{x}}}\;dx=\answer{\frac{20}{3} \, \sqrt{13} - \frac{11}{2} \, \sqrt{6}}\]
\end{problem}}%}

%%%%%%%%%%%%%%%%%%%%%%


\latexProblemContent{
\begin{problem}

Use the Fundamental Theorem of Calculus to evaluate the integral.

\expandafter\input{\file@loc Integrals/2311-Compute-Integral-0011.HELP.tex}

\[\int_{7}^{11} {-\frac{x^{2} - 10 \, x + 21}{7 \, \sqrt{x}}}\;dx=\answer{-\frac{256}{105} \, \sqrt{11} + \frac{32}{15} \, \sqrt{7}}\]
\end{problem}}%}

%%%%%%%%%%%%%%%%%%%%%%


\latexProblemContent{
\begin{problem}

Use the Fundamental Theorem of Calculus to evaluate the integral.

\expandafter\input{\file@loc Integrals/2311-Compute-Integral-0011.HELP.tex}

\[\int_{5}^{6} {\frac{x + 3}{7 \, \sqrt{x}}}\;dx=\answer{\frac{10}{7} \, \sqrt{6} - \frac{4}{3} \, \sqrt{5}}\]
\end{problem}}%}

%%%%%%%%%%%%%%%%%%%%%%


\latexProblemContent{
\begin{problem}

Use the Fundamental Theorem of Calculus to evaluate the integral.

\expandafter\input{\file@loc Integrals/2311-Compute-Integral-0011.HELP.tex}

\[\int_{5}^{14} {-\frac{x^{2} - 1}{9 \, \sqrt{x}}}\;dx=\answer{-\frac{382}{45} \, \sqrt{14} + \frac{8}{9} \, \sqrt{5}}\]
\end{problem}}%}

%%%%%%%%%%%%%%%%%%%%%%


\latexProblemContent{
\begin{problem}

Use the Fundamental Theorem of Calculus to evaluate the integral.

\expandafter\input{\file@loc Integrals/2311-Compute-Integral-0011.HELP.tex}

\[\int_{2}^{15} {\frac{x^{2} - 25}{\sqrt{x}}}\;dx=\answer{40 \, \sqrt{15} + \frac{242}{5} \, \sqrt{2}}\]
\end{problem}}%}

%%%%%%%%%%%%%%%%%%%%%%


\latexProblemContent{
\begin{problem}

Use the Fundamental Theorem of Calculus to evaluate the integral.

\expandafter\input{\file@loc Integrals/2311-Compute-Integral-0011.HELP.tex}

\[\int_{6}^{10} {\frac{x^{2} - 1}{8 \, \sqrt{x}}}\;dx=\answer{\frac{19}{4} \, \sqrt{10} - \frac{31}{20} \, \sqrt{6}}\]
\end{problem}}%}

%%%%%%%%%%%%%%%%%%%%%%


\latexProblemContent{
\begin{problem}

Use the Fundamental Theorem of Calculus to evaluate the integral.

\expandafter\input{\file@loc Integrals/2311-Compute-Integral-0011.HELP.tex}

\[\int_{8}^{9} {-\frac{x^{2} + x - 30}{10 \, \sqrt{x}}}\;dx=\answer{-\frac{436}{75} \, \sqrt{2} + \frac{162}{25}}\]
\end{problem}}%}

%%%%%%%%%%%%%%%%%%%%%%


\latexProblemContent{
\begin{problem}

Use the Fundamental Theorem of Calculus to evaluate the integral.

\expandafter\input{\file@loc Integrals/2311-Compute-Integral-0011.HELP.tex}

\[\int_{1}^{12} {-\frac{x^{2} - 10 \, x + 16}{3 \, \sqrt{x}}}\;dx=\answer{-\frac{32}{5} \, \sqrt{3} + \frac{386}{45}}\]
\end{problem}}%}

%%%%%%%%%%%%%%%%%%%%%%


\latexProblemContent{
\begin{problem}

Use the Fundamental Theorem of Calculus to evaluate the integral.

\expandafter\input{\file@loc Integrals/2311-Compute-Integral-0011.HELP.tex}

\[\int_{3}^{8} {-\frac{x^{2} - 64}{10 \, \sqrt{x}}}\;dx=\answer{-\frac{311}{25} \, \sqrt{3} + \frac{512}{25} \, \sqrt{2}}\]
\end{problem}}%}

%%%%%%%%%%%%%%%%%%%%%%


\latexProblemContent{
\begin{problem}

Use the Fundamental Theorem of Calculus to evaluate the integral.

\expandafter\input{\file@loc Integrals/2311-Compute-Integral-0011.HELP.tex}

\[\int_{2}^{5} {\frac{x + 2}{10 \, \sqrt{x}}}\;dx=\answer{\frac{11}{15} \, \sqrt{5} - \frac{8}{15} \, \sqrt{2}}\]
\end{problem}}%}

%%%%%%%%%%%%%%%%%%%%%%


\latexProblemContent{
\begin{problem}

Use the Fundamental Theorem of Calculus to evaluate the integral.

\expandafter\input{\file@loc Integrals/2311-Compute-Integral-0011.HELP.tex}

\[\int_{3}^{8} {-\frac{x + 6}{4 \, \sqrt{x}}}\;dx=\answer{\frac{7}{2} \, \sqrt{3} - \frac{26}{3} \, \sqrt{2}}\]
\end{problem}}%}

%%%%%%%%%%%%%%%%%%%%%%


\latexProblemContent{
\begin{problem}

Use the Fundamental Theorem of Calculus to evaluate the integral.

\expandafter\input{\file@loc Integrals/2311-Compute-Integral-0011.HELP.tex}

\[\int_{4}^{8} {-\frac{x^{2} + 6 \, x + 8}{4 \, \sqrt{x}}}\;dx=\answer{-\frac{184}{5} \, \sqrt{2} + \frac{96}{5}}\]
\end{problem}}%}

%%%%%%%%%%%%%%%%%%%%%%


%%%%%%%%%%%%%%%%%%%%%%


\latexProblemContent{
\begin{problem}

Use the Fundamental Theorem of Calculus to evaluate the integral.

\expandafter\input{\file@loc Integrals/2311-Compute-Integral-0011.HELP.tex}

\[\int_{5}^{7} {-\frac{x^{2} - 2 \, x - 24}{7 \, \sqrt{x}}}\;dx=\answer{\frac{566}{105} \, \sqrt{7} - \frac{134}{21} \, \sqrt{5}}\]
\end{problem}}%}

%%%%%%%%%%%%%%%%%%%%%%


\latexProblemContent{
\begin{problem}

Use the Fundamental Theorem of Calculus to evaluate the integral.

\expandafter\input{\file@loc Integrals/2311-Compute-Integral-0011.HELP.tex}

\[\int_{6}^{10} {\frac{x^{2} - 7 \, x}{5 \, \sqrt{x}}}\;dx=\answer{-\frac{4}{3} \, \sqrt{10} + \frac{68}{25} \, \sqrt{6}}\]
\end{problem}}%}

%%%%%%%%%%%%%%%%%%%%%%


\latexProblemContent{
\begin{problem}

Use the Fundamental Theorem of Calculus to evaluate the integral.

\expandafter\input{\file@loc Integrals/2311-Compute-Integral-0011.HELP.tex}

\[\int_{4}^{7} {-\frac{x - 4}{7 \, \sqrt{x}}}\;dx=\answer{\frac{10}{21} \, \sqrt{7} - \frac{32}{21}}\]
\end{problem}}%}

%%%%%%%%%%%%%%%%%%%%%%


\latexProblemContent{
\begin{problem}

Use the Fundamental Theorem of Calculus to evaluate the integral.

\expandafter\input{\file@loc Integrals/2311-Compute-Integral-0011.HELP.tex}

\[\int_{2}^{14} {\frac{x^{2} - 11 \, x + 10}{7 \, \sqrt{x}}}\;dx=\answer{-\frac{64}{105} \, \sqrt{14} - \frac{104}{105} \, \sqrt{2}}\]
\end{problem}}%}

%%%%%%%%%%%%%%%%%%%%%%


\latexProblemContent{
\begin{problem}

Use the Fundamental Theorem of Calculus to evaluate the integral.

\expandafter\input{\file@loc Integrals/2311-Compute-Integral-0011.HELP.tex}

\[\int_{8}^{14} {\frac{x^{2} - 49}{8 \, \sqrt{x}}}\;dx=\answer{-\frac{49}{20} \, \sqrt{14} + \frac{181}{10} \, \sqrt{2}}\]
\end{problem}}%}

%%%%%%%%%%%%%%%%%%%%%%


\latexProblemContent{
\begin{problem}

Use the Fundamental Theorem of Calculus to evaluate the integral.

\expandafter\input{\file@loc Integrals/2311-Compute-Integral-0011.HELP.tex}

\[\int_{4}^{10} {\frac{x + 6}{7 \, \sqrt{x}}}\;dx=\answer{\frac{8}{3} \, \sqrt{10} - \frac{88}{21}}\]
\end{problem}}%}

%%%%%%%%%%%%%%%%%%%%%%


\latexProblemContent{
\begin{problem}

Use the Fundamental Theorem of Calculus to evaluate the integral.

\expandafter\input{\file@loc Integrals/2311-Compute-Integral-0011.HELP.tex}

\[\int_{3}^{11} {\frac{x - 1}{7 \, \sqrt{x}}}\;dx=\answer{\frac{16}{21} \, \sqrt{11}}\]
\end{problem}}%}

%%%%%%%%%%%%%%%%%%%%%%


\latexProblemContent{
\begin{problem}

Use the Fundamental Theorem of Calculus to evaluate the integral.

\expandafter\input{\file@loc Integrals/2311-Compute-Integral-0011.HELP.tex}

\[\int_{2}^{7} {\frac{x^{2} + 2 \, x}{3 \, \sqrt{x}}}\;dx=\answer{\frac{434}{45} \, \sqrt{7} - \frac{64}{45} \, \sqrt{2}}\]
\end{problem}}%}

%%%%%%%%%%%%%%%%%%%%%%


\latexProblemContent{
\begin{problem}

Use the Fundamental Theorem of Calculus to evaluate the integral.

\expandafter\input{\file@loc Integrals/2311-Compute-Integral-0011.HELP.tex}

\[\int_{4}^{7} {\frac{x - 6}{3 \, \sqrt{x}}}\;dx=\answer{-\frac{22}{9} \, \sqrt{7} + \frac{56}{9}}\]
\end{problem}}%}

%%%%%%%%%%%%%%%%%%%%%%


\latexProblemContent{
\begin{problem}

Use the Fundamental Theorem of Calculus to evaluate the integral.

\expandafter\input{\file@loc Integrals/2311-Compute-Integral-0011.HELP.tex}

\[\int_{8}^{10} {\frac{x - 1}{6 \, \sqrt{x}}}\;dx=\answer{\frac{7}{9} \, \sqrt{10} - \frac{10}{9} \, \sqrt{2}}\]
\end{problem}}%}

%%%%%%%%%%%%%%%%%%%%%%


\latexProblemContent{
\begin{problem}

Use the Fundamental Theorem of Calculus to evaluate the integral.

\expandafter\input{\file@loc Integrals/2311-Compute-Integral-0011.HELP.tex}

\[\int_{7}^{8} {-\frac{x^{2} + 13 \, x + 42}{9 \, \sqrt{x}}}\;dx=\answer{\frac{2464}{135} \, \sqrt{7} - \frac{5368}{135} \, \sqrt{2}}\]
\end{problem}}%}

%%%%%%%%%%%%%%%%%%%%%%


\latexProblemContent{
\begin{problem}

Use the Fundamental Theorem of Calculus to evaluate the integral.

\expandafter\input{\file@loc Integrals/2311-Compute-Integral-0011.HELP.tex}

\[\int_{7}^{15} {\frac{x - 8}{7 \, \sqrt{x}}}\;dx=\answer{-\frac{6}{7} \, \sqrt{15} + \frac{34}{21} \, \sqrt{7}}\]
\end{problem}}%}

%%%%%%%%%%%%%%%%%%%%%%


\latexProblemContent{
\begin{problem}

Use the Fundamental Theorem of Calculus to evaluate the integral.

\expandafter\input{\file@loc Integrals/2311-Compute-Integral-0011.HELP.tex}

\[\int_{7}^{9} {\frac{x^{2} - 16}{4 \, \sqrt{x}}}\;dx=\answer{\frac{31}{10} \, \sqrt{7} + \frac{3}{10}}\]
\end{problem}}%}

%%%%%%%%%%%%%%%%%%%%%%


\latexProblemContent{
\begin{problem}

Use the Fundamental Theorem of Calculus to evaluate the integral.

\expandafter\input{\file@loc Integrals/2311-Compute-Integral-0011.HELP.tex}

\[\int_{7}^{11} {-\frac{x + 7}{\sqrt{x}}}\;dx=\answer{-\frac{64}{3} \, \sqrt{11} + \frac{56}{3} \, \sqrt{7}}\]
\end{problem}}%}

%%%%%%%%%%%%%%%%%%%%%%


\latexProblemContent{
\begin{problem}

Use the Fundamental Theorem of Calculus to evaluate the integral.

\expandafter\input{\file@loc Integrals/2311-Compute-Integral-0011.HELP.tex}

\[\int_{5}^{10} {\frac{x + 6}{9 \, \sqrt{x}}}\;dx=\answer{\frac{56}{27} \, \sqrt{10} - \frac{46}{27} \, \sqrt{5}}\]
\end{problem}}%}

%%%%%%%%%%%%%%%%%%%%%%


\latexProblemContent{
\begin{problem}

Use the Fundamental Theorem of Calculus to evaluate the integral.

\expandafter\input{\file@loc Integrals/2311-Compute-Integral-0011.HELP.tex}

\[\int_{6}^{9} {-\frac{x^{2} - 16}{9 \, \sqrt{x}}}\;dx=\answer{-\frac{88}{45} \, \sqrt{6} - \frac{2}{15}}\]
\end{problem}}%}

%%%%%%%%%%%%%%%%%%%%%%


\latexProblemContent{
\begin{problem}

Use the Fundamental Theorem of Calculus to evaluate the integral.

\expandafter\input{\file@loc Integrals/2311-Compute-Integral-0011.HELP.tex}

\[\int_{5}^{7} {\frac{x - 2}{\sqrt{x}}}\;dx=\answer{\frac{2}{3} \, \sqrt{7} + \frac{2}{3} \, \sqrt{5}}\]
\end{problem}}%}

%%%%%%%%%%%%%%%%%%%%%%


\latexProblemContent{
\begin{problem}

Use the Fundamental Theorem of Calculus to evaluate the integral.

\expandafter\input{\file@loc Integrals/2311-Compute-Integral-0011.HELP.tex}

\[\int_{6}^{10} {\frac{x^{2} - 1}{10 \, \sqrt{x}}}\;dx=\answer{\frac{19}{5} \, \sqrt{10} - \frac{31}{25} \, \sqrt{6}}\]
\end{problem}}%}

%%%%%%%%%%%%%%%%%%%%%%


\latexProblemContent{
\begin{problem}

Use the Fundamental Theorem of Calculus to evaluate the integral.

\expandafter\input{\file@loc Integrals/2311-Compute-Integral-0011.HELP.tex}

\[\int_{6}^{9} {\frac{x - 6}{10 \, \sqrt{x}}}\;dx=\answer{\frac{4}{5} \, \sqrt{6} - \frac{9}{5}}\]
\end{problem}}%}

%%%%%%%%%%%%%%%%%%%%%%


\latexProblemContent{
\begin{problem}

Use the Fundamental Theorem of Calculus to evaluate the integral.

\expandafter\input{\file@loc Integrals/2311-Compute-Integral-0011.HELP.tex}

\[\int_{5}^{12} {\frac{x^{2} + 7 \, x - 30}{5 \, \sqrt{x}}}\;dx=\answer{\frac{16}{3} \, \sqrt{5} + \frac{536}{25} \, \sqrt{3}}\]
\end{problem}}%}

%%%%%%%%%%%%%%%%%%%%%%


\latexProblemContent{
\begin{problem}

Use the Fundamental Theorem of Calculus to evaluate the integral.

\expandafter\input{\file@loc Integrals/2311-Compute-Integral-0011.HELP.tex}

\[\int_{1}^{14} {\frac{x^{2} - 25}{7 \, \sqrt{x}}}\;dx=\answer{\frac{142}{35} \, \sqrt{14} + \frac{248}{35}}\]
\end{problem}}%}

%%%%%%%%%%%%%%%%%%%%%%


\latexProblemContent{
\begin{problem}

Use the Fundamental Theorem of Calculus to evaluate the integral.

\expandafter\input{\file@loc Integrals/2311-Compute-Integral-0011.HELP.tex}

\[\int_{5}^{7} {-\frac{x - 7}{\sqrt{x}}}\;dx=\answer{\frac{28}{3} \, \sqrt{7} - \frac{32}{3} \, \sqrt{5}}\]
\end{problem}}%}

%%%%%%%%%%%%%%%%%%%%%%


\latexProblemContent{
\begin{problem}

Use the Fundamental Theorem of Calculus to evaluate the integral.

\expandafter\input{\file@loc Integrals/2311-Compute-Integral-0011.HELP.tex}

\[\int_{2}^{13} {-\frac{x^{2} - 36}{5 \, \sqrt{x}}}\;dx=\answer{\frac{22}{25} \, \sqrt{13} - \frac{352}{25} \, \sqrt{2}}\]
\end{problem}}%}

%%%%%%%%%%%%%%%%%%%%%%


\latexProblemContent{
\begin{problem}

Use the Fundamental Theorem of Calculus to evaluate the integral.

\expandafter\input{\file@loc Integrals/2311-Compute-Integral-0011.HELP.tex}

\[\int_{3}^{5} {-\frac{x^{2} - 9}{10 \, \sqrt{x}}}\;dx=\answer{\frac{4}{5} \, \sqrt{5} - \frac{36}{25} \, \sqrt{3}}\]
\end{problem}}%}

%%%%%%%%%%%%%%%%%%%%%%


\latexProblemContent{
\begin{problem}

Use the Fundamental Theorem of Calculus to evaluate the integral.

\expandafter\input{\file@loc Integrals/2311-Compute-Integral-0011.HELP.tex}

\[\int_{1}^{9} {-\frac{x^{2} - 11 \, x + 10}{9 \, \sqrt{x}}}\;dx=\answer{\frac{808}{135}}\]
\end{problem}}%}

%%%%%%%%%%%%%%%%%%%%%%


\latexProblemContent{
\begin{problem}

Use the Fundamental Theorem of Calculus to evaluate the integral.

\expandafter\input{\file@loc Integrals/2311-Compute-Integral-0011.HELP.tex}

\[\int_{1}^{13} {-\frac{1}{10} \, x^{\frac{3}{2}}}\;dx=\answer{-\frac{169}{25} \, \sqrt{13} + \frac{1}{25}}\]
\end{problem}}%}

%%%%%%%%%%%%%%%%%%%%%%


\latexProblemContent{
\begin{problem}

Use the Fundamental Theorem of Calculus to evaluate the integral.

\expandafter\input{\file@loc Integrals/2311-Compute-Integral-0011.HELP.tex}

\[\int_{7}^{11} {-\frac{x + 2}{8 \, \sqrt{x}}}\;dx=\answer{-\frac{17}{12} \, \sqrt{11} + \frac{13}{12} \, \sqrt{7}}\]
\end{problem}}%}

%%%%%%%%%%%%%%%%%%%%%%


\latexProblemContent{
\begin{problem}

Use the Fundamental Theorem of Calculus to evaluate the integral.

\expandafter\input{\file@loc Integrals/2311-Compute-Integral-0011.HELP.tex}

\[\int_{1}^{12} {-\frac{x - 3}{6 \, \sqrt{x}}}\;dx=\answer{-\frac{2}{3} \, \sqrt{3} - \frac{8}{9}}\]
\end{problem}}%}

%%%%%%%%%%%%%%%%%%%%%%


\latexProblemContent{
\begin{problem}

Use the Fundamental Theorem of Calculus to evaluate the integral.

\expandafter\input{\file@loc Integrals/2311-Compute-Integral-0011.HELP.tex}

\[\int_{2}^{6} {\frac{x^{2} - 81}{5 \, \sqrt{x}}}\;dx=\answer{-\frac{738}{25} \, \sqrt{6} + \frac{802}{25} \, \sqrt{2}}\]
\end{problem}}%}

%%%%%%%%%%%%%%%%%%%%%%


\latexProblemContent{
\begin{problem}

Use the Fundamental Theorem of Calculus to evaluate the integral.

\expandafter\input{\file@loc Integrals/2311-Compute-Integral-0011.HELP.tex}

\[\int_{7}^{10} {\frac{x^{2} - 16 \, x + 63}{5 \, \sqrt{x}}}\;dx=\answer{\frac{178}{15} \, \sqrt{10} - \frac{1064}{75} \, \sqrt{7}}\]
\end{problem}}%}

%%%%%%%%%%%%%%%%%%%%%%


\latexProblemContent{
\begin{problem}

Use the Fundamental Theorem of Calculus to evaluate the integral.

\expandafter\input{\file@loc Integrals/2311-Compute-Integral-0011.HELP.tex}

\[\int_{3}^{5} {-\frac{x - 2}{5 \, \sqrt{x}}}\;dx=\answer{\frac{2}{15} \, \sqrt{5} - \frac{2}{5} \, \sqrt{3}}\]
\end{problem}}%}

%%%%%%%%%%%%%%%%%%%%%%


\latexProblemContent{
\begin{problem}

Use the Fundamental Theorem of Calculus to evaluate the integral.

\expandafter\input{\file@loc Integrals/2311-Compute-Integral-0011.HELP.tex}

\[\int_{4}^{12} {-\frac{x^{2} - 49}{\sqrt{x}}}\;dx=\answer{\frac{404}{5} \, \sqrt{3} - \frac{916}{5}}\]
\end{problem}}%}

%%%%%%%%%%%%%%%%%%%%%%


\latexProblemContent{
\begin{problem}

Use the Fundamental Theorem of Calculus to evaluate the integral.

\expandafter\input{\file@loc Integrals/2311-Compute-Integral-0011.HELP.tex}

\[\int_{3}^{11} {\frac{x - 4}{3 \, \sqrt{x}}}\;dx=\answer{-\frac{2}{9} \, \sqrt{11} + 2 \, \sqrt{3}}\]
\end{problem}}%}

%%%%%%%%%%%%%%%%%%%%%%


\latexProblemContent{
\begin{problem}

Use the Fundamental Theorem of Calculus to evaluate the integral.

\expandafter\input{\file@loc Integrals/2311-Compute-Integral-0011.HELP.tex}

\[\int_{5}^{6} {\frac{x^{2} + 13 \, x + 36}{7 \, \sqrt{x}}}\;dx=\answer{\frac{692}{35} \, \sqrt{6} - \frac{376}{21} \, \sqrt{5}}\]
\end{problem}}%}

%%%%%%%%%%%%%%%%%%%%%%


\latexProblemContent{
\begin{problem}

Use the Fundamental Theorem of Calculus to evaluate the integral.

\expandafter\input{\file@loc Integrals/2311-Compute-Integral-0011.HELP.tex}

\[\int_{2}^{10} {-\frac{x - 5}{2 \, \sqrt{x}}}\;dx=\answer{\frac{1}{6} \, \left(10 \, \sqrt{10}\right) - \frac{13}{3} \, \sqrt{2}}\]
\end{problem}}%}

%%%%%%%%%%%%%%%%%%%%%%


\latexProblemContent{
\begin{problem}

Use the Fundamental Theorem of Calculus to evaluate the integral.

\expandafter\input{\file@loc Integrals/2311-Compute-Integral-0011.HELP.tex}

\[\int_{3}^{10} {\frac{x^{2} + 15 \, x + 50}{2 \, \sqrt{x}}}\;dx=\answer{120 \, \sqrt{10} - \frac{334}{5} \, \sqrt{3}}\]
\end{problem}}%}

%%%%%%%%%%%%%%%%%%%%%%


\latexProblemContent{
\begin{problem}

Use the Fundamental Theorem of Calculus to evaluate the integral.

\expandafter\input{\file@loc Integrals/2311-Compute-Integral-0011.HELP.tex}

\[\int_{3}^{12} {-\frac{x^{2} + 3 \, x - 4}{\sqrt{x}}}\;dx=\answer{-\frac{728}{5} \, \sqrt{3}}\]
\end{problem}}%}

%%%%%%%%%%%%%%%%%%%%%%


\latexProblemContent{
\begin{problem}

Use the Fundamental Theorem of Calculus to evaluate the integral.

\expandafter\input{\file@loc Integrals/2311-Compute-Integral-0011.HELP.tex}

\[\int_{6}^{8} {-\frac{x^{2} - 3 \, x + 2}{6 \, \sqrt{x}}}\;dx=\answer{\frac{16}{15} \, \sqrt{6} - \frac{68}{15} \, \sqrt{2}}\]
\end{problem}}%}

%%%%%%%%%%%%%%%%%%%%%%


\latexProblemContent{
\begin{problem}

Use the Fundamental Theorem of Calculus to evaluate the integral.

\expandafter\input{\file@loc Integrals/2311-Compute-Integral-0011.HELP.tex}

\[\int_{8}^{15} {\frac{x^{2} - 5 \, x + 4}{\sqrt{x}}}\;dx=\answer{48 \, \sqrt{15} - \frac{208}{15} \, \sqrt{2}}\]
\end{problem}}%}

%%%%%%%%%%%%%%%%%%%%%%


\latexProblemContent{
\begin{problem}

Use the Fundamental Theorem of Calculus to evaluate the integral.

\expandafter\input{\file@loc Integrals/2311-Compute-Integral-0011.HELP.tex}

\[\int_{2}^{10} {\frac{x^{2} - 25}{5 \, \sqrt{x}}}\;dx=\answer{-2 \, \sqrt{10} + \frac{242}{25} \, \sqrt{2}}\]
\end{problem}}%}

%%%%%%%%%%%%%%%%%%%%%%


\latexProblemContent{
\begin{problem}

Use the Fundamental Theorem of Calculus to evaluate the integral.

\expandafter\input{\file@loc Integrals/2311-Compute-Integral-0011.HELP.tex}

\[\int_{7}^{14} {\frac{x^{2} - 1}{\sqrt{x}}}\;dx=\answer{\frac{382}{5} \, \sqrt{14} - \frac{88}{5} \, \sqrt{7}}\]
\end{problem}}%}

%%%%%%%%%%%%%%%%%%%%%%


\latexProblemContent{
\begin{problem}

Use the Fundamental Theorem of Calculus to evaluate the integral.

\expandafter\input{\file@loc Integrals/2311-Compute-Integral-0011.HELP.tex}

\[\int_{8}^{9} {\frac{x^{2} - 4}{5 \, \sqrt{x}}}\;dx=\answer{-\frac{176}{25} \, \sqrt{2} + \frac{366}{25}}\]
\end{problem}}%}

%%%%%%%%%%%%%%%%%%%%%%


\latexProblemContent{
\begin{problem}

Use the Fundamental Theorem of Calculus to evaluate the integral.

\expandafter\input{\file@loc Integrals/2311-Compute-Integral-0011.HELP.tex}

\[\int_{5}^{9} {-\frac{x + 10}{\sqrt{x}}}\;dx=\answer{\frac{70}{3} \, \sqrt{5} - 78}\]
\end{problem}}%}

%%%%%%%%%%%%%%%%%%%%%%


\latexProblemContent{
\begin{problem}

Use the Fundamental Theorem of Calculus to evaluate the integral.

\expandafter\input{\file@loc Integrals/2311-Compute-Integral-0011.HELP.tex}

\[\int_{1}^{12} {\frac{1}{5} \, x^{\frac{3}{2}}}\;dx=\answer{\frac{576}{25} \, \sqrt{3} - \frac{2}{25}}\]
\end{problem}}%}

%%%%%%%%%%%%%%%%%%%%%%


\latexProblemContent{
\begin{problem}

Use the Fundamental Theorem of Calculus to evaluate the integral.

\expandafter\input{\file@loc Integrals/2311-Compute-Integral-0011.HELP.tex}

\[\int_{3}^{9} {-\frac{x + 4}{\sqrt{x}}}\;dx=\answer{10 \, \sqrt{3} - 42}\]
\end{problem}}%}

%%%%%%%%%%%%%%%%%%%%%%


\latexProblemContent{
\begin{problem}

Use the Fundamental Theorem of Calculus to evaluate the integral.

\expandafter\input{\file@loc Integrals/2311-Compute-Integral-0011.HELP.tex}

\[\int_{1}^{6} {-\frac{x + 5}{9 \, \sqrt{x}}}\;dx=\answer{-\frac{14}{9} \, \sqrt{6} + \frac{32}{27}}\]
\end{problem}}%}

%%%%%%%%%%%%%%%%%%%%%%


\latexProblemContent{
\begin{problem}

Use the Fundamental Theorem of Calculus to evaluate the integral.

\expandafter\input{\file@loc Integrals/2311-Compute-Integral-0011.HELP.tex}

\[\int_{7}^{13} {\frac{x + 10}{3 \, \sqrt{x}}}\;dx=\answer{\frac{86}{9} \, \sqrt{13} - \frac{74}{9} \, \sqrt{7}}\]
\end{problem}}%}

%%%%%%%%%%%%%%%%%%%%%%


\latexProblemContent{
\begin{problem}

Use the Fundamental Theorem of Calculus to evaluate the integral.

\expandafter\input{\file@loc Integrals/2311-Compute-Integral-0011.HELP.tex}

\[\int_{7}^{11} {\frac{x^{2} - 5 \, x - 50}{10 \, \sqrt{x}}}\;dx=\answer{-\frac{662}{75} \, \sqrt{11} + \frac{778}{75} \, \sqrt{7}}\]
\end{problem}}%}

%%%%%%%%%%%%%%%%%%%%%%


\latexProblemContent{
\begin{problem}

Use the Fundamental Theorem of Calculus to evaluate the integral.

\expandafter\input{\file@loc Integrals/2311-Compute-Integral-0011.HELP.tex}

\[\int_{6}^{13} {-\frac{x^{2} + 3 \, x - 28}{5 \, \sqrt{x}}}\;dx=\answer{-\frac{188}{25} \, \sqrt{13} - \frac{148}{25} \, \sqrt{6}}\]
\end{problem}}%}

%%%%%%%%%%%%%%%%%%%%%%


\latexProblemContent{
\begin{problem}

Use the Fundamental Theorem of Calculus to evaluate the integral.

\expandafter\input{\file@loc Integrals/2311-Compute-Integral-0011.HELP.tex}

\[\int_{6}^{10} {-\frac{x + 3}{4 \, \sqrt{x}}}\;dx=\answer{-\frac{19}{6} \, \sqrt{10} + \frac{5}{2} \, \sqrt{6}}\]
\end{problem}}%}

%%%%%%%%%%%%%%%%%%%%%%


\latexProblemContent{
\begin{problem}

Use the Fundamental Theorem of Calculus to evaluate the integral.

\expandafter\input{\file@loc Integrals/2311-Compute-Integral-0011.HELP.tex}

\[\int_{2}^{10} {\frac{x - 7}{10 \, \sqrt{x}}}\;dx=\answer{-\frac{11}{15} \, \sqrt{10} + \frac{19}{15} \, \sqrt{2}}\]
\end{problem}}%}

%%%%%%%%%%%%%%%%%%%%%%


\latexProblemContent{
\begin{problem}

Use the Fundamental Theorem of Calculus to evaluate the integral.

\expandafter\input{\file@loc Integrals/2311-Compute-Integral-0011.HELP.tex}

\[\int_{6}^{8} {\frac{x^{2} - 10 \, x + 24}{8 \, \sqrt{x}}}\;dx=\answer{-\frac{14}{5} \, \sqrt{6} + \frac{76}{15} \, \sqrt{2}}\]
\end{problem}}%}

%%%%%%%%%%%%%%%%%%%%%%


\latexProblemContent{
\begin{problem}

Use the Fundamental Theorem of Calculus to evaluate the integral.

\expandafter\input{\file@loc Integrals/2311-Compute-Integral-0011.HELP.tex}

\[\int_{8}^{9} {\frac{x^{2} - 49}{4 \, \sqrt{x}}}\;dx=\answer{\frac{181}{5} \, \sqrt{2} - \frac{246}{5}}\]
\end{problem}}%}

%%%%%%%%%%%%%%%%%%%%%%


\latexProblemContent{
\begin{problem}

Use the Fundamental Theorem of Calculus to evaluate the integral.

\expandafter\input{\file@loc Integrals/2311-Compute-Integral-0011.HELP.tex}

\[\int_{4}^{10} {\frac{x^{2} + x - 2}{4 \, \sqrt{x}}}\;dx=\answer{\frac{32}{3} \, \sqrt{10} - \frac{38}{15}}\]
\end{problem}}%}

%%%%%%%%%%%%%%%%%%%%%%


\latexProblemContent{
\begin{problem}

Use the Fundamental Theorem of Calculus to evaluate the integral.

\expandafter\input{\file@loc Integrals/2311-Compute-Integral-0011.HELP.tex}

\[\int_{5}^{12} {-\frac{x^{2} - 4}{2 \, \sqrt{x}}}\;dx=\answer{\sqrt{5} - \frac{248}{5} \, \sqrt{3}}\]
\end{problem}}%}

%%%%%%%%%%%%%%%%%%%%%%


\latexProblemContent{
\begin{problem}

Use the Fundamental Theorem of Calculus to evaluate the integral.

\expandafter\input{\file@loc Integrals/2311-Compute-Integral-0011.HELP.tex}

\[\int_{6}^{7} {\frac{x^{2} + 3 \, x - 4}{5 \, \sqrt{x}}}\;dx=\answer{\frac{128}{25} \, \sqrt{7} - \frac{92}{25} \, \sqrt{6}}\]
\end{problem}}%}

%%%%%%%%%%%%%%%%%%%%%%


\latexProblemContent{
\begin{problem}

Use the Fundamental Theorem of Calculus to evaluate the integral.

\expandafter\input{\file@loc Integrals/2311-Compute-Integral-0011.HELP.tex}

\[\int_{1}^{10} {\frac{x^{2} + 11 \, x + 30}{2 \, \sqrt{x}}}\;dx=\answer{\frac{260}{3} \, \sqrt{10} - \frac{508}{15}}\]
\end{problem}}%}

%%%%%%%%%%%%%%%%%%%%%%


\latexProblemContent{
\begin{problem}

Use the Fundamental Theorem of Calculus to evaluate the integral.

\expandafter\input{\file@loc Integrals/2311-Compute-Integral-0011.HELP.tex}

\[\int_{7}^{10} {\frac{x^{2} - 8 \, x - 20}{3 \, \sqrt{x}}}\;dx=\answer{-\frac{160}{9} \, \sqrt{10} + \frac{866}{45} \, \sqrt{7}}\]
\end{problem}}%}

%%%%%%%%%%%%%%%%%%%%%%


\latexProblemContent{
\begin{problem}

Use the Fundamental Theorem of Calculus to evaluate the integral.

\expandafter\input{\file@loc Integrals/2311-Compute-Integral-0011.HELP.tex}

\[\int_{2}^{13} {-\frac{x + 6}{5 \, \sqrt{x}}}\;dx=\answer{-\frac{62}{15} \, \sqrt{13} + \frac{8}{3} \, \sqrt{2}}\]
\end{problem}}%}

%%%%%%%%%%%%%%%%%%%%%%


\latexProblemContent{
\begin{problem}

Use the Fundamental Theorem of Calculus to evaluate the integral.

\expandafter\input{\file@loc Integrals/2311-Compute-Integral-0011.HELP.tex}

\[\int_{8}^{13} {-\frac{x^{2} + x - 30}{\sqrt{x}}}\;dx=\answer{-\frac{244}{15} \, \sqrt{13} - \frac{872}{15} \, \sqrt{2}}\]
\end{problem}}%}

%%%%%%%%%%%%%%%%%%%%%%


\latexProblemContent{
\begin{problem}

Use the Fundamental Theorem of Calculus to evaluate the integral.

\expandafter\input{\file@loc Integrals/2311-Compute-Integral-0011.HELP.tex}

\[\int_{2}^{12} {-\frac{x^{2} - x - 12}{10 \, \sqrt{x}}}\;dx=\answer{-\frac{128}{25} \, \sqrt{3} - \frac{178}{75} \, \sqrt{2}}\]
\end{problem}}%}

%%%%%%%%%%%%%%%%%%%%%%


\latexProblemContent{
\begin{problem}

Use the Fundamental Theorem of Calculus to evaluate the integral.

\expandafter\input{\file@loc Integrals/2311-Compute-Integral-0011.HELP.tex}

\[\int_{7}^{12} {\frac{x - 4}{6 \, \sqrt{x}}}\;dx=\answer{\frac{5}{9} \, \sqrt{7}}\]
\end{problem}}%}

%%%%%%%%%%%%%%%%%%%%%%


\latexProblemContent{
\begin{problem}

Use the Fundamental Theorem of Calculus to evaluate the integral.

\expandafter\input{\file@loc Integrals/2311-Compute-Integral-0011.HELP.tex}

\[\int_{1}^{5} {\frac{x^{2} - 25}{9 \, \sqrt{x}}}\;dx=\answer{-\frac{40}{9} \, \sqrt{5} + \frac{248}{45}}\]
\end{problem}}%}

%%%%%%%%%%%%%%%%%%%%%%


\latexProblemContent{
\begin{problem}

Use the Fundamental Theorem of Calculus to evaluate the integral.

\expandafter\input{\file@loc Integrals/2311-Compute-Integral-0011.HELP.tex}

\[\int_{1}^{7} {\frac{x^{2} - 6 \, x + 9}{3 \, \sqrt{x}}}\;dx=\answer{\frac{16}{5} \, \sqrt{7} - \frac{24}{5}}\]
\end{problem}}%}

%%%%%%%%%%%%%%%%%%%%%%


\latexProblemContent{
\begin{problem}

Use the Fundamental Theorem of Calculus to evaluate the integral.

\expandafter\input{\file@loc Integrals/2311-Compute-Integral-0011.HELP.tex}

\[\int_{5}^{14} {-\frac{x^{2} - x - 72}{3 \, \sqrt{x}}}\;dx=\answer{\frac{1124}{45} \, \sqrt{14} - \frac{412}{9} \, \sqrt{5}}\]
\end{problem}}%}

%%%%%%%%%%%%%%%%%%%%%%


\latexProblemContent{
\begin{problem}

Use the Fundamental Theorem of Calculus to evaluate the integral.

\expandafter\input{\file@loc Integrals/2311-Compute-Integral-0011.HELP.tex}

\[\int_{2}^{11} {\frac{x - 5}{7 \, \sqrt{x}}}\;dx=\answer{-\frac{8}{21} \, \sqrt{11} + \frac{26}{21} \, \sqrt{2}}\]
\end{problem}}%}

%%%%%%%%%%%%%%%%%%%%%%


\latexProblemContent{
\begin{problem}

Use the Fundamental Theorem of Calculus to evaluate the integral.

\expandafter\input{\file@loc Integrals/2311-Compute-Integral-0011.HELP.tex}

\[\int_{5}^{14} {\frac{x + 4}{5 \, \sqrt{x}}}\;dx=\answer{\frac{52}{15} \, \sqrt{14} - \frac{34}{15} \, \sqrt{5}}\]
\end{problem}}%}

%%%%%%%%%%%%%%%%%%%%%%


\latexProblemContent{
\begin{problem}

Use the Fundamental Theorem of Calculus to evaluate the integral.

\expandafter\input{\file@loc Integrals/2311-Compute-Integral-0011.HELP.tex}

\[\int_{6}^{7} {-\frac{x^{2} - 25}{10 \, \sqrt{x}}}\;dx=\answer{\frac{76}{25} \, \sqrt{7} - \frac{89}{25} \, \sqrt{6}}\]
\end{problem}}%}

%%%%%%%%%%%%%%%%%%%%%%


\latexProblemContent{
\begin{problem}

Use the Fundamental Theorem of Calculus to evaluate the integral.

\expandafter\input{\file@loc Integrals/2311-Compute-Integral-0011.HELP.tex}

\[\int_{4}^{12} {\frac{x + 7}{10 \, \sqrt{x}}}\;dx=\answer{\frac{22}{5} \, \sqrt{3} - \frac{10}{3}}\]
\end{problem}}%}

%%%%%%%%%%%%%%%%%%%%%%


\latexProblemContent{
\begin{problem}

Use the Fundamental Theorem of Calculus to evaluate the integral.

\expandafter\input{\file@loc Integrals/2311-Compute-Integral-0011.HELP.tex}

\[\int_{3}^{4} {-\frac{x^{2} + 16 \, x + 64}{4 \, \sqrt{x}}}\;dx=\answer{\frac{409}{10} \, \sqrt{3} - \frac{1328}{15}}\]
\end{problem}}%}

%%%%%%%%%%%%%%%%%%%%%%


\latexProblemContent{
\begin{problem}

Use the Fundamental Theorem of Calculus to evaluate the integral.

\expandafter\input{\file@loc Integrals/2311-Compute-Integral-0011.HELP.tex}

\[\int_{4}^{6} {\frac{x^{2} + 11 \, x + 24}{7 \, \sqrt{x}}}\;dx=\answer{\frac{76}{5} \, \sqrt{6} - \frac{2512}{105}}\]
\end{problem}}%}

%%%%%%%%%%%%%%%%%%%%%%


\latexProblemContent{
\begin{problem}

Use the Fundamental Theorem of Calculus to evaluate the integral.

\expandafter\input{\file@loc Integrals/2311-Compute-Integral-0011.HELP.tex}

\[\int_{1}^{8} {\frac{x^{2} - 4}{7 \, \sqrt{x}}}\;dx=\answer{\frac{176}{35} \, \sqrt{2} + \frac{38}{35}}\]
\end{problem}}%}

%%%%%%%%%%%%%%%%%%%%%%


\latexProblemContent{
\begin{problem}

Use the Fundamental Theorem of Calculus to evaluate the integral.

\expandafter\input{\file@loc Integrals/2311-Compute-Integral-0011.HELP.tex}

\[\int_{2}^{11} {-\frac{x - 9}{\sqrt{x}}}\;dx=\answer{\frac{32}{3} \, \sqrt{11} - \frac{50}{3} \, \sqrt{2}}\]
\end{problem}}%}

%%%%%%%%%%%%%%%%%%%%%%


\latexProblemContent{
\begin{problem}

Use the Fundamental Theorem of Calculus to evaluate the integral.

\expandafter\input{\file@loc Integrals/2311-Compute-Integral-0011.HELP.tex}

\[\int_{1}^{13} {\frac{x^{2} - 100}{9 \, \sqrt{x}}}\;dx=\answer{-\frac{662}{45} \, \sqrt{13} + \frac{998}{45}}\]
\end{problem}}%}

%%%%%%%%%%%%%%%%%%%%%%


\latexProblemContent{
\begin{problem}

Use the Fundamental Theorem of Calculus to evaluate the integral.

\expandafter\input{\file@loc Integrals/2311-Compute-Integral-0011.HELP.tex}

\[\int_{5}^{7} {\frac{x - 4}{3 \, \sqrt{x}}}\;dx=\answer{-\frac{10}{9} \, \sqrt{7} + \frac{14}{9} \, \sqrt{5}}\]
\end{problem}}%}

%%%%%%%%%%%%%%%%%%%%%%


\latexProblemContent{
\begin{problem}

Use the Fundamental Theorem of Calculus to evaluate the integral.

\expandafter\input{\file@loc Integrals/2311-Compute-Integral-0011.HELP.tex}

\[\int_{1}^{4} {\frac{x^{2} - 1}{3 \, \sqrt{x}}}\;dx=\answer{\frac{52}{15}}\]
\end{problem}}%}

%%%%%%%%%%%%%%%%%%%%%%


\latexProblemContent{
\begin{problem}

Use the Fundamental Theorem of Calculus to evaluate the integral.

\expandafter\input{\file@loc Integrals/2311-Compute-Integral-0011.HELP.tex}

\[\int_{6}^{15} {-\frac{x^{2} - 16}{9 \, \sqrt{x}}}\;dx=\answer{-\frac{58}{9} \, \sqrt{15} - \frac{88}{45} \, \sqrt{6}}\]
\end{problem}}%}

%%%%%%%%%%%%%%%%%%%%%%


\latexProblemContent{
\begin{problem}

Use the Fundamental Theorem of Calculus to evaluate the integral.

\expandafter\input{\file@loc Integrals/2311-Compute-Integral-0011.HELP.tex}

\[\int_{7}^{14} {-\frac{x^{2} + 18 \, x + 80}{10 \, \sqrt{x}}}\;dx=\answer{-\frac{1016}{25} \, \sqrt{14} + \frac{659}{25} \, \sqrt{7}}\]
\end{problem}}%}

%%%%%%%%%%%%%%%%%%%%%%


\latexProblemContent{
\begin{problem}

Use the Fundamental Theorem of Calculus to evaluate the integral.

\expandafter\input{\file@loc Integrals/2311-Compute-Integral-0011.HELP.tex}

\[\int_{3}^{12} {-\frac{x^{2} + 7 \, x - 30}{4 \, \sqrt{x}}}\;dx=\answer{-\frac{187}{5} \, \sqrt{3}}\]
\end{problem}}%}

%%%%%%%%%%%%%%%%%%%%%%


\latexProblemContent{
\begin{problem}

Use the Fundamental Theorem of Calculus to evaluate the integral.

\expandafter\input{\file@loc Integrals/2311-Compute-Integral-0011.HELP.tex}

\[\int_{1}^{12} {\frac{x + 4}{8 \, \sqrt{x}}}\;dx=\answer{4 \, \sqrt{3} - \frac{13}{12}}\]
\end{problem}}%}

%%%%%%%%%%%%%%%%%%%%%%


\latexProblemContent{
\begin{problem}

Use the Fundamental Theorem of Calculus to evaluate the integral.

\expandafter\input{\file@loc Integrals/2311-Compute-Integral-0011.HELP.tex}

\[\int_{3}^{5} {-\frac{x^{2} - 49}{10 \, \sqrt{x}}}\;dx=\answer{\frac{44}{5} \, \sqrt{5} - \frac{236}{25} \, \sqrt{3}}\]
\end{problem}}%}

%%%%%%%%%%%%%%%%%%%%%%


\latexProblemContent{
\begin{problem}

Use the Fundamental Theorem of Calculus to evaluate the integral.

\expandafter\input{\file@loc Integrals/2311-Compute-Integral-0011.HELP.tex}

\[\int_{4}^{10} {\frac{x^{2} - 8 \, x + 7}{2 \, \sqrt{x}}}\;dx=\answer{\frac{1}{30} \, \left(10 \, \sqrt{10}\right) + \frac{14}{15}}\]
\end{problem}}%}

%%%%%%%%%%%%%%%%%%%%%%


\latexProblemContent{
\begin{problem}

Use the Fundamental Theorem of Calculus to evaluate the integral.

\expandafter\input{\file@loc Integrals/2311-Compute-Integral-0011.HELP.tex}

\[\int_{5}^{14} {-\frac{x - 6}{3 \, \sqrt{x}}}\;dx=\answer{\frac{8}{9} \, \sqrt{14} - \frac{26}{9} \, \sqrt{5}}\]
\end{problem}}%}

%%%%%%%%%%%%%%%%%%%%%%


\latexProblemContent{
\begin{problem}

Use the Fundamental Theorem of Calculus to evaluate the integral.

\expandafter\input{\file@loc Integrals/2311-Compute-Integral-0011.HELP.tex}

\[\int_{7}^{10} {\frac{x - 9}{10 \, \sqrt{x}}}\;dx=\answer{-\frac{17}{15} \, \sqrt{10} + \frac{4}{3} \, \sqrt{7}}\]
\end{problem}}%}

%%%%%%%%%%%%%%%%%%%%%%


\latexProblemContent{
\begin{problem}

Use the Fundamental Theorem of Calculus to evaluate the integral.

\expandafter\input{\file@loc Integrals/2311-Compute-Integral-0011.HELP.tex}

\[\int_{5}^{10} {-\frac{x^{2} - 25}{9 \, \sqrt{x}}}\;dx=\answer{\frac{10}{9} \, \sqrt{10} - \frac{40}{9} \, \sqrt{5}}\]
\end{problem}}%}

%%%%%%%%%%%%%%%%%%%%%%


\latexProblemContent{
\begin{problem}

Use the Fundamental Theorem of Calculus to evaluate the integral.

\expandafter\input{\file@loc Integrals/2311-Compute-Integral-0011.HELP.tex}

\[\int_{7}^{12} {\frac{x^{2} + 4 \, x - 21}{2 \, \sqrt{x}}}\;dx=\answer{\frac{28}{15} \, \sqrt{7} + \frac{238}{5} \, \sqrt{3}}\]
\end{problem}}%}

%%%%%%%%%%%%%%%%%%%%%%


\latexProblemContent{
\begin{problem}

Use the Fundamental Theorem of Calculus to evaluate the integral.

\expandafter\input{\file@loc Integrals/2311-Compute-Integral-0011.HELP.tex}

\[\int_{1}^{5} {-\frac{x^{2} - 81}{2 \, \sqrt{x}}}\;dx=\answer{76 \, \sqrt{5} - \frac{404}{5}}\]
\end{problem}}%}

%%%%%%%%%%%%%%%%%%%%%%


\latexProblemContent{
\begin{problem}

Use the Fundamental Theorem of Calculus to evaluate the integral.

\expandafter\input{\file@loc Integrals/2311-Compute-Integral-0011.HELP.tex}

\[\int_{7}^{10} {-\frac{x - 1}{2 \, \sqrt{x}}}\;dx=\answer{-\frac{7}{3} \, \sqrt{10} + \frac{4}{3} \, \sqrt{7}}\]
\end{problem}}%}

%%%%%%%%%%%%%%%%%%%%%%


\latexProblemContent{
\begin{problem}

Use the Fundamental Theorem of Calculus to evaluate the integral.

\expandafter\input{\file@loc Integrals/2311-Compute-Integral-0011.HELP.tex}

\[\int_{3}^{12} {\frac{x^{2} - 16}{\sqrt{x}}}\;dx=\answer{\frac{398}{5} \, \sqrt{3}}\]
\end{problem}}%}

%%%%%%%%%%%%%%%%%%%%%%


\latexProblemContent{
\begin{problem}

Use the Fundamental Theorem of Calculus to evaluate the integral.

\expandafter\input{\file@loc Integrals/2311-Compute-Integral-0011.HELP.tex}

\[\int_{8}^{11} {-\frac{x^{2} + 7 \, x + 10}{\sqrt{x}}}\;dx=\answer{-\frac{1796}{15} \, \sqrt{11} + \frac{2488}{15} \, \sqrt{2}}\]
\end{problem}}%}

%%%%%%%%%%%%%%%%%%%%%%


\latexProblemContent{
\begin{problem}

Use the Fundamental Theorem of Calculus to evaluate the integral.

\expandafter\input{\file@loc Integrals/2311-Compute-Integral-0011.HELP.tex}

\[\int_{7}^{13} {\frac{x + 8}{10 \, \sqrt{x}}}\;dx=\answer{\frac{37}{15} \, \sqrt{13} - \frac{31}{15} \, \sqrt{7}}\]
\end{problem}}%}

%%%%%%%%%%%%%%%%%%%%%%


\latexProblemContent{
\begin{problem}

Use the Fundamental Theorem of Calculus to evaluate the integral.

\expandafter\input{\file@loc Integrals/2311-Compute-Integral-0011.HELP.tex}

\[\int_{4}^{12} {\frac{x + 6}{4 \, \sqrt{x}}}\;dx=\answer{10 \, \sqrt{3} - \frac{22}{3}}\]
\end{problem}}%}

%%%%%%%%%%%%%%%%%%%%%%


\latexProblemContent{
\begin{problem}

Use the Fundamental Theorem of Calculus to evaluate the integral.

\expandafter\input{\file@loc Integrals/2311-Compute-Integral-0011.HELP.tex}

\[\int_{2}^{4} {\frac{x^{2} - 25}{8 \, \sqrt{x}}}\;dx=\answer{\frac{121}{20} \, \sqrt{2} - \frac{109}{10}}\]
\end{problem}}%}

%%%%%%%%%%%%%%%%%%%%%%


\latexProblemContent{
\begin{problem}

Use the Fundamental Theorem of Calculus to evaluate the integral.

\expandafter\input{\file@loc Integrals/2311-Compute-Integral-0011.HELP.tex}

\[\int_{8}^{10} {-\frac{x - 5}{5 \, \sqrt{x}}}\;dx=\answer{\frac{1}{15} \, \left(10 \, \sqrt{10}\right) - \frac{28}{15} \, \sqrt{2}}\]
\end{problem}}%}

%%%%%%%%%%%%%%%%%%%%%%


\latexProblemContent{
\begin{problem}

Use the Fundamental Theorem of Calculus to evaluate the integral.

\expandafter\input{\file@loc Integrals/2311-Compute-Integral-0011.HELP.tex}

\[\int_{6}^{14} {\frac{x - 3}{6 \, \sqrt{x}}}\;dx=\answer{\frac{1}{18} \, \left(6 \, \sqrt{6}\right) + \frac{5}{9} \, \sqrt{14}}\]
\end{problem}}%}

%%%%%%%%%%%%%%%%%%%%%%


