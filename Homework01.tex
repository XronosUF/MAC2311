\documentclass[•]{ximera}

%\usepackage{amsmath}

\title{Homework 1}
\author{people}

\begin{abstract}
Location here
\end{abstract}

\begin{document}
\maketitle

%%%%%%%%%%%%%%%%%%%%%%
%\tagged{Cat@One, Cat@Two, Cat@Three, Cat@Four, Cat@Five, Ans@ShortAns, Type@Compute, Topic@Limit, Sub@Rational}{

\begin{problem}
Determine if the limit approaches a finite number, infinity, or does not exist. (If the limit does not exist, write DNE)
\[\lim_{x\to{{-4}}}\dfrac{{{x^{2} + 2 \, x - 8}}}{{{x - 2}}}=\answer{0}\]
\end{problem}%}
%%%%%%%%%%%%%%%%%%%%%%



%%%%%%%%%%%%%%%%%%%%%%
%\tagged{Cat@One, Cat@Two, Cat@Three, Cat@Four, Cat@Five, Ans@ShortAns, Type@Compute, Topic@Limit, Sub@Rational}{

\begin{problem}
Determine if the limit approaches a finite number, infinity, or does not exist. (If the limit does not exist, write DNE)
\[\lim_{x\to{{-2}}}\dfrac{{{x^{2} + 3 \, x + 2}}}{{{x + 1}}}=\answer[tolerance=0.1]{0}\]
\end{problem}%}
%%%%%%%%%%%%%%%%%%%%%


%%%%%%%%%%%%%%%%%%%%%%
%\tagged{Cat@One, Cat@Two, Cat@Three, Cat@Four, Cat@Five, Ans@ShortAns, Type@Compute, Topic@Limit, Sub@Rational}{

\begin{problem}
Determine if the limit approaches a finite number, infinity, or does not exist. (If the limit does not exist, write DNE)
\begin{expandable}
For this problem, consider factoring the numerator and denominator.  Then simplify the fraction to make computing the limit easier.
\begin{expandable}
MORE EXPLOSIONS!!!!
\end{expandable}
\end{expandable}


\[\lim_{x\to{{-3}}}\dfrac{{{x^{2} + x - 6}}}{{{x^{2} + 5 \, x + 6}}}=\answer{{{5}}}\]
\end{problem}%}
%%%%%%%%%%%%%%%%%%%%%








\end{document}
