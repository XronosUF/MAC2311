\documentclass[]{ximera}
%\usepackage{PackageLoader}
%\Verbosetrue
%\maketagssuperstrict
%\maketagsstrict
%\maketagsrelax
%\makeatletter

\title{Xronos Introduction}
\author{Xronos UF}
\date{\today}
\begin{abstract}
Abstract
\end{abstract}

%\newcounter{testcounter}
%\setcounter{testcounter}{0}
\begin{document}
\maketitle


{\huge Welcome to Xronos!!}\\
\large
This intro will show you the basics of how to use the Xronos HW system.  It works like many other online HW systems, but is homegrown and tailored to UF classes.  Let's look at some examples:


\begin{problem}

Is the following multiplication correct?
\[(x+2)^2 = x^2+4\]
\begin{multipleChoice}
\choice{Yes!}
\choice[correct]{No!}
\end{multipleChoice}
\end{problem}

\begin{problem}
What is the correct multiplication?
\[(x+2)^2=\answer{x^2+4x+4}\]


\end{problem}


What about using interesting symbols?

\begin{problem}
Simplify the expression $\sqrt{8}$.  Hint:  Try entering in ``sqrt''... 

\[\sqrt{8}=\answer{2\sqrt{2}}\]

\end{problem}

\begin{problem}
What is the area of a circle with radius 5? \\ Hint: Tasty dessert without an ``e''... \\

The area of the circle is $\answer{25\pi}$.

\end{problem}


Most other symbols are fairly intuitive, and most can be checked.  Look at what happens when you start typing!  A little bubble tells you what is being written!  Pretty helpful!  Feel free to ask if you're still not sure.\\

Now lets look at some actual HW problems:

\begin{problem}
Determine if the limit approaches a finite number, infinity, or does not exist. (If the limit does not exist, write DNE)
\input{./Limits/2311-Compute-Limit-0005.HELP.tex}
\[\lim_{x\to{-2}}\dfrac{x^{2} + 7 \, x + 10}{x^{2} + 5 \, x + 6}=\answer{3}\]
\end{problem}


\begin{problem}
Determine if the limit approaches a finite number, infinity, or does not exist. (If the limit does not exist, write DNE)
\input{./Limits/2311-Compute-Limit-0007.HELP.tex}
\[\lim_{x\to{+\infty}}\dfrac{x - 5}{x^{3} - 11 \, x^{2} + 39 \, x - 45}=\answer{0}\]

\end{problem}


\begin{problem}
Determine if the limit approaches a finite number, infinity, or does not exist. (If the limit does not exist, write DNE)
\input{./Limits/2311-Compute-Limit-0007.HELP.tex}
\[\lim_{x\to{-\infty}}\dfrac{x^{3} - 7 \, x^{2} + 8 \, x + 16}{x^{3} - 9 \, x^{2} + 24 \, x - 16}=\answer{1}\]

\end{problem}

\begin{problem}
Determine if the limit approaches a finite number, infinity, or does not exist. (If the limit does not exist, write DNE)
\input{./Limits/2311-Compute-Limit-0007.HELP.tex}
\[\lim_{x\to{-\infty}}\dfrac{x^{3} + 6 \, x^{2} + 12 \, x + 8}{x^{2} + 6 \, x + 8}=\answer{+\infty}\]

\end{problem}








\end{document}
